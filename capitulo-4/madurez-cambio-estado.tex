\subsection{Madurez y cambio de estado}
\label{subsec:cap4-madurez-cambio-estado}
La madurez de los individuos pertenecientes a las poblaciones de $H(x,y)$, $L(x,y)$ y $P(x,y)$
indica la proximidad de que estos alcancen el siguiente estado de su etapa de desarrollo.

Sea $R(k_{i})$ la tasa de desarrollo (ver ecuación \eqref{eq:schoolfield}) de un individuo para
una temperatura de $k_{i}$ Kelvin en un instante $i$, se considera que ha alcanzado su máximo nivel
de madurez y se encuentra listo para pasar al siguiente estado de su cuando :

\begin{equation}
\label{eq:condicion-cambio-estado}
    \sum_{i=0}^{N} R(k_{i}) \geq 1
\end{equation}

Donde $N$ es la cantidad de días que le toma al individuo pasar de un estado a otro. Para $R(k)$ se
se aplican los parámetros correspondientes al estado actual del individuo. De forma extendida,
para que un individuo,

\begin{description}[style=multiline,leftmargin=1.5cm]
\item[$h(x,y)$], perteneciente a $H(x,y)$ pase a ser una larva, $l(x,y)$, que forma
parte de $L(x,y)$ debe cumplirse \eqref{eq:condicion-cambio-estado}.

\item[$l(x,y)$], perteneciente a $L(x,y)$ pase a ser una pupa, $p(x,y)$, que forma
parte de $P(x,y)$ debe cumplirse \eqref{eq:condicion-cambio-estado}.

\item[$p(x,y)$], perteneciente a $P(x,y)$ pase a ser un adulto, $a(x,y)$, que forma
parte de $A(x,y)$ debe cumplirse \eqref{eq:condicion-cambio-estado}.
\end{description}




