\subsection{Zonificación}
\label{subsec:cap4-zonificacion}
Cada entorno puede contar con factores que lo hagan más o menos apto para el desarrollo,
mortalidad, alimentación, disperción, y reproducción de individuos. En esta sección, con el fin de
simplificar ciertos aspectos muy especificos que se encuentran fuera del alcance de este trabajo,
realizaremos ciertas hipótesis generales, justificadas para nuestro caso de aplicación, pero puede
requerir una revisión en el caso general. Estas hipotesis son, los valores observados en un
conjuto de puntos de control, pertenecientes a una zona, permiten la caracterización de dicha zona como más o menos apta para desarrollo, mortalidad, alimentación, disperción, y reproducción de
individuos. Tambien consideramos que el tamaño de la zona, y por ende la cantidad de puntos de
control que pertenecena ella, influye en la caracterización de las zonas.

La zonificación surge ante necesidad de dividir el espacio de estudio de una forma más granular,
para identificar a los individuos que pertenecen a zonas aptas y los que no. Los puntos de control
distribuidos en un área de estudio nos permite estimar cuales zonas los niveles de riesgo e
infestación correspondiente a la abundacia de larvas por litro que se pueden observar.

En la \tabref{tab:cap4-puntaje-zona} se puden observar los rangos definidos para cada tipo de
zona, en donde $u(x,y)$ es la densidad relativa en un radio, $r$ , donde el valor de la densidad
realtiva es calculado mediante la ecuación \eqref{eq:interpolacion-idw}. Las límites para las
zonas fueron determinados clasificando los valores, de las hembras reproductivas en, grupos
múltiplos de cinco. No se estableció  un límite superior para las zonas óptimas debido a que los
valores mayores a el mínimo establecido, 70 larvas por litro, pertenecen a la misma categoría.

El tamaño del radio,$r$, es un parametro ajustable del modelo, mientras más grande sea el tamaño
del radio, más puntos serán incluidos para el cálculo, lo que gerará que las zonas tiendan a ser
similares.
Para el calculo de las hembras adultas y reproductivas para la clasificación de las zonas se
tuvieron en cuenta los siguientes criterios:

\begin{itemize}
    \item Solo el 50 \% de las larvas observadas son hembras.
    \item La temperatura media utilizada es de 25 \textcelsius.
    \item La mortalidad díaria natural bajo optimas condiciones,a 25 \textcelsius, es de 0,01
    según \eqref{eq:mortalidad-natural-larvas}.
    \item La tasa de desarrollo, a 25 \textcelsius, de la larva hasta su emergencia a adulto es de
    $11.57$ días \citep{rueda1990temperature}.
    \item El 32,10 \% de las hembras adultas no ovipone \citep{osoriopontificia}.
\end{itemize}


\begin{table}
    \begin{minipage}{\textwidth}
\begin{center}
    \caption{\label{tab:cap4-puntaje-zona} Clasificación de las zonas de acuerdo a la densidad de larvas por litro.}
    \begin{tabular}{p{3cm} c c c c}
        \\
                     & Mínimo$^a$ & Máximo$^a$ & Hembras     & Hembras$^c$ \\
        Tipo de zona & $u(x,y)$   & $u(x,y)$   & Adultas$^b$ & Reproductivas$^c$ \\
        \hline
        \hline\\
        Pésima  & 0  & 19 & 8  & 5 \\
        Mala    & 20 & 35 & 15 & 10\\
        Regular & 36 & 51 & 22 & 15\\
        Buena   & 52 & 69 & 30 & 20\\
        Óptima  & 70 & --$^d$ & --$^d$ & --$^d$\\
    \end{tabular}
    \footnotetext[1]{Rango mínimo y máximo de $u(x,y)$ permitido para el tipo de zona.}
    \footnotetext[2]{Cantidad máxima de hembras adultas, al final del periodo de desarrollo.}
    \footnotetext[3]{Cantidad de hembras adultas con capacidad de oviponer.}
    \footnotetext[4]{No se estableció un límite superior para las zonas óptimas. }
\end{center}
    \end{minipage}
\end{table}


\subsection{Sitios de reproducción}
\label{subsec:cap4-sitios de reproduccion}
Sea $BS$ el número de sitios de reproducción agrupados como una sola. La $BS$ variable ambiental
determina el tamaño de la población de equilibrio en el modelo determinista
\citep{otero2006stochastic}. Las diferentes condiciones ambientales deben ser
representadas por diferentes valores del parámetro de $BS$ \citet{otero2006stochastic}.

Se considera a $BS(x,y)$ como el valor de $BS$, asociado a $(x,y)$, partiendo de las hipotesis
realizadas en la \secref{subsec:cap4-zonificacion} podemos considerar que el valor de $BS(x,y)$ se
encuentra influenciado por la densidad $u(x,y)$, de ese modo a medida que $u(x,y)$ varie, lo debe
hacer el valor de $BS(x,y)$. Para dicha estimación utilizamos la interpolación de Lagrange.

Sea $bs$ la función a interpolar, sean $u_0$, $u_1$,...,$u_m$ las densidades conocidas de $bs$ y
sean $bs_0$, $bs_1$,...,$bs_m$ los valores que toma la función para dichas densidades, el polinomio interpolador de grado n de Lagrange es un polinomio de la forma :

\begin{equation}
\label{eq:sitios-reproduccion-x-y}
    bs(u(x,y)) = \sum_{i=0}^{n} bs_{i} * l_{i}(u(x,y))
\end{equation}

donde $l_j(u(x,y))$ son los llamados polinomios de Lagrange, que se calculan de este modo:

\begin{equation}
\label{eq:sitios-reproduccion-x-y}
    l_{i}(u(x,y)) = \prod_{j \neq i} \cfrac{u(x,y) - u_{j}}{u_{i} - u_{j}}
\end{equation}

Consideramos un polinomio de grado 3, con los para $u_0$, $u_1$ y $u_2$ igual a $19$, $51$ y $70$,
correspondientes a zonas del tipo pésima, regular y óptima. Los valores conocidos de las densidades
$19$, $51$ y $70$ son $bs_{min}$, $bs_{med}$ y $bs_{max}$, estos son parámetros configurables del
modelo donde $bs_{min}$ representa el menor $BS$ observado, $bs_{max}$ representa el mayor $BS$
observado y  $bs_{med}$ es el valor medio exitente entre $bs_{max}$ y $bs_{min}$.
