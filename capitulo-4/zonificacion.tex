\subsection{Zonificación}
\label{subsec:cap4-zonificacion}
Cada entorno puede contar con factores que lo hagan más o menos apto para el desarrollo,
mortalidad, alimentación y reproducción de individuos. La zonificación surge ante necesidad de
dividir el espacio de estudio de una forma más granular, para identificar a los individuos que
pertenecen a zonas aptas y los que no.

Los puntos de control distribuidos en un área de estudio nos permite estimar cuales zonas los
niveles de riesgo e infestación correspondiente a la abundacia de larvas por litro que se pueden
observar.

La clasificación de las zonas se realizó teniendo en cuenta los siguientes criterios,
\begin{itemize}
    \item Solo el 50 \% de las larvas observadas son hembras.
    \item La temperatura media utilizada es de 25 \textcelsius.
    \item La mortalidad díaria natural bajo optimas condiciones es de 0,01 según
    \eqref{eq:mortalidad-natural-larvas}
    \item La tasa de desarrollo de la larva hasta su emergencia a adulto es de $11.57$ días
    \citep{rueda1990temperature}.
    \item El 32,10 \% de las hembras adultas no ovipone \citep{osoriopontificia}.
\end{itemize}

En la \tabref{tab:puntaje-zona} se puden observar los rangos definidos para cada tipo de zona, En
donde $u(x,y)$ es la densidad relativa en un radio de 250 metros, cuyo valor es calculado mediante
la ecuación \eqref{eq:interpolacion-idw}. El tamaño del radio es un parametro ajustable del
modelo, mientras más grande sea el tamaño del radio, más puntos serán incluidos para el cálculo,
lo que gerará que las zonas tiendan a ser similares.

\begin{table}
\centering
    \label{tab:puntaje-zona}
    \begin{tabular}{p{3cm} p{5cm}}
        Tipo de zona & Rango \\
        \hline
        Óptima & $67 \leq u(x,y)$ \\
        Buena  & $41 \leq u(x,y) < 60 $  \\
        Normal & $23 \leq u(x,y) < 40$\\
        Mala   & $11 \leq u(x,y) < 22$\\
        Pésima &  $0 \leq u(x,y) < 10 $ \\
    \end{tabular}
    \caption{Clasificación de las zonas de acuerdo a la densidad de larvas por litro}
\end{table}
