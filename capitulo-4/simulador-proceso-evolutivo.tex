\section{Simulador del proceso evolutivo}
\label{sec:cap4-simulador-proceso-evolutivo}
La simulación del prorceso evolutivo del mosquito del aedes aegypti requiere principalmente los
puntos de control pertenecientes el área de estudio y los datos climatológiocs del correspondientes
a dicha área. El proceso evolutivo consta de varios pasos, el desarrollo, la mortalidad,
la ovipostura y dispersión de los individuos de la población.

El proceso inicia al procesar los puntos de control pertenecientes a la muestra utilizada para el
estudio. La población inicial se encuentra compuestas por las larvas observadas en la muestra
utilizada para el estudio.

La población de estudio, $\vec{m}$, se encuentra compuesta por un conjunto $m_{j}(x,y)$ que representan a los individuos, $h_{j}(x,y)$, $l_{j}(x,y)$, $p_{j}(x,y)$, $am_{j}(x,y)$,
$an_{j}(x,y)$ y $ap_{j}(x,y)$ que corresponden a las poblaciones de $H(x,y)$, $L(x,y)$, $P(x,y)$,
$AM(x,y)$, $AN(x,y)$ y $AP(x,y)$ respectivamente. Todos los individuos,$m_{j}(x,y)$, que se
encuenten en una misma $(x, y)$ pertenencen a una misma colonia.

El periodo de estudio, $\vec{k}$, se ecuentra conformado por un conjuto de temperaturas $k_{i}$
que representan las temperaturas diarias.

\begin{algorithm}
\caption{Simulación del proceso evolutivo}
\label{alg:simulador-evolutivo}
\begin{algorithmic}[1]
    \REQUIRE $\vec{k}\neq \emptyset \land \vec{m} \neq \emptyset$
    \ENSURE $\vec{m'}$
    \FORALL{$k{i} \in \vec{k}$ }
        \STATE $\vec{huevos} \Leftarrow \emptyset$
        \FORALL{$m_{j}(x, y) \in \vec{m}$}
            \STATE $desarrollar(m_{j}(x, y), k_{i})$
            \IF{$regular(m_{j}(x, y), k_{i})$}
                \STATE \COMMENT{Se elimina $m_{j}$ si es un candidato.}
                \STATE $m_{j}(x, y) \Leftarrow \varnothing $
            \ELSIF{$esta\_maduro(m_{j}(x, y), k_{i})$}
                \STATE $ cambiar\_estado(m_{j}(x, y)) $
            \ELSIF{$se\_reproduce(m_{j}(x, y), k_{i})$}
                \STATE $\vec{huevos} \Leftarrow \vec{huevos} + oviponer(m_{j}(x, y))$
            \ENDIF
        \ENDFOR

        \IF{$\vec{huevos} \neq \emptyset$}
            \STATE \COMMENT{Si ovipone se extiende la población}
            \STATE $\vec{m} \Leftarrow  \vec{m} + \vec{huevos}$
        \ENDIF
    \ENDFOR
    \RETURN $\vec{m}$
\end{algorithmic}
\end{algorithm}

El \algref{alg:simulador-evolutivo}, describe al simulador como un proceso iterativo cuyo objetivo
es simular los efectos de cada $k_{i}$ para cada individuo $m_{j}$ que pertenezca a la población.

El proceso $desarrollar(m_{j}, k_{i})$, es el proceso por el cual se calculan las tasas de
desarrollo correspondientes para cada $m_{j}$ . Estas son obtenidas medidante el modelo de
\cite{sharpe1977reaction} presentado en la \secref{subsec:cap4-tasas de desarrollo}.

El desarrollo en las etapas inmaduras ($H(x,y)$, $L(x,y)$ y $P(x,y)$) es realizado con el fin de
simular los efectos de $k_{i}$ en la maduración de $m_{j}$. Sin embargo para los adultos ($AN(x,y)$
y $AP(x,y)$), es realizado con el fin de simular los efectos de $k_{i}$ en la duración del ciclo
gonotrófico de $m_{j}$ y la dispersión del vector, presentada en la
\secref{subsec:cap4-vuelo-dispersion}.

El proceso $regular(m_{j}, k_{i})$ como principal objetivo, determinar la cantidad de individuos
que deben ser eliminados de la población, debido a la mortalidad diaria y seleccionar los
candidatos para dicha eliminación. Como se mencionó anteriormente, en la
\secref{subsec:cap4-mortalidad}, la tasa de mortalidad diaria depende del estado en el que
encuentre el inidividuo. La selección de canidatos para la eliminación se realiza de forma
aleatoria para los individuos que provenientes de una misma colonia $(x, y)$.

El proceso $esta\_maduro(m_{j}, k_{i})$ se encarga de validar que individuo haya completado su
desarrollo, una vez completo se procede a realizar el cambio de estado mediante el proceso
$cambiar\_estado(m_{j})$. El proceso de maduración y cambio de estado fue presentado previamente
en la \secref{subsec:cap4-madurez-cambio-estado}.

El proceso $se\_reproduce(m_{j}, k_{i})$, que solo es válido para hembras adultas se encarga de
validar $m_{j}$ haya completado su ciclo gonotrófico, una vez finalizado debe generar una tanda de
huevos mediante $oviponer(m_{j}$. La duración del ciclo gonotrófico y la oviposutra fue presentado
anteriormente en la \secref{subsec:cap4-ciclo-gontrofico-ovipostura}.
