%!TEX root = ../tesis.tex
\section{Caracteristicas Biológicas}
\label{sec:caracteristicas-biologicas}

El ciclo de vida de Ae. aegypti comprende el estado de huevo, cuatro
estadios larvales, la pupa y el adulto.

\subsection{Huevos}
El desarrollo embriológico generalmente se completa en 48 horas
si el ambiente es húmedo y cálido, pero puede prolongarse hasta
5 días a temperaturas más bajas.

1) Los huevos son depositados individualmente en las paredes de los
recipientes por encima del nivel del agua.
2) Una vez que se ha completado el desarrollo embrionario los
huevos son capaces de resistir largos períodos de desecación,
que pueden prolongarse por más de un año.
3) Cuando los huevos son mojados, se genera un estímulo para la eclosión.
4) Algunos huevos hacen eclosión en los primeros 15 minutos de
contacto con el agua, al tiempo que otros pueden no responder hasta
que han sido mojados varias veces (Nelson, 1958).


\subsection{Larvas}
El desarrollo larval a 14 C es irregular y la mortalidad relativamente
alta. Por debajo de esa temperatura, las larvas eclosionadas no alcanzan
el estado adulto. Experimentalmente se determinó que el desarrollo cero
se sitúa en 13,3oC, con un umbral inferior de desarrollo ubicado entre
9 y 10oC y una constante térmica de 2.741 grados día, considerando el
rango comprendido entre 16 y 32oC. Por su parte, la temperatura más alta
que permite el desarrollo es 36oC, con una menor duración del estado
larval que a 30-34oC (Bar-Zeev, 1958). Christophers (1960) señala que la
actividad del insecto disminuye abruptamente por debajo de 15oC hasta
inhibición bajo medias diarias de 12oC.

En condiciones óptimas el período larval puede durar 5 días pero
comúnmente se extiende de 7 a 14 días.

\subsection{Pupa}
El estado de pupa demora de 2 a 3 días. Siccha y Pérez (2006) indicaron
una duración del ciclo de 26,83
(10-47) días a 20oC, de 17,59 (9-29) días a 25oC, y de 9,75 (5-16) días
a 30oC. La duración promedio de los cuatro estados larvales sucesivos
y la pupa, expresada como porcentaje del tiempo ocupado por la larva
hasta que llega a adulto, son 14,6, 13,9, 17,5, 33,3 y 20,6, respectivamente
(Baz-Zeed, 1958).

\subsection{Siccha & Pérez (2006)}
Se registró los días que
demoraron en cambiar cada uno de los estadíos larvales y pupa hasta la
emergencia de los adultos. La duración del ciclo fue 26.83 (10-47) días
para 20ºC, 17.59 (9-29) para 25ºC, y 9.75 (5-16) días para 30ºC.

\subsection{ Temperature-Dependent Development and Survival Rates of
Culex quinquefasciatus tus and Aedes aegypti (Diptera: Culicidae)}

Los porcentajes (y rango) del tiempo total de desarrollo, en todo el rango
de temperaturas, que pasó en cada etapa inmadura de Cx. quinquefasciatus
fueron: 14\% (11,9 a 18,4) para el primer estadio, el 13\% (10,1 a 14,5)
para el segundo estadio, el 17\% (14,7 a 20,4) en el tercer estadio, el
33\% (27,3 a 37,9) en el cuarto estadio, y el 23\% (19,2-26,2) para la
fase de pupa.

Para Ae. aegypti, los porcentajes fueron: 19\% (13,9 a 23,1) para el primer
estadio, el 14\% (11,3 a 21,3) para el segundo estadio, el 17\% (13,0 a 27,0)
en el tercer estadio, el 27\% (24,2 a 28,5) de cuarto estadio y el 23\%
(15,3 a 28,5) para la fase de pupa.


\subsection{Vuelo de un mosquito adulto}
Los machos rondan como voladores solitarios aunque es más común
que lo hagan en grupos pequeños (Bates, 1970; Kettle, 1993) atraídos
por los mismos huéspedes vertebrados que las hembras.

Vuelan en sentido contrario al viento, desplazándose mediante lentas
corrientes de aire, siguen los olores y gases emitidos por el
huésped (CO2), al estar cerca utilizan estímulos visuales para
localizarlo mientras sus receptores táctiles y térmicos las guían
hacia el sitio donde se posan.

Por lo general, la hembra de Ae. aegypti no se desplaza más allá de
5,000 m de distancia de radio de vuelo en toda su vida, permanece
físicamente en donde emergió, siempre y cuando no halla algún factor
que la perturbe o no disponga de huéspedes, sitios de reposo y de
postura. En caso de no haber recipientes adecuados, la hembra grávida
es capaz de volar hasta tres kilómetros en busca de este sitio.
Los machos suelen dispersarse en menor magnitud que las hembras

En un ensayo de laboratorio, con viento en calma mosquitos
Ae. aegypti se desplazan a 17 cm/s, al introducir viento en contra
de 33 cm/s, incrementaron su velocidad para contrarrestarle disminuyendo
su avance a 16 cm/s, lo que implica un esfuerzo de desplazamiento
como si hubieran volado a 49 cm/s, por consiguiente volar con
viento de mayor velocidad le representa un mayor esfuerzo y suelen
hacerlo (Kettle, 1993). Cuando los mosquitos están inactivos, sin
aparearse, alimentarse, ni desplazándose, buscan sitios oscuros y
tranquilos para reposar en superficies verticales o debajo de algún
mueble, frecuentemente en el interior de la vivienda, donde encuentran
condiciones adecuadas de iluminación, temperatura, humedad y viento,
evitando cualquier factor que los perturben (Nelson, 1986).
