\section{Larvitrampas}
\label{sec:densidad-vectorial-larvitrampas}
Antes de la utilización de la larvitrampa, ésta debe cepillarse y flamearse,
luego mantenerla sumergida en agua durante no menos de tres días, para
asegurarse que el agua no contenga residuos de sustancias que puedan actuar
como larvicida. De esta manera, además, se garantiza la destrucción de
algún huevo del mosquito que estuviese previamente en el neumático o en
larvitrampas ya utilizadas.

\subsection{Especificaciones para la colocación e inspección}
Instalarla a una altura de 50 cm (del suelo a la base de la larvitrampa).
Protegerla de la luz directa del sol, el aire, la lluvia, en lugares a
media luz o completamente a la sombra. No deben ubicarse cercanas a depósitos
de agua. Debe evitarse su colocación en lugares completamente pavimentados,
u otros que tengan mucha refracción de la luz. Debe estar visible para la
hembra del mosquito. Protegerla de niños y animales domésticos (perros,
gatos, roedores, etc.)

\subsection{Forma de revisión}
Se establece una rutina semanal para revisar las larvitrampas, para lo
cual, una vez por semana debe vaciarse todo su contenido cuidadosamente
(para que no quede ninguna larva en sus paredes) en un recipiente adecuado
para realizar la inspección. En caso de ser positivas, se registra como
tal y las larvas serán colectadas en tubos para ser enviadas al laboratorio
para su determinación taxonómica. Luego, el dispositivo se lava y se acondiciona
para ser colocadas nuevamente siguiendo las especificaciones ya descritas.

\subsection{Consideración final}
Tener en cuenta que en verano, con condiciones más favorables para el
desarrollo de esta especie, las larvas pueden alcanzar el estadio de
adulto entre 6 y 7 días desde la ovipostura, por lo que es necesaria la
inspección de todas las larvitrampas en los tiempos indicados a fin de
evitar que alguna de ellas se transformen en criaderos de adultos.
