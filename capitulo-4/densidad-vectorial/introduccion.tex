\section{Introducción}
\label{sec:densidad-vectorial-introduccion}

Tres indicadores entomológicos son recomendados por la OMS (Organización
mundial de la salud) para estimar la densidad del vector del dengue: El
Índice de Casa, Índice de Recipiente e Índice de Breteau. Muchos
programas de control del dengue usan los índices larvarios como indicadores
de las densidades poblaciones de Aedes aegypti, para dirigir y focalizar
espacial y temporalmente las acciones de control del vector. Sin embargo:

Los indicadores tradicionales son poco confiables porque
\begin{itemize}
    \item Se basan en búsqueda de fases inmaduras por lo que representan
        una estimación indirecta de las poblaciones de mosquitos adultos.
    \item No reflejan la asociación que existe entre las densidades de
        mosquitos o cantidad y/o tipo de recipientes presentes, con los
        riesgos de transmisión de dengue.
    \item Proporcionan poca o nula información de aquellas viviendas en
        las que existe un mayor riesgo de presencia de mosquitos.
\end{itemize}

Además estos indicadores no reflejan las poblaciones de adultos ni estiman
riesgo entomológico, lo cual es muy importante en la transmisión del dengue.
Por lo cual, a la fecha sólo son recomendados para detectar la calidad de
las acciones (control de calidad) realizadas por el personal de control
larvario. Existen numerosos métodos e indicadores para determinar las
poblaciones de Aedes aegypti en la etapa de huevo, larva, pupa o adulto;
uno de los métodos más prácticos, eficientes y económicos es el monitoreo
de poblaciones de este vector por medio de ovitrampas. Las ovitrampas han
sido usadas desde 1965 en la vigilancia del Aedes aegypti (L), como un
instrumento para determinar la distribución del mosquito, medir la fluctuación
estacional de las poblaciones y para evaluar la eficacia de la aplicación
de insecticidas; además, como una estrategia de muestreo presencia-ausencia,
lo cual permite una estimación de la densidad mediante la proporción de
muestras positivas y son especialmente útiles para la detección temprana
de reinfestaciones. \cite{cenaprece2013}

Las técnicas tradicionales de vigilancia de A. aegypti usan los índices
aédicos de recipientes, de viviendas y de Breteau para determinar el grado
de infestación, dispersión y densidad del mosquito en una zona y tiempo
determinados. Estos índices se fundamentan en la detección visual de formas
inmaduras del vector dentro de recipientes domésticos, técnica considerada
poco sensible por la habilidad de las larvas para escapar y su capacidad de
permanecer sumergidas por largos períodos de tiempo (3, 5). Asi mismo, la
proporción de viviendas y recipientes infestados con A. aegypti no provee
información fehaciente sobre la densidad poblacional al registrar como
positivo un recipiente o casa sin tener en cuenta la cantidad de formas
inmaduras presentes, lo cual quiere decir que para el índice es igual
si hay una o cientos de ellas.
