\section{Introducción}
\label{sec:densidad-vectorial-introduccion}
Las técnicas tradicionales de vigilancia de A. aegypti usan los índices
aédicos de recipientes, de viviendas y de Breteau para determinar el grado
de infestación, dispersión y densidad del mosquito en una zona y tiempo
determinados. Estos índices se fundamentan en la detección visual de formas
inmaduras del vector dentro de recipientes domésticos, técnica considerada
poco sensible por la habilidad de las larvas para escapar y su capacidad de
permanecer sumergidas por largos períodos de tiempo (3, 5). Asi mismo, la
proporción de viviendas y recipientes infestados con A. aegypti no provee
información fehaciente sobre la densidad poblacional al registrar como
positivo un recipiente o casa sin tener en cuenta la cantidad de formas
inmaduras presentes, lo cual quiere decir que para el índice es igual
si hay una o cientos de ellas [1].
