\section{Ovitrampa}
\label{sec:densidad-vectorial-ovitrampa}
Son recipientes que ofrecen a las hembras de Aedes aegypti un lugar colocar
los huevos. Detecta la presencia de huevos y por lo tanto actividad de
ovipostura. Las ovitrampas consisten en frascos de plástico o pequeñas
macetas plásticas de unos 500 ml de color oscuro preferentemente, en cuyo
interior, se coloca una pieza plana de madera (baja-lengua o similar).
Asimismo, también pueden construirse con un pote de vidrio de boca ancha,
de aprox. medio litro, pintado de negro por fuera y equipado con una paleta
de cartón o madera (baja-lengua) sujeta verticalmente al interior, con su
lado áspero mirando hacia adentro. Las dimensiones del recipiente no son
críticas pero todos los frascos a usar en un estudio particular deben ser
idénticos. Al frasco se le deberá agregar 250 ml de agua limpia.


\subsection{Especificaciones para la colocación e inspección}
La colocación debe realizarse en lugares representativos del municipio,
especialmente en las zonas donde se produjeron casos de dengue autóctonos
o importados. Respecto al número de ovitrampas a colocar, se sugiere no
menor a 10 por localidad. La idea es mantener el mismo circuito (mismos
lugares de colocación), un modelo a “escala ciudad”, para tener la idea de
la "presencia" relacionada con la distribución geográfica del vector, se
basa en el criterio que la información sea independiente. O sea que sea
improbable (más bien imposible) que una hembra pueda poner huevos en dos
ovitrampas contiguas. Además la instalación debería basarse en la capacidad
operativa de trabajo, y para ello se pueden colocar las ovitrampas en una
grilla con puntos más o menos equidistantes de aproximadamente 400 metros
de lado.

Cada ovitrampa se coloca en un lugar accesible, protegido donde predomine
la sombra y haya cierto grado de humedad (ambiente sombreado). Debe asegurarse
la presencia de moradores al retirarla.Sobre un plano de la localidad o
sector a muestrear se seleccionarán los puntos donde se colocarán las
ovitrampas. Una variante sería colocar una por Unidad Sanitaria que el
municipio posea, asumiendo que la ubicación de las mismas brindará una
visión representativa del conjunto. Conviene tener presente que en este
caso, el muestreo puede no ser representativo de viviendas regulares.

Las ovitrampas deben ser inspeccionadas semanalmente y en el caso de detectar
paletas con huevos, cuando no puedan ser leídos en el nivel local, se deberán
remitir para su lectura a los laboratorios de entomología más cercanos,
(Divisiones de Zoonosis Urbanas, División de Zoonosis Rurales, CEPAVE, etc.).
La remisión será en un sobre o bolsita plástica, con los datos para georreferenciar.
La vigilancia entomológica se debe realizar en forma continua anual. Es
importante destacar que una vez detectada la presencia de Aedes Aegypti por
cualquiera de los sistemas de monitoreo (larvitrampas u ovitrampas) se deben
realizar las acciones inmediatas de control focal en la comunidad.

\subsection{Consideración final}
Es importante añadir un identificador a cada ovitrampa que permita la
identificarlas fácilmente. El rótulo se debe colocar sobre la baja-lengua
o paleta de la ovitrampa, debe estar debidamente escrito (con lápiz) el
número y/o código de la ovitrampa. También se rotulará el frasco sobre
su pared con tinta indeleble. Se recomienda numerar cada una de las paletas
o baja-lenguas y agregarle iniciales para identificar el municipio y
detallar en el protocolo común los datos de cada una (lugar físico por.
ej. calle, barrio y zona del municipio como también la fecha del retiro
de las mismas de su lugar para su posterior envío).
