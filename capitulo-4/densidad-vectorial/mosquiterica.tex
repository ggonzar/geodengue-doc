\section{Mosquitérica genérica}
Es un dispositivos de oviposturas más recientes, cuenta con un sencillo
diseño y de fácil fabricación debido a que los materiales necesarios para
su construcción son fácilmente accesibles.

Los materiales para la construcción son los siguientes :
\begin{itemize}
    \item Una botella de plástico de 2 litros.
    \item Tijeras.
    \item Una lija para madera.
    \item Un rollo de cinta aislante.
    \item Una pieza de tul o gasa para sellar la boquilla (cuello).
    \item Un poco de arroz o alpiste.
    \item 250 ml de agua no clorada.
\end{itemize}

Hay que cortar el cilindro en dos partes, de modo que la porción de la
boca quede en forma contraria, formando un embudo. Debe retirarse el tapón
de la botella. Asimismo, se debe retirar con cuidado el anillo de precinto
y almacenarlo, también se utilizará en la mosquitérica.

Se debe lijar bien dentro del "embudo". Esto servirá para aumentar el área
de evaporación, y será más fácil para el mosquito localizar la  trampa.
Hay que colocar la  tela en el cuello y asegurarlo con el anillo. Hay que
tener en cuenta que debe existir un tejido muy fino, de modo que las larvas
no puedan pasar;


Hay que colocar en la parte inferior de la botella la comida que se eligió,
puede ser granos de alpiste o tres granos de arroz, pero siempre triturados.
Se inserta la porción de cuello hacia abajo sobre la parte inferior de la
botella; Se usa cinta adhesiva para asegurar las dos partes, sobre el lado
exterior.

Se pone el agua sin cloro en la mosquitérica a pocos centímetros por encima
del cuello. Si no dispone de agua sin cloro, se usa agua de tomar del grifo
y se deja que repose durante dos días.


\subsection{Especificaciones para la colocación e inspección}
Las técnicas de colocación e inspección mencionadas anteriormente se aplican
a este método.

\subsection{Consideración final}
Actualmente la mosquitérica es utilizada en reemplazo a los antiguos
dispositivos de ovipostura en los últimos trabajos realizados en sudamérica
para el control del vector del dengue por presentar resultados más exactos.
