\section{Índices de Stegomia}
\label{sec:densidad-vectorial-indices-stegomia}

La Organización mundial de la salud (OMS) recomienda la utlización de tres
indicadores entomológicos, generalmente conocidos como índices de Stegomia,
para estimar la densidad del vector, Índice de casas (I.C.), Índice de
Recipientes (I.R.) e Índice de Breteau (I.B.).Estos indices son calculados a
partir de muestreos de larvas y recipientes.

\subsubsection{Índice de Casas}
El índice de Casas (I.C.) es el valor numérico que especifíca el porcentaje
de viviendas infestadas con larvas, pupas o ambos estados de desarrollo
del mosquito transmisor del dengue. Para este indice, se analizan los
contenedores de las viviendas y alrededores.

\begin{equation}
I.C. = \frac{\text{número de casas infestadas} * 100}{\text{número de casas inspecionadas}}
\end{equation}

Donde :
\begin{itemize}
\item El número de casas infestadas : cantidad de casas que
cuentan con al menos un contenedor que alberga a larvas o pupas de Aedes
aegypti.
\item número de casas inspeccionadas : El total de casas analizadas para
el estudio.
\end{itemize}

Es el de uso más generalizado para la distribución de medición de
la población larvaria. Es el índice más rápido y simple para examinar la
población larval. Puede ser utilizado para proporcionar una indicación
rápida de la distribución del mosquito en una área determinada. Sus
defectos son que no tiene en cuenta el número de envases positivos por
yarda ni la productividad de esos envases.

\subsubsection{Índice de Recipientes}
El índice de Recipientes es un valor numérico que consiste en el
porcentaje de recipientes que contienen agua y están infestados con las
larvas y/ó crisálidas del mosquito transmisor del virus del dengue.
El índice del envase proporciona una indicación más detallada de la
abundancia de la población larvaria.

\begin{equation}
I.R. = \frac{\text{número de contenedores positivos} * 100}{\text{número de contenedores inspecionadas}}
\end{equation}

Donde :
\begin{itemize}
\item número de contenedores positivos : La cantidad de contenedores en los
cuales se observan larvas o pupas de Aedes aegypti.
\item número de contenedores  inspeccionadas : El total de contenedores
analizadas para el estudio.
\end{itemize}

Las encuestas larvales que utilizan el índice del envase son mucho más
lentas a realizar que las encuestas sobre el índice de la casa, pues
requieren generalmente que todos los envases en una premisa puedan ser
examinadas para las etapas no maduras y los detalles guardados de envases
positivos y negativos. El índice del envase no proporciona ninguna información
en la productividad de diversos envases.

\subsubsection{Índice de Breteau}
El índice de Breteau (I.B.) es un valor numérico que define el número de
insectos en desarrollo que se encuentran en las viviendas humanas por
la cantidad del total inspeccionado.

\begin{equation}
I.B. = \frac{\text{número de contenedores positivos} * 100}{\text{número de casas inspecionadas}}
\end{equation}
Donde :
\begin{itemize}
\item número de contenedores positivos : La cantidad de contenedores en los
cuales se observan larvas o pupas de Aedes aegypti.
\item número de casas inspecionadas : El total de casas analizadas para
el estudio.
\end{itemize}

La determinación correcta requiere de una encuesta completa de todos los
envases en una premisa que pueda hacer este tipo de ennumeración. Los
datos se utilizan para determinar el índice de la casa. Usando la
combinación del índice de Breteau y el índice de la casa, es fácil
determinar si el problema es extenso dentro de un área ó se enfoca a
unas viviendas.

\subsubsection{Probelmatica}
Los indicadores tradicionales son poco confiables porque
\begin{itemize}
    \item Se basan en búsqueda de fases inmaduras por lo que representan
        una estimación indirecta de las poblaciones de mosquitos adultos.
    \item No reflejan la asociación que existe entre las densidades de
        mosquitos o cantidad y/o tipo de recipientes presentes, con los
        riesgos de transmisión de dengue.
    \item Proporcionan poca o nula información de aquellas viviendas en
        las que existe un mayor riesgo de presencia de mosquitos.
\end{itemize}

Además estos indicadores no reflejan las poblaciones de adultos ni estiman
riesgo entomológico, lo cual es muy importante en la transmisión del dengue.
Por lo cual, a la fecha sólo son recomendados para detectar la calidad de
las acciones (control de calidad) realizadas por el personal de control
larvario. Existen numerosos métodos e indicadores para determinar las
poblaciones de Aedes aegypti en la etapa de huevo, larva, pupa o adulto;
uno de los métodos más prácticos, eficientes y económicos es el monitoreo
de poblaciones de este vector por medio de ovitrampas. Las ovitrampas han
sido usadas desde 1965 en la vigilancia del Aedes aegypti (L), como un
instrumento para determinar la distribución del mosquito, medir la fluctuación
estacional de las poblaciones y para evaluar la eficacia de la aplicación
de insecticidas; además, como una estrategia de muestreo presencia-ausencia,
lo cual permite una estimación de la densidad mediante la proporción de
muestras positivas y son especialmente útiles para la detección temprana
de reinfestaciones. \cite{cenaprece2013}

Las técnicas tradicionales de vigilancia de A. aegypti usan los índices
aédicos de recipientes, de viviendas y de Breteau para determinar el grado
de infestación, dispersión y densidad del mosquito en una zona y tiempo
determinados. Estos índices se fundamentan en la detección visual de formas
inmaduras del vector dentro de recipientes domésticos, técnica considerada
poco sensible por la habilidad de las larvas para escapar y su capacidad de
permanecer sumergidas por largos períodos de tiempo (3, 5). Asi mismo, la
proporción de viviendas y recipientes infestados con A. aegypti no provee
información fehaciente sobre la densidad poblacional al registrar como
positivo un recipiente o casa sin tener en cuenta la cantidad de formas
inmaduras presentes, lo cual quiere decir que para el índice es igual
si hay una o cientos de ellas.
