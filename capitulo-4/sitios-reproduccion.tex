\subsection{Sitios de reproducción}
\label{subsec:cap4-sitios de reproduccion}
Sea $BS$ el número de sitios de reproducción agrupados como una solo para una zona. La variable
ambiental, $BS$ ,determina el tamaño de la población de equilibrio en el modelo determinista
\cite{otero2006stochastic}. Las diferentes condiciones ambientales deben ser
representadas por diferentes valores del parámetro de $BS$ \cite{otero2006stochastic}.

Se considera a $BS(x,y)$ como el valor de $BS$, asociado a $(x,y)$, partiendo de las hipotesis
realizadas en la \secref{subsec:cap4-zonificacion} podemos considerar que el valor de $BS(x,y)$ se
encuentra influenciado por $u(x,y)$, de ese modo a medida que $u(x,y)$ varíe, lo debe hacer el
valor de $BS(x,y)$. Para el cálculo de $BS$ relativo a $(x,y)$ se utiliza el método interpolador
de Lagrange.

Sea $bs$ la función a interpolar, sean $u_0$, $u_1$,...,$u_m$ las densidades conocidas de $bs$ y
sean $bs_0$, $bs_1$,...,$bs_m$ los valores que toma la función para dichas densidades, el polinomio interpolador de grado n de Lagrange es un polinomio de la forma :

\begin{equation}
\label{eq:sitios-reproduccion-x-y}
    bs(u(x,y)) = \sum_{i=0}^{n} bs_{i} * l_{i}(u(x,y))
\end{equation}

donde $l_j(u(x,y))$ son los llamados polinomios de Lagrange, que se calculan de este modo:

\begin{equation}
\label{eq:sitios-reproduccion-x-y}
    l_{i}(u(x,y)) = \prod_{j \neq i} \cfrac{u(x,y) - u_{j}}{u_{i} - u_{j}}
\end{equation}

Consideramos un polinomio de grado 3, con los parámetros $u_0$, $u_1$ y $u_2$ igual a $19$, $51$ y
$70$, correspondientes a zonas del tipo pésima, regular y óptima. Los valores, $bs_{min}$,
$bs_{med}$ y $bs_{max}$ respectivamente, estos son parámetros configurables del modelo
donde $bs_{min}$ representa el menor $BS$ observado, $bs_{max}$ representa el mayor $BS$ observado
y  $bs_{med}$ es el valor medio existente entre $bs_{max}$ y $bs_{min}$.
