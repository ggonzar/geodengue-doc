\section{Proceso Evolutivo}

Definimos un proceso evolutivo, para describir el conjunto de transformaciones o cambios, a través del tiempo, 
a lo que es sometido un antepasado común para generar nuevos descendientes y diversificar la población. Mediante 
este proceso se  busca reproducir el comportamiento, del mosquito transmisor del dengue \textit{Aedes Aegigty},
ante las variaciones climáticas a las que se encuentran sometidos, simulando las variaciones correspondientes a 
sus características biológicas, de forma a generar la suficiente información para realizar los análisis
correspondientes.

\subsection{Definición del problema}

El problema puede ser definido como, la necesidad de encontrar una función $PE$ que permita someter a los
individuos a ciertas variaciones climáticas teniendo en cuenta sus características biológicas descriptas en la 
sección \ref{sec:caracteristicas-biologicas}.

Sean:
%lista de definiciones de las variables básicas
\begin{description}[style=multiline,leftmargin=1.5cm]
    \item[$t_{i}$] utilizada para representar el instante de tiempo mediante a una estructura de datos compleja
    compuesta por datos climáticos y otras propiedades complementarias correspondientes al periodo tiempo
    individual $i$.\addsymbol{symbol:ti}
    
    \item[$p_{i}$] define a un individuo, que es equivalente a la unidad del \textit{Aedes Aegypti}, se encuentra
    representada como una estructura de datos compleja compuesta de propiedades como su  madurez y expectativa de
    vida, sexo y estado \addsymbol{symbol:pi}.

    \item[$\tau(p_{j})$] \addsymbol{symbol:estado-pj} es utilizada para representar las etapas o estados del ciclo
    de desarrollo del individuo $p_{j}$.
    \begin{align*}
        \tau (p_{j}) = [HUEVO, LARVA, PUPA, ADULTO]
    \end{align*}

    \item[$S(p_{j})$] \addsymbol{symbol:sexo-pj} Indicador binario del sexo de $p_{j}$, el cual toma el valor 1
    si $p_{j}$ es $HEMBRA$, y 0 en caso de que sea $MACHO$.

    \item[$\eta (p_{j})$] \addsymbol{symbol:ma-pj} variable numérica asignada a $p_{j}$ para representar el nivel
    de madurez. Su valor varía entre 0 y 100.
    
    \item[$\xi (p_{j})$] \addsymbol{symbol:ex-pj} variable numérica asignada a $p_{j}$ para representar la
    expectativa de vida. Su valor varía entre 0 y 100.
    
    \item[$T$] variable que define el rango de tiempo del estudio, se representa como una colección de $N$
    instantes $t_{i}$, donde $N$ es el tamaño del periodo de estudio \addsymbol{symbol:T}.
        \begin{align*}
            T = t_1,t_2,t_t,\ldots,t_N , & & 1 \leq i \leq N
        \end{align*}

    \item[$P$] la población, se define como una colección de $M$ individuos $p_{i}$, en donde $M$ es el tamaño de
    la población\addsymbol{symbol:P}.
    \begin{align*}
        P = p_1,p_2,p_t,\ldots,p_M,  & & 1 \leq j \leq M
    \end{align*}
    
\end{description}

Podemos definir $PE$ como una función dependiente, necesariamente, de $p_{j}$ y $t_{i}$. Donde esta se encarga
de modificar las propiedades como $\xi(p_{j})$, $\eta(p_{j})$ y $\tau(p_{j})$ de cada individuo $p_{j}$ que sea
sometido a sometido a las variaciones climáticas definidas en $t_{i}$, teniendo en  cuenta las características
biológicas de $p_{i}$ para cada $S(p_{j})$.

\begin{align*}
PE (t_{i}, p_{j}) = p_{j}' & & \forall t_{i} \in T\\
& & \forall p_{j} \in P 
\end{align*}

Donde $p_{j}'$ denota al j-esmimo individuo de la población $p_{j}$, luego de ser sometido a $PE$ para modificar
sus propiedades.  
