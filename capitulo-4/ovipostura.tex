\subsection{Ciclo gonotrófico y Ovipostura}
\label{subsec:cap4-ciclo-gontrofico-ovipostura}
El Aedes aegypti puede alimentarse más de una vez para cada ovoposición \cite{scott1993detection},
especialmente si es perturbado antes de finalizar si alimentación. En \cite{osoriopontificia},
observa que  22.56 \% de la población no realizó ninguna toma de sangre, por ende no generó
ninguna tanda de huevos,  43.6 \% realizaron una toma de sangre, de las cuales el 21,9\% de los
mosquitos no realizaron ovoposturas, 16,5 \% realizaron dos tomas de sangre, 8,42 \% realizaron
tres tomas de sangre, 6,9 \% realizaron cuatro tomas de sangre y 2,02 \% realizaron cinco tomas de
sangre. La cantidad de huevos que pone una hembra luego de las alimentaciones sanguineas
correspondientes, fue tomada de \cite{otero2006stochastic}, que considera que cada hembra pone un
número fijo de huevos (63) en cada oviposición.

El ciclo gonotrófico de los mosquitos es el nombre que se le adjudico al período que existe desde
que el mosquito realiza una alimentación sangínea - ovipostura - hasta una nueva alimentación.
Como se mencionó anteriormente en \secref{subsec:cap4-tasas de desarrollo}, la tasa de desarrollo
del ciclo gonotrófico puede estimarse mediante la versión simplificada del modelo de Sharpe y DeMichele \cite{sharpe1977reaction}, popuesta por Schoolfield en \cite{schoolfield1981non}.

Sea $R(k_{i})$ la tasa de desarrollo (ver ecuación \eqref{eq:schoolfield}) del ciclo gonotrófico
de una hembra (nulípara, parida), para una temperatura de $k_{i}$ Kelvin en un instante $i$, se
considera que un día es de ovipostura si se cumple :

\begin{equation}
\label{eq:ciclo-gonotrofico-ovipostura}
    \sum_{i=0}^{N} R(k_{i}) \geq 1
\end{equation}

Donde $N$ es la duración en días del ciclo gonotrófico de la hembra. Para $R(k)$ se aplican los
parámetros correspondientes a hembras nulíparas y paridas de acuerdo al estado de la hembra.
En \cite{edman1987host} se observó que hembras nulíparas de Aedes aegypti poseen un proceso de
digestión es más lento en las hembras paridas y por ende el ciclo gonotrófico de las mismas tiene
a ser más largo.
