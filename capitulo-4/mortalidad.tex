\subsection{Mortalidad}
\label{subsec:cap4-mortalidad}
La mortalidad de los individuos depende de la etapa del ciclo de desarrollo en el que se encuentren
los individuos de una población.

El modelo considera que todas las poblaciones cuentan con una cantidad entera de individuos, por
lo que a la hora de reducir una población se espera que sea mediante la sustracción de una
cantidad entera.La utilización de las tasas de desarrollo dan como resultado cantidades no enteras, por lo que es necesario un pequeño ajuste dado por el siguiente operador de redondeo :

\begin{equation}
\label{eq:operador-redondeo}
\rho(n) = \left\{
\begin{array}{l l}
   round(n) & \quad  \forall n < round(n)\\
   n & \quad  \forall n > round(n)\\
\end{array} \right.
\end{equation}

Donde $round(n)$ es la función redondea, $n$, al entero más cercano. Para $n > round(n)$, la parte
no entera de $n$ queda acumulada para la siguiente iteración.

\subsubsection{Mortalidad de huevos}
La tasa de mortalidad de los huevos se encuentra definida como una constante, $me = 0.01$,
$1/\text{días}$, independiente de la temperatura\citep{otero2006stochastic}.

\begin{equation}
    M_{H(x,y)} = \rho(me * H(x,y))
\end{equation}

Donde $M_{H(x,y)}$ es la cantidad de huevos que deben ser eliminados de la población $H(x,y)$.

\subsubsection{Mortalidad de larvas}
En \citet{otero2006stochastic} la mortalidad de las larvas, se encuentra dividida en dos
contribuciones. La primera contribución representa la mortalidad natural bajo óptimas condiciones
y se encuentra influneciada únicamente de la temperatura. Esta tasa se encuentra definida por :

\begin{equation}
 \begin{array}{l l}
    ml(k) = 0.01 + 0.9725 * exp\bigg( \frac{-(k - 278)}{2.7035}\bigg) &\quad  \forall k, 278 K \geq k \geq 303 K\\
\end{array}
\end{equation}

La segunda contribución es la mortadiad denso dependiente de las larvas. Este mecanismo de
regulación puede estar realacionado con procesos concurrentes, como las limitaciones de los
alimentos, las interacciones químicas, presencia de depredadores especializados en el sitio de
reproducción y mucho más\citep{otero2006stochastic}. Esta se encuentra definida por :

\begin{equation}
  \alpha (x,y) = \alpha _{0}/BS(x,y)
\end{equation}

Donde $\alpha _{0}$ está asociado a la capacidad de carga de un solo lugar de reproducción y
$BS(x,y)$ es el número de sitios de reproducción. El valor de $alpha _{0}$ puede ser instalado en
los valores observados en la región que se está simulando.

Tomando ambas contribuciones, la mortalidad natural bajo óptimas condiciones y la denso
dependiente, la mortalidad de las larvas queda definida como :
\begin{equation}
    M_{L(x,y)}(k) = \rho(ml(k) * L(x,y) + \alpha (x,y) * L(x,y) *(L(x,y) - 1))
\end{equation}

Donde $M_{L(x,y)}$ es la cantidad de larvas que deben ser eliminadas de la población $L(x,y)$.

\subsubsection{Mortalidad de las pupas}
La tasa de mortalidad de las pupas se encuentra definida como una función influneciada únicamente
de la temperatura. \citep{otero2006stochastic}.

\begin{equation}
 \begin{array}{l l}
    mp(k) = 0.01 + 0.9725 * exp\bigg( \frac{-(k - 278)}{2.7035}\bigg) &\quad  \forall k, 278 K \geq k \geq 303 K\\
\end{array}
\end{equation}

Además de la mortalidad diaria en la fase de pupa, existe una importante mortalidad adicional
asociada con la emergencia sin éxito de adultos, solo el 83\%  de las pupas alcanzan la maduración
y emergerán como mosquitos adultos, por lo tanto, el factor de supervivencia es de $ef=0.83$
\citep{otero2006stochastic}.

\begin{equation}
    M_{P(x,y)}(k) = \rho(P(x,y) * (mp + (1 - ef) * R(k)))
\end{equation}

Donde $M_{P(x,y)}$ es la cantidad de pupas que deben ser eliminadas de la población $P(x,y)$.

\subsubsection{Mortalidad de adultos}
La tasa de mortalidad de los adultos se encuentra definida como una constante, $ma = 0.09$,
$1/\text{días}$, independiente de la temperatura\citep{otero2006stochastic}.

\begin{equation}
    M_{A(x,y)} = \rho(ma * A(x,y))
\end{equation}

Donde $M_{A(x,y)}$ es la cantidad de adultos que deben ser eliminados de la población $A(x,y)$.
