
\subsection{Operadores básicos}
\label{subsec:operadores-basicos}
Durante el proceso evolutivo existen un conjunto de operadores básicos que son aplicables a todos los individuos
$p_{j}$ de la población, que son utilizados por $PE(t_{i},p_{j})$ para modificar las características del 
individuo.

\subsubsection{Tasa de desarrollo}
En \cite{rueda1990temperature} para describir el efecto de la temperatura en la tasa media de desarrollo,
se utilizó el modelo no lineal de Sharpe \& DeMichele definido en \cite{sharpe1977reaction} para procesos
poiquilotermos\footnote{La poiquilotermia o ectotermia es un término aplicado a ciertos animales con temperatura
corporal variable} dependientes de la temperatura. El modelo de Sharpe \& DeMichele con inhibición de alta temperatura
se encuentra descripto, en \cite{rueda1990temperature}, por la siguiente ecuación\addsymbol{symbol:sharpe-demichele}.

\begin{equation} \label{eq:sharpe-demichele}
   r(k)  = \cfrac{ RH025 \cfrac{k}{298.15} * 
    exp \Bigg[
            \cfrac{HA}{1.987} \bigg(\cfrac{1}{298.15} - \cfrac{1}{k}\bigg)
        \Bigg]}
    {1 + exp\Bigg[\cfrac{HH}{1.987} \bigg(\cfrac{1}{TH}- \cfrac{1}{k}\bigg)\Bigg]} 
\end{equation}

Donde $r(k)$ representa la tasa de desarrollo media ($dias^-1$) para una temperatura $K$, que se encuentra en
la escala de kelvin. Los parámetros $RH025$, $HA$, $TH$, y $HH$  son estimados por la ecuación de regresión 
no lineal de \cite{wagner1984modeling}. Una vez determinados los parámetros, la ecuación puede utilizarse para
calcular tasas de desarrollo a cualquier temperatura\cite{rueda1990temperature}. En \cite{rueda1990temperature} 
se calcula los valores de $RH025$, $HA$, $TH$, y $HH$ para los 4 estadios larvales y cada etapa del ciclo de vida 
del \textit{Aedes Aegypti} (Ver tabla \ref{tab:tasa-desarrollo}).

\begin{table}
\begin{minipage}{\paperwidth}
\begin{tabular}{p{6cm} c c c c c }
Estado                     & $RH025$    & $HA$   & $TH$   & $HH$      & $R^2$   \\
\hline
Primer estadio larval      & 0.68007 & 28,033.83 & 304.33 & 72,404.07 & 0.95 \\
Segundo estadio larval     & 1.24508 & 36,400.55 & 301.78 & 81,383.14 & 0.96 \\
Tercer estadio larval      & 1.06144 & 41,192.69 & 301.29 & 60,832.62 & 0.97 \\
Cuarto estadio larval      & 0.57065 & 34,455.89 & 301.44 & 45,543.49 & 0.97 \\
Larva\footnote{Del primer al cuarto estadio larval} & 0.20429 & 36,072.78 & 301.56 & 59,147.51 & 0.97 \\
Pupa                       & 0.74423 & 19,246.42 & 302.68 & 5,954.35 & 0.98  \\
Total\footnote{Eclosión de los huevos a la emergencia de adultos}& 0.15460 & 33,255.57 & 301.67 & 50,543.49 & 0.98 \\

\end{tabular}
\end{minipage}
\caption{ \label{tab:tasa-desarrollo} Parámetros de ajuste estimados de la ecuación \eqref{eq:sharpe-demichele}, 
para la tasa de desarrollo del Aedes Aegypti (Tomado de \cite{rueda1990temperature})}
 
\end{table}

\subsubsection{Zonificación}
Los datos con los que cuenta $t_{i}$ corresponden a datos climáticos de la ciudad en la que se encuentra la
$P$, de esta forma todos los $p_{j}$ comparten las características de $t_{i}$. Si bien todos los individuos
comparten la mismas características de $t_{i}$, el nivel de impacto,de $t_{i}$, en cada uno de ellos, no
necesariamente será el mismo para todos, debido a que el entorno en el que se encuentra cada $p_{j}$ es 
distinto. Cada entorno puede contar con factores que lo hagan apto para el buen desarrollo de los individuos
como, así también con otros que causen una alta mortalidad.

La zonificación surge ante necesidad de dividir el espacio de estudio de una forma más granular, para identificar
a los individuos que pertenecen a zonas aptas y los que no.
\begin{equation}
Z (t_{i}, p_{j}) = t_{i} -  R \cfrac{t_{i}}{u(p_{j})}
\end{equation}

Donde $Z (t_{i}, p_{j})$ el valor de la temperatura relativa correspondiente a la zona de $p_{j}$, R es el rango
inferior correspondiente a la zona de $p_{j}$ (Tabla \ref{tab:puntaje-zona}) y $u(p_{i})$ el valor predicho para 
$p_{i}$ mediante la ecuación \eqref{eq:interpolacion-idw}.

\begin{table}
\centering
    \label{tab:puntaje-zona}
    \begin{tabular}{p{3cm} p{3cm}}
        Tipo de zona & Rango \\
        \hline
        Óptima & $60 \leq u(p_{i})$ \\
        Buena  & $30 \leq u(p_{i}) < 60 $  \\
        Normal & $20 \leq u(p_{i}) < 30$\\
        Mala   & $8  \leq u(p_{i}) < 20$\\
        Pésima &  $0 \leq u(p_{i}) < 8 $ \\
    \end{tabular}
    \caption{División de zonas}
\end{table}


\subsubsection{Madurez y cambio de estado}
La madurez $\eta (p_{j})$ de un individuo $p_{j}$, que se encuentra en un estado $\tau_{k}$, indica el nivel de
proximidad al siguiente estado $\tau_{k+1}$. Se encuentra representada por un número que varía entre
cero y cien, la función $\eta (t_{i}, p_{j})$, que se encarga de incrementar el nivel de madurez de $p{j}$
que es sometido a $t_{i}$ \addsymbol{symbol:n_ti_pj}.

\begin{equation}
\eta (t_{i}, p_{j}) = \left\{
  \begin{array}{l l}
    \eta (t_{i}, p_{j}) = 0 & \quad \forall i = 0 \\
    \eta (t_{i-1},p_{j}) + \cfrac{1}{\omega(t_{i}, p_{j}) * 24} & \quad \forall i \neq 0
  \end{array} \right.
\end{equation}

Donde 
\begin{equation}
    \omega(t_{i}, p_{j}) = 1/r(Z(t_{i}, p_{j}) + 273.15)
\end{equation}

Cuando $\eta(p_{i})$ del individuo $p_{j}$ alcanza su máximo valor, cien, el individuo ya se encuentra listo para
pasar del estado $\tau_{k}$ al siguiente estado $\tau_{k+1}$, a este proceso se lo denomina como cambio de estado.

La velocidad con la que incrementa el valor de $\eta(p_{j})$ depende de las condiciones a las condiciones
descritas  por $t_{i}$ y las características de la zona en la que se encuentra $p_{j}$.

\subsubsection{Mortalidad}
\label{subsec:mortalidad}
La mortalidad de los individuos depende de la etapa del ciclo de desarrollo en la que se encuentren. En 
\cite{otero2006stochastic} se define la mortalidad de huevos como una constante independiente a la temperatura:

\begin{equation}
 \begin{array}{l l}
    \upsilon(t_{i}, p_{j}) = 0.01 & \quad \forall T,  278 k \geq T \geq 303 k\\
              & \quad \forall \tau(p_{j}) = HUEVO 
\end{array}
\end{equation}

En el caso de las larvas \cite{otero2006stochastic} define que se encuentra influenciada dos métodos. La primera
define como una ecuación dependiente de temperatura. La segunda depende de la densidad poblacional del habitad de la
larva.
\begin{equation}
 \begin{array}{l l}
\upsilon(t_{i}, p_{j}) = 0.01 + 0.9725 * exp\bigg[ \frac{-(T - 278)}{2.7035}\bigg] &\quad  \forall T, 278 k \geq T \geq 303 k\\
              & \quad \forall \tau(p_{j}) = LARVA
 
\end{array}
\end{equation}

En el caso de la pupa \cite{otero2006stochastic}, la mortalidad intrínseca de una pupa se ha considerado como 
una ecuación dependiente de temperatura. Además de la mortalidad diaria en la fase de pupa, existe una importante 
mortalidad adicional, solo el 83\% \cite{otero2006stochastic} de las pupas alcanzan la maduración y emergerán como
mosquitos adultos, por lo tanto, el factor de supervivencia es de 0.83.

\begin{equation}
 \begin{array}{l l}

  \upsilon(t_{i}, p_{j}) = 0.83 \Bigg[ 0.01 + 0.9725 * exp\bigg( \frac{-(T - 278)}{2.7035}\bigg)\Bigg] &\quad  \forall T, 278 k \geq T \geq 303 k\\
              & \quad \forall \tau(p_{j}) = PUPA
\end{array}
\end{equation}

Para la etapa adulto, \cite{otero2006stochastic} la define como una constante independiente de la temperatura.
\begin{equation}
 \begin{array}{l l}
    \upsilon(t_{i}, p_{j}) = 0.09 &\quad  \forall T, 278 k \geq T \geq 303 k\\
              & \quad \forall \tau(p_{j}) = ADULTO
\end{array}
\end{equation}

La mortalidad y supervivencia de los individuos se encuentra expresada mediante la variable de expectativa de
vida. La expectativa de vida es un valor numérico que indica la vitalidad del individuos, esta varía de acuerdo
a las condiciones climáticas a las que es sometido el individuos durante el proceso evolutivo.

\begin{equation}
\xi (t_{i}, p_{j}) = \left\{
  \begin{array}{l l}
    100 & \quad \forall i = 0 \\
    \xi (t_{i-1}, p_{j}) - \cfrac{1}{\upsilon(Z(t_{i}, p_{j}),p_{j}) * 24} & \quad \forall i \neq 0
  \end{array} \right.
\end{equation}

El valor de $\xi (t_{i}, p_{j})$ representa el porcentaje de vitalidad del
individuo $p_{j}$ luego de ser sometido a las variaciones del instante
$t_{i}$ perteneciente al periodo $T$ de estudio. La expectativa de vida
para cada individuo $p_{j} \in P$, en el instante $t_{0}$ se describe como
$\xi (t_{0}, p_{j})= 100$. A medida que $p_{j}$ sea sometido a varios
$t_{i}$ la expectativa de vida irá disminuyendo, si $\xi (t_{i}, p_{j})= 0$ el
individuo se ha quedado sin expectativa de vida, por lo que se debe proceder
a eliminar $p_{j}$ de $P$ mediante el proceso de reducción de la población
$\theta (t_{i}, p_{j})$ \addsymbol{symbol:t_ti_pj}.

\begin{equation}
\theta (t_{i}, p_{j}) = \left\{
  \begin{array}{l l}
    p_{j} = null & \quad \xi(t_{i}, p_{j}) == 0 \\
    p_{j} & \quad \text{en caso contrario}
  \end{array} \right.
\end{equation}
