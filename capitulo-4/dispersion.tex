
\subsection{Vuelo y dispersión}
\label{subsec:cap4-vuelo-dispersion}
El aedes aegypti es un mosquito doméstico que generalmente esta confinado a las casas donde se
cria \citep{luevano1993ciclo}, tiende a permanecer físicamente en donde emergió, siempre y cuando
no exista algún factor que la perturbe o no disponga de huéspedes, sitios de reposo y de postura
\citep{ThironIzcazaJ2003}. Por lo general mosquito no sobrepasa los 50 a 100 metros durante su vida
\citep{cabezas2005dengue}. En caso de no contar con sitios adecuados de ovipostura y disponibilidad
de alimento tienden a dispersarme una mayor distancia, hasta tres kilómetros, en busca de
mejores condiciones \citep{ThironIzcazaJ2003}. Los mosquito tienen la particularidad de que vuelan
en sentido contrario al la dirección al viento \citep{ThironIzcazaJ2003} y a una velocidad máxima
de $2 km/h$\citep{kaufmann2004flight}.

Exiten varios factores externos, que influyen el la dispersión del mosquito, disponiblidad de
alimentos, criadreos, sitios de reposo. Para las estrategias de control de Aedes aegypti en zonas
urbanas donde existen brotes de dengue y fiebre amarilla se asume que los mosquitos tienen un
rango de vuelo durante su vida de 50 a 100 metros\citep{dengueUruguayCap8}.

La función más importante del adulto de Aedes aegypti es la reproducción y secundariamente la
dispersión de la especie. Para Aedes aegypti el transporte pasivo de huevos y larvas en
recipientes ha tenido mayor trascendencia en su distribución, en la que el hombre ha participado
en forma determinante en comparación con la dispersión activa propia de la especie
\citep{ThironIzcazaJ2003}.

Partiendo de las hipotesis realizadas en la \secref{subsec:cap4-zonificacion} podemos considerar a
que la dispersión encuentra influenciado por la densidad $u(x,y)$, de ese modo a medida que
$u(x,y)$ varie, la dispersión debe ajustarse a su tipo de zona. De forma simplificada definimos que
la dispersión de un adulto que se encuentre en zonas del tipo \textit{Regular}, \textit{Buena} u
\textit{Óptima} se encuentra entre 0 y 100 metros de vuelo. Para las hembras adultas, que
pertenezcan a zonas del tipo \textit{Mala} o \textit{Pésima} se tiene una dispersión entre 100 a
3000 metros de vuelo, de este modo, las hembras adultas que se encuentren en zonas menos aptas
tenderán a desplzarce en busca de mejores condiciones.
