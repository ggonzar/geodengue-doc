%!TEX root = ../tesis.tex
\chapter{Solución propuesta}

  %~ * Introducción
El dengue ha sido considerado un problema de salud pública mundial, que afecta aproximadamente a
2.500 millones de personas que viven en zonas en riesgo de dengue y más de 100 países que han
informado de la presencia de esta enfermedad en su territorio \citet{gustavo2006dengue}. A lo
largo de los años se han desarrollado multiples modelos destinados a, la vigilancia, monitoreo y
la predicción de la transmisión del dengue.

En este capítulo se presentan las metodologias empleadas para la identificación de focos de dengue
y los modelos utilizados para la simulación del comportamiento de los vectores del dengue en
diferentes escenarios climáticos.


%~ * Identificación de focos de Riesgo
%!TEX root = ../tesis.tex
\section{Identificación de focos de infestación}
\label{sec:cap4-identificacion-focos}

El análisis de la distribución espacial y temporal de las poblaciones del vector, puede llegar a
jugar un papel importante en la planificación y evaluación de medidas orientadas a la disminución
de las poblaciones del vector y en consecuencia, reducir los casos de dengue
\cite{dengueUruguayCap1, cenaprece2013,nino2008uso}. Los SIG constituyen una herramienta esencial
para el análisis de la distribución espacial de las poblaciones
\cite{vgomesAegis2001,petric2012surveillance}, permitiendo obtener mejores resultados en
combinación con las metodologías de vigilancia entomológica y medicas\cite{petric2012surveillance}.

Las metodologías de vigilancia entomológica basadas en la distribución geográfica de larvitrampas
u ovitrampas, como las presentadas en
\cite{NINO2011,petric2012surveillance, journal.pone.0054167,nino2008uso}, permiten generar
información regionalizada sobre la abundancia poblacional del vector \cite{NINO2011}. Los datos
sobre los mosquitos atrapados deben mantenerse para crear un registro histórico de las especies de
mosquitos que se encuentran en asociación con diferentes hábitats y patógenos para permitir la
detección temprana de las adaptaciones \cite{petric2012surveillance}.

Para la identificación de focos de infestación se propone, utilizar la metodología de vigilancia
entomológica basada en el uso de larvitrampas propuesta en \cite{NINO2011}. En donde las
larvitrampas, o puntos de control, deben ser distribuidas geográficamente para generar información
regionalizada que, mediante técnicas de interpolación espacial, permiten obtener mapas de
interpolación donde se puede apreciar los niveles de infestación del vector, y el riesgo
correspondiente a la abundacia de mosqutios observada en el área de estudio \cite{NINO2011, nino2008uso, journal.pone.0054167, albierispatial}. El hecho de contar con esta información
regionalizada permitirá a las autoridades pertinentes definir y planificar mejor las medidas de
prevención y control a realizarce para reducir los niveles de infestación en las zonas criticas
\cite{NINO2011, nino2008uso, petric2012surveillance}.

La selección del método de interpolación se realizó teniendo en cuenta el factor humano en la
distribución de los puntos de control. En \cite{villatoro2007comparacion} se realiza una
comparación de los interpoladores IDW y Kriging, donde los autores señalan que el método Kriging
fue más preciso y eficiente que el IDW, aunque la diferencia entre ambos métodos no fue muy amplia.
Sin embargo, cuando el distanciamiento, es muy grande, los variogramas no son posibles de obtener,
entonces el Kriging deja de ser una opción y comparativamente el IDW se perfila como el mejor
\cite{villatoro2007comparacion}. El método seleccionado finalmente fue el IDW, debido que la
distribución del los puntos de control no será perfecta, inclusive, en algunas localidades la
distribución no será de forma uniforme.

Los mapas de interpolación resultantes, indican, con mayor detalle que los índices aédicos
tradicionales, los lugares específicos donde sería necesario tomar medidas de prevención y
control de acuerdo al grado de infestación, permitiendo así una mayor racionalización de tiempo y
recursos \cite{NINO2011}.

\section{Modelos de simulación de la ecología del vector}
\label{sec:cap4-modelo-simulacion}

La comunidad académica internacional reconoce que, el comportamiento, las tasas de desarrollo, la
mortalidad y la dispersión geográfica de los vectores transmisores del dengue, se encuentran
influenciados por las variaciones climáticas y del medio que los rodea.

Según \citet{velez2013hacia} una primera aproximación al entendimiento puede ser adquirida
mediante el diseño de modelos matemáticos que integren los múltiples factores de riesgo, obtenidos
como un producto entre factores de amenaza y vulnerabilidad, propios de cada población. Estos
modelos, cuya principal tarea sería profundizar el entendimiento de las dinámicas de transimsión,
además pueden actuar como un simulador del comportamiento de los vectores del dengue en diferentes
escenarios climáticos, entomológicos, sociodemográficos y culturales futuros. En
\citet{velez2013hacia} los autores señalan, que además de la significativa ganancia en
conocimiento, los modelos matemáticos podrían, entonces, contribuir a la detección temprana y
oportuna del momento de ocurrencia y el orden de magnitud de brotes epidémicos, lo cual brindaría
una importante capacidad de anticipación para la prevención de eventos que generan significativas
alteraciones en la salud y, en casos serios, la muerte de muchos pacientes en nuestro medio.

La metodología presentada en la \secref{sec:cap4-identificacion-focos} permite generar información
regionalizada sobre los focos de infestación, mediante la utilización de puntos de control, que
puede ser combinada con información ambiental, demográfica o epidemiológica, y así obtener modelos
detallados que tengan la capacidad de monitorear, simular o predecir la transmisión del dengue
\citet{NINO2011}.

La propuesta de modelo de simulación,presentada en esta sección, consiste en, un conjuto de
submodelos que buscan estimar la tasa de desarrollo, mortalidad, reporducción y dispersión del
vector del dengue ante las variaciones climáticas a las que es sometido, con el fin de generar
información suficiente para contribuir con la detección temparana de posibles brotes
epidemiologicos.

\subsection{Modelo matemático del ciclo de vida}
\label{subsec:cap4-modelo-matematico-ciclo-vida}
El modelo considera un espacio bi-dimensional, con un sistema de coordenadas geográficas $(x,y)$,
para expresar todas las posiciones sobre el plano, correspondientes a la longitud y latitud. Si
consideramos a $X_{i}$ como una etapa del ciclo de vida del Aedes Aegypti, correspondiente a una
población de mosquitos, entonces, $X_{i}(x,y)$ representa a $X_{i}$ en las coordenadas geográficas
$(x,y)$.

Al igual que en \citet{otero2006stochastic}, consideramos cinco poblaciones diferentes:
los huevos $H(x,y)$, larvas $L(x,y)$, pupas $P(x,y)$, hembras adultas nulíparas\footnote{Hembras
que no han ovipuesto.} $AN(x,y)$ y hembras adultas paridas\footnote{Hembras que han ovipuesto al
menos una vez.} $AP(x,y)$.

La evolución de las poblaciones, $H(x,y)$, $L(x,y)$, $P(x,y)$, $AN(x,y)$ y $AP(x,y)$ , se ven
afectadas por los siguientes eventos: muerte de huevos, eclosión de huevos, muerte de larvas,
emergencia de pupas, muerte de pupas, emergencia de adultos, muerte de adultos, ovipostura de
hembras nulíparas, ovipostura de hembras paridas y dispersión de los adultos (machos y hembras).
Según \citet{otero2006stochastic} los eventos se producen a tasas que dependen no sólo de valores
de la población, sino también de la temperatura, que a su vez es una función de tiempo, por lo
tanto, la dependencia de la temperatura introduce una dependencia del tiempo en las tasas de
eventos. La evolución es modelada como un proceso interativo diario, en donde la ocurrencia de los
eventos mencionados anteriormente estan sujetos a las temperaturas $K$ reportadas diariamente en
un periodo de tiempo $T$.

\subsection{Tasas de desarrollo}
\label{subsec:cap4-tasas de desarrollo}
En el modelo se cuenta con 4 tasas de desarrollos correspondientes a : la eclosión de huevos,
emergencia a pupas, emergencia a adultos y el ciclo gonotrófico. Estos valores son obtenidos
mediante el modélo no lineal de Sharpe y DeMichele, presentado en \cite{sharpe1977reaction}, para procesos poiquilotermos\footnote{ La poiquilotermia o ectotermia es un término aplicado a ciertos
animales con temperatura corporal variable}, donde el el proceso de maduración es controlado por
una enzima que actúa en un rango de temperatura determinado, la enzima se desactiva a las bajas temperaturas, $TL$, y altas $TH$.

\begin{equation} \label{eq:sharpe-demichele}
   R(k)  = R(298K) *\cfrac{ \cfrac{k}{298K} *
    exp \Bigg[
            \cfrac{\Delta H_{A}}{R} \bigg(\cfrac{1}{298K} - \cfrac{1}{k}\bigg)
        \Bigg]}
    {1 + exp\Bigg[\cfrac{\Delta H_{H}}{R} \bigg(\cfrac{1}{T_{H}}- \cfrac{1}{k}\bigg)\Bigg] +  exp\Bigg[\cfrac{\Delta H_{L}}{R} \bigg(\cfrac{1}{T_{L}}- \cfrac{1}{k}\bigg)\Bigg]}
\end{equation}

Donde $R(k)$ representa la tasa de desarrollo media ($dias^{-1}$) para una temperatura $K$,en la
escala de Kelvin; $T_{H}$ , $T_{L}$ son temperaturas absolutas, en la escala de Kelvin, mientras
que $H_{A}$, $H_{H}$ y $H_{L}$ son entalpías termodinámicas características del organismo, y $R$,
igual $1,987202$ $cal/K.mol$, es la constante universal de los gases.

Schoolfield presentó, en \cite{schoolfield1981non}, un modelo simplificado con inhibición de altas
temperaturas, con una única alta temperatura de desactivación. El modelo se encuentra definido
como :

\begin{equation} \label{eq:schoolfield}
   R(k)  = R(298K) *\cfrac{ \cfrac{k}{298K} *
    exp \Bigg[
            \cfrac{\Delta H_{A}}{R} \bigg(\cfrac{1}{298K} - \cfrac{1}{k}\bigg)
        \Bigg]}
    {1 + exp\Bigg[\cfrac{\Delta H_{H}}{R} \bigg(\cfrac{1}{T_{1/2}}- \cfrac{1}{k}\bigg)\Bigg] }
\end{equation}

Donde $T_{1/2}$ es la temperatura cuando la mitad de la enzima se desactiva, debido a la alta
temperatura. Los parámetros $R(298K)$, $H_{A}$, $T_{1/2}$, y $H_{H}$ son estimados mediante la
de regresión no lineal de Wagner, presentado en \cite{wagner1984modeling}. Para este trabajo,
al igual que \cite{rueda1990temperature, otero2006stochastic}, se adopta el modelo de Schoolfield,
ya que, según \cite{otero2006stochastic}, es lo suficientemente flexible para el ajuste de los
datos biológicos disponibles. Los parámetros deben calcularse para cada etapa de desarrollo, una
vez determinados, la ecuación puede utilizarse para calcular tasas de desarrollo a cualquier
temperatura \cite{rueda1990temperature}.

\subsection{Madurez y cambio de estado}
\label{subsec:cap4-madurez-cambio-estado}
La madurez de los individuos pertenecientes a las poblaciones de $H(x,y)$, $L(x,y)$ y $P(x,y)$
indica la proximidad de que estos alcancen el siguiente estado de su etapa de desarrollo.

Sea $R(k_{i})$ la tasa de desarrollo (ver ecuación \eqref{eq:schoolfield}) de un individuo para
una temperatura de $k_{i}$ Kelvin en un instante $i$, se considera que ha alcazado su máximo nivel
de madurez y se encuentra listo para pasar al siguiente estado de su cuando :

\begin{equation}
\label{eq:condicion-cambio-estado}
    \sum_{i=0}^{N} R(k_{i}) \geq 1
\end{equation}

Donde $N$ es la cantidad de días que le toma al individuo pasar de un estado a otro. Para $R(k)$ se
se aplican los parámetros correspondientes al estado actual del individuo. De forma extendida,
para que un individuo,

\begin{description}[style=multiline,leftmargin=1.5cm]
\item[$h(x,y)$], perteneciente a $H(x,y)$ pase a ser una larva, $l(x,y)$, que forma
parte de $L(x,y)$ debe cumplirse \eqref{eq:condicion-cambio-estado}.

\item[$l(x,y)$], perteneciente a $L(x,y)$ pase a ser una pupa, $p(x,y)$, que forma
parte de $P(x,y)$ debe cumplirse \eqref{eq:condicion-cambio-estado}.

\item[$p(x,y)$], perteneciente a $P(x,y)$ pase a ser un adulto, $a(x,y)$, que forma
parte de $A(x,y)$ debe cumplirse \eqref{eq:condicion-cambio-estado}.
\end{description}





\subsection{Mortalidad}
\label{subsec:cap4-mortalidad}
La mortalidad de los individuos depende de la etapa del ciclo de desarrollo en el que se encuentren
los individuos de una población.

El modelo considera que todas las poblaciones cuentan con una cantidad entera de individuos, por
lo que a la hora de reducir una población se espera que sea mediante la sustracción de una
cantidad entera.La utilización de las tasas de desarrollo dan como resultado cantidades no enteras, por lo que es necesario un pequeño ajuste dado por el siguiente operador de redondeo :

\begin{equation}
\label{eq:operador-redondeo}
\rho(n) = \left\{
\begin{array}{l l}
   round(n) & \quad  \forall n < round(n)\\
   n & \quad  \forall n > round(n)\\
\end{array} \right.
\end{equation}

Donde $round(n)$ es la función redondea, $n$, al entero más cercano. Para $n > round(n)$, la parte
no entera de $n$ queda acumulada para la siguiente iteración.

\subsubsection{Mortalidad de huevos}
La tasa de mortalidad de los huevos se encuentra definida como una constante, $me = 0.01$,
$1/\text{días}$, independiente de la temperatura\citep{otero2006stochastic}.

\begin{equation}
    M_{H(x,y)} = \rho(me * H(x,y))
\end{equation}

Donde $M_{H(x,y)}$ es la cantidad de huevos que deben ser eliminados de la población $H(x,y)$.

\subsubsection{Mortalidad de larvas}
En \citet{otero2006stochastic} la mortalidad de las larvas, se encuentra dividida en dos
contribuciones. La primera contribución representa la mortalidad natural bajo óptimas condiciones
y se encuentra influneciada únicamente de la temperatura. Esta tasa se encuentra definida por :

\begin{equation}
 \begin{array}{l l}
    ml(k) = 0.01 + 0.9725 * exp\bigg( \frac{-(k - 278)}{2.7035}\bigg) &\quad  \forall k, 278 K \geq k \geq 303 K\\
\end{array}
\end{equation}

La segunda contribución es la mortadiad denso dependiente de las larvas. Este mecanismo de
regulación puede estar realacionado con procesos concurrentes, como las limitaciones de los
alimentos, las interacciones químicas, presencia de depredadores especializados en el sitio de
reproducción y mucho más\citep{otero2006stochastic}. Esta se encuentra definida por :

\begin{equation}
  \alpha (x,y) = \alpha _{0}/BS(x,y)
\end{equation}

Donde $\alpha _{0}$ está asociado a la capacidad de carga de un solo lugar de reproducción y
$BS(x,y)$ es el número de sitios de reproducción. El valor de $alpha _{0}$ puede ser instalado en
los valores observados en la región que se está simulando.

Tomando ambas contribuciones, la mortalidad natural bajo óptimas condiciones y la denso
dependiente, la mortalidad de las larvas queda definida como :
\begin{equation}
    M_{L(x,y)}(k) = \rho(ml(k) * L(x,y) + \alpha (x,y) * L(x,y) *(L(x,y) - 1))
\end{equation}

Donde $M_{L(x,y)}$ es la cantidad de larvas que deben ser eliminadas de la población $L(x,y)$.

\subsubsection{Mortalidad de las pupas}
La tasa de mortalidad de las pupas se encuentra definida como una función influneciada únicamente
de la temperatura. \citep{otero2006stochastic}.

\begin{equation}
 \begin{array}{l l}
    mp(k) = 0.01 + 0.9725 * exp\bigg( \frac{-(k - 278)}{2.7035}\bigg) &\quad  \forall k, 278 K \geq k \geq 303 K\\
\end{array}
\end{equation}

Además de la mortalidad diaria en la fase de pupa, existe una importante mortalidad adicional
asociada con la emergencia sin éxito de adultos, solo el 83\%  de las pupas alcanzan la maduración
y emergerán como mosquitos adultos, por lo tanto, el factor de supervivencia es de $ef=0.83$
\citep{otero2006stochastic}.

\begin{equation}
    M_{P(x,y)}(k) = \rho(P(x,y) * (mp + (1 - ef) * R(k)))
\end{equation}

Donde $M_{P(x,y)}$ es la cantidad de pupas que deben ser eliminadas de la población $P(x,y)$.

\subsubsection{Mortalidad de adultos}
La tasa de mortalidad de los adultos se encuentra definida como una constante, $ma = 0.09$,
$1/\text{días}$, independiente de la temperatura\citep{otero2006stochastic}.

\begin{equation}
    M_{A(x,y)} = \rho(ma * A(x,y))
\end{equation}

Donde $M_{A(x,y)}$ es la cantidad de adultos que deben ser eliminados de la población $A(x,y)$.

\subsection{Ciclo gonotrófico y Ovipostura}
\label{subsec:cap4-ciclo-gontrofico-ovipostura}
El Aedes aegypti puede alimentarse más de una vez para cada ovoposición \cite{scott1993detection},
especialmente si es perturbado antes de finalizar si alimentación. En \cite{osoriopontificia},
observa que  22.56 \% de la población no realizó ninguna toma de sangre, por ende no generó
ninguna tanda de huevos,  43.6 \% realizaron una toma de sangre, de las cuales el 21.9\% de los
mosquitos no realizaron ovoposturas, 16.5 \% realizaron dos tomas de sangre, 8.42 \% realizaron
tres tomas de sangre, 6.9 \% realizaron cuatro tomas de sangre y 2.02 \% realizaron cinco tomas de
sangre. La cantidad de huevos en cada oviposición, luego de las alimentaciones sanguineas
correspondientes, varía entre 30 y 100 unidades \cite{luevano1993ciclo, beltran2001bionomia,cabezas2005dengue}.

El ciclo gonotrófico de los mosquitos es el nombre que se le adjudico al período que existe desde
que el mosquito realiza una alimentación sangínea - ovipostura - hasta una nueva alimentación.
Como se mencionó anteriormente en \secref{subsec:cap4-tasas de desarrollo}, la tasa de desarrollo
del ciclo gonotrófico puede estimarse mediante la versión simplificada del modelo de Sharpe y DeMichele \cite{sharpe1977reaction}, popuesta por Schoolfield en \cite{schoolfield1981non}.

Sea $R(k_{i})$ la tasa de desarrollo (ver ecuación \eqref{eq:schoolfield}) del ciclo gonotrófico
de una hembra (nulípara, parida), para una temperatura de $k_{i}$ Kelvin en un instante $i$, se
considera que un día es de ovipostura si se cumple :

\begin{equation}
\label{eq:ciclo-gonotrofico-ovipostura}
    \sum_{i=0}^{N} R(k_{i}) \geq 1
\end{equation}

Donde $N$ es la duración en días del ciclo gonotrófico de la hembra. Para $R(k)$ se aplican los
parámetros correspondientes a hembras nulíparas y paridas de acuerdo al estado de la hembra.
En \cite{edman1987host} se observó que hembras nulíparas de Aedes aegypti poseen un proceso de
digestión es más lento en las hembras paridas y por ende el ciclo gonotrófico de las mismas tiene
a ser más largo.

\subsection{Zonificación}
\label{subsec:cap4-zonificacion}
Cada entorno puede contar con factores que lo hagan más o menos apto para el desarrollo,
mortalidad, alimentación, disperción, y reproducción de individuos. En esta sección, con el fin de
simplificar ciertos aspectos muy especificos que se encuentran fuera del alcance de este trabajo,
realizaremos ciertas hipótesis generales, justificadas para nuestro caso de aplicación, pero puede
requerir una revisión en el caso general. Estas hipotesis son, los valores observados en un
conjuto de puntos de control, pertenecientes a una zona, permiten la caracterización de dicha zona como más o menos apta para desarrollo, mortalidad, alimentación, disperción, y reproducción de
individuos. Tambien consideramos que el tamaño de la zona, y por ende la cantidad de puntos de
control que pertenecena ella, influye en la caracterización de las zonas.

La zonificación surge ante necesidad de dividir el espacio de estudio de una forma más granular,
para identificar a los individuos que pertenecen a zonas aptas y los que no. Los puntos de control
distribuidos en un área de estudio nos permite estimar cuales zonas los niveles de riesgo e
infestación correspondiente a la abundacia de larvas por litro que se pueden observar.

En la \tabref{tab:cap4-puntaje-zona} se puden observar los rangos definidos para cada tipo de
zona, en donde $u(x,y)$ es la densidad relativa en un radio, $r$ , donde el valor de la densidad
realtiva es calculado mediante la ecuación \eqref{eq:interpolacion-idw}. Las límites para las
zonas fueron determinados clasificando los valores, de las hembras reproductivas en, grupos
múltiplos de cinco. No se estableció  un límite superior para las zonas óptimas debido a que los
valores mayores a el mínimo establecido, 70 larvas por litro, pertenecen a la misma categoría.

El tamaño del radio,$r$, es un parametro ajustable del modelo, mientras más grande sea el tamaño
del radio, más puntos serán incluidos para el cálculo, lo que gerará que las zonas tiendan a ser
similares.
Para el calculo de las hembras adultas y reproductivas para la clasificación de las zonas se
tuvieron en cuenta los siguientes criterios:

\begin{itemize}
    \item Solo el 50 \% de las larvas observadas son hembras.
    \item La temperatura media utilizada es de 25 \textcelsius.
    \item La mortalidad díaria natural bajo optimas condiciones,a 25 \textcelsius, es de 0,01
    según \eqref{eq:mortalidad-natural-larvas}.
    \item La tasa de desarrollo, a 25 \textcelsius, de la larva hasta su emergencia a adulto es de
    $11.57$ días \citep{rueda1990temperature}.
    \item El 32,10 \% de las hembras adultas no ovipone \citep{osoriopontificia}.
\end{itemize}


\begin{table}
    \begin{minipage}{\textwidth}
\begin{center}
    \caption{\label{tab:cap4-puntaje-zona} Clasificación de las zonas de acuerdo a la densidad de larvas por litro.}
    \begin{tabular}{p{3cm} c c c c}
        \\
                     & Mínimo$^a$ & Máximo$^a$ & Hembras     & Hembras$^c$ \\
        Tipo de zona & $u(x,y)$   & $u(x,y)$   & Adultas$^b$ & Reproductivas$^c$ \\
        \hline
        \hline\\
        Pésima  & 0  & 19 & 8  & 5 \\
        Mala    & 20 & 35 & 15 & 10\\
        Regular & 36 & 51 & 22 & 15\\
        Buena   & 52 & 69 & 30 & 20\\
        Óptima  & 70 & --$^d$ & --$^d$ & --$^d$\\
    \end{tabular}
    \footnotetext[1]{Rango mínimo y máximo de $u(x,y)$ permitido para el tipo de zona.}
    \footnotetext[2]{Cantidad máxima de hembras adultas, al final del periodo de desarrollo.}
    \footnotetext[3]{Cantidad de hembras adultas con capacidad de oviponer.}
    \footnotetext[4]{No se estableció un límite superior para las zonas óptimas. }
\end{center}
    \end{minipage}
\end{table}


\subsection{Vuelo y dispersión}
\label{sec:cap4-vuelo-dispersion}
El aedes aegypti es un mosquito doméstico que generalmente esta confinado a las casas donde se
cria \citep{luevano1993ciclo}, tiende a permanecer físicamente en donde emergió, siempre y cuando
no exista algún factor que la perturbe o no disponga de huéspedes, sitios de reposo y de postura
\citep{ThironIzcazaJ2003}. Por lo general mosquito no sobrepasa los 50 a 100 metros durante su vida
\citep{cabezas2005dengue}. En caso de no contar con sitios adecuados de ovipostura y disponibilidad
de alimento tienden a dispersarme una mayor distancia, hasta tres kilómetros, en busca de
mejores condiciones \citep{ThironIzcazaJ2003}. Los mosquito tienen la particularidad de que vuelan
en sentido contrario al la dirección al viento \citep{ThironIzcazaJ2003} y a una velocidad máxima
de $2 km/h$\citep{kaufmann2004flight}.

Exiten varios factores externos, que influyen el la dispersión del mosquito, disponiblidad de
alimentos, criadreos, sitios de reposo. Para las estrategias de control de Aedes aegypti en zonas
urbanas donde existen brotes de dengue y fiebre amarilla se asume que los mosquitos tienen un
rango de vuelo durante su vida de 50 a 100 metros\citep{dengueUruguayCap8}.

La función más importante del adulto de Aedes aegypti es la reproducción y secundariamente la
dispersión de la especie. Para Aedes aegypti el transporte pasivo de huevos y larvas en
recipientes ha tenido mayor trascendencia en su distribución, en la que el hombre ha participado
en forma determinante en comparación con la dispersión activa propia de la especie
\citep{ThironIzcazaJ2003}.

Para la dipsersión de los mosquitos consideramos a la zonificación como una aproximación válida
para la caracterización del habitad del mosquito. Para zonas de riegos calificadas como Malas o
Pésimas el rango de vuelo del mosquito se encuetra definido entre 100 metros a 3 kilometros y para
las demás zonas que no cumplan con dicho criterio el rango de vuelo se encuentra entre 0 y 100
metros. De este modo, los mosquitos que se encuentren en zonas menos aptas tenderán a desplzarce
en busca de mejores condiciones.







  %~ * Resultado del conteo de larvas
  %~ * Datos de origen
  %~ * Interpolación
  %~ * Representación
%~ * Predicción de Focos
%\section{Proceso Evolutivo}

Definimos un proceso evolutivo, para describir el conjunto de transformaciones o cambios, a través del tiempo, 
a lo que es sometido un antepasado común para generar nuevos descendientes y diversificar la población. Mediante 
este proceso se  busca reproducir el comportamiento, del mosquito transmisor del dengue \textit{Aedes Aegigty},
ante las variaciones climáticas a las que se encuentran sometidos, simulando las variaciones correspondientes a 
sus características biológicas, de forma a generar la suficiente información para realizar los análisis
correspondientes.

\subsection{Definición del problema}

El problema puede ser definido como, la necesidad de encontrar una función $PE$ que permita someter a los
individuos a ciertas variaciones climáticas teniendo en cuenta sus características biológicas descriptas en la 
sección \ref{sec:caracteristicas-biologicas}.

Sean:
%lista de definiciones de las variables básicas
\begin{description}[style=multiline,leftmargin=1.5cm]
    \item[$t_{i}$] utilizada para representar el instante de tiempo mediante a una estructura de datos compleja
    compuesta por datos climáticos y otras propiedades complementarias correspondientes al periodo tiempo
    individual $i$.\addsymbol{symbol:ti}
    
    \item[$p_{i}$] define a un individuo, que es equivalente a la unidad del \textit{Aedes Aegypti}, se encuentra
    representada como una estructura de datos compleja compuesta de propiedades como su  madurez y expectativa de
    vida, sexo y estado \addsymbol{symbol:pi}.

    \item[$\tau(p_{j})$] \addsymbol{symbol:estado-pj} es utilizada para representar las etapas o estados del ciclo
    de desarrollo del individuo $p_{j}$.
    \begin{align*}
        \tau (p_{j}) = [HUEVO, LARVA, PUPA, ADULTO]
    \end{align*}

    \item[$S(p_{j})$] \addsymbol{symbol:sexo-pj} Indicador binario del sexo de $p_{j}$, el cual toma el valor 1
    si $p_{j}$ es $HEMBRA$, y 0 en caso de que sea $MACHO$.

    \item[$\eta (p_{j})$] \addsymbol{symbol:ma-pj} variable numérica asignada a $p_{j}$ para representar el nivel
    de madurez. Su valor varía entre 0 y 100.
    
    \item[$\xi (p_{j})$] \addsymbol{symbol:ex-pj} variable numérica asignada a $p_{j}$ para representar la
    expectativa de vida. Su valor varía entre 0 y 100.
    
    \item[$T$] variable que define el rango de tiempo del estudio, se representa como una colección de $N$
    instantes $t_{i}$, donde $N$ es el tamaño del periodo de estudio \addsymbol{symbol:T}.
        \begin{align*}
            T = t_1,t_2,t_t,\ldots,t_N , & & 1 \leq i \leq N
        \end{align*}

    \item[$P$] la población, se define como una colección de $M$ individuos $p_{i}$, en donde $M$ es el tamaño de
    la población\addsymbol{symbol:P}.
    \begin{align*}
        P = p_1,p_2,p_t,\ldots,p_M,  & & 1 \leq j \leq M
    \end{align*}
    
\end{description}

Podemos definir $PE$ como una función dependiente, necesariamente, de $p_{j}$ y $t_{i}$. Donde esta se encarga
de modificar las propiedades como $\xi(p_{j})$, $\eta(p_{j})$ y $\tau(p_{j})$ de cada individuo $p_{j}$ que sea
sometido a sometido a las variaciones climáticas definidas en $t_{i}$, teniendo en  cuenta las características
biológicas de $p_{i}$ para cada $S(p_{j})$.

\begin{align*}
PE (t_{i}, p_{j}) = p_{j}' & & \forall t_{i} \in T\\
& & \forall p_{j} \in P 
\end{align*}

Donde $p_{j}'$ denota al j-esmimo individuo de la población $p_{j}$, luego de ser sometido a $PE$ para modificar
sus propiedades.  

%
\subsection{Operadores básicos}

Durante el proceso evolutivo existen un conjunto de operadores básicos que son aplicables a todos los individuos
$p_{j}$ de la población, que son utilizados por $PE(t_{i},p_{j})$ para modificar las características del 
individuo.

\subsubsection{Tasa de desarrollo}
En \cite{rueda1990temperature} para describir el efecto de la temperatura en la tasa media de desarrollo,
se utilizó el modelo no lineal de Sharpe \& DeMichele definido en \cite{sharpe1977reaction} para procesos
poiquilotermos\footnote{La poiquilotermia o ectotermia es un término aplicado a ciertos animales con temperatura
corporal variable} dependientes de la temperatura. El modelo de Sharpe \& DeMichele con inhibición de alta temperatura
se encuentra descripto, en \cite{rueda1990temperature}, por la siguiente ecuación.

\begin{equation} \label{eq:sharpe-demichele}
   r(k)  = \cfrac{ RH025 \cfrac{k}{298.15} * 
    exp \Bigg[
            \cfrac{HA}{1.987} \bigg(\cfrac{1}{298.15} - \cfrac{1}{k}\bigg)
        \Bigg]}
    {1 + exp\Bigg[\cfrac{HH}{1.987} \bigg(\cfrac{1}{TH}- \cfrac{1}{k}\bigg)\Bigg]} 
\end{equation}

Donde $r(k)$ representa la tasa de desarrollo media ($dias^-1$) para una temperatura $K$, que se encuentra en
la escala de kelvin. Los parámetros $RH025$, $HA$, $TH$, y $HH$  son estimados por la ecuación de regresión 
no lineal de \cite{wagner1984modeling}. Una vez determinados los parámetros, la ecuación puede utilizarse para
calcular tasas de desarrollo a cualquier temperatura\cite{rueda1990temperature}. En \cite{rueda1990temperature} 
se calcula los valores de $RH025$, $HA$, $TH$, y $HH$ para los 4 estadios larvales y cada etapa del ciclo de vida 
del \textit{Aedes Aegypti} (Ver tabla \ref{tab:tasa-desarrollo}).

\begin{table}
\begin{minipage}{\paperwidth}
\begin{tabular}{p{6cm} c c c c c }
Estado                     & $HA$    & $RH025$   & $TH$   & $HH$      & $R^2$   \\
\hline
Primer estadio larval      & 0.68007 & 28,033.83 & 304.33 & 72,404.07 & 0.95 \\
Segundo estadio larval     & 1.24508 & 36,400.55 & 301.78 & 81,383.14 & 0.96 \\
Tercer estadio larval      & 1.06144 & 41,192.69 & 301.29 & 60,832.62 & 0.97 \\
Cuarto estadio larval      & 0.57065 & 34,455.89 & 301.44 & 45,543.49 & 0.97 \\
Larva\footnote{Del primer al cuarto estadio larval} & 0.20429 & 36,072.78 & 301.56 & 59,147.51 & 0.97 \\
Pupa                       & 0.74423 & 19,246.42 & 302.68 & 5,954.35 & 0.98  \\
Total\footnote{Eclosión de los huevos a la emergencia de adultos}& 0.15460 & 33,255.57 & 301.67 & 50,543.49 & 0.98 \\

\end{tabular}
\end{minipage}
\caption{ \label{tab:tasa-desarrollo} Parámetros de ajuste estimados de la ecuación \eqref{eq:sharpe-demichele}, 
para la tasa de desarrollo del Aedes Aegypti (Tomado de \cite{rueda1990temperature})}
 
\end{table}

\subsubsection{Zonificación}
Los datos con los que cuenta $t_{i}$ corresponden a datos climáticos de la ciudad en la que se encuentra la
$P$, de esta forma todos los $p_{j}$ comparten las características de $t_{i}$. Si bien todos los individuos
comparten la mismas características de $t_{i}$, el nivel de impacto,de $t_{i}$, en cada uno de ellos, no
necesariamente será el mismo para todos, debido a que el entorno en el que se encuentra cada $p_{j}$ es 
distinto. Cada entorno puede contar con factores que lo hagan apto para el buen desarrollo de los individuos
como, así también con otros que causen una alta mortalidad.

La zonificación surge ante necesidad de dividir el espacio de estudio de una forma más granular, para identificar
a los individuos que pertenecen a zonas aptas y los que no.

\begin{table}
\centering
    \label{tab:puntaje-zona}
    \begin{tabular}{p{3cm} p{3cm}}
        Tipo de zona & Rango\\
        \hline
         Optima & 60 < $u(p_{i})$ \\
        Buena & 60 > $u(p_{i}) \geq 30$ \\
        Normal & 30 > $u(p_{i}) \geq 20$\\
        Mala & 20 > $u(p_{i}) \geq 8$\\
        Pésima & 8 > $u(p_{i}) \geq 0$ \\
    \end{tabular}
    \caption{División de zonas, donde $u(p_{i})$ se encuentra definida por la ecuación \eqref{eq:interpolacion-idw}}
\end{table}

\subsubsection{Madurez y cambio de estado}
La madurez $\eta (p_{j})$ de un individuo $p_{j}$, que se encuentra en un estado $\tau_{k}$, indica el nivel de
proximidad al siguiente estado $\tau_{k+1}$. Se encuentra representada por un número que varía entre
cero y cien, la función $\eta (t_{i}, p_{j})$, que se encarga de incrementar el nivel de madurez de $p{j}$
que es sometido a $t_{i}$.

\begin{equation}
\eta (t_{i}, p_{j}) = \left\{
  \begin{array}{l l}
    \eta (t_{i}, p_{j}) = 0 & \quad \forall i = 0 \\
    \eta (t_{i-1},p_{j}) + \cfrac{1}{\omega(t_{i}, p_{j}) * 24} & \quad \forall i \neq 0
  \end{array} \right.
\end{equation}

Donde 
\begin{equation}
    \omega(t_{i}, p_{j}) = 1/r(t_{i} + 273.15)
\end{equation}

Cuando $\eta(p_{i})$ del individuo $p_{j}$ alcanza su máximo valor, cien, el individuo ya se encuentra listo para
pasar del estado $\tau_{k}$ al siguiente estado $\tau_{k+1}$, a este proceso se lo denomina como cambio de estado.

La velocidad con la que incrementa el valor de $\eta(p_{j})$ depende de las condiciones a las condiciones
descritas  por $t_{i}$ y las características de la zona en la que se encuentra $p_{j}$.

\subsubsection{Mortalidad}
La mortalidad de los individuos depende de la etapa del ciclo de desarrollo en la que se encuentren. En 
\cite{otero2006stochastic} se define la mortalidad de huevos como una constante independiente a la temperatura:

\begin{equation}
 \begin{array}{l l}
    me = 0.01 & \forall T,  278 k \geq T \geq 303 k
\end{array}
\end{equation}

En el caso de las larvas \cite{otero2006stochastic} define que se encuentra influenciada dos métodos. La primera
define como una ecuación dependiente de temperatura. La segunda depende de la densidad poblacional del habitad de la
larva.
\begin{equation}
 \begin{array}{l l}
 ml = 0.01 + 0.9725 * exp\bigg[ \frac{-(T - 278)}{2.7035}\bigg] & \forall T, 278 k \geq T \geq 303 k
\end{array}
\end{equation}

Para la etapa adulto, \cite{otero2006stochastic} la define como una constante independiente de la temperatura.
\begin{equation}
 \begin{array}{l l}
    ma = 0.09 & \forall T, 278 k \geq T \geq 303 k
\end{array}
\end{equation}

La mortalidad y supervivencia de los individuos se encuentra expresada mediante la variable de expectativa de
vida. La expectativa de vida es un valor numérico que indica la vitalidad del individuos, esta varía de acuerdo
a las condiciones climáticas a las que es sometido el individuos durante el proceso evolutivo.

\begin{equation}
\xi (t_{i}, p_{j}) = \left\{
  \begin{array}{l l}
    100 & \quad \forall i = 0 \\
    \xi (t_{i-1}, p_{j}) - \frac{1}{\upsilon(t_{i}, p_{j}) * 24} & \quad \forall i \neq 0
  \end{array} \right.
\end{equation}

El valor de $\xi (t_{i}, p_{j})$ representa el porcentaje de vitalidad del
individuo $p_{j}$ luego de ser sometido a las variaciones del instante
$t_{i}$ perteneciente al periodo $T$ de estudio. La expectativa de vida
para cada individuo $p_{j} \in P$, en el instante $t_{0}$ se describe como
$\xi (t_{0}, p_{j})= 100$. A medida que $p_{j}$ sea sometido a varios
$t_{i}$ la expectativa de vida irá disminuyendo, si $\xi (t_{i}, p_{j})= 0$ el
individuo se ha quedado sin expectativa de vida, por lo que se debe proceder
a eliminar $p_{j}$ de $P$ mediante el proceso de reducción de la población
$\theta (t_{i}, p_{j})$.

\begin{equation}
\theta (t_{i}, p_{j}) = \left\{
  \begin{array}{l l}
    p_{j} = null & \quad \xi(t_{i}, p_{j}) == 0 \\
    p_{j} & \quad \text{en caso contrario}
  \end{array} \right.
\end{equation}

%
\subsection{Operadores complementarios}
Las etapas inmaduras del \em Aedes Aegigyti\em son principalmente acuáticas
y estaticas, por lo que todas cuentan con características similares, no
así la etapa adulto del mosquito que cuenta con ciertas características
que divergen mucho del comportamiento básico definido. Debido este
comportamiento especifico todos los $p_{j}$ que se encuentren en el estado
\em Adulto \em cuentan con un conjunto de operadores complementarios que
tienen por objetivo, con ayuda de los operadores básicos, describir el
comportamiento del mosquito en su estado adulto.

\subsubsection{Vuelo y búsqueda de alimentos}
Existen diferencias, en cuanto alimentación y vuelo, entre los adultos
machos y hembras. Los rondan en grupos pequeños o solitariamente,
principalmente atraídos por los mismos huéspedes vertebrados que las hembras.

Los mosquito tienen la particularidad de que vuelan en sentido contrario
al la dirección al viento y a una velocidad máxima $VMAX$ de $2 km/h$ según lo
mencionado en \cite{web-site:speedAnimals}.

La distancia $D(t_{i})$ recorrida por el individuo en un instante $t_{i}$
depende de la dirección del viento representada por $\alpha(t_{i}$, la
velocidad del viento $S(t_{i})$

\begin{equation}
 D (t_{i}) = \sqrt{{(\sin(\alpha(t_{i})) * VMAX  - S(t_{i}))}^{2}
  + {(\cos(\alpha(t_{i})) * VMAX} ^{2} }
\end{equation}

Las hembras se alimentan principalmente de sangre, que extraen de cualquier
vertebrado, por sus hábitos domésticos muestran marcada predilección por
la del hombre\cite{ThironIzcazaJ2003}. En cuanto a los machos sus partes
bucales no son aptas para chupar sangre, por lo que se alimentan de
carbohidratos de cualquier fuente accesible como frutos o néctar de flores
que satisface sus requerimientos energéticos.

\subsubsection{Reproducción y postura de huevos}

  %~ * Análisis predictivo
  %~ * Descripción del algoritmo de simulación
  %~ * Representación
  %~ * Resultados
