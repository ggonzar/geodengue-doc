%!TEX root = ../tesis.tex
\chapter{Solución propuesta}

  %~ * Introducción
El dengue ha sido considerado un problema de salud pública mundial, que afecta aproximadamente a
2.500 millones de personas que viven en zonas en riesgo de dengue y más de 100 países que han
informado de la presencia de esta enfermedad en su territorio \citet{gustavo2006dengue}. A lo
largo de los años se han desarrollado multiples modelos destinados a, la vigilancia, monitoreo y
la predicción de la transmisión del dengue.

La comunidad académica internacional reconoce que, el comportamiento, las tasas de desarrollo, la
mortalidad y la dispersión geográfica de los vectores transmisores del dengue, se encuentran
influenciados por las variaciones climáticas y del medio que los rodea. Según
\citet{velez2013hacia} una primera aproximación al entendimiento puede ser adquirida mediante el
diseño de modelos matemáticos que integren los múltiples factores de riesgo, obtenidos como un
producto entre factores de amenaza y vulnerabilidad, propios de cada población. Estos modelos,
cuya principal tarea sería profundizar el entendimiento de las dinámicas de transimsión, además
pueden actuar como un simulador del comportamiento de los vectores del dengue en diferentes
escenarios climáticos, entomológicos, sociodemográficos y culturales futuros. En
\citet{velez2013hacia} los autores señalan, que además de la significativa ganancia en
conocimiento, los modelos matemáticos podrían, entonces, contribuir a la detección temprana y
oportuna del momento de ocurrencia y el orden de magnitud de brotes epidémicos, lo cual brindaría
una importante capacidad de anticipación para la prevención de eventos que generan significativas
alteraciones en la salud y, en casos serios, la muerte de muchos pacientes en nuestro medio.

En este capítulo se presentan las metodologias empleadas para la identificación de focos de dengue
y los modelos utilizados para la simulación del comportamiento de los vectores del dengue en
diferentes escenarios climáticos.



%~ * Identificación de focos de Riesgo
%!TEX root = ../tesis.tex
\section{Identificación de focos de infestación}
\label{sec:cap4-identificacion-focos}

El análisis de la distribución espacial y temporal de las poblaciones del vector, puede llegar a
jugar un papel importante en la planificación y evaluación de medidas orientadas a la disminución
de las poblaciones del vector y en consecuencia, reducir los casos de dengue
\cite{dengueUruguayCap1, cenaprece2013,nino2008uso}. Los SIG constituyen una herramienta esencial
para el análisis de la distribución espacial de las poblaciones
\cite{vgomesAegis2001,petric2012surveillance}, permitiendo obtener mejores resultados en
combinación con las metodologías de vigilancia entomológica y medicas\cite{petric2012surveillance}.

Las metodologías de vigilancia entomológica basadas en la distribución geográfica de larvitrampas
u ovitrampas, como las presentadas en
\cite{NINO2011,petric2012surveillance, journal.pone.0054167,nino2008uso}, permiten generar
información regionalizada sobre la abundancia poblacional del vector \cite{NINO2011}. Los datos
sobre los mosquitos atrapados deben mantenerse para crear un registro histórico de las especies de
mosquitos que se encuentran en asociación con diferentes hábitats y patógenos para permitir la
detección temprana de las adaptaciones \cite{petric2012surveillance}.

Para la identificación de focos de infestación se propone, utilizar la metodología de vigilancia
entomológica basada en el uso de larvitrampas propuesta en \cite{NINO2011}. En donde las
larvitrampas, o puntos de control, deben ser distribuidas geográficamente para generar información
regionalizada que, mediante técnicas de interpolación espacial, permiten obtener mapas de
interpolación donde se puede apreciar los niveles de infestación del vector, y el riesgo
correspondiente a la abundacia de mosqutios observada en el área de estudio \cite{NINO2011, nino2008uso, journal.pone.0054167, albierispatial}. El hecho de contar con esta información
regionalizada permitirá a las autoridades pertinentes definir y planificar mejor las medidas de
prevención y control a realizarce para reducir los niveles de infestación en las zonas criticas
\cite{NINO2011, nino2008uso, petric2012surveillance}.

La selección del método de interpolación se realizó teniendo en cuenta el factor humano en la
distribución de los puntos de control. En \cite{villatoro2007comparacion} se realiza una
comparación de los interpoladores IDW y Kriging, donde los autores señalan que el método Kriging
fue más preciso y eficiente que el IDW, aunque la diferencia entre ambos métodos no fue muy amplia.
Sin embargo, cuando el distanciamiento, es muy grande, los variogramas no son posibles de obtener,
entonces el Kriging deja de ser una opción y comparativamente el IDW se perfila como el mejor
\cite{villatoro2007comparacion}. El método seleccionado finalmente fue el IDW, debido que la
distribución del los puntos de control no será perfecta, inclusive, en algunas localidades la
distribución no será de forma uniforme.

Los mapas de interpolación resultantes, indican, con mayor detalle que los índices aédicos
tradicionales, los lugares específicos donde sería necesario tomar medidas de prevención y
control de acuerdo al grado de infestación, permitiendo así una mayor racionalización de tiempo y
recursos \cite{NINO2011}.

  %~ * Resultado del conteo de larvas
  %~ * Datos de origen
  %~ * Interpolación
  %~ * Representación
%~ * Predicción de Focos
\section{Proceso Evolutivo}

Definimos un proceso evolutivo, para describir el conjunto de transformaciones o cambios, a través del tiempo, 
a lo que es sometido un antepasado común para generar nuevos descendientes y diversificar la población. Mediante 
este proceso se  busca reproducir el comportamiento, del mosquito transmisor del dengue \textit{Aedes Aegigty},
ante las variaciones climáticas a las que se encuentran sometidos, simulando las variaciones correspondientes a 
sus características biológicas, de forma a generar la suficiente información para realizar los análisis
correspondientes.

\subsection{Definición del problema}

El problema puede ser definido como, la necesidad de encontrar una función $PE$ que permita someter a los
individuos a ciertas variaciones climáticas teniendo en cuenta sus características biológicas descriptas en la 
sección \ref{sec:caracteristicas-biologicas}.

Sean:
%lista de definiciones de las variables básicas
\begin{description}[style=multiline,leftmargin=1.5cm]
    \item[$t_{i}$] utilizada para representar el instante de tiempo mediante a una estructura de datos compleja
    compuesta por datos climáticos y otras propiedades complementarias correspondientes al periodo tiempo
    individual $i$.\addsymbol{symbol:ti}
    
    \item[$p_{i}$] define a un individuo, que es equivalente a la unidad del \textit{Aedes Aegypti}, se encuentra
    representada como una estructura de datos compleja compuesta de propiedades como su  madurez y expectativa de
    vida, sexo y estado \addsymbol{symbol:pi}.

    \item[$\tau(p_{j})$] \addsymbol{symbol:estado-pj} es utilizada para representar las etapas o estados del ciclo
    de desarrollo del individuo $p_{j}$.
    \begin{align*}
        \tau (p_{j}) = [HUEVO, LARVA, PUPA, ADULTO]
    \end{align*}

    \item[$S(p_{j})$] \addsymbol{symbol:sexo-pj} Indicador binario del sexo de $p_{j}$, el cual toma el valor 1
    si $p_{j}$ es $HEMBRA$, y 0 en caso de que sea $MACHO$.

    \item[$\eta (p_{j})$] \addsymbol{symbol:ma-pj} variable numérica asignada a $p_{j}$ para representar el nivel
    de madurez. Su valor varía entre 0 y 100.
    
    \item[$\xi (p_{j})$] \addsymbol{symbol:ex-pj} variable numérica asignada a $p_{j}$ para representar la
    expectativa de vida. Su valor varía entre 0 y 100.
    
    \item[$T$] variable que define el rango de tiempo del estudio, se representa como una colección de $N$
    instantes $t_{i}$, donde $N$ es el tamaño del periodo de estudio \addsymbol{symbol:T}.
        \begin{align*}
            T = t_1,t_2,t_t,\ldots,t_N , & & 1 \leq i \leq N
        \end{align*}

    \item[$P$] la población, se define como una colección de $M$ individuos $p_{i}$, en donde $M$ es el tamaño de
    la población\addsymbol{symbol:P}.
    \begin{align*}
        P = p_1,p_2,p_t,\ldots,p_M,  & & 1 \leq j \leq M
    \end{align*}
    
\end{description}

Podemos definir $PE$ como una función dependiente, necesariamente, de $p_{j}$ y $t_{i}$. Donde esta se encarga
de modificar las propiedades como $\xi(p_{j})$, $\eta(p_{j})$ y $\tau(p_{j})$ de cada individuo $p_{j}$ que sea
sometido a sometido a las variaciones climáticas definidas en $t_{i}$, teniendo en  cuenta las características
biológicas de $p_{i}$ para cada $S(p_{j})$.

\begin{align*}
PE (t_{i}, p_{j}) = p_{j}' & & \forall t_{i} \in T\\
& & \forall p_{j} \in P 
\end{align*}

Donde $p_{j}'$ denota al j-esmimo individuo de la población $p_{j}$, luego de ser sometido a $PE$ para modificar
sus propiedades.  


\subsection{Operadores básicos}

Durante el proceso evolutivo existen un conjunto de operadores básicos que son aplicables a todos los individuos
$p_{j}$ de la población, que son utilizados por $PE(t_{i},p_{j})$ para modificar las características del 
individuo.

\subsubsection{Tasa de desarrollo}
En \cite{rueda1990temperature} para describir el efecto de la temperatura en la tasa media de desarrollo,
se utilizó el modelo no lineal de Sharpe \& DeMichele definido en \cite{sharpe1977reaction} para procesos
poiquilotermos\footnote{La poiquilotermia o ectotermia es un término aplicado a ciertos animales con temperatura
corporal variable} dependientes de la temperatura. El modelo de Sharpe \& DeMichele con inhibición de alta temperatura
se encuentra descripto, en \cite{rueda1990temperature}, por la siguiente ecuación.

\begin{equation} \label{eq:sharpe-demichele}
   r(k)  = \cfrac{ RH025 \cfrac{k}{298.15} * 
    exp \Bigg[
            \cfrac{HA}{1.987} \bigg(\cfrac{1}{298.15} - \cfrac{1}{k}\bigg)
        \Bigg]}
    {1 + exp\Bigg[\cfrac{HH}{1.987} \bigg(\cfrac{1}{TH}- \cfrac{1}{k}\bigg)\Bigg]} 
\end{equation}

Donde $r(k)$ representa la tasa de desarrollo media ($dias^-1$) para una temperatura $K$, que se encuentra en
la escala de kelvin. Los parámetros $RH025$, $HA$, $TH$, y $HH$  son estimados por la ecuación de regresión 
no lineal de \cite{wagner1984modeling}. Una vez determinados los parámetros, la ecuación puede utilizarse para
calcular tasas de desarrollo a cualquier temperatura\cite{rueda1990temperature}. En \cite{rueda1990temperature} 
se calcula los valores de $RH025$, $HA$, $TH$, y $HH$ para los 4 estadios larvales y cada etapa del ciclo de vida 
del \textit{Aedes Aegypti} (Ver tabla \ref{tab:tasa-desarrollo}).

\begin{table}
\begin{minipage}{\paperwidth}
\begin{tabular}{p{6cm} c c c c c }
Estado                     & $HA$    & $RH025$   & $TH$   & $HH$      & $R^2$   \\
\hline
Primer estadio larval      & 0.68007 & 28,033.83 & 304.33 & 72,404.07 & 0.95 \\
Segundo estadio larval     & 1.24508 & 36,400.55 & 301.78 & 81,383.14 & 0.96 \\
Tercer estadio larval      & 1.06144 & 41,192.69 & 301.29 & 60,832.62 & 0.97 \\
Cuarto estadio larval      & 0.57065 & 34,455.89 & 301.44 & 45,543.49 & 0.97 \\
Larva\footnote{Del primer al cuarto estadio larval} & 0.20429 & 36,072.78 & 301.56 & 59,147.51 & 0.97 \\
Pupa                       & 0.74423 & 19,246.42 & 302.68 & 5,954.35 & 0.98  \\
Total\footnote{Eclosión de los huevos a la emergencia de adultos}& 0.15460 & 33,255.57 & 301.67 & 50,543.49 & 0.98 \\

\end{tabular}
\end{minipage}
\caption{ \label{tab:tasa-desarrollo} Parámetros de ajuste estimados de la ecuación \eqref{eq:sharpe-demichele}, 
para la tasa de desarrollo del Aedes Aegypti (Tomado de \cite{rueda1990temperature})}
 
\end{table}

\subsubsection{Zonificación}
Los datos con los que cuenta $t_{i}$ corresponden a datos climáticos de la ciudad en la que se encuentra la
$P$, de esta forma todos los $p_{j}$ comparten las características de $t_{i}$. Si bien todos los individuos
comparten la mismas características de $t_{i}$, el nivel de impacto,de $t_{i}$, en cada uno de ellos, no
necesariamente será el mismo para todos, debido a que el entorno en el que se encuentra cada $p_{j}$ es 
distinto. Cada entorno puede contar con factores que lo hagan apto para el buen desarrollo de los individuos
como, así también con otros que causen una alta mortalidad.

La zonificación surge ante necesidad de dividir el espacio de estudio de una forma más granular, para identificar
a los individuos que pertenecen a zonas aptas y los que no.

\begin{table}
\centering
    \label{tab:puntaje-zona}
    \begin{tabular}{p{3cm} p{3cm}}
        Tipo de zona & Rango\\
        \hline
         Optima & 60 < $u(p_{i})$ \\
        Buena & 60 > $u(p_{i}) \geq 30$ \\
        Normal & 30 > $u(p_{i}) \geq 20$\\
        Mala & 20 > $u(p_{i}) \geq 8$\\
        Pésima & 8 > $u(p_{i}) \geq 0$ \\
    \end{tabular}
    \caption{División de zonas, donde $u(p_{i})$ se encuentra definida por la ecuación \eqref{eq:interpolacion-idw}}
\end{table}

\subsubsection{Madurez y cambio de estado}
La madurez $\eta (p_{j})$ de un individuo $p_{j}$, que se encuentra en un estado $\tau_{k}$, indica el nivel de
proximidad al siguiente estado $\tau_{k+1}$. Se encuentra representada por un número que varía entre
cero y cien, la función $\eta (t_{i}, p_{j})$, que se encarga de incrementar el nivel de madurez de $p{j}$
que es sometido a $t_{i}$.

\begin{equation}
\eta (t_{i}, p_{j}) = \left\{
  \begin{array}{l l}
    \eta (t_{i}, p_{j}) = 0 & \quad \forall i = 0 \\
    \eta (t_{i-1},p_{j}) + \cfrac{1}{\omega(t_{i}, p_{j}) * 24} & \quad \forall i \neq 0
  \end{array} \right.
\end{equation}

Donde 
\begin{equation}
    \omega(t_{i}, p_{j}) = 1/r(t_{i} + 273.15)
\end{equation}

Cuando $\eta(p_{i})$ del individuo $p_{j}$ alcanza su máximo valor, cien, el individuo ya se encuentra listo para
pasar del estado $\tau_{k}$ al siguiente estado $\tau_{k+1}$, a este proceso se lo denomina como cambio de estado.

La velocidad con la que incrementa el valor de $\eta(p_{j})$ depende de las condiciones a las condiciones
descritas  por $t_{i}$ y las características de la zona en la que se encuentra $p_{j}$.

\subsubsection{Mortalidad}
La mortalidad de los individuos depende de la etapa del ciclo de desarrollo en la que se encuentren. En 
\cite{otero2006stochastic} se define la mortalidad de huevos como una constante independiente a la temperatura:

\begin{equation}
 \begin{array}{l l}
    me = 0.01 & \forall T,  278 k \geq T \geq 303 k
\end{array}
\end{equation}

En el caso de las larvas \cite{otero2006stochastic} define que se encuentra influenciada dos métodos. La primera
define como una ecuación dependiente de temperatura. La segunda depende de la densidad poblacional del habitad de la
larva.
\begin{equation}
 \begin{array}{l l}
 ml = 0.01 + 0.9725 * exp\bigg[ \frac{-(T - 278)}{2.7035}\bigg] & \forall T, 278 k \geq T \geq 303 k
\end{array}
\end{equation}

Para la etapa adulto, \cite{otero2006stochastic} la define como una constante independiente de la temperatura.
\begin{equation}
 \begin{array}{l l}
    ma = 0.09 & \forall T, 278 k \geq T \geq 303 k
\end{array}
\end{equation}

La mortalidad y supervivencia de los individuos se encuentra expresada mediante la variable de expectativa de
vida. La expectativa de vida es un valor numérico que indica la vitalidad del individuos, esta varía de acuerdo
a las condiciones climáticas a las que es sometido el individuos durante el proceso evolutivo.

\begin{equation}
\xi (t_{i}, p_{j}) = \left\{
  \begin{array}{l l}
    100 & \quad \forall i = 0 \\
    \xi (t_{i-1}, p_{j}) - \frac{1}{\upsilon(t_{i}, p_{j}) * 24} & \quad \forall i \neq 0
  \end{array} \right.
\end{equation}

El valor de $\xi (t_{i}, p_{j})$ representa el porcentaje de vitalidad del
individuo $p_{j}$ luego de ser sometido a las variaciones del instante
$t_{i}$ perteneciente al periodo $T$ de estudio. La expectativa de vida
para cada individuo $p_{j} \in P$, en el instante $t_{0}$ se describe como
$\xi (t_{0}, p_{j})= 100$. A medida que $p_{j}$ sea sometido a varios
$t_{i}$ la expectativa de vida irá disminuyendo, si $\xi (t_{i}, p_{j})= 0$ el
individuo se ha quedado sin expectativa de vida, por lo que se debe proceder
a eliminar $p_{j}$ de $P$ mediante el proceso de reducción de la población
$\theta (t_{i}, p_{j})$.

\begin{equation}
\theta (t_{i}, p_{j}) = \left\{
  \begin{array}{l l}
    p_{j} = null & \quad \xi(t_{i}, p_{j}) == 0 \\
    p_{j} & \quad \text{en caso contrario}
  \end{array} \right.
\end{equation}


\subsection{Operadores complementarios}
Las etapas inmaduras del \em Aedes Aegigyti\em son principalmente acuáticas
y estaticas, por lo que todas cuentan con características similares, no
así la etapa adulto del mosquito que cuenta con ciertas características
que divergen mucho del comportamiento básico definido. Debido este
comportamiento especifico todos los $p_{j}$ que se encuentren en el estado
\em Adulto \em cuentan con un conjunto de operadores complementarios que
tienen por objetivo, con ayuda de los operadores básicos, describir el
comportamiento del mosquito en su estado adulto.

\subsubsection{Vuelo y búsqueda de alimentos}
Existen diferencias, en cuanto alimentación y vuelo, entre los adultos
machos y hembras. Los rondan en grupos pequeños o solitariamente,
principalmente atraídos por los mismos huéspedes vertebrados que las hembras.

Los mosquito tienen la particularidad de que vuelan en sentido contrario
al la dirección al viento y a una velocidad máxima $VMAX$ de $2 km/h$ según lo
mencionado en \cite{web-site:speedAnimals}.

La distancia $D(t_{i})$ recorrida por el individuo en un instante $t_{i}$
depende de la dirección del viento representada por $\alpha(t_{i}$, la
velocidad del viento $S(t_{i})$

\begin{equation}
 D (t_{i}) = \sqrt{{(\sin(\alpha(t_{i})) * VMAX  - S(t_{i}))}^{2}
  + {(\cos(\alpha(t_{i})) * VMAX} ^{2} }
\end{equation}

Las hembras se alimentan principalmente de sangre, que extraen de cualquier
vertebrado, por sus hábitos domésticos muestran marcada predilección por
la del hombre\cite{ThironIzcazaJ2003}. En cuanto a los machos sus partes
bucales no son aptas para chupar sangre, por lo que se alimentan de
carbohidratos de cualquier fuente accesible como frutos o néctar de flores
que satisface sus requerimientos energéticos.

\subsubsection{Reproducción y postura de huevos}

  %~ * Análisis predictivo
  %~ * Descripción del algoritmo de simulación
  %~ * Representación
  %~ * Resultados
