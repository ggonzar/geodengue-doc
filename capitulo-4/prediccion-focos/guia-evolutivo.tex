\section{Proceso Evolutivo}


\textbf{Definición 1.} \em Instante \em : Se utiliza el término instante
    para representar a una estrucutra de datos compleja compuesta por
    datos climáticos correspondientes a periodos de tiempos individuales
    $t_{i}$ y otras propiedades complementarias.


\textbf{Definición 2.}\em Peridodo \em : Un periodo es una estrucutra de
    datos que representa a un conjunto de instantes $t_{i}$. El periodo
    $T$ \addsymbol{symbol:P} se define como una colección de $N$ instantes
    $t_{i}$, donde $N$ es el tamaño del periodo de estudio.

    \begin{align*}
        T = t_1,t_2,t_t,\ldots,t_N , & & 1 \leq i \leq N
    \end{align*}


\textbf{Definición 3.} \em Estado \em : Se utliza el término estado
    para representar las etapas del ciclo de desarrollo del \em Aedes Aegypti\em.
    \begin{align*}
        \tau = [huevo, larva, pupa, adulto]
    \end{align*}

\textbf{Definición 3.} \em Indiviuduo\em : Se utliza el término individuo
    para representar la unidad del \em Aedes Aegypti \em cualesquiera de sus
    estados, en forma de una estrucutra de datos compleja compuesta de
    propiedades como su madurez y espectativa de vida.

\textbf{Definición 4.} \em Población\em : La población $P$ es una estuctura
    de datos que representa a un grupo de individuos $p_{i}$. Se define como
    una colección de $M$ individuos $p_{i}$, en donde $M$ es el tamaño de
    la población.

    \begin{align*}
        P = p_1,p_2,p_t,\ldots,p_M,  & & 1 \leq i \leq M
    \end{align*}


El proceso evolutivo consiste en somenter los datos, de las muestras obtneidas
mediante la utilización de dispositivos de ovipostura, a las multiples
variaciones climáticas y las caracteristicas biológicas del inividuo.

Cada $p_{i}$ que pertenzca a $P$ es sometido a las variaciones climáticas de
cada $t_{i}$ que pertenezca a $T$ siempre y cuando $p_{i}$ aún forme parte
de la población $P$.

\section{Madurez y mortalidad}
\subsection{Madurez}
Para cada $p_{i}$ existe asociado un atributo madurez \addsymbol{symbol:mIi} es
un valor numérico(entre 0 y 100) que varía de acuerdo a las condiciones
climáticas a las que es sometido el mosquito. Cuando la madurez es igual
a 100 el mosquito ya se encuentra listo para un cambio de estado.

\begin{equation}
\eta (t_{i}, p_{j}) = \left\{
  \begin{array}{l l}
    0 & \quad \forall i = 0 \\
    \eta (t_{i-1}, p_{j}) + \frac{1}{\omega(t_{i}) * 24} & \quad \forall i \neq 0
  \end{array} \right.
\end{equation}

Cambio de estado

\begin{equation}
\tau_{k} (t_{i}, p_{j}) = \left\{
  \begin{array}{l l}
    \tau_{k+1} & \quad \forall  \eta(t_{i}, p_{j}) = 100 \\
    \tau_{k} & \quad \text{en caso contrario}
  \end{array} \right.
\end{equation}



\subsection{Mortalidad}
La mortalidad y supervivencia de los individuos se encuentra expresada
mediante la variable de expectativa de vida. La expectativa de vida es
un valor numérico que indica la vitalidad del individuos, esta varía de
acuerdo a las condiciones climáticas a las que es sometido el individuos
durante el proceso evolutivo.

\begin{equation}
\xi (t_{i}, p_{j}) = \left\{
  \begin{array}{l l}
    100 & \quad \forall i = 0 \\
    \xi (t_{i-1}, p_{j}) - \frac{1}{\upsilon(t_{i}) * 24} & \quad \forall i \neq 0
  \end{array} \right.
\end{equation}

El valor de $\xi (t_{i}, p_{j})$ representa el porcentaje de vitalidad del
inidividuo $p_{j}$ luego de ser somenetido a las variaciones del instante
$t_{i}$ perteneciente al periodo $T$ de estudio. La espectativa de vida
para cada individuo $p_{j} \in P$, en el isntante $t_{0}$ se describe como
$\xi (t_{0}, p_{j})= 100$. A medida que $p_{j}$ sea sometido a varios
$t_{i}$ la espectativa de vida irá disminuyendo, si $\xi (t_{i}, p_{j})= 0$ el
idividuo se ha quedado sin espectativa de vida, por lo que se debe proceder
a eliminar $p_{j}$ de $P$ mediante el proceso de reducción de la población
$\theta (t_{i}, p_{j})$.

\begin{equation}
\theta (t_{i}, p_{j}) = \left\{
  \begin{array}{l l}
    p_{j} = null & \quad \xi(t_{i}, p_{j}) == 0 \\
    p_{j} & \quad \text{en caso contrario}
  \end{array} \right.
\end{equation}


\section{esta maduro}
\section{pone huevos}
\begin{equation}
f(T_{i}) = \left\{
  \begin{array}{l l}
    \text{poner huevos}(T_{i}) & \quad \text{si $estado$ es  Adulto}\\
    false & \quad \text{en caso contrario}
  \end{array} \right.
\end{equation}

\section{para el adulto}
\section{vuelo}
\section{reproducción}
\section{poner huevos}
\section{buscar alimentos}
