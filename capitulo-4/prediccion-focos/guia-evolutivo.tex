\section{Proceso Evolutivo}
El proceso evolutivo, es el encargado de realizar los cambio o transformación
de forma gradual de los individuos de la población inicial, sometiéndolos a
variaciones climáticas durante un periodo de tiempo. Mediante este proceso se 
busca reproducir el comportamiento, del mosquito transmisor del dengue \textit{Aedes Aegigty}, 
ante las variaciones climáticas a las que se encuentran sometidos, de forma a
generar la suficiente información para realizar los análisis correspondientes.

\subsection{Definición del problema}

El problema puede ser definido como, la necesidad de encontrar una función $PE$ que permita someter a los
individuos a ciertas variaciones climáticas teniendo en cuenta sus características biológicas.

Sean:
%lista de definiciones de las variables básicas
\begin{description}[style=multiline,leftmargin=1.5cm]
    \item[$t_{i}$] utilizada para representarel instante de tiempo mediante a una estructura de datos compleja
    compuesta por datos climáticos y otras propiedades complementarias correspondientes al periodo tiempo
    individual $i$.\addsymbol{symbol:ti}
    
    \item[$p_{i}$] define a un individuo, que es equivalente a la unidad del \textit{Aedes Aegypti}, se encuentra
    representada como una estructura de datos compleja compuesta de propiedades como su  madurez y expectativa de
    vida, sexo y estado.

    \item[$\tau(p_{j})$] \addsymbol{symbol:estado-pj} es utilizada para representar las etapas o estados del ciclo
    de desarrollo del individuo $p_{j}$.
    \begin{align*}
        \tau (p_{j}) = [HUEVO, LARVA, PUPA, ADULTO]
    \end{align*}

    \item[$S(p_{j})$] \addsymbol{symbol:sexo-pj} Indicador binario del sexo de $p_{j}$, el cual toma el valor 1
    si $p_{j}$ es $HEMBRA$, y 0 en caso de que sea $MACHO$.

    \item[$\eta (p_{j})$] \addsymbol{symbol:ma-pj} variable numérica asignada a $p_{j}$ para representar el nivel
    de madurez. Su valor varía entre 0 y 100.
    
    \item[$\xi (p_{j})$] \addsymbol{symbol:ex-pj} variable numérica asignada a $p_{j}$ para representar la
    expectativa de vida. Su valor varía entre 0 y 100.
    
    \item[$T$] variable que define el rango de tiempo del estudio, se representa como una colección de $N$
    instantes $t_{i}$, donde $N$ es el tamaño del periodo de estudio \addsymbol{symbol:T}.
        \begin{align*}
            T = t_1,t_2,t_t,\ldots,t_N , & & 1 \leq i \leq N
        \end{align*}

    \item[$P$] la población, se define como una colección de $M$ individuos $p_{i}$, en donde $M$ es el tamaño de
    la población.
    \begin{align*}
        P = p_1,p_2,p_t,\ldots,p_M,  & & 1 \leq j \leq M
    \end{align*}
    
\end{description}

Podemos definir $PE$ como una función dependiente, necesariamente, de $p_{j}$ y $t_{i}$. Donde esta se encarga
de modificar las propiedades como $\xi(p_{j})$, $\eta(p_{j})$ y $\tau(p_{j})$ de cada individuo $p_{j}$ que sea
sometido a sometido a las variaciones climáticas definidas en $t_{i}$, teniendo en  cuenta las características
biológicas de $p_{i}$ para cada $S(p_{j})$.

\begin{align*}
PE (t_{i}, p_{j}) = p_{j}' & & \forall t_{i} \in T\\
& & \forall p_{j} \in P 
\end{align*}

Donde $p_{j}'$ denota al j-esmimo individuo de la población $p_{j}$, luego de ser sometido a $PE$ para modificar
sus propiedades.  

\subsection{Operadores básicos}

Durante el proceso evolutivo existen un conjunto de operadores básicos que son aplicables a todos los individuos
$p_{j}$ de la población, que son utilizados por $PE(t_{i},p_{j})$ para modificar las características del 
individuo.

\subsubsection{Zonificación}
Los datos con los que cuenta $t_{i}$ corresponden a datos climáticos de la ciudad en la que se encuentra la
$P$, de esta forma todos los $p_{j}$ comparten las características de $t_{i}$. Si bien todos los individuos
comparten la mismas características de $t_{i}$, el nivel de impacto,de $t_{i}$, en cada uno de ellos, no
necesariamente será el mismo para todos, debido a que el entorno en el que se encuentra cada $p_{j}$ es 
distinto. Cada entorno puede contar con factores que lo hagan apto para el buen desarrollo de los individuos
como, así también con otros que causen una alta mortalidad.

La zonificación surge ante necesidad de dividir el espacio de estudio de una forma más granular, para identificar
a los individuos que pertenecen a zonas aptas y los que no. 


\subsubsection{Madurez y cambio de estado}
La madurez $\eta (p_{j})$ de un individuo $p_{j}$, que se encuentra en un estado $\tau_{k}$, indica el nivel de
proximidad al siguiente estado $\tau_{k+1}$. Se encuentra representada por un número que varía entre
cero y cien, la función $\eta (t_{i}, p_{j})$, que se encarga de incrementar el nivel de madurez de $p{j}$
que es sometido a $t_{i}$.

\begin{equation}
\eta (t_{i}, p_{j}) = \left\{
  \begin{array}{l l}
    \eta (p_{j}) = 0 & \quad \forall i = 0 \\
    \eta (p_{j}) + \frac{1}{\omega(t_{i}, p_{j}) * 24} & \quad \forall i \neq 0
  \end{array} \right.
\end{equation}

Cuando $\eta(p_{i})$ del individuo $p_{j}$ alcanza su máximo valor, cien, el individuo ya se encuentra listo para
pasar del estado $\tau_{k}$ al siguiente estado $\tau_{k+1}$, a este proceso se lo denomina como cambio de estado.

La velocidad con la que incrementa el valor de $\eta(p_{j})$ depende de las condiciones a las condiciones
descritas  por $t_{i}$ y las características de la zona en la que se encuentra $p_{j}$.

\subsubsection{Mortalidad}
La mortalidad y supervivencia de los individuos se encuentra expresada mediante la variable de expectativa de
vida. La expectativa de vida es un valor numérico que indica la vitalidad del individuos, esta varía de acuerdo
a las condiciones climáticas a las que es sometido el individuos durante el proceso evolutivo.

\begin{equation}
\xi (t_{i}, p_{j}) = \left\{
  \begin{array}{l l}
    100 & \quad \forall i = 0 \\
    \xi (t_{i-1}, p_{j}) - \frac{1}{\upsilon(t_{i}, p_{j}) * 24} & \quad \forall i \neq 0
  \end{array} \right.
\end{equation}

El valor de $\xi (t_{i}, p_{j})$ representa el porcentaje de vitalidad del
individuo $p_{j}$ luego de ser sometido a las variaciones del instante
$t_{i}$ perteneciente al periodo $T$ de estudio. La expectativa de vida
para cada individuo $p_{j} \in P$, en el instante $t_{0}$ se describe como
$\xi (t_{0}, p_{j})= 100$. A medida que $p_{j}$ sea sometido a varios
$t_{i}$ la expectativa de vida irá disminuyendo, si $\xi (t_{i}, p_{j})= 0$ el
individuo se ha quedado sin expectativa de vida, por lo que se debe proceder
a eliminar $p_{j}$ de $P$ mediante el proceso de reducción de la población
$\theta (t_{i}, p_{j})$.

\begin{equation}
\theta (t_{i}, p_{j}) = \left\{
  \begin{array}{l l}
    p_{j} = null & \quad \xi(t_{i}, p_{j}) == 0 \\
    p_{j} & \quad \text{en caso contrario}
  \end{array} \right.
\end{equation}

\subsection{Operadores complementarios}
Las etapas inmaduras del \em Aedes Aegigyti\em son principalmente acuáticas
y estaticas, por lo que todas cuentan con características similares, no
así la etapa adulto del mosquito que cuenta con ciertas características
que divergen mucho del comportamiento básico definido. Debido este
comportamiento especifico todos los $p_{j}$ que se encuentren en el estado
\em Adulto \em cuentan con un conjunto de operadores complementarios que
tienen por objetivo, con ayuda de los operadores básicos, describir el
comportamiento del mosquito en su estado adulto.

\subsubsection{Vuelo y búsqueda de alimentos}
Existen diferencias, en cuanto alimentación y vuelo, entre los adultos
machos y hembras. Los rondan en grupos pequeños o solitariamente,
principalmente atraídos por los mismos huéspedes vertebrados que las hembras.

Los mosquito tienen la particularidad de que vuelan en sentido contrario
al la dirección al viento y a una velocidad máxima $VMAX$ de $2 km/h$ según lo
mencionado en \cite{web-site:speedAnimals}.

La distancia $D(t_{i})$ recorrida por el individuo en un instante $t_{i}$
depende de la dirección del viento representada por $\alpha(t_{i}$, la
velocidad del viento $S(t_{i})$

\begin{equation}
 D (t_{i}) = \sqrt{{(\sin(\alpha(t_{i})) * VMAX  - S(t_{i}))}^{2}
  + {(\cos(\alpha(t_{i})) * VMAX} ^{2} }
\end{equation}

Las hembras se alimentan principalmente de sangre, que extraen de cualquier
vertebrado, por sus hábitos domésticos muestran marcada predilección por
la del hombre\cite{ThironIzcazaJ2003}. En cuanto a los machos sus partes
bucales no son aptas para chupar sangre, por lo que se alimentan de
carbohidratos de cualquier fuente accesible como frutos o néctar de flores
que satisface sus requerimientos energéticos.

\subsubsection{Reproducción y postura de huevos}
