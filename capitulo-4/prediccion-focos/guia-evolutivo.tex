\section{Proceso Evolutivo}

Se debe explicar lo básico y dar una idea de que se busca y como funciona.


\textbf{Definición 1.} \em Instante \em : Se utiliza el término instante
    para representar a una estrucutra de datos compleja compuesta por
    datos climáticos correspondientes a periodos de tiempos individuales
    $t_{i}$ y otras propiedades complementarias.


\textbf{Definición 2.}\em Peridodo \em : Un periodo es una estrucutra de
    datos que representa a un conjunto de instantes $t_{i}$. El periodo
    $T$ \addsymbol{symbol:P} se define como una colección de $N$ instantes
    $t_{i}$, donde $N$ es el tamaño del periodo de estudio.

    \begin{align*}
        T = t_1,t_2,t_t,\ldots,t_N , & & 1 \leq i \leq N
    \end{align*}


\textbf{Definición 3.} \em Estado \em : Se utliza el término estado
    para representar las etapas del ciclo de desarrollo del \em Aedes Aegypti\em.
    \begin{align*}
        \tau = [huevo, larva, pupa, adulto]
    \end{align*}

\textbf{Definición 4.} \em Sexo \em : Representa el sexo del \em Aedes
    Aegypti\em. Los valores posibles son $MACHO$ y $HEMBRA$.

\textbf{Definición 4.} \em Indiviuduo\em : Se utliza el término individuo
    para representar la unidad del \em Aedes Aegypti \em cualesquiera de sus
    estados, en forma de una estrucutra de datos compleja compuesta de
    propiedades como su madurez y espectativa de vida.

\textbf{Definición 4.} \em Población\em : La población $P$ es una estuctura
    de datos que representa a un grupo de individuos $p_{i}$. Se define como
    una colección de $M$ individuos $p_{i}$, en donde $M$ es el tamaño de
    la población. Contiene a todos los individuos si importar el sexo  y el
    estado en que $\tau_{k}$ en el que se encuentren.

    \begin{align*}
        P = p_1,p_2,p_t,\ldots,p_M,  & & 1 \leq i \leq M
    \end{align*}

La población inical se genera a partir de los valores obtenidos mediante
las mediciones o muestreos realizados con dispositivos de ovipostura.

Cada $p_{i}$ que pertenzca a $P$ es sometido a las variaciones climáticas de
cada $t_{i}$ que pertenezca a $T$ siempre y cuando $p_{i}$ aún forme parte
de la población $P$.

\section{Operadores básicos}
Durante el proceso evolutivo para cada individuo $p_{j}$ exiten un conjuto
de operadores básicos que son aplicables a todos los individuos sin importar
el sexo y el estado $\tau_{k}$ en el que se encuentre.

hablar de las tablas y la zonificación


\subsection{Madurez}
Para cada $p_{i}$ existe asociado un atributo madurez \addsymbol{symbol:mIi} es
un valor numérico(entre 0 y 100) que varía de acuerdo a las condiciones
climáticas a las que es sometido el mosquito. Cuando la madurez es igual
a 100 el mosquito ya se encuentra listo para un cambio de estado.

\begin{equation}
\eta (t_{i}, p_{j}) = \left\{
  \begin{array}{l l}
    0 & \quad \forall i = 0 \\
    \eta (t_{i-1}, p_{j}) + \frac{1}{\omega(t_{i}) * 24} & \quad \forall i \neq 0
  \end{array} \right.
\end{equation}

\subsection{Cambio de estado}

\begin{equation}
\tau_{k} (t_{i}, p_{j}) = \left\{
  \begin{array}{l l}
    \tau_{k+1} & \quad \forall  \eta(t_{i}, p_{j}) = 100 \\
    \tau_{k} & \quad \text{en caso contrario}
  \end{array} \right.
\end{equation}


\subsection{Mortalidad}
La mortalidad y supervivencia de los individuos se encuentra expresada
mediante la variable de expectativa de vida. La expectativa de vida es
un valor numérico que indica la vitalidad del individuos, esta varía de
acuerdo a las condiciones climáticas a las que es sometido el individuos
durante el proceso evolutivo.

\begin{equation}
\xi (t_{i}, p_{j}) = \left\{
  \begin{array}{l l}
    100 & \quad \forall i = 0 \\
    \xi (t_{i-1}, p_{j}) - \frac{1}{\upsilon(t_{i}) * 24} & \quad \forall i \neq 0
  \end{array} \right.
\end{equation}

El valor de $\xi (t_{i}, p_{j})$ representa el porcentaje de vitalidad del
inidividuo $p_{j}$ luego de ser somenetido a las variaciones del instante
$t_{i}$ perteneciente al periodo $T$ de estudio. La espectativa de vida
para cada individuo $p_{j} \in P$, en el isntante $t_{0}$ se describe como
$\xi (t_{0}, p_{j})= 100$. A medida que $p_{j}$ sea sometido a varios
$t_{i}$ la espectativa de vida irá disminuyendo, si $\xi (t_{i}, p_{j})= 0$ el
idividuo se ha quedado sin espectativa de vida, por lo que se debe proceder
a eliminar $p_{j}$ de $P$ mediante el proceso de reducción de la población
$\theta (t_{i}, p_{j})$.

\begin{equation}
\theta (t_{i}, p_{j}) = \left\{
  \begin{array}{l l}
    p_{j} = null & \quad \xi(t_{i}, p_{j}) == 0 \\
    p_{j} & \quad \text{en caso contrario}
  \end{array} \right.
\end{equation}

\section{Operadores complementarios}
Las etapas inmaduras del \em Aedes Aegigyti\em son principalmente acuaticas
y estaticas, por lo que todas cuentan con caracteristicas similares, no
así la etapa adulto del mosquito que cuenta con ciertas caracteristicas
que divergen mucho del comportamiento básico definido. Debido este
comportamiento especifico todos los $p_{j}$ que se encuentren en el estado
\em Adulto \em cuentan con un conjuto de operadores complementarios que
tienen por objetivo, con ayuda de los operaodres básicos, describir el
comportamiento del mosquito en su estado adulto.

\subsection{Vuelo y búsqueda de alimentos}
Exiten diferencias, en cuanto alimentación y vuelo, entre los adultos
machos y hembras. Los rondan en grupos pequeños o solitariamente,
principalmente atraídos por los mismos huéspedes vertebrados que las hembras.

Los mosquito tienen la particularidad de que vuelan en sentido contrario
al la dirección al viento y a una velocidad máxima $VMAX$ de $2 km/h$ según lo
mencionado en \cite{web-site:speedAnimals}.

La distancia $D(t_{i})$ recorrida por el individuo en un instante $t_{i}$
depende de la dirección del viento representada por $\alpha(t_{i}$, la
velocidad del viento $S(t_{i})$

\begin{equation}
 D (t_{i}) = \sqrt{{(\sin(\alpha(t_{i})) * VMAX  - S(t_{i}))}^{2}
  + {(\cos(\alpha(t_{i})) * VMAX} ^{2} }
\end{equation}

Las hembras se alimentan principalmente de sangre, que extraen de cualquier
vertebrado, por sus hábitos domésticos muestran marcada predilección por
la del hombre\cite{ThironIzcazaJ2003}. En cuanto a los machos sus partes
bucales no son aptas para chupar sangre, por lo que se alimentan de
carbohidratos de cualquier fuente accesible como frutos o néctar de flores
que satisface sus requerimientos energéticos.

\subsection{Reproducción y postura de huevos}
