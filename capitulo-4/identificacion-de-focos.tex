%!TEX root = ../tesis.tex
\section{Identificación de focos de infestación}

El proceso de identificación de focos de infestación, consiste en utilizar un conjunto de puntos con valores
conocidos, denominados puntos de control, obtenidos mediante los dispositivos  de ovipostura para generar
información geoespacial que permite determinar las zonas con mayor densidad poblacional de \textit{Aedes Aegipty}.El
objetivo principal es analizar la situación actual y la propagación del vector de la enfermedad y para aplicar
medidas correctivas en las zonas más afectadas y preventivas en las zonas aledañas.

En \cite{NINO2011} se define la metodología de vigilancia y localización de los focos de infestación del 
\textit{Aedes Aegipty} mediante la utilización de larvitrampas y técnicas de interpolación espacial, las 
cuales permiten estimar la abundancia vectorial de forma continua en el espacio a partir del conteo de individuos
colectados en el área de estudio. 

Sean las siguientes variables básicas:
%lista de definiciones de las variables básicas
\begin{description}[style=multiline,leftmargin=1.5cm]

    \item[($x_{i},y_{i}$)] \addsymbol{symbol:xy_i} Par de valores que representan un punto y que sirven para
        georeferenciar a un dispositivo de control.  

    \item[$d_{i}$] \addsymbol{symbol:d_i} Define a una muestra (dispositivo de control), que tiene asociado un
    valor $z_{i}$ y una ubicación representado por ($x_{i},y_{i}$).

    \item[$z_{i}$] \addsymbol{symbol:z_i} Valor numérico que representa la cantidad de individuos contabilizados
    en el dispositivo de ovipostura $d_{i}$.

    \item[$D$] \addsymbol{symbol:D} Conjunto de muestras $d_{i}$ sobre las cuales se realiza el estudio.
        \begin{align*}
            D = d_1,d_2,d_3,\ldots,d_N
        \end{align*}
        
\end{description}

Podemos estimar la abundancia vectorial de $D$ mediante el método de interpolación $I(D, d'_j)$ 
\addsymbol{symbol:Id_i}, que se encarga de predecir el valor de $z_j$ para cada $d'_j$ desconocido.
