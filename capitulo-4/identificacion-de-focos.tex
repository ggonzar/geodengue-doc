%!TEX root = ../tesis.tex
\subsection{Identificación de focos de infestación}
\label{sec:cap4-identificacion-focos}

En \citet{NINO2011} se ha diseñado una metodología de vigilancia entomológica que localiza focos
de infestación de A. aegypti, mediante el uso de larvitrampas, para generar información
regionalizada sobre la abundancia larval y técnicas de interpolación espacial que permiten
visualizar de forma continua el nivel de infestación vectorial.

Las larvitrampas, o puntos de control, distribuidas geográficamente permiten, mediante técnicas de
interpolación espacial, obtener mapas de interpolación donde se puede apreciar los niveles de
infestación del vector del dengue y el riesgo correspondiente a la abundacia de mosqutios
observada en el área de estudio. El hecho de contar con esta información regionalizada permitirá
a las autoridades pertinentes definir y planificar mejor las medidas de prevención y control a
realizarce para reducir los niveles de infestación en las zonas criticas. Según \citet{NINO2011},
los mapas de superficie generados con esta metodología indican, con mayor detalle que los índices
aédicos tradicionales, los lugares específicos donde sería necesario tomar medidas de prevención y
control de acuerdo al grado de infestación, permitiendo así una mayor racionalización de tiempo y
recursos.
