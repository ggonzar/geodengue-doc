
\subsection{Operadores complementarios}
Las etapas inmaduras del \em Aedes Aegigyti\em son principalmente acuáticas y estaticas, por lo que todas cuentan con
características similares, no así la etapa adulto del mosquito que cuenta con ciertas características que divergen
mucho del comportamiento básico definido. Debido este comportamiento especifico todos los $p_{j}$ que se encuentren en
el estado \em Adulto \em cuentan con un conjunto de operadores complementarios que tienen por objetivo, con ayuda de
los operadores básicos, describir el comportamiento del mosquito en su estado adulto.

\subsubsection{Vuelo y búsqueda de alimentos}
Existen diferencias, en cuanto alimentación y vuelo, entre los adultos machos y hembras. Los rondan en grupos pequeños
o solitariamente, principalmente atraídos por los mismos huéspedes vertebrados que las hembras.

Los mosquito tienen la particularidad de que vuelan en sentido contrario al la dirección al viento y a una velocidad máxima $VMAX$ de $2 km/h$ según lo mencionado en \cite{web-site:speedAnimals}.

La distancia $D(t_{i})$ recorrida por el individuo en un instante $t_{i}$  depende de la dirección del viento
representada por $\alpha(t_{i}$, la velocidad del viento $S(t_{i})$

\begin{equation}
 D (t_{i}) = \sqrt{{(\sin(\alpha(t_{i})) * VMAX  - S(t_{i}))}^{2}
  + {(\cos(\alpha(t_{i})) * VMAX} ^{2} }
\end{equation}

\subsubsection{Oviposición}
\label{subsec:cap4-oviposicion}
El ciclo gonotrófico se encuentra descripto, en \cite{otero2006stochastic}, por el modelo no lineal de Sharpe
\& DeMichele, presentado anteriormente en la sección \ref{subsec:operadores-basicos}. En la tabla
\ref{tab:ciclo-gonotrofico} se describen los coeficientes tomados de \cite{otero2006stochastic} aplicables al
ciclo gonotrófico de las hembras adultas, para la ecuación \eqref{eq:sharpe-demichele}.

\begin{table}
\begin{minipage}{\paperwidth}
\begin{tabular}{p{6cm} c c c c }
Estado                     & $RH025$    & $HA$   & $TH$   & $HH$   \\
\hline
Ciclo gonotrófico\footnote{Hembras adultas nuliperas}    & 0.216 & 15725 & 447.2 & 1756481 \\
Ciclo gonotrófico\footnote{Hembras adultas que pusieron huevos}      & 0.372 & 15725 & 447.2 & 1756481 \\
\end{tabular}
\end{minipage}
\caption{ \label{tab:ciclo-gonotrofico} Coeficientes para el modelo de maduración enzimática
\eqref{eq:sharpe-demichele}, (Tomado de \citet{otero2006stochastic})}
\end{table}
