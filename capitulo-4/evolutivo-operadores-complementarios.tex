
\subsection{Operadores complementarios}
Las etapas inmaduras del \em Aedes Aegigyti\em son principalmente acuáticas
y estaticas, por lo que todas cuentan con características similares, no
así la etapa adulto del mosquito que cuenta con ciertas características
que divergen mucho del comportamiento básico definido. Debido este
comportamiento especifico todos los $p_{j}$ que se encuentren en el estado
\em Adulto \em cuentan con un conjunto de operadores complementarios que
tienen por objetivo, con ayuda de los operadores básicos, describir el
comportamiento del mosquito en su estado adulto.

\subsubsection{Vuelo y búsqueda de alimentos}
Existen diferencias, en cuanto alimentación y vuelo, entre los adultos
machos y hembras. Los rondan en grupos pequeños o solitariamente,
principalmente atraídos por los mismos huéspedes vertebrados que las hembras.

Los mosquito tienen la particularidad de que vuelan en sentido contrario
al la dirección al viento y a una velocidad máxima $VMAX$ de $2 km/h$ según lo
mencionado en \cite{web-site:speedAnimals}.

La distancia $D(t_{i})$ recorrida por el individuo en un instante $t_{i}$
depende de la dirección del viento representada por $\alpha(t_{i}$, la
velocidad del viento $S(t_{i})$

\begin{equation}
 D (t_{i}) = \sqrt{{(\sin(\alpha(t_{i})) * VMAX  - S(t_{i}))}^{2}
  + {(\cos(\alpha(t_{i})) * VMAX} ^{2} }
\end{equation}

Las hembras se alimentan principalmente de sangre, que extraen de cualquier
vertebrado, por sus hábitos domésticos muestran marcada predilección por
la del hombre\cite{ThironIzcazaJ2003}. En cuanto a los machos sus partes
bucales no son aptas para chupar sangre, por lo que se alimentan de
carbohidratos de cualquier fuente accesible como frutos o néctar de flores
que satisface sus requerimientos energéticos.

\subsubsection{Reproducción y postura de huevos}
