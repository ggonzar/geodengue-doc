\section{Conteo de larvas mediante procesamiento digital de imágenes}
Llamamos procesamiento digital de imágenes (PDI) al conjunto de técnicas de para alterar o extraer
información de imágenes digitales. Existe una inmensa cantidad de contenido de imágenes digitales 
(cámaras de seguridad, fotografías personales, contenido web de blogs, noticias y eventos, etc.). 
De dicha cantidad de datos en formato de imágenes digitales existe cada vez más mayor necesidad
de procesarlos para generar el contenido deseado. 
Una aplicación práctica sería aplicar un filtro visual a una imagen. Existen distintos tipos de 
filtros gráficos como vintage, por puntos, tipo antigua, poster y otros que mediante un algoritmo 
PDI puede ser aplicados a una imagen. Como ejemplo; tomar una fotografía actual y aplicar un filtro de foto antigua. Esta aplicación práctica representa el procesamiento para la alteración de imágenes.

La aplicación práctica menos más compleja es el procesamiento de imágenes para la extracción de 
información. Es más sencillo rotar una imagen que realizar un análisis de cuantas personas se 
encuentran en dicha imagen. Análogo al proceso de alterar imágenes existen varias técnicas para la
extracción de información a partir de objetos visuales digitales. Una aplicación práctica sería
identificar el número de chapa de un vehículo al cual se le tomó una fotografía al infringir una 
ley de tránsito.

Para este trabajo utilizamos PDI para resolver y automatizar un proceso fundamental para nuestro
sistema; Realizar el conteo de larvas en un recipiente de ovipostura. El conteo de larvas se realiza 
de la siguiente manera:

\begin{enumerate}
    \item Se toma una (o más) fotografía(s) al dispositivo de ovipostura del cual se quiere conocer el 
    número de larvas.
    \item Se guarda la fotografía en el servidor para ser analizado.
    \item Un algoritmo que utiliza técnicas de PDI analiza la imagen y obtiene la cantidad de larvas
    contenidas en la misma.
\end{enumerate}

En el modelo propuesto utilizamos el algoritmo de Otsu explicado en la siguiente sección para la
segmentación mediante umbrales y un algoritmo de no aproximación para la identificación de contornos
para las imágenes.

\subsection{Método Otsu}
Dentro de el área de PDI (procesamiento digital de imágenes) existen diferentes métodos de análisis
de imágenes con distintas características y para distintos fines. Ya sea que necesitemos realizar
reconocimiento de rostros, formas o figuras, o realizar cualquier tipo de procesamiento con las 
imágenes existe un método o un conjunto de procesos que cumplen la función que necesitamos.

Uno de los métodos dentro del campo de PDI es el método de Otsu. Este es un método de segmentación por
umbralización. La umbralización es el proceso de establecer umbrales (valores máximos y mínimos) que se
aplican al análisis de una imagen. Cada píxel es discriminado según los valores umbrales para determinar
a qué grupo pertenece. Un ejemplo práctico conceptual sería pasar una imagen en colores a una misma
imagen en blanco y negro. Para ello necesitamos definir un umbral, un valor que nos determine si 1 píxel
analizado pertenece al grupo de los ”blancos” o de los ”negros” dependiedo de su color actual. Sí el
color del píxel es un rojo intenso entonces lo incluimos en el conjunto de los ”negros” y si el color es
más bien claro, como blanco o gris claro entonces lo agregamos al grupo de los ”blancos”.

Este método lo utilizamos en nuestro trabajo para, a partir de una fotografía de las larvas, hacer un
traspaso a una imagen a escala de grises, la cual puede ser analizada por un algoritmo de conteo y así
realizar el conteo de las larvas.

En primer lugar definimos 2 conjuntos. El conjunto de objetos dentro de la imagen(e: las larvas) y el
fondo(ej: el recipiente con el agua que contiene las larvas). Luego definimos el umbral ”T” que
determina si un píxel pertenece a un grupo u otro según su nivel de gris ”g”. La función ”f” retorna 1 
o 0 por cada píxel (par de coordenadas ”x” e ”y”).

\begin{equation}
f(x, y) =\left\{
  \begin{array}{l l}
    1 & \quad g > T\\
    0 & \quad g \leq T
  \end{array} \right.
\end{equation}

Como se observa en la ecuación se analiza simplemente el nivel de gris en cada píxel y se determina si
forma parte de un objeto de estudio o si forma parte del fondo.
