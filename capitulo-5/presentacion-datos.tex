\subsection{Presentación de los datos}
La herramienta está diseñada para realizar difrentes tipos de analisis estadisticos y geográficos
para facilitar la información necesaria, para ello se consulta los eventos registrados en la base
de datos.

El software de tratamiento de datos geográficos obtiene la información de una fuente
de datos y los combina o transforma para que puedan ser analizados visualmente. El
resultado puede ser un mapa, una imagen o simplemente información de una región o
punto geográfico determinado.

Análisis. El potencial analítico que provee la información generada es amplio y diverso. La
primera alternativa de análisis y la más intuitiva es la identificación de los focos de la
enfermedad. Además de ser el objetivo principal de la solución propuesta es una de las
alternativas de mayor importancia ya que la identificación de focos de la enfermedad permite :
    \begin{itemize}
        \item Visualizar el estado actual y la distribución poblacional del mosquito \textit{Aedes Aegipty}.
        \item Realizar planes preventivos sobre zonas o regiones más afectadas.
        \item Implementar estrategias de fumigación y limpieza según zonas más afectadas.
        \item Otros.
    \end{itemize}

Calculo de indices adeicos
Análisis de tasas de desarrollo
Análisis de tasas de mortalidad
Análisis de duración del ciclo gonotrófico
Análisis de dispersión
Interpolación espacial para identificación de focos en eventos por simulación.
Historial de muestreo
