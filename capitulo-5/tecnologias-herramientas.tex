\section{Tecnologías y herramientas utilizadas}

Una vez analizadas las funcionalidades del sistema, se tomo la decisión de implementar la herramienta
como aplicación web por su orientación a múltiples usuarios y a la localización geográfica de contenidos,
es decir, que su objetivo está orientada a un uso online cuyo contenido depende de las aportaciones de
diferentes usuarios. Así también se evita un problema de las aplicaciones standalone, que obligan a que
se tenga que instalar el software en los equipos cliente, con la consiguiente dificultad de acceso a la
herramienta. En el caso de las aplicaciones web tan solo se necesita un navegador web para su acceso,
aunque en este caso ha de ser lo suficientemente reciente para soportar las tecnologías que se emplean.

Las tecnologías utilizadas en el desarrollo de la aplicación fueron <TODO>

para el procesamiento digital de imágenes se utilizó el paquete <TODO> de python

Como servidor de aplicaciones web se utilizó apache, con respecto al almacenamiento de datos, 
se decidió usar el sistema gestor de bases de datos PostgreSQL , por ser de uso libre y por 
cuestiones referentes a los datos geográficos ya que dispone de la extensión PostGIS para su manejo.


En lo referente a las tecnologías GIS empleadas, puesto que la aplicación ofrecerá la posibilidad 
de visualizar, consultar y modificar información geográfica, los estándares de OGC con los que trabaja, directa o indirectamente, son GML, WFS, WMS, SLD,
Filter Encoding y documentos de contexto de mapas (Web Map Context Documents, WMC).

El servidor de información geográfica Geoserver. 


El Cliente de información geográfica Mapbuilder. Genera mapas y realiza peticiones, 
interactuando con el servidor GIS a través de los estándares OGC. Es una Librería que 
implementa un entorno para crear el contenido de una página web mediante el procesamiento 
de documentos XML usando la tecnología AJAX.


\subsection{GeoServer}
El proyecto GeoServer en una aplicación Java que integra la versión 1.0 de WFS,
1.1.1 de WMS y 1.0 de WCS para poder publicar información bien directamente permi-
tiendo su manipulación, bien en forma de imágenes o mapas. Geoserver tiene en cuenta
cuatro metas principales en el ámbito de desarrollo, ordenadas por importancia:
Cumplimiento de los estándares: El proyecto GeoServer intenta promover la estandarización, 
y soportar tantos estándares como sea posible, para permitir a todos compartir su información 
geoespacial rápidamente y de una forma interoperable, disminuyendo así las barreras entre 
proveedores de información geográfica.
Soporte para diferentes formatos información: Para hacer un producto útil, GeoServer intenta 
traducir todos los formatos de datos de información geográfica en uno solo. Sin embargo, 
el soporte para varios formatos de datos es una de las prioridades. Actualmente soporta 
eficientemente almacenamiento en los formatos shapefile, PostGIS, DB2, Oracle, ArcSDE y, 
además ofrece servicio para soportes en prueba como MySQL, Vector Product Format Library 
(VPF), Web Feature Server (WFS) y MapInfo. Y en cuanto a los formatos de salida que Geoserver
puede generar como respuesta a peticiones WFS-T y WMS se encuentran, entre
otros:
• JPEG: Joint Photographic Experts Group, algoritmo de compresión de imágenes con pérdida 
de información.
• PNG: Portable Network Graphics, algoritmo de compresión de imágenes sin
pérdida de información.
• SVG: Scaleable Vector Graphics, lenguaje para describir gráficos vectoriales
bidimensionales.
• GML: Geography Markup Language.
• PDF: Portable Document Format, formato de almacenamiento de documentos, desarrollado 
por la empresa Adobe Systems.
• shapefiles: formato propietario abierto de datos espaciales desarrollado por la compañía ESRI,
originalmente creado para su producto ArcView GIS, pero actualmente se ha convertido en formato estándar
de facto por la importancia que los productos ESRI tienen en el mercado SIG. Un Shapefile es un formato
vectorial de almacenamiento digital donde se guarda la localización de los elementos geográficos y los
atributos asociados a ellos. El formato carece de capacidad para almacenar información topológica.

• KML/KMZ: Keyhole Markup Language, lenguaje de marcado basado en XML para representar datos geográficos
en tres dimensiones, desarrollado para ser manejado con Google Earth. 

Fácil de usar: Fácil de instalar, configurar y ejecutar para organizaciones con pocos recursos técnicos. Orientado para organizaciones con experiencia técnica mínima.

Eficiencia: El procesado de información geográfica normalmente requiere muchas cargas computacionales y
de ancho de banda, GeoServer intenta minimizar ambas. Desde el punto de vista del usuario, GeoServer es
una herramienta necesaria para mostrar mapas en las paginas web, donde el usuario puede hacer zoom,
cambiar la vista y hacer operaciones soportadas por los especificaciones WMS y WFS de OGC.

Es usado en conjunto con clientes como MapBuilder, con el objetivo de conectar facil y54
2. Tecnologías y software GIS eficientement bases de datos geográficas con clientes GIS. 
El uso de estándares permite combinar la información del GeoServer fácilmente con otra información 
geográfica
