\section{Tecnologías y herramientas utilizadas}
Como se mencionó anteriormente, se tomo la desicion de implementar la herramienta como una
aplicación web. En esta sección presentaremos las herramientas y tecnologias empleadas para el
desarrollo.

La capa de presentación fue desarrollada puramente en javascript, html y css donde la comunciación
con la capa de servicios se ecuentra dada por peticiones Ajax. Para la visualización e interacción
como mapas se utilizó OpenLayers. Como servidor web se utilizó
Apache\footnote{https://httpd.apache.org/} en su versión 2, aunque se podría utilizarce cualquier
otro servidor web HTTP.

La capa de negocios, se encuentra desarrollada en Python, en donde para la capa de servicios se
utilizó Flask\footnote{http://flask.pocoo.org/} un framework minimalista para Python. Para llevar
a cabo análisis y procesamientos complejos se utilizó la extensión de Python
NumPy\footnote{http://www.numpy.org/} que agrega mayor soporte para vectores y matrices,
constituyendo una biblioteca de funciones matemáticas de alto nivel para operar con esos vectores
o matrices. Para el procesamiento digital de imágenes se utilizó el paquete OpenVc de python. Como
servidor web se utilizó Apache con su módulo
mod\_wsgi\footnote{http://www.modwsgi.org/} que proporciona una interfaz compatible con
WSGI \footnote{WSGI es una especificación para una interfaz simple y universal entre los
servidores web y aplicaciones web o frameworks para el lenguaje programación Python.}

Como servidor WMS se utiliza GeoServer, que intenta promover la estandarización, y soportar tantos
estándares como sea posible, para permitir a todos compartir su información geoespacial
rápidamente y de una forma interoperable, disminuyendo así las barreras entre proveedores de
información geográfica.

Con respecto al almacenamiento de datos, se decidió utilizar el sistema gestor de bases de datos
PostgreSQL\footnote{http://www.postgresql.org/}, y para la manipulación de datos geográficos se utilizó PostGIS\footnote{http://postgis.net/} que es módulo que añade soporte de objetos
geográficos a PostgreSQL.

