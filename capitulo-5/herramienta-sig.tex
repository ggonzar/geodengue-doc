\section{Descripción funcional}
GeoDengue es un sistema de información geográfico-sanitaria basado en una arquitectura cliente
servidor. Esta se presenta efectiva y concretamente todos los requerimientos que describen como el
software debe comportarse para cumplir con sus finalidades y objetivos. La aplicación se encuentra
dividida en sub módulos para, la gestión de datos de entrada, el simulador del proceso evolutivo y
presentación de los datos.


\subsection{Gestión de datos de entrada}
%Muestras
El sistema requiere principalmente de 2 datos de entrada, los putnos de control y los datos
climatólogicos correspondientes a el periodo de tiempo para la simulación

El área de estudio se encuentra determinada por una muestra, que representan a un conjunto de
puntos de control georeferenciados en una zona determinada. Los puntos de control que pertenecen a
una muestra deben ser establecidos en un mismo instante, siendo así la fecha de instalación y
fecha de recolección, de los puntos de control de la muestra, iguales.


Los puntos de control, como resultado de la medición realizada, permiten obtener la densidad de
larvas asociada con la información geográfica.
%Gestión de puntos de control
\subsection{Simulador del proceso evolutivo}

La simulación del prorceso evolutivo del mosquito del aedes aegypti requiere principalmente los
puntos de control pertenecientes el área de estudio y los datos climatológiocs del correspondientes
a dicha área. El proceso evolutivo consta de varios pasos, el desarrollo, la mortalidad,
la ovipostura y dispersión de los individuos de la población.

El proceso inicia al procesar los puntos de control pertenecientes a la muestra utilizada para el
estudio. La población inicial se encuentra compuestas por las larvas observadas en la muestra
utilizada para el estudio. Una vez generada la población inicial se obtienen los datos
climatológicos correspondientes, desde la fecha de recolección de los puntos de contro, hasta un
periodo máximo de 30 días. Se establece un periodo máximo debido los serivicios de predicción del
clima utilizados establecen dicho tope.

\begin{algorithm}
\caption{Simulación del proceso evolutivo}
\label{alg:simulador-evolutivo}
\begin{algorithmic}[1]
    \REQUIRE $\vec{k}\neq \emptyset \land \vec{m} \neq \emptyset$
    \ENSURE $\vec{m'}$
    \FORALL{$k{i} \in \vec{k}$ }
        \STATE $\vec{huevos} \Leftarrow \emptyset$
        \FORALL{$m{j} \in \vec{m}$}
            \STATE $desarrollar(m_{j}, k_{i})$
            \IF{$regular(m_{j}, k_{i})$}
                \STATE \COMMENT{Se elimina $m_{j}$ si es un candidato.}
                \STATE $m_{j} \Leftarrow \varnothing $
            \ELSIF{$esta\_maduro(m_{j}, k_{i})$}
                \STATE $ cambiar\_estado(m_{j}) $
            \ELSIF{$se\_reproduce(m_{j}, k_{i})$}
                \STATE $\vec{huevos} \Leftarrow \vec{huevos} + oviponer(m_{j})$
            \ENDIF
        \ENDFOR

        \IF{$\vec{huevos} \neq \emptyset$}
            \STATE \COMMENT{Si ovipone se extiende la población}
            \STATE $\vec{m} \Leftarrow  \vec{m} + \vec{huevos}$
        \ENDIF
    \ENDFOR
    \RETURN $\vec{m}$
\end{algorithmic}
\end{algorithm}

El \algref{alg:simulador-evolutivo}, describe al simulador como un proceso iterativo cuyo objetivo
es simular los efectos de cada $k_{i}$ para cada individuo $m_{j}$ que pertenezca a la población.

El proceso $desarrollar(m_{j}, k_{i})$, es el proceso por el cual se calculan las tasas de
desarrollo correspondientes para cada $m_{j}$ . Estas son obtenidas medidante el modelo de
\citet{sharpe1977reaction} presentado en la \secref{subsec:cap4-tasas de desarrollo}.

El desarrollo en las etapas inmaduras (Huevo, Larva y Pupa) es realizado con el fin de simular los
efectos de $k_{i}$ en la maduración de $m_{j}$. Sin embargo para el adulto, es realizado con el fin
de simular los efectos de $k_{i}$ en la duración del ciclo gonotrófico de $m_{j}$ y la dispersión
del vector presentada en la \secref{subsec:cap4-vuelo-dispersion}.


\begin{algorithm}
\caption{$desarrollar(m_{j}, k_{i})$}
\label{alg:desarrollo}
\begin{algorithmic}[1]
    \REQUIRE $ k{i} \neq \varnothing \land m{j} \neq \varnothing$
    \ENSURE $d_{j}$
            \STATE $d_{j} \Leftarrow d_{j -1} + R(k_{i})$
            \IF{$m_{j}\ es\ adulto$}
            \STATE $volar(m_{j}, k_{i})$
            \ENDIF
    \RETURN $d_{j}$
\end{algorithmic}
\end{algorithm}

El proceso $regular(m_{j}, k_{i})$ como principal objetivo, determinar la cantidad de individuos
que deben ser eliminados de la población, debido a la mortalidad diaria y seleccionar los
candidatos para dicha eliminación. Como se mencionó anteriormente, en la
\secref{subsec:cap4-mortalidad}, la tasa de mortalidad diaria depende del estado en el que s
encuentre el inidividuo. La selección de canidatos para la eliminación se realiza de forma
aleatoria para los individuos que provenientes de una misma colonia $(x, y)$.


El proceso $esta\_maduro(m_{j}, k_{i})$ se encarga de validar que individuo haya completado su
desarrollo, una vez completo se procede a realizar el cambio de estado mediante el proceso
$cambiar\_estado(m_{j})$. El proceso de maduración y cambio de estado fue presentado previamente
en la \secref{subsec:cap4-madurez-cambio-estado}.

El proceso $se\_reproduce(m_{j}, k_{i})$, que solo es válido para hembras adultas se encarga de
validar $m_{j}$ haya completado su ciclo gonotrófico, una vez finalizado debe generar una tanda de
huevos mediante $oviponer(m_{j}$. La duración del ciclo gonotrófico y la oviposutra fue presentado
anteriormente en la \secref{subsec:cap4-ciclo-gontrofico-oviposturan}.
