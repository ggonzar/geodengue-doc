
\section{Descripción funcional}
En esta sección presentamos los requerimientos del sistema, geográfico-sanitario, y el
comportamiento que debe adoptar con el fin de cumplir sus finalidades y objetivos. Se encuentra
dividido en procesos básicos que tienen como finalidad, la gestión de datos de entrada, la
simulación del proceso evolutivo y el análisis de los datos.

La gestión de datos de entrada es la encargada de procesar la información obtenida a partir de los
puntos de control, mediante el conteo de larvas, y registrarlos en la base de datos del sistema,
donde los datos se encuentran agrupados por áreas de muestreo.

El simulador del proceso evolutivo es el encargado de, a partir de la información obtenida de los
puntos de control, simular los efectos de las variaciones climáticas en el desarrollo, mortalidad
y dispersión del vector, donde, los efectos son registrados en la base de datos del sistema.

El proceso de análisis de datos es el encargado de procesar la información, generada por los
puntos de control y el simulador del proceso evolutivo, y presentarlos en de forma clara para que
puedan ser interpretados por el usuario final de forma sencilla.

\subsection{Muestreo y puntos de control}
La utlización de larvitrampas como puntos de control, permite estimar con mayor precisión la
abundancia del Aedes aegypti al proporcionar información fidedigna sobre el número de ejemplares
que habitan en distintos puntos del área de estudio en comparación con la presencia o ausencia de
larvas en recipientes o viviendas usados en los índices aédicos tradicionales
\citet{NINO2011}, que fueron presentados en la \secref{sec:densidad-vectorial-indices-stegomia}.

Con el fin de acotar el área de estudio, los puntos de control deben agruparse en regiones y estar
asociados al periodo en el que se realizó el muestreo, estos conjutos son denominados muestras. Se
considera que todos los puntos de control pertenecientes a una muestra cuentan con la misma fecha
de instalación y recolección. De forma a que una muestra es la resultante de una medición realizada
en una region en un periodo determinado.

La definición de las muestras, establecimiento y recolección los puntos de control pertenecientes
a las mismas, corresponden a uno de los ejes principales y son un prerequisito para ejecutar o
iniciar algún proceso.

\subsection{Conteo de larvas}
Con el muestreo se busca estimar la abundancia poblacional del Aedes Aegypti en una región, donde,
como parte de proceso de recolección de los datos se debe realizar un conteo para determinar la
cantidad de lavas asociadas a cada punto de control que pertenezca a la muestra. De forma que el
conteo de larva se torna un trabajo tedioso.

Con el fin de agilizar y facilitar este proceso, se realiza el conteo de larvas mediante
procesamiento digital de imágenes. Llamamos procesamiento digital de imágenes, o PDI (por sus
siglas), al conjunto de técnicas aplicadas para alterar o extraer información de imágenes digitales
\cite{moreira2009implementacion, ortiz2013procesamiento}. El procesamiento de imágenes ayuda a
analizar, deducir y tomar decisiones \cite{ortiz2013procesamiento}. Se han desarrollado
herramientas aplicables a las áreas de Medicina, Fisiología, Biometría, Astronomía, Ciencias
Ambientales, Robótica, Metalúrgica, Física, Electrónica y Biología \cite{ortiz2013procesamiento, santillan2008deteccion, moreira2009implementacion}.

Para que el conteo de larvas mediante PDI sea aplicable, se debe contar con una cámara para
digitalizar la imagen y posteriormente procesarla. Las imágenes obtenidas, por lo general no son
utilizadas directamente, estas son sometidas a un preprocesamiento con el fin de corregir las
variaciones en intensidad debidas al ruido, por deficiencias en la iluminación o la obtención de
imágenes de bajo contraste \citep{santillan2008deteccion}.

\begin{figure}
\begin{minipage}{\textwidth}
    \begin{tabular}{c c }
        \initbox
        \num\putindeepbox[7pt]{\includegraphics[width=0.4\textwidth]{capitulo-5/graphics/bandeja-muestra.jpg}} &
        \num\putindeepbox[7pt]{\includegraphics[width=0.4\textwidth]{capitulo-5/graphics/larvas-dengue.jpg}} \\
    \end{tabular}

    \caption{\label{fig:cap5-conteo-pdi-bandejas} Bandejas de plástico utilizadas como contenedor
    de larvas.}

    \footnotetext[1]{Bandeja de muestra vacía, sin larvas.}
    \footnotetext[2]{Bandeja con larvas correspondientes al un punto de control.}
\end{minipage}
\end{figure}

Existen diferentes métodos para el análisis de imágenes que nos permiten realizar el conteo de
larvas teniendo en cuenta las características de las larvas. Presentaremos el método y las técnicas
utilizadas para la realizar el conteo de forma trivial, pero dista mucho de ser trivial y un
análisis más profundo escapa al alcance de este proyecto.

El método seleccionado para realizar el conteo fue el de Otsu, ya que no requiere supervición
humana ni información previa de la imagen antes de su procesamiento \cite{santillan2008deteccion}.
Este método se emplea cuando hay una clara diferencia entre los objetos a extraer respecto del
fondo de la escena \citep{santillan2008deteccion}. Con la finalidad resaltar las características
de las larvas, estas deben depositarse en una bandeja de plástico de color blanco para
posteriormente realizar la captura (\figref{fig:cap5-conteo-pdi-bandejas}), con esto se consigue
un contraste entre el fondo las larvas observadas.

\begin{figure}
\begin{minipage}{\textwidth}
    \begin{tabular}{c c }
        \initbox
        \num\putindeepbox[7pt]{\includegraphics[width=0.4\textwidth]{capitulo-5/graphics/larvas-original.png}} &
        \num\putindeepbox[7pt]{\includegraphics[width=0.4\textwidth]{capitulo-5/graphics/larvas-otsu.png}} \\
    \end{tabular}

    \caption{\label{fig:cap5-larvas-otsu} Transformación de la una imagen mediante el método de
    Otsu.}

    \footnotetext[1]{Imagen original.}
    \footnotetext[2]{Imagen luego de la umbralización de Otsu.}
\end{minipage}
\end{figure}

La imagen se transforma a escala de grises, se ajusta el histograma para mejorar su contraste y se
calcula su umbral mediante el método de Otsu (\figref{fig:cap5-larvas-otsu}). Se definen 2
conjuntos, el primero corresponde a los objetos dentro de la imagen, en este caso las larvas, y el
segundo al el fondo, recipiente con el agua que contiene las larvas.  Se analiza simplemente el
nivel de gris correspondiente a cada píxel y se determina si forma parte de un objeto de estudio o
si forma parte del fondo.

\subsection{Simulación del proceso evolutivo}
La simulación del prorceso evolutivo del mosquito del aedes aegypti, presentado previamente en la
\secref{sec:cap4-simulador-proceso-evolutivo}, es el encargado de simular los efectos temperatura
en los siguientes eventos : eclosión de huevos, motalidad de larvas, emergencia de pupas, muerte
de pupas, emergencia de adultos, muerte de adultos, ovipostura y dispersión de los adultos
(\figref{fig:cap-5-proceso-evolutivo}).

\begin{figure}
\centering
\includegraphics[width=0.8\textwidth]{capitulo-5/graphics/proceso-evolutivo.png}
\caption{\label{fig:cap-5-proceso-evolutivo} Eventos del simulador del proceso evolutivo.}
\end{figure}

La población inicial es obtenida mediante la cantidad de larvas observadas en los puntos de
control que corresponden a la muestra utilizada para el estudio. Por cada larva observada,
en un punto de cotrol ubicado en las coordenadas geográficas, $(x, y)$, se inicializa un individuo
con las mismas coordenadas del punto de control de origen.

Las efectos de la tempratura, $k_{i}$, en cada individuo, $m_{j}$, son registrados mediante un
proceso que se encarga de almacenar, en una base de datos, la información correspondiente, de cada
$m_{j}$, para su posterior análisis.

