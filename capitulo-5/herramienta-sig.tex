\section{Descripción funcional}
GeoDengue es un sistema de información geográfico-sanitaria basado en una arquitectura cliente
servidor. Esta se presenta efectiva y concretamente todos los requerimientos que describen como el
software debe comportarse para cumplir con sus finalidades y objetivos. La aplicación se encuentra
dividida en sub módulos para, la gestión de datos de entrada, el simulador del proceso evolutivo y
presentación de los datos.


\subsection{Gestión de datos de entrada}
%Muestras
El sistema requiere principalmente de 2 datos de entrada, los putnos de control y los datos
climatólogicos correspondientes a el periodo de tiempo para la simulación

El área de estudio se encuentra determinada por una muestra, que representan a un conjunto de
puntos de control georeferenciados en una zona determinada. Los puntos de control que pertenecen a
una muestra deben ser establecidos en un mismo instante, siendo así la fecha de instalación y
fecha de recolección, de los puntos de control de la muestra, iguales.


Los puntos de control, como resultado de la medición realizada, permiten obtener la densidad de
larvas asociada con la información geográfica.
