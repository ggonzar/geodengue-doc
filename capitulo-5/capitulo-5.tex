%##Capítulo 5. GeoDengue
\chapter{GeoDengue}
%* Introducción
A finales de los años 90, la información geográfica era mayoritariamente tratada en
supercomputadores, usada casi siempre para mantener registros internos de administraciones y el
software que se utilizaba para su manejo era stand-alone \citep{vgomesAegis2001}. Sin embargo, con
la aparición de internet, la demanda de acceso a la información ha crecido considerablemente
obligando a los sistemas de información geográfica a modificar su paradigma para ofrecer
información de forma distribuida.

Las autoridades sanitarias, en sus tareas de vigilancia en Salud Publica, tienen en los GIS una
herramienta fundamental para conocer como se extiende una enfermedad, estudiar su posible relacion
con un potencial foco de riesgo, o localizar un brote epidemico\citep{vgomesAegis2001}. En esta
sección se presenta a GeoDengue una herramienta para identificación de focos de dengue en un
sistema de información georeferenciada.

%
El modelo solución para este problema se constituye de 4 partes fundamentales:

\begin{enumerate}[style=multiline,leftmargin=1.5cm]
    \item Establecimiento de puntos de control. Como primer paso es necesario fabricar los dispositivos de
     ovipostura (puntos de control, muestras) y distribuirlos en la zona o región que se desea analizazr. En el
     capítulo 5 se describe la estratégia implementada como mejor opción para la distribución de puntos de control.
     Además en el Anexo A se describe con detallada precisión la fabricación de los dispositivos de control;
     materiales, procedimiento, costos y recursos humanos necesarios
    
    \item Conteo. Una vez depositado el conjunto de muestras es necesario hacer un seguimiento para realizar el conteo de larvas que habitan en el dispositivo de control. Este conteo puede llevarse a cabo a los 7 días luego de que el dispositivo de ovipostura fue depositado. Es importante resaltar que un número mayor a 7 días podría significar que potenciales mosquitos hayan podido escapar del dispositivo de control por eso es necesario realizar el conteo antes de los 7 días o al séptimo día inclusive y luego vaciar el recipiente para volver a utilizarlo. 
    
    \item Procesamiento. El paso 3 y 4 constituyen básicamente el registro, procesamiento y análisis de los datos obtenidos de los conteos realizados. Ej: De un conjunto $D$ de 150 muestras, debemos tener 150 valores $z_{i}$ que representen los valores de la cantidad de larvas encontradas en cada dispositivo $d_{i}$. Al conjunto de valores $z_{i}$ se aplica un algoritmo de interpolación espacial que se encarga de generar una matriz de valores a partir de los datos de entrada. Este conjunto de valores contenidos en la matriz es utilizado para generar un mapa térmico que representa la propagación y distribución de las larvas en el contexto geoespacial.

    \item Análisis. El potencial analítico que provee la información generada es amplio y diverso. La primera alternativa de análisis y la más intuitiva es la identificación de los focos de la enfermedad. Además de ser el objetivo principal de la solución propuesta es una de las alternativas de mayor importancia ya que la identificación de focos de la enfermedad permite :
    \begin{itemize}
        \item Visualizar el estado actual y la distribución poblacional del mosquito \textit{Aedes Aegipty}.
        \item Realizar planes preventivos sobre zonas o regiones más afectadas.
        \item Implementar estrategias de fumigación y limpieza según zonas más afectadas.
        \item Otros.
    \end{itemize}
\end{enumerate}

\section{Descripción funcional}
GeoDengue es un sistema de información geográfico-sanitaria basado en una arquitectura cliente
servidor. Esta se presenta efectiva y concretamente todos los requerimientos que describen como el
software debe comportarse para cumplir con sus finalidades y objetivos. La aplicación se encuentra
dividida en sub módulos para, la gestión de datos de entrada, el simulador del proceso evolutivo y
presentación de los datos.


\subsection{Gestión de datos de entrada}
%Muestras
El sistema requiere principalmente de 2 datos de entrada, los putnos de control y los datos
climatólogicos correspondientes a el periodo de tiempo para la simulación

El área de estudio se encuentra determinada por una muestra, que representan a un conjunto de
puntos de control georeferenciados en una zona determinada. Los puntos de control que pertenecen a
una muestra deben ser establecidos en un mismo instante, siendo así la fecha de instalación y
fecha de recolección, de los puntos de control de la muestra, iguales.


Los puntos de control, como resultado de la medición realizada, permiten obtener la densidad de
larvas asociada con la información geográfica.

\section{Presentación de los datos}

%* Diseño y Arquitectura
\section{Diseño y Arquitectura}
GeoDengue está basada en una arquitectura, de tres capas, cliente-servidor, en el que las tareas
se reparten entre los proveedores de recursos o servicios, denominados servidores, y los
demandantes, llamados clientes. La primera capa, la de presentación, es la que se encarga de
interactuar con el usuario final, la segunda capa es la de negocios, esta se encarga de procesar
las solicitudes realizadas por la capa de presentación y definir las reglas que deben aplicase en
para cada solicitud. Por último, se encuentra la capa de datos, donde se almacenan los datos,
porcesan las peticiones de la capa de negocios para persistir o recuperar información.

\begin{figure}
\centering
\includegraphics[width=0.9\textwidth]{capitulo-5/graphics/arquitectura-completa.png}
\caption{\label{fig:arquitectura-completa}Arquitectura de interacción de componentes de GeoDengue.}
\end{figure}

Se optó por un enfoque web debido a la practicidad de estas aplicaciones, el usuario final solo
debe contar con un navegador web. Estas deberían funcionar igual independientemente de la versión
del sistema operativo instalado en el cliente. Las aplicaciones web son catalogadas como
aplicaciones de bajo consumo, debido a que la mayor parte de la aplicación no se encuentra en
nuestro ordenador, y muchas de las tareas de procesamiento que realiza el software no consumen
recursos nuestros porque se realizan en el servidor.

La capa de presentación se encuentra diseñada como un Single Page Application o SPA (en español
Aplicación de una sola página). Un SPA es una aplicación que se ejecuta en una única página, donde
la navegación se realiza mediante cargas parciales, sin recargar el sitio completamente. La
comunicación con el servidor se realiza mendiante peticiones Ajax, para ello se cuenta con una
arquitectura basada en servicios REST.


%* Tecnologías y herramientas utilizadas
\section{Tecnologías y herramientas utilizadas}
Como se mencionó anteriormente, se tomo la desicion de implementar la herramienta como una
aplicación web. En esta sección presentaremos las herramientas y tecnologias empleadas para el
desarrollo.

La capa de presentación fue desarrollada puramente en javascript, html y css donde la comunciación
con la capa de servicios se ecuentra dada por peticiones ajax. Como servidor web se utilizó
Apache\footnote{https://httpd.apache.org/} en su versión 2, aunque se podría utilizarce cualquier
otro servidor web HTTP.

La capa de negocios, se encuentra desarrollada en Python, en donde para la capa de servicios se
utilizó Flask\footnote{http://flask.pocoo.org/} un framework minimalista para Python. Para llevar
a cabo análisis y procesamientos complejos se utilizó la extensión de Python
NumPy\footnote{http://www.numpy.org/} que agrega mayor soporte para vectores y matrices,
constituyendo una biblioteca de funciones matemáticas de alto nivel para operar con esos vectores
o matrices. Como servidor web se utilizó Apache con su módulo
mod\_wsgi\footnote{http://www.modwsgi.org/} que proporciona una interfaz compatible con
WSGI \footnote{WSGI es una especificación para una interfaz simple y universal entre los
servidores web y aplicaciones web o frameworks para el lenguaje programación Python.}

Con respecto al almacenamiento de datos, se decidió utilizar el sistema gestor de bases de datos
PostgreSQL\footnote{http://www.postgresql.org/}, y para la manipulación de datos geográficos se utilizó PostGIS\footnote{http://postgis.net/} que es módulo que añade soporte de objetos
geográficos a PostgreSQL.


Falta hablar de geoserver como mapserver

Falta hablar de como se obtienen los datos climáticos

%\section{Algoritmos y estructuras de datos utilizadas}


