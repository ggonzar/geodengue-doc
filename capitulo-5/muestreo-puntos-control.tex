\subsection{Muestreo y puntos de control}
Las larvitrmpas o puntos de control, deben ser distribuidas geográficamente para la obtener la
información correspondiente a la abundancia poblacional del Aedes aegypti en una región. Con el
fin de acotar el área de estudio, los puntos de control deben agruparse en regiones y estar
asociados al periodo en el que se realizó el muestreo. Al agrupamiento de puntos de control en una
región y un periodo determinado, se denomina muestra, donde para realizar el muestreo en una área
de estudio se debe realizar principalmente los siguientes pasos :

\begin{itemize}
\item \textit{Instalación} : consiste en la distribución geográfica de los puntos de control en el área de estudio. En este paso se deben registrar las coordenadas geográficas del punto de control y su fecha de instalación.

\item \textit{Selección del periodo de revisión} : se debe establecer el periodo de revisión de los puntos de control, de forma a evitar que estos se transformen en criaderos de mosquitos.

\item \textit{Recolección} : una vez finalizado el periodo de revisión, se debe proceder realizar la recolección de los datos, registrando la cantidad de larvas de mosquitos observadas y la fecha de recolección.

\end{itemize}

Se considera que todos los puntos de control pertenecientes a una muestra cuentan con la misma
fecha de instalación y recolección. De este modo la muestra resultante corresponde a una medición
realizada en un área de estudio, para un mismo periodo de tiempo. La definición de las muestras,
el establecimiento y recolección los puntos de control, corresponden al eje principal del sistema,
ya que son considerados un requisito previo para los procesos de simulación y análisis de datos.
