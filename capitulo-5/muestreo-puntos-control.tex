\subsection{Muestreo y puntos de control}
Con el fin de acotar el área de estudio, los puntos de control deben agruparse en regiones y estar
asociados al periodo en el que se realizó el muestreo, estos conjuntos son denominados muestras. Se
considera que todos los puntos de control pertenecientes a una muestra cuentan con la misma fecha
de instalación y recolección. De forma a que una muestra es la resultante de una medición realizada
en una región en un periodo determinado.

La definición de las muestras, el establecimiento y recolección los puntos de control,
corresponden al eje principal del sistema, ya que es prerequisito para los procesos de simulación
y análisis de datos.
