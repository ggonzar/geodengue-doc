\subsection{Muestreo y puntos de control}
Con el fin de acotar el área de estudio, los puntos de control deben agruparse en regiones y estar
asociados al periodo en el que se realizó el muestreo, estos conjutos son denominados muestras. Se
considera que todos los puntos de control pertenecientes a una muestra cuentan con la misma fecha
de instalación y recolección. De forma a que una muestra es la resultante de una medición realizada
en una region en un periodo determinado.

La definición de las muestras, establecimiento y recolección los puntos de control pertenecientes
a las mismas, corresponden a uno de los ejes principales y son un prerequisito para ejecutar o
iniciar algún proceso.
