\subsection{Muestreo y puntos de control}
La utlización de larvitrampas como puntos de control, permite estimar con mayor precisión la
abundancia del Aedes aegypti al proporcionar información fidedigna sobre el número de ejemplares
que habitan en distintos puntos del área de estudio en comparación con la presencia o ausencia de
larvas en recipientes o viviendas usados en los índices aédicos tradicionales
\citet{NINO2011}, que fueron presentados en la \secref{sec:densidad-vectorial-indices-stegomia}.

Con el fin de acotar el área de estudio, los puntos de control deben agruparse en regiones y estar
asociados al periodo en el que se realizó el muestreo, estos conjutos son denominados muestras. Se
considera que todos los puntos de control pertenecientes a una muestra cuentan con la misma fecha
de instalación y recolección. De forma a que una muestra es la resultante de una medición realizada
en una region en un periodo determinado.

La definición de las muestras, establecimiento y recolección los puntos de control pertenecientes
a las mismas, corresponden a uno de los ejes principales y son un prerequisito para ejecutar o
iniciar algún proceso.
