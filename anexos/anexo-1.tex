\chapter{Especificaciones de dispositivos de ovipostura}

\section{Larvitrampas}

\subsection{Especificaciones para la colocación e inspección}
Instalarla a una altura de 50 cm (del suelo a la base de la larvitrampa). Protegerla de la luz
directa del sol, el aire, la lluvia, en lugares a media luz o completamente a la sombra. No deben
ubicarse cercanas a depósitos de agua. Debe evitarse su colocación en lugares completamente
pavimentados, u otros que tengan mucha refracción de la luz. Debe estar visible para la
hembra del mosquito. Protegerla de niños y animales domésticos (perros, gatos, roedores, etc.).

Un prerrequisito para cualquier tipo de larvitrampa en secciones de llantas es que facilite la
inspección visual del agua insitu o que transfiera fácilmente los contenidos a otro recipiente
para que sean examinados.

\subsection{Forma de revisión}
Se establece una rutina semanal para revisar las larvitrampas, para lo cual, una vez por semana
debe vaciarse todo su contenido cuidadosamente (para que no quede ninguna larva en sus paredes) en
un recipiente adecuado para realizar la inspección. En caso de ser positivas, se registra como tal
y las larvas serán colectadas en tubos para ser enviadas al laboratorio para su determinación
taxonómica. Luego, el dispositivo se lava y se acondiciona para ser colocadas nuevamente siguiendo
las especificaciones ya descritas.

Antes de la utilización de la larvitrampa, ésta debe cepillarse y flamearse, luego mantenerla
sumergida en agua durante no menos de tres días, para asegurarse que el agua no contenga residuos
de sustancias que puedan actuar como larvicida. De esta manera, además, se garantiza la
destrucción de algún huevo del mosquito que estuviese previamente en el neumático o en
larvitrampas ya utilizadas.

\subsection{Consideración final}


Tener en cuenta que en verano, con condiciones más favorables para el desarrollo de esta especie,
las larvas pueden alcanzar el estadio de adulto entre 6 y 7 días desde la ovipostura, por lo que
es necesaria la inspección de todas las larvitrampas en los tiempos indicados a fin de evitar que
alguna de ellas se transformen en criaderos de adultos.

\section{Ovitrampa}

\subsection{Especificaciones para la colocación e inspección}
La colocación debe realizarse en lugares representativos del municipio,
especialmente en las zonas donde se produjeron casos de dengue autóctonos
o importados. Respecto al número de ovitrampas a colocar, se sugiere no
menor a 10 por localidad. La idea es mantener el mismo circuito (mismos
lugares de colocación), un modelo a “escala ciudad”, para tener la idea de
la "presencia" relacionada con la distribución geográfica del vector, se
basa en el criterio que la información sea independiente. O sea que sea
improbable (más bien imposible) que una hembra pueda poner huevos en dos
ovitrampas contiguas. Además la instalación debería basarse en la capacidad
operativa de trabajo, y para ello se pueden colocar las ovitrampas en una
grilla con puntos más o menos equidistantes de aproximadamente 400 metros
de lado.

Cada ovitrampa se coloca en un lugar accesible, protegido donde predomine
la sombra y haya cierto grado de humedad (ambiente sombreado). Debe asegurarse
la presencia de moradores al retirarla.Sobre un plano de la localidad o
sector a muestrear se seleccionarán los puntos donde se colocarán las
ovitrampas. Una variante sería colocar una por Unidad Sanitaria que el
municipio posea, asumiendo que la ubicación de las mismas brindará una
visión representativa del conjunto. Conviene tener presente que en este
caso, el muestreo puede no ser representativo de viviendas regulares.

Las ovitrampas deben ser inspeccionadas semanalmente y en el caso de detectar
paletas con huevos, cuando no puedan ser leídos en el nivel local, se deberán
remitir para su lectura a los laboratorios de entomología más cercanos,
(Divisiones de Zoonosis Urbanas, División de Zoonosis Rurales, CEPAVE, etc.).
La remisión será en un sobre o bolsita plástica, con los datos para georreferenciar.
La vigilancia entomológica se debe realizar en forma continua anual. Es
importante destacar que una vez detectada la presencia de Aedes Aegypti por
cualquiera de los sistemas de monitoreo (larvitrampas u ovitrampas) se deben
realizar las acciones inmediatas de control focal en la comunidad.

\subsection{Consideración final}
Es importante añadir un identificador a cada ovitrampa que permita la
identificarlas fácilmente. El rótulo se debe colocar sobre la baja-lengua
o paleta de la ovitrampa, debe estar debidamente escrito (con lápiz) el
número y/o código de la ovitrampa. También se rotulará el frasco sobre
su pared con tinta indeleble. Se recomienda numerar cada una de las paletas
o baja-lenguas y agregarle iniciales para identificar el municipio y
detallar en el protocolo común los datos de cada una (lugar físico por.
ej. calle, barrio y zona del municipio como también la fecha del retiro
de las mismas de su lugar para su posterior envío).

\section{Mosquitérica genérica}

Los materiales para la construcción son los siguientes :
\begin{itemize}
    \item Una botella de plástico de 2 litros.
    \item Tijeras.
    \item Una lija para madera.
    \item Un rollo de cinta aislante.
    \item Una pieza de tul o gasa para sellar la boquilla (cuello).
    \item Un poco de arroz o alpiste.
    \item 250 ml de agua no clorada.
\end{itemize}

Hay que cortar el cilindro en dos partes, de modo que la porción de la
boca quede en forma contraria, formando un embudo. Debe retirarse el tapón
de la botella. Asimismo, se debe retirar con cuidado el anillo de precinto
y almacenarlo, también se utilizará en la mosquitérica.

Se debe lijar bien dentro del "embudo". Esto servirá para aumentar el área
de evaporación, y será más fácil para el mosquito localizar la  trampa.
Hay que colocar la  tela en el cuello y asegurarlo con el anillo. Hay que
tener en cuenta que debe existir un tejido muy fino, de modo que las larvas
no puedan pasar;


Hay que colocar en la parte inferior de la botella la comida que se eligió,
puede ser granos de alpiste o tres granos de arroz, pero siempre triturados.
Se inserta la porción de cuello hacia abajo sobre la parte inferior de la
botella; Se usa cinta adhesiva para asegurar las dos partes, sobre el lado
exterior.

Se pone el agua sin cloro en la mosquitérica a pocos centímetros por encima
del cuello. Si no dispone de agua sin cloro, se usa agua de tomar del grifo
y se deja que repose durante dos días.

\subsection{Especificaciones para la colocación e inspección}
Las técnicas de colocación e inspección mencionadas anteriormente se aplican
a este método.

\subsection{Consideración final}
Actualmente la mosquitérica es utilizada en reemplazo a los antiguos
dispositivos de ovipostura en los últimos trabajos realizados en sudamérica
para el control del vector del dengue por presentar resultados más exactos.
