%----------------------1----------------------------------------
\begin{frame}[c,plain]{}
    \begin{center}
    Anexos.
    \end{center}
\end{frame}



\begin{frame}[c]{Ovitrampas.}
  \begin{center}
    \includegraphics[width=6cm]{../book/capitulo-3/graphics/ovitrampa.jpg}
  \end{center}
\end{frame}

\begin{frame}[c]{Ritmo reproductivo básico}
      El ritmo básico de reproducción de una infección es el número promedio de casos nuevos que genera un caso dado lo largo de un período infeccioso.
  \begin{center}
      \begin{equation}
      \begin{array}{l l}
        R_{o} = \beta * N / \gamma & \quad \text{Si $R_{o} > 1$ $\implies$ Hay epidemia.}\\
      \end{array}
      \end{equation}
  \end{center}
  Donde :
    \begin{itemize}
      \item $\beta$ : Tasa de infección.
      \item $1/\gamma$ : Extensión del periodo infeccioso.
      \item $N$ : La población total (Susceptible + Infectada + Recuperada).
    \end{itemize}
\end{frame}

\begin{frame}[c]{Stack tecnológico.}
\includegraphics[width=\textwidth]{../book/capitulo-5/graphics/stack-tecnologias.png}
\end{frame}

\begin{frame}[c]{Conteo de larvas con PDI.}
    \begin{figure}
    \begin{subfigure}[b]{0.45\textwidth}
        \includegraphics[width=4.5cm]{../book/capitulo-5/graphics/bandeja-muestra.jpg}
        \caption{Bandeja vacía.}
    \end{subfigure}
    ~~~~
    \begin{subfigure}[b]{0.45\textwidth}
        \includegraphics[width=4.5cm]{../book/capitulo-5/graphics/larvas-dengue.jpg}
        \caption{Bandeja con larvas.}
    \end{subfigure}
    \end{figure}
\end{frame}

\begin{frame}[c]{Conteo de larvas con PDI (2).}
    \begin{figure}
    \begin{subfigure}[t]{0.45\textwidth}
        \includegraphics[width=4.5cm]{../book/capitulo-5/graphics/larvas-original.png}
        \caption{Imagen original.}
    \end{subfigure}
    ~~~~
    \begin{subfigure}[t]{0.45\textwidth}
        \includegraphics[width=4.5cm]{../book/capitulo-5/graphics/larvas-otsu.png}
        \caption{Imagen luego de la umbralización.}
    \end{subfigure}
    \end{figure}
\end{frame}

\begin{frame}[t]{Interpolación Espacial.}
  \begin{center}
   \begin{columns}[T]
        \begin{column}[T]{3.5cm}
            \includegraphics[width=4cm]{./graphics/interpolacion-ej.png}
        \end{column}
        \begin{column}[T]{7cm}
        \begin{equation}\label{eq:interpolacion-idw}
         u(x,y) = \sum_{i=1}^{N} w_i(x,y) * u_{i}
        \end{equation}
        Donde :
        \begin{equation}
        w_i(X) =  \dfrac{d((x,y), (x_i,y_i))^{-p}}{\sum_{j=1}^{N} d((x,y), (x_j,y_j))^{-p}}
        \end{equation}
        \end{column}
    \end{columns}
  \end{center}
    \text{Ejemplo:}
    $u(x,y) = w_1(x,y) * 49 + w_2(x,y) * 20 + w_3(x,y) * 55 + w_4(x,y) * 16 $
\end{frame}

\begin{frame}[c]{Interpolación Espacial (Villatoro et al., 2007).}
  \begin{center}
     \begin{itemize}
     \item Kriging fue más preciso y eficiente que el IDW, aunque la diferencia entre ambos métodos no fue muy amplia.

    \item Cuando el distanciamiento, es muy grande, los variogramas no son posibles de obtener, entonces el Kriging deja de ser una opción y comparativamente el IDW se perfila como el mejor.

    \end{itemize}
  \end{center}
\end{frame}
{
\usebackgroundtemplate{\includegraphics[width=\paperwidth]{./graphics/huevos.png}}
\begin{frame}[c]{Modelos de simulación de la ecología del vector.\\\textit{Mortalidad de los huevos(Otero et al., 2006).}}
  % poseen una gran resistencia
  \begin{center}
      \begin{equation}
          M_{H(x,y)} = me * H(x,y)
      \end{equation}
  \end{center}
  Donde :
    \begin{itemize}
      \item $me$ : Tasa de mortalidad diaria igual a $0,01\  \text{días}^{-1}$.
      \item $H(x, y)$ : Cantidad de huevos observados en $(x,y)$.
      \item $M_{H(x,y)}$ : Cantidad de huevos a eliminar.
    \end{itemize}
\end{frame}
}

{
\usebackgroundtemplate{\includegraphics[width=\paperwidth]{./graphics/larva.png}}
\begin{frame}[c]{Simulación del proceso evolutivo. \\\textit{Mortalidad de larvas (Otero et al., 2006)}}
  Mortalidad bajo óptimas condiciones de las larvas:
  \begin{center}
    \begin{equation}
    \label{eq:mortalidad-natural-larvas}
        ml(k) = 0.01 + 0.9725 * exp\bigg( \frac{-(k - 278)}{2.7035}\bigg)
    \end{equation}
  \end{center}
  Donde :
    \begin{itemize}
      \item $k$ : Temperatura en Kelvin.
    \end{itemize}
\end{frame}

\begin{frame}[c]{Modelos de simulación de la ecología del vector.\\\textit{Mortalidad de larvas (Otero et al., 2006).}}
  \begin{center}
      \begin{equation}
      M_{L(x,y)}(k) = ml(k) * L(x,y) + \bigg(\frac{\alpha _{0}}{BS(x,y)}\bigg) * L(x,y) *(L(x,y) - 1)
    \end{equation}
  \end{center}
  Donde:
 \begin{itemize}
      \item $k$ : Temperatura en Kelvin.
      \item $L(x, y)$ : Cantidad de larvas observadas en $(x,y)$.
      \item $\alpha _{0}$ : Capacidad de carga de un solo lugar de reproducción.
      \item $BS(x,y)$ : Es el número de sitios de reproducción en $(x,y)$ .
      \item $M_{L(x,y)}$ : Cantidad de larvas a eliminar.
    \end{itemize}
\end{frame}
}


{
\usebackgroundtemplate{\includegraphics[width=\paperwidth]{./graphics/pupas.png}}

\begin{frame}[c]{Simulación del proceso evolutivo. \\\textit{Mortalidad de las pupas (Otero et al., 2006)}}
  Mortalidad bajo óptimas condiciones de la pupa:
  \begin{center}
    \begin{equation}
    \label{eq:mortalidad-natural-pupas}
        mp(k) = 0.01 + 0.9725 * exp\bigg( \frac{-(k - 278)}{2.7035}\bigg)
    \end{equation}
  \end{center}
  Donde :
    \begin{itemize}
      \item $k$ : Temperatura en Kelvin.
    \end{itemize}
\end{frame}

\begin{frame}[c]{Modelos de simulación de la ecología del vector.\\\textit{Mortalidad de las pupas (Otero et al., 2006).}}
  \begin{center}
    \begin{equation}
        M_{P(x,y)}(k) = P(x,y) * (mp(k) + (1 - ef) * R(k))
    \end{equation}
  \end{center}
  Donde :
    \begin{itemize}
      \item $k$ : Temperatura en Kelvin.
      \item $ef$ : El factor de supervivencia es de $0,83$.
      \item $R(k)$ : La tasa de desarrollo de la pupa.
      \item $P(x, y)$ : Cantidad de pupas observadas en $(x,y)$.
      \item $M_{P(x,y)}$ : Cantidad de pupas a eliminar.
    \end{itemize}
\end{frame}
}


{
\usebackgroundtemplate{\includegraphics[width=\paperwidth]{./graphics/adulto.png}}
\begin{frame}[c]{Modelos de simulación de la ecología del vector.\\\textit{Mortalidad de adultos (Otero et al., 2006).}}
 % 10% diario, 50% semanal, 95% a final del primer mes.
  % Es constante, a pesar de su alta mortalidad, si la población
  % es lo sifucientemente grande, puede llegar a causar una epidemia
  \begin{center}
    \begin{equation}
        M_{A(x,y)} = ma * A(x,y)
    \end{equation}
  \end{center}
  Donde:
    \begin{itemize}
      \item $ma$ : Tasa de mortalidad diaria igual a $0,09$ \ $1/\text{días}$.
      \item $A(x, y)$ : Cantidad de adultos observados en $(x,y)$.
      \item $M_{A(x,y)}$ : Cantidad de adultos a eliminar.
    \end{itemize}
\end{frame}
}

\begin{frame}[t]{Coeficientes del modelo de Sharpe y DeMichele.}
    \begin{itemize}
      \item $\Delta H_{A}$ : Es la entalpía de activación de la reacción que es catalizada por la enzima ($cal\ mol^{-1}$).
      \item $\Delta H_{H}$ : es el cambio de entalpía asociado con la alta temperatura de la inactivación de la enzima ($cal\ mol^{-1}$).
      \item $T_{1/2}$ : es la temperatura cuando la mitad de la enzima se desactiva, a causa de la alta temperatura ($K$).
      \item $R$ : es la constante universal de los gases 1,987 $cal\ mol^{-1}$ .
    \end{itemize}
\end{frame}

\begin{frame}[c]{Entalpía.}
 $\Delta H$ es cantidad de energía absorbida o cedida por un sistema termodinámico, es decir, la cantidad de energía que un sistema intercambia con su entorno.
    \begin{center}
        \begin{equation}
            \Delta H = H_{final} - H_{inicial}
        \end{equation}
    \end{center}
    Donde:
    \begin{itemize}
      \item $H_{final}$ :En una reacción química es la entalpía de los productos.
      \item $H_{inicial}$ :En una reacción química es la entalpía de los reactivos.
    \end{itemize}
\end{frame}
\begin{frame}[c]{Reactivos y productos.}
Los reactivos son aquellos compuestos de una reacción química que combinados con otros compuestos, permiten obtener compuestos diferentes(productos). Los productos son el resultado de mezcla de los reactivos.
\begin{center}
\includegraphics[width=7cm]{./graphics/reactivos-productos.jpg}
\end{center}
\end{frame}

\begin{frame}[t]{Coeficientes del modelo de Sharpe y DeMichele.}
\begin{table}
\begin{minipage}{\textwidth}
    \centering
    \small
    \caption{ \label{tab:coef-sharpe-demichele} Coeficientes para el modelo simplificado de Sharpe y DeMichele, con inhibición de altas temperaturas presentado por Schoolfield.}
    \begin{tabular}{l c r r r r }
        \hline \\
        Ciclo de desarrollo    & $R(298K)$ & $\Delta H_{A}$ & $\Delta H_{H}$ & $\Delta T_{1/2}$  \\
        \hline
        \hline
        Eclosión de los huevos$^a$ & 0,24000 & 10798,00 &  100000,00  & 14184,000\\
        Desarrollo larvario$^b$    & 0,20429 & 36072,78 &   59147,51  &   301,560\\
        Desarrollo pupal$^b$       & 0,74423 & 19246,42 &    5954,35  &   302,687\\
        Ciclo gonotrófico (AN)$^c$ & 0,21600 & 15725,00 & 1756481,00  &   447,200\\
        Ciclo gonotrófico (AP)$^c$ & 0,37200 & 15725,00 & 1756481,00  &   447,200\\
    \end{tabular}
    \footnotetext[1]{Coeficientes Otero et al., 2006.}
    \footnotetext[2]{Coeficientes Rueda et al.,1990.}
    \footnotetext[3]{Coeficientes, para AP y AN, tomados Otero et al., 2006.}
\end{minipage}
\end{table}
\end{frame}


\begin{frame}[t]{Mapas de interpolación a 20 \textcelsius.}
    \begin{figure}
    \begin{subfigure}[b]{0.45\textwidth}
        \includegraphics[width=\textwidth]{../book/capitulo-6/graphics/raster/temp-20-0.png}
        \caption{ Primer día de simulación.}
    \end{subfigure}
    ~~~~
    \begin{subfigure}[b]{0.45\textwidth}
        \includegraphics[width=\textwidth]{../book/capitulo-6/graphics/raster/temp-20-38.png}
        \caption{Día número 50 de simulación.}
    \end{subfigure}
    \end{figure}
\end{frame}

\begin{frame}[t]{Mapas de interpolación a 24 \textcelsius.}
    \begin{figure}
    \begin{subfigure}[b]{0.45\textwidth}
        \includegraphics[width=4.5cm]{../book/capitulo-6/graphics/raster/temp-24-0.png}
        \caption{ Primer día de simulación.}
    \end{subfigure}
    ~~~~
    \begin{subfigure}[b]{0.45\textwidth}
        \includegraphics[width=4.5cm]{../book/capitulo-6/graphics/raster/temp-24-49.png}
        \caption{Día número 50 de simulación.}
    \end{subfigure}
    \end{figure}
\end{frame}

\begin{frame}[t]{Mapas de interpolación a 27 \textcelsius.}
    \begin{figure}
    \begin{subfigure}[b]{0.45\textwidth}
        \includegraphics[width=\textwidth]{../book/capitulo-6/graphics/raster/temp-27-0.png}
        \caption{ Primer día de simulación.}
    \end{subfigure}
    ~~~~
    \begin{subfigure}[b]{0.45\textwidth}
        \includegraphics[width=\textwidth]{../book/capitulo-6/graphics/raster/temp-27-49.png}
        \caption{Día número 50 de simulación}
    \end{subfigure}
    \end{figure}
\end{frame}

\begin{frame}[t]{Mapas de interpolación a 34 \textcelsius.}
    \begin{figure}
    \begin{subfigure}[b]{0.45\textwidth}
        \includegraphics[width=\textwidth]{../book/capitulo-6/graphics/raster/temp-34-0.png}
        \caption{ Primer día de simulación.}
    \end{subfigure}
    ~~~~
    \begin{subfigure}[b]{0.45\textwidth}
        \includegraphics[width=\textwidth]{../book/capitulo-6/graphics/raster/temp-34-42.png}
        \caption{Día número 50 de simulación.}
    \end{subfigure}
    \end{figure}
\end{frame}


