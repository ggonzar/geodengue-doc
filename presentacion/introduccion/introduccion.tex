%----------------------1----------------------------------------
\begin{frame}[t]{Motivaci\'on y Definici\'on del Problema}
\begin{columns}[t]
    \begin{column}[T]{5cm}
          %\includegraphics[height=6cm]{./Imagenes/vision.jpg}
     \end{column}

     \begin{column}[T]{5cm} % alternative top-align that's better for graphics
    La visi\'on es uno de los sentidos m\'as vers\'atiles, y que m\'as informaci\'on nos aporta acerca de nuestro entorno. \\
    La visi\'on computacional trata de emular la capacidad que tiene el cerebro humano para percibir nuestro entorno mediante herramientas computacionales.
     \end{column}
     \end{columns}


\end{frame}

%------------------------------2-------------------------------
\begin{frame}[t]{Motivaci\'on y Definici\'on del Problema. Aplicaciones de Visi\'on computacional}
\centering
\begin{columns}[t]
    \begin{column}[T]{2cm}
    \centering
    \begin{figure}
     \centering
          %\includegraphics[height=2cm]{./Imagenes/vision_robotica.jpg}
          \caption*{Rob\'otica}
    \end{figure}
     \end{column}

     \begin{column}[T]{2cm} % alternative top-align that's better for graphics
      \begin{figure}
       \centering
           % \includegraphics[height=2cm]{./Imagenes/video_vigilancia.jpg}
            \caption*{Procesamiento de Video}
      \end{figure}
     \end{column}

     \begin{column}[T]{3,5cm}
      \begin{figure}
       \centering
          %\includegraphics[height=2cm]{./Imagenes/conteo_celulas.jpg}
          \caption*{Medicina}
      \end{figure}
     \end{column}
     \end{columns}


\end{frame}
