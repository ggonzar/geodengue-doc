%--------------------------1------------------------------------
\begin{frame}[c]{Propuesta}
    \begin{center}
    Modelar y construir una herramienta que permita realizar estudios epidemiológicos de forma
    cartográfica, especializada para el particular caso del dengue, mediante la simulación de
    comportamiento del Aedes aegypti sometido a las variaciones climáticas correspondientes a la
    región con el fin de identificar focos de infestación.
    \end{center}
\end{frame}

%--------------------------2------------------------------------

\begin{frame}[c]{Objetivo General}
    \begin{center}
    Diseñar un modelo que permita analizar la extensión del vector del dengue y estudiar su posible
    relación con un potencial foco de riesgo, de forma a realizar una predicción de posibles focos
    de riesgo.
    \end{center}
\end{frame}


%--------------------------3------------------------------------
\begin{frame}[t]{Objetivos Especificos}
    \begin{center}

        \begin{itemize}
        \item Analizar nuevos métodos de muestreo de la abundancia poblacional del vector, con el fin de apoyar la lucha preventiva contra la enfermedad.

        \item Diseñar un modelo e implementar un sistema computacional mediante el cual se puedan procesar y presentar los resultados obtenidos, en un sistema de información geográfica.

        \item Diseñar el modelo de forma paramétrica y escalable, para que sea aplicable y extensible a cualquier región o área geográfica de estudio.

        \item Generar información relevante que pueda ayudar a las autoridades pertinentes para toma de decisiones en la lucha contra el dengue.
        \end{itemize}
    \end{center}
\end{frame}
