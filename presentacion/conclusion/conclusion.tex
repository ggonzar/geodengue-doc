%----------------------1----------------------------------------
\begin{frame}[t]{Conclusiones}
    \begin{itemize}
        \item El modelo y la herramienta resultante, son aplicables en cualquier región o área de estudio.

        \item Los mapas de interpolación permiten apreciar los niveles de infestación y el riesgo correspondiente a la abundancia de mosquitos.

        \item La instalación y recolección de puntos de control requiere trabajo de campo.

        \item La configuración del simulador del proceso evolutivo requiere de datos ecológicos.

        \item Los resultados obtenidos, utilizando configuraciones del material bibliográfico, son considerados una aproximación válida.

    \end{itemize}
\end{frame}

\begin{frame}[t]{Conclusiones}
    \begin{itemize}
        \item Existen diferencias entre los resultados obtenidos y los observados por expertos en laboratorios.

        \item Las variaciones pueden deberse a los rasgos característicos de las cepas de mosquito.

        \item Se pueden obtener mejores resultados ajustando las configuraciones con los datos ecológicos del Paraguay.

        \item El potencial analítico de la información generada es infinito.
    \end{itemize}
\end{frame}
