\documentclass[oneside]{beamer}


%*****************************************************************************%
%   Theme para la Presentación                                                %
%*****************************************************************************%
\usetheme[pageofpages=de,% String used between the current page and thetotal page count.
          bullet=circle,% Use circles instead of squares for bullets.
          titleline=true,% Show a line below the frame title.
          alternativetitlepage=true,% Use the fancy title page.
          titlepagelogo=../book/graphics/logo.png,% Logo for the first page.
          ]{Torino}
% Nouvelle is a green and red alternative to the chameleon color theme.
\usecolortheme{freewilly}
\providecommand*{\eqcolon}{\mathrel{=\!\raise.16ex\hbox{\footnotesize\!:}}} % para las definiciones

%*****************************************************************************%
%   Paquetes principales                                                      %
%*****************************************************************************%
\usepackage{caption}
\usepackage{amssymb, amsmath}
\usepackage{amsthm}
\usepackage{amsfonts}
\usepackage{float}
\usepackage{afterpage}
\usepackage{dsfont}
\usepackage{url}
\usepackage{color}
\usepackage{lettrine}
\usepackage{algorithm}
\usepackage{algpseudocode}
\usepackage{graphicx}
\usepackage{subfig}
\usepackage{caption}
\usepackage{multimedia}
\usepackage{verbatim} % comentarios
\usepackage{pdfpages}
\usepackage[spanish]{babel}
\usepackage[utf8]{inputenc}

%Para determinar la fecha actual
\newcommand{\monthname}{\ifcase\month\or Enero\or Febrero\or
      Marzo\or Abril\or Mayo\or Junio\or Julio\or Agosto\or Septiembre\or
      Octubre\or Noviembre\or Diciembre\fi}

\newcommand{\thismonth}{\monthname,\ \the\year}


\author{\  \\ Maximiliano Báez González}
\title{Modelo predictivo de focos de dengue aplicado a Sistemas de Información Geográfica}
\institute{Facultad Politécnica- UNA}
%pone la fecha de generación como portada
\date{\thismonth}


\begin{document}
\begin{frame}[t,plain]
\titlepage
\end{frame}

%----------------------------1----------------------------------
\begin{frame}[t]{Motivación y Definición del Problema.}
  \begin{center}
    \includegraphics[width=10cm]{./graphics/dengue-intro.png}
  \end{center}
\end{frame}

%----------------------------2----------------------------------

\begin{frame}[t]{Motivación y Definición del Problema.\\\textit{Criaderos del Aedes aegypti.}}
\begin{center}
    \includegraphics[width=9cm]{./graphics/criaderos.jpg}
    \end{center}
\end{frame}

%----------------------------3----------------------------------


%----------------------------4----------------------------------

\begin{frame}[t]{Motivación y Definición del Problema.\\\textit{Dengue en Paraguay.}}
  \begin{center}
    \begin{itemize}
    \item El Paraguay desde el año 2009 es considerado un país endémico.

    \item El clima, subtropical, favorece la aparición y desarrollo del dengue.

    \item En las últimas décadas se han observado un crecimiento considerable de las notificaciones de posibles casos de dengue, algunas con derivaciones fatales.

    \item Monitorear el comportamiento el vector con técnicas tradicionales de vigilancia.

    \item Control vectorial para la disminución de las poblaciones de mosquitos.
    \end{itemize}
  \end{center}
\end{frame}

%----------------------------5----------------------------------

\begin{frame}[t]{Motivación y Definición del Problema.\\\textit{Dengue en Paraguay.}}
  \begin{center}
  \begin{table}
      \begin{minipage}{\textwidth}
          \begin{center}
          \caption{Histórico de casos de dengue notificados, confirmados y con derivación fatal en Paraguay.}
          \begin{tabular}{l c c r r}
              \hline
              Año & Periodo (inicio / fin) & Notificados & Confirmados & Muertes\\
              \hline
              \hline
              2014 & 29-12-13 / 31-05-14 & 10.541 & 1.052 & 2\\
              2013 & 30-12-12 / 21-12-13 & 153.793 & 131.306 & 70\\
              2012 & 01-01-12 / 22-12-12 & 37.815 & 30.588 & 11\\
              2011 & 03-01-11 / 29-12-11 & 53.397 & 42.264 & 62\\
              2010 & 11-10-09 / 25-12-10 & 21.951 & 13.760 & --$^a$
          \end{tabular}
          \footnotetext[1]{No se encontraron datos sobre muertes en el periodo.}
          \end{center}
      \end{minipage}
  \end{table}
  \end{center}
\end{frame}

%----------------------------6----------------------------------

\begin{frame}[t]{Motivación y Definición del Problema.\\\textit{Vigilancia Entomológica en Paraguay.}}

    \begin{itemize}
      \item Para estimar la densidad del vector, la OMS ha recomendado los siguientes indicadores entomológicos : Índice de Casa, Índice de Recipiente e Índice de Breteau.

      \item  Los indices tradicionales, se fundamentan en la detección de la presencia de formas inmaduras del vector dentro de recipientes domésticos.

      \item Son considerados una pobre indicación de la producción de mosquitos adultos.

    \end{itemize}
\end{frame}

%----------------------------7----------------------------------

\begin{frame}[t]{Motivación y Definición del Problema.\\\textit{Vigilancia Entomológica en Paraguay.}}
    \begin{itemize}
      \item No reflejan la asociación que existe entre las densidades de mosquitos y tipo de recipientes presentes.

      \item Proporciona poca o nula información de aquellas viviendas en las que existe un mayor riesgo de presencia de mosquitos.

      \item Sólo son recomendados para detectar la calidad de las acciones realizadas por el personal de control larvario.

      \item Existen numerosos métodos e indicadores más prácticos, eficientes y económicos, como larvitrampas y ovitrampas.

    \end{itemize}
\end{frame}


\begin{frame}[t]{Motivación y Definición del Problema.\\\textit{Larvitrampas.}}
  \begin{center}
    \includegraphics[width=9cm]{../book/anexos/graphics/construccion-larvitrampa.png}
  \end{center}
\end{frame}

\begin{frame}[t]{Motivación y Definición del Problema.\\\textit{Larvitrampas.}}
  \begin{center}
    \begin{columns}[c]
        \begin{column}[c]{3cm}
          \includegraphics[width=\textwidth]{../book/anexos/graphics/disenho-1.png}

          \includegraphics[width=\textwidth]{../book/anexos/graphics/disenho-2.png}
        \end{column}
        \begin{column}[c]{7cm}
          \begin{itemize}
            \item Criadero artificial y controlado.
            \item Se basan en la detección del vector en su etapa larval.
            \item Brinda información sobre los patrones de actividad espacial y estacional de ovipostura.
            \item Permiten reconocer las condiciones climáticas favorables para la eclosión y desarrollo larvario.
            \item Materiales reciclados como materia prima.
          \end{itemize}
        \end{column}
      \end{columns}
  \end{center}
\end{frame}

%----------------------------6----------------------------------
\begin{frame}[t]{Motivación y Definición del Problema.}
  \begin{itemize}

    \item Las autoridades sanitarias del Paraguay no cuentan con datos computables, geográficamente, relacionados con a casos reportados, confirmados, sospechosos y fatales de dengue que permitan realizar análisis estadísticos y espaciales.

    \item  Se deben optar por nuevas metodologías que permitan generar información para el análisis sin la necesidad de grandes requerimientos.

    \item Las metodologías de vigilancia entomológica basadas en el uso de larvitrampas y ovitrampas permiten generar información regionalizada sobre el estado y la distribución de la población del vector.

  \end{itemize}
\end{frame}


%----------------------1----------------------------------------
\begin{frame}[t]{Propuesta}

\end{frame}

%----------------------1----------------------------------------
\begin{frame}[t]{Objetivo General}

\end{frame}


%----------------------2----------------------------------------
\begin{frame}[t]{Objetivos Especificos}

\end{frame}

%----------------------1----------------------------------------
\begin{frame}[t]{Resultados y discusión. Entorno de pruebas}
\begin{itemize}
    \item Los parámetros de configuración fueron tomados del material bibliográfico de apoyo.
    \item Se generaron aleatoriamente 1.146 larvas para 25 puntos de control.
    \item El periodo de simulación fue de 50 días, a temperatura constante.
    \item La dirección del viento seleccionada fue la suroeste.
    \item Comparar las tasas de desarrollo obtenidas con los resultados obtenidos por expertos en laboratorio.
    \end{itemize}
\end{frame}

%--------------------------2------------------------------------

\begin{frame}[t]{Resultados y discusión. Tasas de desarrollo de los huevos}
    \includegraphics[width=\textwidth]{./graphics/huevos-desarrollo.png}
\end{frame}

\begin{frame}[t]{Resultados y discusión. Tasas de desarrollo de las larvas}
    \includegraphics[width=\textwidth]{./graphics/larvas-desarrollo.png}
\end{frame}

\begin{frame}[c]{Resultados y discusión. Tasas de desarrollo de las pupas}
    \includegraphics[width=\textwidth]{./graphics/pupas-desarrollo.png}
\end{frame}

\begin{frame}[t]{Resultados y discusión. Ciclo gonotrófico.}
    \includegraphics[width=\textwidth]{./graphics/ciclo-gonotrofico-temperatura.png}
\end{frame}

%--------------------------2------------------------------------

\begin{frame}[t]{Resultados y discusión. Crecimiento de la población.}
\begin{center}
    \includegraphics[width=9cm]{../paper/graphics/evolucion-poblacion-all.png}
\end{center}
\end{frame}

\begin{frame}[t]{Resultados y discusión. Crecimiento de la población de adultos.}
\begin{center}
    \includegraphics[width=9cm]{../paper/graphics/evolucion-poblacion-adultos.png}
\end{center}
\end{frame}

%--------------------------2------------------------------------

\begin{frame}[t]{Resultados y discusión. Mapas de interpolación a 20 \textcelsius}
    \begin{figure}
    \begin{subfigure}[b]{0.45\textwidth}
        \includegraphics[width=\textwidth]{./graphics/inicial.png}
        \caption{ Primer día de simulación.}
    \end{subfigure}
    ~~~~
    \begin{subfigure}[b]{0.45\textwidth}
        \includegraphics[width=\textwidth]{./graphics/temp-20-final.png}
        \caption{Día número 50 de simulación.}
    \end{subfigure}
    \end{figure}
\end{frame}

\begin{frame}[t]{Resultados y discusión. Mapas de interpolación a 24 \textcelsius}
    \begin{figure}
    \begin{subfigure}[b]{0.45\textwidth}
        \includegraphics[width=\textwidth]{./graphics/inicial.png}
        \caption{ Primer día de simulación.}
    \end{subfigure}
    ~~~~
    \begin{subfigure}[b]{0.45\textwidth}
        \includegraphics[width=\textwidth]{./graphics/temp-24-final.png}
        \caption{Día número 50 de simulación.}
    \end{subfigure}
    \end{figure}
\end{frame}

\begin{frame}[t]{Resultados y discusión. Mapas de interpolación a 27 \textcelsius}
    \begin{figure}
    \begin{subfigure}[b]{0.45\textwidth}
        \includegraphics[width=\textwidth]{./graphics/inicial.png}
        \caption{ Primer día de simulación.}
    \end{subfigure}
    ~~~~
    \begin{subfigure}[b]{0.45\textwidth}
        \includegraphics[width=\textwidth]{./graphics/temp-27-final.png}
        \caption{Día número 50 de simulación}
    \end{subfigure}
    \end{figure}
\end{frame}

\begin{frame}[t]{Resultados y discusión. Mapas de interpolación a 30 \textcelsius}
    \begin{figure}
    \begin{subfigure}[b]{0.45\textwidth}
        \includegraphics[width=\textwidth]{./graphics/inicial.png}
        \caption{ Primer día de simulación.}
    \end{subfigure}
    ~~~~
    \begin{subfigure}[b]{0.45\textwidth}
        \includegraphics[width=\textwidth]{./graphics/temp-30-final.png}
        \caption{Día número 50 de simulación.}
    \end{subfigure}
    \end{figure}
\end{frame}


\begin{frame}[t]{Resultados y discusión. Mapas de interpolación a 34 \textcelsius}
    \begin{figure}
    \begin{subfigure}[b]{0.45\textwidth}
        \includegraphics[width=\textwidth]{./graphics/inicial.png}
        \caption{ Primer día de simulación.}
    \end{subfigure}
    ~~~~
    \begin{subfigure}[b]{0.45\textwidth}
        \includegraphics[width=\textwidth]{./graphics/temp-34-final.png}
        \caption{Día número 50 de simulación.}
    \end{subfigure}
    \end{figure}
\end{frame}

%----------------------1----------------------------------------
\begin{frame}[t]{Conclusiones}
    \begin{itemize}
        \item El modelo y la herramienta resultante, son aplicables en cualquier región o área de estudio.

        \item Los mapas de interpolación permiten apreciar los niveles de infestación y el riesgo correspondiente a la abundancia de mosquitos.

        \item La instalación y recolección de puntos de control requiere trabajo de campo.

        \item La configuración del simulador del proceso evolutivo requiere de datos ecológicos.

        \item Los resultados obtenidos, utilizando configuraciones del material bibliográfico, son considerados una aproximación válida.

    \end{itemize}
\end{frame}

\begin{frame}[t]{Conclusiones}
    \begin{itemize}
        \item Existen diferencias entre los resultados obtenidos y los observados por expertos en laboratorios.

        \item Las variaciones pueden deberse a los rasgos característicos de las cepas de mosquito.

        \item Se pueden obtener mejores resultados ajustando las configuraciones con los datos ecológicos del Paraguay.

        \item El potencial analítico de la información generada es infinito.
    \end{itemize}
\end{frame}

%----------------------1----------------------------------------
\begin{frame}[t]{Trabajos futuros}

\end{frame}

%----------------------1----------------------------------------
\begin{frame}[c]{Aportes.}
    \begin{itemize}
        \item Una base de conocimiento sobre métodos y modelos matemáticos para la lucha contra el dengue aplicables al Paraguay.
        \item Escala de clasificación de los niveles de infestación.
        \item Conteo de larvas mediante el proceso digital de imágenes.
        \item Un modelo y una herramienta genérica aplicable a cualquier área geográfica de estudio.
        \item Un simulador del proceso evolutivo del ciclo de vida del vector.
        \item Una herramienta web para la administración y análisis de los resultados.
    \end{itemize}
\end{frame}

\begin{frame}[t]{Aportes.}
    \begin{itemize}
        \item Desarrollo de este trabajo como un proyecto de código abierto :

        \begin{itemize}
            \item https://github.com/mbaez/geodengue
            \item https://github.com/mbaez/geodengue-doc
        \end{itemize}
    \end{itemize}
\end{frame}

%----------------------1----------------------------------------
\begin{frame}{}
    \begin{center}
    Gracias por su atención.
    \end{center}
\end{frame}



\end{document}
