\section{Modelo propuesto}
Los métodos de muestreo, como larvitrampas y ovitrampas resultan eficientes y económicos para
determinar determinar la distribución espacial y temporal de Aedes aegypti y otros mosquitos
\cite{dengueUruguayCap1, cenaprece2013}. La distribución geográfica de larvitrampas, consideradas
como puntos de control, permiten generar información regionalizada sobre el estado de las
poblaciones del vector \cite{NINO2011}, en donde esta información puede ser combinada con
información ambiental, demográfica o epidemiológica, con el fin de obtener modelos detallados que
tengan la capacidad de monitorear, simular el comportamiento del vector y en consecuencia,
predecir una posible epidemia del dengue.

El modelo considera un espacio bi-dimensional, con un sistema de coordenadas geográficas $(x,y)$,
para expresar todas las posiciones sobre el plano, correspondientes a la longitud y latitud. Si
consideramos a $m_{i}$ como a un individuo que se encuentra en una etapa del ciclo de vida del
Aedes aegypti, correspondiente a una población de mosquitos, entonces, $m_{i}(x,y)$ representa a
$m_{i}$ en las coordenadas geográficas $(x,y)$.

La evolución de las poblaciones, se ven afectadas por los siguientes eventos: muerte de huevos,
eclosión de huevos, muerte de larvas, emergencia de pupas, muerte de pupas, emergencia de adultos,
muerte de adultos, ovipostura de hembras nulíparas\footnote{Hembras que no han ovipuesto.},
ovipostura de hembras paridas\footnote{Hembras que han ovipuesto al menos una vez.} y dispersión
de los adultos (machos y hembras). Según \cite{otero2006stochastic} los eventos se producen a
tasas que dependen no sólo de valores de la población, sino también de la temperatura, que a su
vez es una función de tiempo, por lo tanto, la dependencia de la temperatura introduce una
dependencia del tiempo en las tasas de eventos.

\subsection{Tasas de desarrollo}
En el modelo se cuenta con 4 tasas de desarrollos correspondientes a : la eclosión de huevos,
emergencia a pupas, emergencia a adultos y el ciclo gonotrófico. Estos valores son obtenidos
mediante el modelo no lineal de Sharpe y DeMichele, presentado en \cite{sharpe1977reaction}, para
procesos poiquilotermos\footnote{La poiquilotermia o ectotermia es un término aplicado a ciertos
animales con temperatura corporal variable}, donde el proceso de maduración es controlado por
una enzima que actúa en un rango de temperatura determinado, la enzima se desactiva a las bajas
temperaturas, y altas. Schoolfield presentó, en \cite{schoolfield1981non}, un versión simplificada
del modelo de Sharpe y DeMichele con inhibición de altas temperaturas, con una única alta
temperatura de desactivación.

\begin{equation} \label{eq:schoolfield}
   R(k)  = R(298K) *\cfrac{ \cfrac{k}{298K} *
    exp \Bigg[
            \cfrac{\Delta H_{A}}{R} \bigg(\cfrac{1}{298K} - \cfrac{1}{k}\bigg)
        \Bigg]}
    {1 + exp\Bigg[\cfrac{\Delta H_{H}}{R} \bigg(\cfrac{1}{T_{1/2}}- \cfrac{1}{k}\bigg)\Bigg] }
\end{equation}

Donde $R(k)$ representa la tasa de desarrollo media ($dias^{-1}$) para una temperatura $K$,en la
escala de Kelvin; $T_{1/2}$ es la temperatura cuando la mitad de la enzima se desactiva, debido a
la alta temperatura, mientras que $H_{A}$, $H_{H}$ y $H_{L}$ son entalpías termodinámicas
características del organismo, y $R$, igual $1,987202$ $cal/K.mol$, es la constante universal de
los gases. Los parámetros $R(298K)$, $H_{A}$, $T_{1/2}$, y $H_{H}$ son estimados mediante la
de regresión no lineal de Wagner, presentado en \cite{wagner1984modeling}. Según
\cite{otero2006stochastic}, el modelo simplificado de Schoolfield, es lo suficientemente
flexible para el ajuste de los datos biológicos disponibles. Los parámetros deben calcularse para
cada etapa de desarrollo, una vez determinados, la ecuación puede utilizarse para calcular tasas
de desarrollo a cualquier temperatura \cite{rueda1990temperature}.

\subsection{Zonificación}
\label{subsec:cap4-zonificacion}
Cada entorno puede contar con factores que lo hagan más o menos apto para el desarrollo,
mortalidad, alimentación, dispersión, y reproducción de individuos. Con el fin de
simplificar ciertos aspectos muy específicos que se encuentran fuera del alcance de este trabajo,
se realizan ciertas hipótesis generales, justificadas para este caso de aplicación, pero puede
requerir una revisión en caso general. Estas hipótesis son, los valores observados en un
conjunto de puntos de control, pertenecientes a una zona, permiten la caracterización de dicha
zona como más o menos apta para desarrollo, mortalidad, alimentación, dispersión, y reproducción de
individuos. También consideramos que el tamaño de la zona, y por ende la cantidad de puntos de
control que pertenecen a ella, influye en la caracterización de las zonas.

Para determinar el tipo de zona de un individuo $m_{i}$ ubicado en $(x,y)$, primero se estima la
densidad relativa de larvas, para $m_{i}(x,y)$, utilizando interpolación espacial y posteriormente
se la clasifica utilizando una escala. Si consideramos a $u(x,y)$ el valor interpolado para
$m_{i}(x,y)$, entonces la densidad relativa de larvas de $m_{i}(x,y)$ es igual a $u(x,y)$.

La forma general de encontrar un valor interpolado $u$ en un punto $(x,y)$ basado en un conjunto de
muestras $u_i = u (x_i)$ para $i = 0,1, ..., N$ utilizando IDW, es una función de interpolación:

\begin{equation}\label{eq:interpolacion-idw}
 u(x,y) = \sum_{i=1}^{N} w_i(X) * u_{i}
\end{equation}

Donde :
\begin{equation}
w_i(X) =  \dfrac{d(X, X_i)^{-p}}{\sum_{j=1}^{N} d(X, X_i)^{-p}}
\end{equation}

Siendo $d(X, X_i)$ una función que determina la distancia existente entre $X$ y $X_{i}$, donde $p$
es el exponente de ponderación.  El valor estimado de $u(x,y)$ es utilizado para clasificar la
zona como \textit{Pésima}, \textit{Mala}, \textit{Regular}, \textit{Buena} u \textit{Óptima} con
influencia positiva el desarrollo, alimentación, dispersión, y reproducción de individuos y
negativamente para la mortalidad.

\begin{table}[!hptb]
\begin{threeparttable}
    \begin{minipage}[b]{0.5\textwidth}
    \caption{\label{tab:cap4-puntaje-zona} Escala de clasificación de las zonas de acuerdo a la densidad relativa de larvas.}
    \footnotesize
    \begin{tabular}{l c c c c}
        \hline \\
                     & Mínimo\tnote{a} & Máximo\tnote{a} & Hembras     & Hembras \\
        Tipo de zona & $u(x,y)$   & $u(x,y)$   & Adultas\tnote{b} & Reproductivas \tnote{c}\\
        \hline
        \hline\\
        Pésima  & 0  & 19 & 8  & 5 \\
        Mala    & 20 & 35 & 15 & 10\\
        Regular & 36 & 51 & 22 & 15\\
        Buena   & 52 & 69 & 30 & 20\\
        Óptima  & 70 & --\tnote{d} & --\tnote{d} & --\tnote{d}\\
        \hline
    \end{tabular}
    \begin{tablenotes}[flushleft]\footnotesize
    \item[a]{Rango mínimo y máximo de $u(x,y)$ permitido para el tipo de zona.}
    \item[b]{Cantidad máxima de hembras adultas, al final del periodo de desarrollo.}
    \item[c]{Cantidad de hembras adultas con capacidad de oviponer.}
    \item[d]{No se estableció un límite superior para las zonas óptimas. }
    \end{tablenotes}
    \end{minipage}
    \end{threeparttable}
\end{table}

En la \tabref{tab:cap4-puntaje-zona} se pueden observar los rangos definidos para cada tipo de
zona, en donde $u$ es la densidad relativa de larvas en las coordenadas $(x,y)$. Los límites para
las zonas fueron determinados clasificando los valores, de las hembras reproductivas en, grupos
múltiplos de cinco. No se estableció un límite superior para las zonas óptimas debido a que los
valores mayores a el mínimo establecido, 70 larvas por dispositivo, pertenecen a la misma categoría

\subsection{Mortalidad}
\label{subsec:cap4-mortalidad}
La mortalidad de los individuos depende de la etapa del ciclo de desarrollo en el que se encuentren
los individuos de una población.

\subsubsection{Mortalidad de huevos}
La tasa de mortalidad de los huevos se encuentra definida como una constante, $me = 0.01$,
$1/\text{días}$, independiente de la temperatura \cite{otero2006stochastic}.

\begin{equation}
    M_{H(x,y)} = me * H(x,y)
\end{equation}

Donde $M_{H(x,y)}$ es la cantidad de huevos que deben ser eliminados de la población $H(x,y)$.

\subsubsection{Mortalidad de larvas}
La mortalidad de las larvas, según \cite{otero2006stochastic}, se encuentra dividida en dos
contribuciones. La primera contribución representa la mortalidad natural bajo óptimas condiciones
y se encuentra influenciada únicamente de la temperatura \cite{otero2006stochastic}. Esta tasa se
encuentra definida por :

\begin{equation}
\label{eq:mortalidad-natural-larvas}
    ml(k) = 0.01 + 0.9725 * exp\bigg( \frac{-(k - 278)}{2.7035}\bigg)
\end{equation}

La segunda contribución es la mortalidad denso dependiente de las larvas \cite{otero2006stochastic}
. Este mecanismo de regulación puede estar relacionado con procesos concurrentes, como las
limitaciones de los alimentos, las interacciones químicas, presencia de depredadores
especializados en el sitio de reproducción y mucho más \cite{otero2006stochastic}. Esta se
encuentra definida por :

\begin{equation}
  \alpha (x,y) = \alpha _{0}/BS(x,y)
\end{equation}

Donde $\alpha _{0}$ está asociado a la capacidad de carga de un solo lugar de reproducción y
$BS(x,y)$ es el número de sitios de reproducción en $(x,y)$. El valor de $\alpha _{0}$ puede
ajustado a los valores observados en la región que se está simulando. Se considera que el valor de
$BS(x,y)$ se encuentra influenciado por $u(x,y)$, de ese modo a medida que $u(x,y)$ varíe, lo debe
hacer el valor de $BS(x,y)$. Para el cálculo de $BS$ relativo a $(x,y)$ se utiliza el método
interpolador de Lagrange.

\begin{equation}
\label{eq:sitios-reproduccion-x-y}
    bs(u(x,y)) = \sum_{i=0}^{n} bs_{i} * l_{i}(u(x,y))
\end{equation}

donde $l_j(u(x,y))$ son los llamados polinomios de Lagrange, que se calculan de este modo:

\begin{equation}
\label{eq:sitios-reproduccion-x-y}
    l_{i}(u(x,y)) = \prod_{j \neq i} \cfrac{u(x,y) - u_{j}}{u_{i} - u_{j}}
\end{equation}

Consideramos un polinomio de tercer grado, con los parámetros $u_0$, $u_1$ y $u_2$ igual a $19$,
$51$ y $70$, correspondientes a zonas del tipo \textit{Pésima}, \textit{Regular} y \textit{Óptima}.
Los valores, $bs_{min}$, $bs_{med}$ y $bs_{max}$ respectivamente, estos son parámetros
configurables del modelo donde $bs_{min}$ representa el menor $BS$ observado, $bs_{max}$
representa el mayor $BS$ observado y $bs_{med}$ es el valor medio existente entre $bs_{max}$ y
$bs_{min}$.

Tomando ambas contribuciones, la mortalidad natural bajo óptimas condiciones y la denso
dependiente, la mortalidad de las larvas queda definida como :
\begin{equation}
    M_{L(x,y)}(k) = ml(k) * L(x,y) + \alpha (x,y) * L(x,y) * (L(x,y) - 1)
\end{equation}

Donde $M_{L(x,y)}$ es la cantidad de larvas que deben ser eliminadas de la población $L(x,y)$.

\subsubsection{Mortalidad de las pupas}
La tasa de mortalidad de las pupas se encuentra definida como una función influenciada únicamente
de la temperatura \cite{otero2006stochastic}.

\begin{equation}
\label{eq:mortalidad-natural-pupas}
    mp(k) = 0.01 + 0.9725 * exp\bigg( \frac{-(k - 278)}{2.7035}\bigg)
\end{equation}

Además de la mortalidad diaria en la fase de pupa, existe una importante mortalidad adicional
asociada con la emergencia sin éxito de adultos, solo el 83 \%  de las pupas alcanzan la maduración
y emergerán como mosquitos adultos, por lo tanto, el factor de supervivencia es de $ef = 0.83$
\cite{otero2006stochastic}.

\begin{equation}
    M_{P(x,y)}(k) = P(x,y) * (mp + (1 - ef) * R(k))
\end{equation}

Donde $M_{P(x,y)}$ es la cantidad de pupas que deben ser eliminadas de la población $P(x,y)$.

\subsubsection{Mortalidad de adultos}
La tasa de mortalidad de los adultos se encuentra definida como una constante, $ma = 0.09$,
$1/\text{días}$, independiente de la temperatura \cite{otero2006stochastic}.

\begin{equation}
    M_{A(x,y)} = ma * A(x,y)
\end{equation}

Donde $M_{A(x,y)}$ es la cantidad de adultos que deben ser eliminados de la población $A(x,y)$.

\subsection{Madurez y cambio de estado}
La madurez de los individuos pertenecientes a las poblaciones de $H(x,y)$, $L(x,y)$ y $P(x,y)$
indica la proximidad de que estos alcancen el siguiente estado de su etapa de desarrollo. Sea
$R(k_{i})$ la tasa de desarrollo de un individuo para una temperatura de $k_{i}$ Kelvin en un
instante $i$, se considera que ha alcanzado su máximo nivel de madurez y se encuentra listo para
pasar al siguiente estado de su cuando $\sum_{i=0}^{N} R(k_{i}) \geq 1$ , en donde $N$ es la
cantidad de días que le toma al individuo pasar de un estado a otro. Para $R(k)$ se se aplican los
parámetros correspondientes al estado actual del individuo.

\subsection{Ciclo gonotrófico y Ovipostura}
El ciclo gonotrófico de los mosquitos es el nombre que se le adjudicó al período que existe desde
que el mosquito realiza una alimentación sanguínea - ovipostura - hasta una nueva alimentación.
Como se mencionó anteriormente, la tasa de desarrollo del ciclo gonotrófico puede estimarse
mediante la versión simplificada del modelo de Sharpe y DeMichele \cite{sharpe1977reaction},
propuesta por Schoolfield en \cite{schoolfield1981non}.

Sea $R(k_{i})$ la tasa de desarrollo del ciclo gonotrófico de una hembra (nulípara o parida), para
una temperatura de $k_{i}$ Kelvin en un instante $i$, se considera que un día es de ovipostura si
se cumple $\sum_{i=0}^{N} R(k_{i}) \geq 1$, en donde $N$ es la duración en días del ciclo
gonotrófico de la hembra adulta. Para $R(k)$ se aplican los parámetros correspondientes a hembras
nulíparas y paridas de acuerdo al estado de la hembra. En \cite{edman1987host} se observó que
hembras nulíparas de Aedes aegypti poseen un proceso de digestión más lento en las hembras paridas
y por ende el ciclo gonotrófico de las mismas tiende a ser más largo. La cantidad de huevos en
cada oviposición, luego de las alimentaciones sanguíneas correspondientes, varía entre 30 y 100
unidades \cite{luevano1993ciclo, beltran2001bionomia,cabezas2005dengue}.

\subsection{Vuelo y dispersión}
El Aedes aegypti es un mosquito doméstico que generalmente esta confinado a las casas donde se
cría \cite{luevano1993ciclo}, tiende a permanecer físicamente en donde emergió, siempre y cuando
no exista algún factor que la perturbe o no disponga de huéspedes, sitios de reposo y de postura
\cite{ThironIzcazaJ2003}. Por lo general, el mosquito, no sobrepasa los 50 a 100 metros durante su
vida \cite{cabezas2005dengue}. En caso de no contar con sitios adecuados de ovipostura y
disponibilidad de alimento tienden a dispersarme una mayor distancia, hasta tres kilómetros, en
busca de mejores condiciones \cite{ThironIzcazaJ2003}. Los mosquitos tienen la particularidad de
volar en sentido contrario a la dirección al viento \cite{ThironIzcazaJ2003,web-site:speedAnimals}
y a una velocidad máxima de 2 kilómetros por hora \cite{web-site:speedAnimals,kaufmann2004flight}.

Partiendo de las hipótesis realizadas en la podemos considerar que la dispersión se encuentra
influenciada por el valor de $u(x,y)$, de ese modo a medida que $u(x,y)$ varíe, la dispersión debe
ajustarse a su tipo de zona. De forma simplificada definimos que la dispersión de un adulto que se
encuentre en zonas del tipo \textit{Regular}, \textit{Buena} u \textit{Óptima} se encuentra entre
0 y 100 metros de vuelo. Para las hembras adultas, que pertenezcan a zonas del tipo \textit{Mala}
o \textit{Pésima} se tiene una dispersión entre 100 a 3.000 metros de vuelo, de este modo, las
hembras adultas que se encuentren en zonas menos aptas tenderán a desplazarse en busca de mejores
condiciones.

\subsection{Simulación del proceso evolutivo}
La simulación del proceso evolutivo del mosquito del Aedes aegypti, es el encargado de simular los
efectos temperatura en el ciclo de vida del mosquito. Estos efectos pueden desencadenar en una
serie de eventos en el individuo como : eclosión de huevos, mortalidad de larvas, emergencia de
pupas, muerte de pupas, emergencia de adultos, muerte de adultos, ovipostura y dispersión de los
adultos.

La población inicial es obtenida mediante la cantidad de larvas observadas en los puntos de
control que corresponden a la muestra utilizada para el estudio. Por cada larva observada, en un
punto de control ubicado en las coordenadas geográficas, $(x, y)$, se inicializa un individuo con
las mismas coordenadas del punto de control de origen.

El simulador del proceso evolutivo, que es considerado como un proceso iterativo, el proceso
inicia tomando como parámetros de entrada la población inicial, y el periodo de simulación,
representado por $T$. El proceso se ejecutará siempre y cuando se cumplan las siguientes
condiciones : el periodo de simulación, $T$, no haya finalizado y que la población cuente con
individuos para su procesamiento. En el caso de que no se cumplan algunas de las condiciones
mencionadas anteriormente, el proceso de simulación finalizará.

El desarrollo de los individuos se encarga de calcular las tasas de desarrollo correspondientes
para cada etapa de su ciclo de vida , con el fin de estimar su desarrollo considerando las
condiciones climáticas. El cambio de estado es consecuencia de la finalización de la etapa de
desarrollo del individuo, donde el individuo ya está listo para pasar a la siguiente etapa de su
ciclo de desarrollo.

La regulación de la población es la encargada de calcular las tasas de mortalidad diaria,
correspondientes a cada etapa del ciclo de desarrollo del individuo, con el fin de reducir el
tamaño de la población debido a la mortalidad diaria de los individuos.

Si el individuo en cuestión corresponde a una hembra adulta inseminada, entonces esta se encuentra
en fase reproductiva. La postura de huevos se realiza respetando la tasa de desarrollo del ciclo
gonotrófico de las hembras . Si la hembra adulta ovipone, los huevos son añadidos a la población
como individuos en un estado inicial de \textit{HUEVO}.
