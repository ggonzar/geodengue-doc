\section{Introduction}
El Aedes aegypti es el principal vector del dengue en América, y afecta a más de 2,500 millones
de personas que viven en zonas en riesgo \cite{world2009dengue, gustavo2006dengue}. Actualmente la enfermedad es considerada un problema grave para la salud pública a nivel mundial
\cite{dengueUruguayCap1, world2009dengue, DIBO2005}, siendo catalogada como ejemplo de una
enfermedad que puede constituir una emergencia de salud pública de interés internacional con
implicaciones para la seguridad sanitaria \cite{dengueUruguayCap1, world2009dengue}, debido a la
necesidad de interrumpir la infección y la rápida propagación de la epidemia más allá de las
fronteras \cite{world2009dengue}.

Las autoridades sanitarias del Paraguay no cuentan con datos computables, geográficamente,
relacionados con a casos reportados, confirmados, sospechosos y fatales de dengue que permitan
realizar análisis estadísticos y espaciales, como regresiones geográficamente ponderadas para
determinar la relación existente entre los casos de dengue y variables como : datos
climatológicos, criaderos de mosquitos, e indices de infestación larvaria. Para realizar este
trabajo de análisis, primero de deben diseñar y desarrollar las plataformas para el registro de la
información, definir las políticas y metodologías de actualización y migración de los
datos, para posteriormente realizar un análisis. Esto implica un trabajo institucional con un
alto costo y cuya ejecución podría llevar un tiempo considerable. Teniendo en cuenta dicho
requerimiento previo, se deben optar por nuevas metodologías que permitan generar información para
el análisis sin la necesidad de requerimientos institucionales para realizar estimaciones válidas.

Las metodologías de vigilancia entomológica basadas en el uso de larvitrampas y ovitrampas
permiten generar información regionalizada sobre el estado y la distribución de la población del
vector. El análisis de la distribución espacial del Aedes aegypti, en los sistemas de información
geográfica, permite a las autoridades sanitarias una mejor definición, planificación y evaluación
de las acciones a realizar para disminuir las poblaciones del vector en las regiones con alta
densidad vectorial.
