\section{Introduction}
El Aedes aegypti es el principal vector del dengue en América, y afecta a más de 2,500 millones
de personas que viven en zonas en riesgo \cite{world2009dengue, gustavo2006dengue}. Actualmente la enfermedad es considerada como un problema grave para la salud pública a nivel mundial
\cite{dengueUruguayCap1, world2009dengue, DIBO2005}, siendo catalogada como ejemplo de una
enfermedad que puede constituir una emergencia de salud pública de interés internacional con
implicaciones para la seguridad sanitaria \cite{dengueUruguayCap1, world2009dengue}, debido a la
necesidad de interrumpir la infección y la rápida propagación de la epidemia más allá de las
fronteras \cite{world2009dengue}.

El Paraguay desde el año 2009 es considerado un país endémico, cuyo clima, subtropical, favorece
la aparición y desarrollo del dengue. Las autoridades sanitarias nacionales llevan a cabo
acciones para la vigilancia entomológica con el fin monitorear la densidad vectorial en zonas
endémicas y no endémicas, mediante técnicas basadas en utilización de indices tradicionales.
Actualmente existen numerosos métodos e indicadores más prácticos, eficientes y económicos para
determinar las poblaciones de Aedes aegypti \cite{cenaprece2013}, como larvitrampas y ovitrampas.

Las metodologías de vigilancia entomológica basadas en el uso de larvitrampas y ovitrampas
resultan eficientes y económicos para determinar determinar la distribución espacial y temporal de
Aedes aegypti y otros mosquitos \cite{dengueUruguayCap1, cenaprece2013}. El análisis de la
distribución espacial del Aedes aegypti, en los sistemas de información geográfica, permite a las
autoridades sanitarias una mejor definición, planificación y evaluación de las acciones a realizar
para disminuir las poblaciones del vector en las regiones con alta densidad vectorial.
