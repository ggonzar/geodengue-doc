
\documentclass[conference]{IEEEtran}

\usepackage[style=numeric, uniquename=full, sorting=none,backend=biber, natbib=true]{biblatex}

%se importan las configuraciones custmizdas realizadas.
\addbibresource{../book/referencias.bib}

\usepackage[spanish]{babel}
\usepackage[utf8]{inputenc}

% *** GRAPHICS RELATED PACKAGES ***
%
\usepackage[pdftex]{graphicx}


% *** MATH PACKAGES ***
%
\usepackage{amsmath,amsthm}
\usepackage{amssymb}
\usepackage{textcomp}

% *** SPECIALIZED LIST PACKAGES ***
\usepackage{algorithm}
%\usepackage{algorithmic}
\usepackage{algpseudocode}


%Traducción al español del paquete algorithmic%
\floatname{algorithm}{Algoritmo}
\renewcommand{\algorithmicrequire}{\textbf{Entrada:}}
\renewcommand{\algorithmicensure}{\textbf{Salida:}}
\renewcommand{\algorithmicend}{\textbf{fin}}
\renewcommand{\algorithmicif}{\textbf{si}}
\renewcommand{\algorithmicthen}{\textbf{entonces}}
\renewcommand{\algorithmicelse}{\textbf{si no}}
\newcommand{\algorithmicelsif}{\textbf{si no, si}}
\renewcommand{\algorithmicfor}{\textbf{para}}
\renewcommand{\algorithmicforall}{\textbf{para todo}}
\renewcommand{\algorithmicdo}{\textbf{hacer}}
 %\renewcommand{\algorithmicendfor}{\algorithmicend\ \algorithmicfor}
% \renewcommand{\algorithmicwhile}{\textbf{mientras}}
% \renewcommand{\algorithmicendwhile}{\algorithmicend\ \algorithmicwhile}
% \renewcommand{\algorithmicloop}{\textbf{repetir}}
% \renewcommand{\algorithmicendloop}{\algorithmicend\ \algorithmicloop}
% \renewcommand{\algorithmicrepeat}{\textbf{repetir}}
% \renewcommand{\algorithmicuntil}{\textbf{hasta que}}
% \renewcommand{\algorithmicprint}{\textbf{imprimir}}
\renewcommand{\algorithmicreturn}{\textbf{retorna}}
% \renewcommand{\algorithmictrue}{\textbf{cierto }}
% \renewcommand{\algorithmicfalse}{\textbf{falso }}
% *** ALIGNMENT PACKAGES ***
%
%\usepackage{array}
%\usepackage{mdwmath}
%\usepackage{mdwtab}

%\usepackage{eqparbox}
\usepackage{threeparttable}
\usepackage{caption}
\usepackage{subcaption}
%\usepackage[caption=false]{caption}
%\usepackage[font=footnotesize]{subfig}

%\usepackage[caption=false,font=footnotesize]{subfig}

% *** FLOAT PACKAGES ***
%
%\usepackage{fixltx2e}
%\usepackage{stfloats}

\newcommand{\figref}[1]{Figura \ref{#1}}
\newcommand{\tabref}[1]{Cuadro \ref{#1}}
\newcommand{\secref}[1]{sección \ref{#1}}
% \newcommand{\algref}[1]{Algoritmo \ref{#1}}

% correct bad hyphenation here
\hyphenation{}


\begin{document}
%
% paper title
% can use linebreaks \\ within to get better formatting as desired
\title{Modelo predictivo de focos de dengue aplicado a Sistemas de Información Geográfica}


% author names and affiliations
% use a multiple column layout for up to three different
% affiliations
\author{\IEEEauthorblockN{Maximiliano Báez González}
    \IEEEauthorblockA{Facultad Politécnica- Universidad Nacional de Asunción\\
    P.O.Box: 2111 SL, San Lorenzo - Central - Paraguay\\
    Email: maxibaezpy@gmail.com
    }
}
% make the title area
\maketitle

\begin{abstract}
%\boldmath
El Dengue es una enfermedad viral transmitida por el mosquito del Aedes aegypti. En Paraguay las autoridades sanitarias llevan a cabo acciones para la vigilancia entomológica con el fin monitorear la densidad vectorial en zonas endémicas y no endémicas, mediante técnicas basadas en utilización de indices tradicionales. Actualmente existen numerosos métodos e indicadores más prácticos, eficientes y económicos para determinar las poblaciones de Aedes aegypti como larvitrampas y ovitrampas. La información regionalizada obtenida de los métodos de muestreo, como larvitrampas y ovitrampas, puede ser combinada con información ambiental, demográfica o epidemiológica, con el fin de obtener modelos detallados que tengan la capacidad de monitorear, simular el comportamiento del vector y en consecuencia, predecir una posible epidemia del dengue. En este trabajo presenta el diseño e implementación de un modelo predictivo para identificar focos de infestación del vector del dengue. El modelo es implementado como un simulador del proceso evolutivo de la ecología del vector, compuesto de un conjunto de submodelos que buscan estimar la tasa de desarrollo, mortalidad, reproducción y dispersión del vector del dengue ante las variaciones climáticas a las que es sometido, donde la población inicial es generada a partir de la información obtenida de las larvitrampas, con el fin de generar información suficiente para contribuir con la detección temprana de posibles brotes epidemiológicos.
\end{abstract}

\begin{IEEEkeywords}
Aedes, dengue, vectores de enfermedades, modelos estadísticos, mapa de riesgo,vigilancia  epidemiológica, SIG, Dinámica de la población de Aedes aegypti.
\end{IEEEkeywords}

% For peerreview papers, this IEEEtran command inserts a page break and
% creates the second title. It will be ignored for other modes.
\IEEEpeerreviewmaketitle

%introduccion
%----------------------------1----------------------------------
\begin{frame}[t]{Motivación y Definición del Problema.}
  \begin{center}
    \includegraphics[width=10cm]{./graphics/dengue-intro.png}
  \end{center}
\end{frame}

%----------------------------2----------------------------------

\begin{frame}[t]{Motivación y Definición del Problema.\\\textit{Criaderos del Aedes aegypti.}}
\begin{center}
    \includegraphics[width=9cm]{./graphics/criaderos.jpg}
    \end{center}
\end{frame}

%----------------------------3----------------------------------


%----------------------------4----------------------------------

\begin{frame}[t]{Motivación y Definición del Problema.\\\textit{Dengue en Paraguay.}}
  \begin{center}
    \begin{itemize}
    \item El Paraguay desde el año 2009 es considerado un país endémico.

    \item El clima, subtropical, favorece la aparición y desarrollo del dengue.

    \item En las últimas décadas se han observado un crecimiento considerable de las notificaciones de posibles casos de dengue, algunas con derivaciones fatales.

    \item Monitorear el comportamiento el vector con técnicas tradicionales de vigilancia.

    \item Control vectorial para la disminución de las poblaciones de mosquitos.
    \end{itemize}
  \end{center}
\end{frame}

%----------------------------5----------------------------------

\begin{frame}[t]{Motivación y Definición del Problema.\\\textit{Dengue en Paraguay.}}
  \begin{center}
  \begin{table}
      \begin{minipage}{\textwidth}
          \begin{center}
          \caption{Histórico de casos de dengue notificados, confirmados y con derivación fatal en Paraguay.}
          \begin{tabular}{l c c r r}
              \hline
              Año & Periodo (inicio / fin) & Notificados & Confirmados & Muertes\\
              \hline
              \hline
              2014 & 29-12-13 / 31-05-14 & 10.541 & 1.052 & 2\\
              2013 & 30-12-12 / 21-12-13 & 153.793 & 131.306 & 70\\
              2012 & 01-01-12 / 22-12-12 & 37.815 & 30.588 & 11\\
              2011 & 03-01-11 / 29-12-11 & 53.397 & 42.264 & 62\\
              2010 & 11-10-09 / 25-12-10 & 21.951 & 13.760 & --$^a$
          \end{tabular}
          \footnotetext[1]{No se encontraron datos sobre muertes en el periodo.}
          \end{center}
      \end{minipage}
  \end{table}
  \end{center}
\end{frame}

%----------------------------6----------------------------------

\begin{frame}[t]{Motivación y Definición del Problema.\\\textit{Vigilancia Entomológica en Paraguay.}}

    \begin{itemize}
      \item Para estimar la densidad del vector, la OMS ha recomendado los siguientes indicadores entomológicos : Índice de Casa, Índice de Recipiente e Índice de Breteau.

      \item  Los indices tradicionales, se fundamentan en la detección de la presencia de formas inmaduras del vector dentro de recipientes domésticos.

      \item Son considerados una pobre indicación de la producción de mosquitos adultos.

    \end{itemize}
\end{frame}

%----------------------------7----------------------------------

\begin{frame}[t]{Motivación y Definición del Problema.\\\textit{Vigilancia Entomológica en Paraguay.}}
    \begin{itemize}
      \item No reflejan la asociación que existe entre las densidades de mosquitos y tipo de recipientes presentes.

      \item Proporciona poca o nula información de aquellas viviendas en las que existe un mayor riesgo de presencia de mosquitos.

      \item Sólo son recomendados para detectar la calidad de las acciones realizadas por el personal de control larvario.

      \item Existen numerosos métodos e indicadores más prácticos, eficientes y económicos, como larvitrampas y ovitrampas.

    \end{itemize}
\end{frame}


\begin{frame}[t]{Motivación y Definición del Problema.\\\textit{Larvitrampas.}}
  \begin{center}
    \includegraphics[width=9cm]{../book/anexos/graphics/construccion-larvitrampa.png}
  \end{center}
\end{frame}

\begin{frame}[t]{Motivación y Definición del Problema.\\\textit{Larvitrampas.}}
  \begin{center}
    \begin{columns}[c]
        \begin{column}[c]{3cm}
          \includegraphics[width=\textwidth]{../book/anexos/graphics/disenho-1.png}

          \includegraphics[width=\textwidth]{../book/anexos/graphics/disenho-2.png}
        \end{column}
        \begin{column}[c]{7cm}
          \begin{itemize}
            \item Criadero artificial y controlado.
            \item Se basan en la detección del vector en su etapa larval.
            \item Brinda información sobre los patrones de actividad espacial y estacional de ovipostura.
            \item Permiten reconocer las condiciones climáticas favorables para la eclosión y desarrollo larvario.
            \item Materiales reciclados como materia prima.
          \end{itemize}
        \end{column}
      \end{columns}
  \end{center}
\end{frame}

%----------------------------6----------------------------------
\begin{frame}[t]{Motivación y Definición del Problema.}
  \begin{itemize}

    \item Las autoridades sanitarias del Paraguay no cuentan con datos computables, geográficamente, relacionados con a casos reportados, confirmados, sospechosos y fatales de dengue que permitan realizar análisis estadísticos y espaciales.

    \item  Se deben optar por nuevas metodologías que permitan generar información para el análisis sin la necesidad de grandes requerimientos.

    \item Las metodologías de vigilancia entomológica basadas en el uso de larvitrampas y ovitrampas permiten generar información regionalizada sobre el estado y la distribución de la población del vector.

  \end{itemize}
\end{frame}


%Materiales y metodos
\section{Modelo propuesto}
Los métodos de muestreo, como larvitrampas y ovitrampas resultan eficientes y económicos para
determinar determinar la distribución espacial y temporal de Aedes aegypti y otros mosquitos
\cite{dengueUruguayCap1, cenaprece2013}. La distribución geográfica de larvitrampas, consideradas
como puntos de control, permiten generar información regionalizada sobre el estado de las
poblaciones del vector \cite{NINO2011}, en donde esta información puede ser combinada con
información ambiental, demográfica o epidemiológica, con el fin de obtener modelos detallados que
tengan la capacidad de monitorear, simular el comportamiento del vector y en consecuencia,
predecir una posible epidemia del dengue.

El modelo considera un espacio bi-dimensional, con un sistema de coordenadas geográficas $(x,y)$,
para expresar todas las posiciones sobre el plano, correspondientes a la longitud y latitud. Si
consideramos a $m_{i}$ como a un individuo que se encuentra en una etapa del ciclo de vida del
Aedes aegypti, correspondiente a una población de mosquitos, entonces, $m_{i}(x,y)$ representa a
$m_{i}$ en las coordenadas geográficas $(x,y)$.

La evolución de las poblaciones, se ven afectadas por los siguientes eventos: muerte de huevos,
eclosión de huevos, muerte de larvas, emergencia de pupas, muerte de pupas, emergencia de adultos,
muerte de adultos, ovipostura de hembras nulíparas\footnote{Hembras que no han ovipuesto.},
ovipostura de hembras paridas\footnote{Hembras que han ovipuesto al menos una vez.} y dispersión
de los adultos (machos y hembras). Según \cite{otero2006stochastic} los eventos se producen a
tasas que dependen no sólo de valores de la población, sino también de la temperatura, que a su
vez es una función de tiempo, por lo tanto, la dependencia de la temperatura introduce una
dependencia del tiempo en las tasas de eventos.

\subsection{Tasas de desarrollo}
En el modelo se cuenta con 4 tasas de desarrollos correspondientes a : la eclosión de huevos,
emergencia a pupas, emergencia a adultos y el ciclo gonotrófico. Estos valores son obtenidos
mediante el modelo no lineal de Sharpe y DeMichele, presentado en \cite{sharpe1977reaction}, para
procesos poiquilotermos\footnote{La poiquilotermia o ectotermia es un término aplicado a ciertos
animales con temperatura corporal variable}, donde el proceso de maduración es controlado por
una enzima que actúa en un rango de temperatura determinado, la enzima se desactiva a las bajas
temperaturas, y altas. Schoolfield presentó, en \cite{schoolfield1981non}, un versión simplificada
del modelo de Sharpe y DeMichele con inhibición de altas temperaturas, con una única alta
temperatura de desactivación.

\begin{equation} \label{eq:schoolfield}
   R(k)  = R(298K) *\cfrac{ \cfrac{k}{298K} *
    exp \Bigg[
            \cfrac{\Delta H_{A}}{R} \bigg(\cfrac{1}{298K} - \cfrac{1}{k}\bigg)
        \Bigg]}
    {1 + exp\Bigg[\cfrac{\Delta H_{H}}{R} \bigg(\cfrac{1}{T_{1/2}}- \cfrac{1}{k}\bigg)\Bigg] }
\end{equation}

Donde $R(k)$ representa la tasa de desarrollo media ($dias^{-1}$) para una temperatura $K$,en la
escala de Kelvin; $T_{1/2}$ es la temperatura cuando la mitad de la enzima se desactiva, debido a
la alta temperatura, mientras que $H_{A}$, $H_{H}$ y $H_{L}$ son entalpías termodinámicas
características del organismo, y $R$, igual $1,987202$ $cal/K.mol$, es la constante universal de
los gases. Los parámetros $R(298K)$, $H_{A}$, $T_{1/2}$, y $H_{H}$ son estimados mediante la
de regresión no lineal de Wagner, presentado en \cite{wagner1984modeling}. Según
\cite{otero2006stochastic}, el modelo simplificado de Schoolfield, es lo suficientemente
flexible para el ajuste de los datos biológicos disponibles. Los parámetros deben calcularse para
cada etapa de desarrollo, una vez determinados, la ecuación puede utilizarse para calcular tasas
de desarrollo a cualquier temperatura \cite{rueda1990temperature}.

\subsection{Zonificación}
\label{subsec:cap4-zonificacion}
Cada entorno puede contar con factores que lo hagan más o menos apto para el desarrollo,
mortalidad, alimentación, dispersión, y reproducción de individuos. Con el fin de
simplificar ciertos aspectos muy específicos que se encuentran fuera del alcance de este trabajo,
se realizan ciertas hipótesis generales, justificadas para este caso de aplicación, pero puede
requerir una revisión en caso general. Estas hipótesis son, los valores observados en un
conjunto de puntos de control, pertenecientes a una zona, permiten la caracterización de dicha
zona como más o menos apta para desarrollo, mortalidad, alimentación, dispersión, y reproducción de
individuos. También consideramos que el tamaño de la zona, y por ende la cantidad de puntos de
control que pertenecen a ella, influye en la caracterización de las zonas.

Para determinar el tipo de zona de un individuo $m_{i}$ ubicado en $(x,y)$, primero se estima la
densidad relativa de larvas, para $m_{i}(x,y)$, utilizando interpolación espacial y posteriormente
se la clasifica utilizando una escala. Si consideramos a $u(x,y)$ el valor interpolado para
$m_{i}(x,y)$, entonces la densidad relativa de larvas de $m_{i}(x,y)$ es igual a $u(x,y)$.

La forma general de encontrar un valor interpolado $u$ en un punto $(x,y)$ basado en un conjunto de
muestras $u_i = u (x_i)$ para $i = 0,1, ..., N$ utilizando IDW, es una función de interpolación:

\begin{equation}\label{eq:interpolacion-idw}
 u(x,y) = \sum_{i=1}^{N} w_i(X) * u_{i}
\end{equation}

Donde :
\begin{equation}
w_i(X) =  \dfrac{d(X, X_i)^{-p}}{\sum_{j=1}^{N} d(X, X_i)^{-p}}
\end{equation}

Siendo $d(X, X_i)$ una función que determina la distancia existente entre $X$ y $X_{i}$, donde $p$
es el exponente de ponderación.  El valor estimado de $u(x,y)$ es utilizado para clasificar la
zona como \textit{Pésima}, \textit{Mala}, \textit{Regular}, \textit{Buena} u \textit{Óptima} con
influencia positiva el desarrollo, alimentación, dispersión, y reproducción de individuos y
negativamente para la mortalidad.

\begin{table}[!hptb]
\begin{threeparttable}
    \begin{minipage}[b]{0.5\textwidth}
    \caption{\label{tab:cap4-puntaje-zona} Escala de clasificación de las zonas de acuerdo a la densidad relativa de larvas.}
    \footnotesize
    \begin{tabular}{l c c c c}
        \hline \\
                     & Mínimo\tnote{a} & Máximo\tnote{a} & Hembras     & Hembras \\
        Tipo de zona & $u(x,y)$   & $u(x,y)$   & Adultas\tnote{b} & Reproductivas \tnote{c}\\
        \hline
        \hline\\
        Pésima  & 0  & 19 & 8  & 5 \\
        Mala    & 20 & 35 & 15 & 10\\
        Regular & 36 & 51 & 22 & 15\\
        Buena   & 52 & 69 & 30 & 20\\
        Óptima  & 70 & --\tnote{d} & --\tnote{d} & --\tnote{d}\\
        \hline
    \end{tabular}
    \begin{tablenotes}[flushleft]\footnotesize
    \item[a]{Rango mínimo y máximo de $u(x,y)$ permitido para el tipo de zona.}
    \item[b]{Cantidad máxima de hembras adultas, al final del periodo de desarrollo.}
    \item[c]{Cantidad de hembras adultas con capacidad de oviponer.}
    \item[d]{No se estableció un límite superior para las zonas óptimas. }
    \end{tablenotes}
    \end{minipage}
    \end{threeparttable}
\end{table}

En la \tabref{tab:cap4-puntaje-zona} se pueden observar los rangos definidos para cada tipo de
zona, en donde $u$ es la densidad relativa de larvas en las coordenadas $(x,y)$. Los límites para
las zonas fueron determinados clasificando los valores, de las hembras reproductivas en, grupos
múltiplos de cinco. No se estableció un límite superior para las zonas óptimas debido a que los
valores mayores a el mínimo establecido, 70 larvas por dispositivo, pertenecen a la misma categoría

\subsection{Mortalidad}
\label{subsec:cap4-mortalidad}
La mortalidad de los individuos depende de la etapa del ciclo de desarrollo en el que se encuentren
los individuos de una población.

\subsubsection{Mortalidad de huevos}
La tasa de mortalidad de los huevos se encuentra definida como una constante, $me = 0.01$,
$1/\text{días}$, independiente de la temperatura \cite{otero2006stochastic}.

\begin{equation}
    M_{H(x,y)} = me * H(x,y)
\end{equation}

Donde $M_{H(x,y)}$ es la cantidad de huevos que deben ser eliminados de la población $H(x,y)$.

\subsubsection{Mortalidad de larvas}
La mortalidad de las larvas, según \cite{otero2006stochastic}, se encuentra dividida en dos
contribuciones. La primera contribución representa la mortalidad natural bajo óptimas condiciones
y se encuentra influenciada únicamente de la temperatura \cite{otero2006stochastic}. Esta tasa se
encuentra definida por :

\begin{equation}
\label{eq:mortalidad-natural-larvas}
    ml(k) = 0.01 + 0.9725 * exp\bigg( \frac{-(k - 278)}{2.7035}\bigg)
\end{equation}

La segunda contribución es la mortalidad denso dependiente de las larvas \cite{otero2006stochastic}
. Este mecanismo de regulación puede estar relacionado con procesos concurrentes, como las
limitaciones de los alimentos, las interacciones químicas, presencia de depredadores
especializados en el sitio de reproducción y mucho más \cite{otero2006stochastic}. Esta se
encuentra definida por :

\begin{equation}
  \alpha (x,y) = \alpha _{0}/BS(x,y)
\end{equation}

Donde $\alpha _{0}$ está asociado a la capacidad de carga de un solo lugar de reproducción y
$BS(x,y)$ es el número de sitios de reproducción en $(x,y)$. El valor de $\alpha _{0}$ puede
ajustado a los valores observados en la región que se está simulando. Se considera que el valor de
$BS(x,y)$ se encuentra influenciado por $u(x,y)$, de ese modo a medida que $u(x,y)$ varíe, lo debe
hacer el valor de $BS(x,y)$. Para el cálculo de $BS$ relativo a $(x,y)$ se utiliza el método
interpolador de Lagrange.

\begin{equation}
\label{eq:sitios-reproduccion-x-y}
    bs(u(x,y)) = \sum_{i=0}^{n} bs_{i} * l_{i}(u(x,y))
\end{equation}

donde $l_j(u(x,y))$ son los llamados polinomios de Lagrange, que se calculan de este modo:

\begin{equation}
\label{eq:sitios-reproduccion-x-y}
    l_{i}(u(x,y)) = \prod_{j \neq i} \cfrac{u(x,y) - u_{j}}{u_{i} - u_{j}}
\end{equation}

Consideramos un polinomio de tercer grado, con los parámetros $u_0$, $u_1$ y $u_2$ igual a $19$,
$51$ y $70$, correspondientes a zonas del tipo \textit{Pésima}, \textit{Regular} y \textit{Óptima}.
Los valores, $bs_{min}$, $bs_{med}$ y $bs_{max}$ respectivamente, estos son parámetros
configurables del modelo donde $bs_{min}$ representa el menor $BS$ observado, $bs_{max}$
representa el mayor $BS$ observado y $bs_{med}$ es el valor medio existente entre $bs_{max}$ y
$bs_{min}$.

Tomando ambas contribuciones, la mortalidad natural bajo óptimas condiciones y la denso
dependiente, la mortalidad de las larvas queda definida como :
\begin{equation}
    M_{L(x,y)}(k) = ml(k) * L(x,y) + \alpha (x,y) * L(x,y) * (L(x,y) - 1)
\end{equation}

Donde $M_{L(x,y)}$ es la cantidad de larvas que deben ser eliminadas de la población $L(x,y)$.

\subsubsection{Mortalidad de las pupas}
La tasa de mortalidad de las pupas se encuentra definida como una función influenciada únicamente
de la temperatura \cite{otero2006stochastic}.

\begin{equation}
\label{eq:mortalidad-natural-pupas}
    mp(k) = 0.01 + 0.9725 * exp\bigg( \frac{-(k - 278)}{2.7035}\bigg)
\end{equation}

Además de la mortalidad diaria en la fase de pupa, existe una importante mortalidad adicional
asociada con la emergencia sin éxito de adultos, solo el 83 \%  de las pupas alcanzan la maduración
y emergerán como mosquitos adultos, por lo tanto, el factor de supervivencia es de $ef = 0.83$
\cite{otero2006stochastic}.

\begin{equation}
    M_{P(x,y)}(k) = P(x,y) * (mp + (1 - ef) * R(k))
\end{equation}

Donde $M_{P(x,y)}$ es la cantidad de pupas que deben ser eliminadas de la población $P(x,y)$.

\subsubsection{Mortalidad de adultos}
La tasa de mortalidad de los adultos se encuentra definida como una constante, $ma = 0.09$,
$1/\text{días}$, independiente de la temperatura \cite{otero2006stochastic}.

\begin{equation}
    M_{A(x,y)} = ma * A(x,y)
\end{equation}

Donde $M_{A(x,y)}$ es la cantidad de adultos que deben ser eliminados de la población $A(x,y)$.

\subsection{Madurez y cambio de estado}
La madurez de los individuos pertenecientes a las poblaciones de $H(x,y)$, $L(x,y)$ y $P(x,y)$
indica la proximidad de que estos alcancen el siguiente estado de su etapa de desarrollo. Sea
$R(k_{i})$ la tasa de desarrollo de un individuo para una temperatura de $k_{i}$ Kelvin en un
instante $i$, se considera que ha alcanzado su máximo nivel de madurez y se encuentra listo para
pasar al siguiente estado de su cuando $\sum_{i=0}^{N} R(k_{i}) \geq 1$ , en donde $N$ es la
cantidad de días que le toma al individuo pasar de un estado a otro. Para $R(k)$ se se aplican los
parámetros correspondientes al estado actual del individuo.

\subsection{Ciclo gonotrófico y Ovipostura}
El ciclo gonotrófico de los mosquitos es el nombre que se le adjudicó al período que existe desde
que el mosquito realiza una alimentación sanguínea - ovipostura - hasta una nueva alimentación.
Como se mencionó anteriormente, la tasa de desarrollo del ciclo gonotrófico puede estimarse
mediante la versión simplificada del modelo de Sharpe y DeMichele \cite{sharpe1977reaction},
propuesta por Schoolfield en \cite{schoolfield1981non}.

Sea $R(k_{i})$ la tasa de desarrollo del ciclo gonotrófico de una hembra (nulípara o parida), para
una temperatura de $k_{i}$ Kelvin en un instante $i$, se considera que un día es de ovipostura si
se cumple $\sum_{i=0}^{N} R(k_{i}) \geq 1$, en donde $N$ es la duración en días del ciclo
gonotrófico de la hembra adulta. Para $R(k)$ se aplican los parámetros correspondientes a hembras
nulíparas y paridas de acuerdo al estado de la hembra. En \cite{edman1987host} se observó que
hembras nulíparas de Aedes aegypti poseen un proceso de digestión más lento en las hembras paridas
y por ende el ciclo gonotrófico de las mismas tiende a ser más largo. La cantidad de huevos en
cada oviposición, luego de las alimentaciones sanguíneas correspondientes, varía entre 30 y 100
unidades \cite{luevano1993ciclo, beltran2001bionomia,cabezas2005dengue}.

\subsection{Vuelo y dispersión}
El Aedes aegypti es un mosquito doméstico que generalmente esta confinado a las casas donde se
cría \cite{luevano1993ciclo}, tiende a permanecer físicamente en donde emergió, siempre y cuando
no exista algún factor que la perturbe o no disponga de huéspedes, sitios de reposo y de postura
\cite{ThironIzcazaJ2003}. Por lo general, el mosquito, no sobrepasa los 50 a 100 metros durante su
vida \cite{cabezas2005dengue}. En caso de no contar con sitios adecuados de ovipostura y
disponibilidad de alimento tienden a dispersarme una mayor distancia, hasta tres kilómetros, en
busca de mejores condiciones \cite{ThironIzcazaJ2003}. Los mosquitos tienen la particularidad de
volar en sentido contrario a la dirección al viento \cite{ThironIzcazaJ2003,web-site:speedAnimals}
y a una velocidad máxima de 2 kilómetros por hora \cite{web-site:speedAnimals,kaufmann2004flight}.

Partiendo de las hipótesis realizadas en la podemos considerar que la dispersión se encuentra
influenciada por el valor de $u(x,y)$, de ese modo a medida que $u(x,y)$ varíe, la dispersión debe
ajustarse a su tipo de zona. De forma simplificada definimos que la dispersión de un adulto que se
encuentre en zonas del tipo \textit{Regular}, \textit{Buena} u \textit{Óptima} se encuentra entre
0 y 100 metros de vuelo. Para las hembras adultas, que pertenezcan a zonas del tipo \textit{Mala}
o \textit{Pésima} se tiene una dispersión entre 100 a 3.000 metros de vuelo, de este modo, las
hembras adultas que se encuentren en zonas menos aptas tenderán a desplazarse en busca de mejores
condiciones.

\subsection{Simulación del proceso evolutivo}
La simulación del proceso evolutivo del mosquito del Aedes aegypti, es el encargado de simular los
efectos temperatura en el ciclo de vida del mosquito. Estos efectos pueden desencadenar en una
serie de eventos en el individuo como : eclosión de huevos, mortalidad de larvas, emergencia de
pupas, muerte de pupas, emergencia de adultos, muerte de adultos, ovipostura y dispersión de los
adultos.

La población inicial es obtenida mediante la cantidad de larvas observadas en los puntos de
control que corresponden a la muestra utilizada para el estudio. Por cada larva observada, en un
punto de control ubicado en las coordenadas geográficas, $(x, y)$, se inicializa un individuo con
las mismas coordenadas del punto de control de origen.

El simulador del proceso evolutivo, que es considerado como un proceso iterativo, el proceso
inicia tomando como parámetros de entrada la población inicial, y el periodo de simulación,
representado por $T$. El proceso se ejecutará siempre y cuando se cumplan las siguientes
condiciones : el periodo de simulación, $T$, no haya finalizado y que la población cuente con
individuos para su procesamiento. En el caso de que no se cumplan algunas de las condiciones
mencionadas anteriormente, el proceso de simulación finalizará.

El desarrollo de los individuos se encarga de calcular las tasas de desarrollo correspondientes
para cada etapa de su ciclo de vida , con el fin de estimar su desarrollo considerando las
condiciones climáticas. El cambio de estado es consecuencia de la finalización de la etapa de
desarrollo del individuo, donde el individuo ya está listo para pasar a la siguiente etapa de su
ciclo de desarrollo.

La regulación de la población es la encargada de calcular las tasas de mortalidad diaria,
correspondientes a cada etapa del ciclo de desarrollo del individuo, con el fin de reducir el
tamaño de la población debido a la mortalidad diaria de los individuos.

Si el individuo en cuestión corresponde a una hembra adulta inseminada, entonces esta se encuentra
en fase reproductiva. La postura de huevos se realiza respetando la tasa de desarrollo del ciclo
gonotrófico de las hembras . Si la hembra adulta ovipone, los huevos son añadidos a la población
como individuos en un estado inicial de \textit{HUEVO}.

%Resultados y discusión
\section{Descripción general del entorno de pruebas}
En esta sección se presentan las parámetros adoptados para la configuración del entorno de
pruebas, que se encuentra dividido en : las características de la población inicial, periodo
simulación y los datos climatológicos, y por ultimo los parámetros de simulador del proceso
evolutivo.

Para las pruebas se generaron aleatoriamente 25 puntos de control que fueron distribuidos
geográficamente de forma aleatoria y no uniforme en un área de total de $3,028 km^{2}$
(Ver \figref{fig:distribucion-puntos}). En los 25 puntos de control fueron distribuidos un total de
1.146 individuos en un estado inicial de larvas.

\begin{figure}[!htpb]
    \centering
    \includegraphics[width=0.45\textwidth]{../book/capitulo-6/graphics/extension-poblacion.png}
    \caption{\label{fig:distribucion-puntos}Distribución geográfica de los 25 puntos de control.}
\end{figure}

El periodo de simulación utilizado fue igual a 50 días con 10 temperaturas constantes :
15\textcelsius , 18\textcelsius , 20\textcelsius , 22\textcelsius , 24\textcelsius , 25\textcelsius
, 26\textcelsius , 27\textcelsius , 30\textcelsius , 34\textcelsius. La dirección al igual que la
temperatura fue establecida como una constante para las pruebas, cuyo valor fue la dirección
suroeste, que genera un ángulo que varía entre $202,5^{\circ}$ a $247,5^{\circ}$.

Los parámetros del simulador del proceso evolutivo, en su mayoría son calculados con datos
biológicos correspondientes al área de estudio. Obtener dichos datos requieren minuciosos estudios
de campo que escapan del alcance de este trabajo.

En cuanto a los sitios de reproducción, los parámetros $bs_{min}$ y $bs_{max}$ fueron
configurados según lo observado en \cite{otero2006stochastic, otero2008stochastic}, siendo $15$ y
$50$ los valores adoptados respectivamente.  El valor de $bs_{med}$ fue establecido, en $32,5$,
realizando un promedio entre $bs_{min}$ y $bs_{max}$.

Los coeficientes para el modelo simplificado de Sharpe y DeMichele, con inhibición de altas
temperaturas de Schoolfield, para el calculo de las tasas de desarrollo media en $dias^{-1}$,
fueron tomados de :  \cite{rueda1990temperature} para el desarrollo larvario y el desarrollo
pupal, y de  \cite{otero2006stochastic} para la eclosión de huevos, ciclo gonotrófico para hembras
nulíperas y paridas.

\section{Resultados y discusión}
En esta sección se presentan los resultados experimentales obtenidos mediante la simulación del
proceso evolutivo de ciclo de vida del Aedes aegypti. En general, para todos los casos, la
población inicial sufre un decrecimiento causada por la mortalidad diaria de los individuos a
temperaturas entre 15 y 34 \textcelsius, y por la emergencia de adultos a temperaturas entre 18 y
34 \textcelsius. La aparición de adultos implica que la población de individuos en etapas
inmaduras (huevos, larvas y pupas), llegaron a completar su ciclo de desarrollo para dar lugar a
mosquitos adultos, por lo tanto la población de individuos en etapas inmaduras tiende a disminuir
y mientras que la población de mosquitos adultos tiende a aumentar. El crecimiento de la población
se debe a que las hembras adultas pertenecientes a la población de mosquitos, culminaron su ciclo
gonotrófico y dieron lugar la ovipostura.

\begin{figure}[!htpb]
    \centering
    \begin{subfigure}[b]{0.45\textwidth}
        \includegraphics[width=\textwidth]{./graphics/evolucion-poblacion-all.png}
        \caption{ Población de mosquitos en etapas inmaduras.}
    \end{subfigure}
    ~~~~
    \begin{subfigure}[b]{0.45\textwidth}
        \includegraphics[width=\textwidth]{./graphics/evolucion-poblacion-adultos.png}
        \caption{ Población mosquitos adultos.}
    \end{subfigure}

\caption{\label{fig:poblacion-all}Análisis del comportamiento de la población de mosquitos en relación al tiempo a 10 temperaturas constantes (15-34 \textcelsius)}
\end{figure}

En la \figref{fig:poblacion-all} se pueden apreciar el crecimiento y decrecimiento de la población
a diferentes temperaturas, en donde se pudo observar que a medida que la temperatura aumenta, las
tasas de desarrollo son menores, motivo por el cual las poblaciones de individuos en etapas
inmaduras, sometidos a temperaturas más elevadas, tienden a disminuir su tamaño rápidamente debido
a que se desarrollan con mayor rapidez, dando lugar a su etapa de adulto. Del mismo modo el ciclo
gonotrófico, para las hembras adultas tienden a disminuir su duración, causando que el intervalo
entre oviposturas disminuya, en consecuencia el tamaño de la población de individuos en etapas
inmaduras aumenta rápidamente.

\begin{figure}[!htpb]
    \centering
    \begin{subfigure}[b]{0.225\textwidth}
        \includegraphics[width=\textwidth]{../book/capitulo-6/graphics/raster/temp-24-0.png}
        \caption{\label{fig:poblacion-mapas-a}Población población inicial.}
    \end{subfigure}
    ~~~~
    \begin{subfigure}[b]{0.225\textwidth}
        \includegraphics[width=\textwidth]{./graphics/temp-20-final.png}
        \caption{ Población final a 20 \textcelsius.}
    \end{subfigure}

    \begin{subfigure}[b]{0.225\textwidth}
        \includegraphics[width=\textwidth]{./graphics/temp-24-final.png}
        \caption{ Población final a 24 \textcelsius.}
    \end{subfigure}
    ~~~~
    \begin{subfigure}[b]{0.225\textwidth}
        \includegraphics[width=\textwidth]{../book/capitulo-6/graphics/raster/temp-27-49.png}
        \caption{ Población final a 27 \textcelsius.}
    \end{subfigure}

    \begin{subfigure}[b]{0.225\textwidth}
        \includegraphics[width=\textwidth]{../book/capitulo-6/graphics/raster/temp-30-35.png}
        \caption{ Población final a 30 \textcelsius.}
    \end{subfigure}
    ~~~~
    \begin{subfigure}[b]{0.225\textwidth}
        \includegraphics[width=\textwidth]{../book/capitulo-6/graphics/raster/temp-34-42.png}
        \caption{ Población final a 34 \textcelsius.}
    \end{subfigure}
\caption{\label{fig:poblacion-mapas-all} Mapas de interpolación de la población de mosquitos y la distribución de las hembras adultas (puntos en azul).}
\end{figure}

En la \figref{fig:poblacion-mapas-all} se puede apreciar los mapas de interpolación de las
poblaciones en un su estado inicial y final del periodo de simulación. El estado inicial
(\figref{fig:poblacion-mapas-a}) es el mismo para todas las temperaturas. Comparando el estado
inicial y los estados finales se puede observar que existe una dispersión de los focos en
dirección al noreste debido a que la dirección del viento utilizada, para las pruebas, era del
suroeste. Este patrón de dispersión se podrá observar en todas las poblaciones que fueron
sometidas a temperaturas que permitan la generación de hembras adultas. La dispersión de los focos
de infestación es consecuencia de la dispersión y ovipostura de las hembras adultas emergentes de
la población.

\subsection{Análisis de la población a temperaturas variables}
Los resultados presentados anteriormente corresponden, a pruebas realizadas a 10 temperaturas
constantes(15-34 \textcelsius) durante un periodo de simulación igual a 50 días. Las variaciones
en la temperatura repercuten en la duración del ciclo de vida del vector, disminuyendo y
aumentando su duración para aquellas temperaturas que resulten más o menos favorables. A medida
que las temperaturas resulten menos favorables, el desarrollo de los individuos se tornará más
lento y a aumentará la mortalidad de los individuos.

Con la finalidad de analizar el efecto de múltiples temperaturas en un mismo periodo de
simulación, la población inicial es sometida a un periodo simulación de 90 días con temperaturas
variables. En la \figref{fig:var-temperatura} se puede observar la variación de la temperatura
correspondiente al periodo de simulación de 90 días. La temperatura promedio observada es de 22,56
\textcelsius, con una mínima de 14,5 \textcelsius y una máxima de 33,5 \textcelsius.

\begin{figure}[!htpb]
    \centering
    \includegraphics[width=0.45\textwidth]{../book/capitulo-6/graphics/temperatura-variable-90.png}
    \caption{\label{fig:var-temperatura}Variación de la temperatura durante un periodo de 90 días.}
\end{figure}

En general se pudo observar una duración de 4,07 días para su fase de huevo, 10,13 días para la
fase larval y 2,74 días para fase pupal. La duración del ciclo gonotrófico correspondiente a
hembras nulíperas y paridas fue de 4,75 y 3,43 días respectivamente.

En la \figref{fig:temp-var-poblacion} se puede observar el comportamiento creciente y decreciente
de la población de mosquitos en relación al tiempo. El decrecimiento de la población de individuos
en etapas inmaduras se encuentra influenciado por la mortalidad diaria y emergencia a adultos,
donde esta última causa el crecimiento de la población de adultos. La oviposición de las hembras
adultas, pertenecientes a la población de adultos, es la causante del crecimiento de la población
de individuos en etapas inmaduras. El decrecimiento de la población de adultos es causada por la
mortalidad diaria.

\begin{figure}[!htbp]
    \centering
    \includegraphics[width=0.45\textwidth]{../book/capitulo-6/graphics/temp-var-90-poblacion.png}
    \caption{\label{fig:temp-var-poblacion} Análisis del comportamiento de la población de mosquitos durante un periodo de 90 días a temperaturas variables.}
\end{figure}

En la \figref{fig:temp-var-generacion} se puede observar comportamiento correspondiente de las
generaciones de mosquitos pertenecientes a la población. En general se pudieron observar 4
generaciones de individuos en etapas inmaduras y 3 generaciones de adultos. La primera generación
corresponde a los individuos pertenecientes a la población inicial, la segunda generación
corresponde a los descendientes la primera generación, producto de la oviposición de las hembras
adultas emergentes de la población inicial. La tercera generación corresponde a los descendientes
de la segunda generación. La cuarta y última generación se encuentra compuesta únicamente por
individuos en etapas inmaduras, ya que ninguno de sus individuos emergió para convertirse en adulto.

\begin{figure}[!htbp]
    \centering
    \begin{subfigure}[b]{0.45\textwidth}
        \includegraphics[width=\textwidth]{../book/capitulo-6/graphics/temp-var-90-generacion-inmaduras.png}
        \caption{\label{fig:temp-var-inmaduras-generacion}Crecimiento generacional de la población de individuos en etapas inmaduras a temperatura variable.}
    \end{subfigure}
    ~~~~
    \begin{subfigure}[b]{0.45\textwidth}
        \includegraphics[width=\textwidth]{../book/capitulo-6/graphics/temp-var-90-generacion-adultos.png}
        \caption{\label{fig:temp-var-adultos-generacion}Crecimiento generacional de la población de adultos a temperatura variable.}
    \end{subfigure}

    \caption{\label{fig:temp-var-generacion} Análisis generacional de la población de mosquitos durante un periodo de 90 días a temperatura variable.}
\end{figure}

%conclusión final
%----------------------1----------------------------------------
\begin{frame}[t]{Conclusiones}
    \begin{itemize}
        \item El modelo y la herramienta resultante, son aplicables en cualquier región o área de estudio.

        \item Los mapas de interpolación permiten apreciar los niveles de infestación y el riesgo correspondiente a la abundancia de mosquitos.

        \item La instalación y recolección de puntos de control requiere trabajo de campo.

        \item La configuración del simulador del proceso evolutivo requiere de datos ecológicos.

        \item Los resultados obtenidos, utilizando configuraciones del material bibliográfico, son considerados una aproximación válida.

    \end{itemize}
\end{frame}

\begin{frame}[t]{Conclusiones}
    \begin{itemize}
        \item Existen diferencias entre los resultados obtenidos y los observados por expertos en laboratorios.

        \item Las variaciones pueden deberse a los rasgos característicos de las cepas de mosquito.

        \item Se pueden obtener mejores resultados ajustando las configuraciones con los datos ecológicos del Paraguay.

        \item El potencial analítico de la información generada es infinito.
    \end{itemize}
\end{frame}


\printbibliography

% that's all folks
\end{document}


