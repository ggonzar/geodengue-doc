
\subsection{Vuelo y dispersión}
El Aedes aegypti es un mosquito doméstico, generalmente esta confinado a las casas donde se
cría \cite{luevano1993ciclo}, tiende a permanecer físicamente en donde emergió, siempre y cuando
no exista algún factor que la perturbe o no disponga de huéspedes, sitios de reposo y de postura
\cite{ThironIzcazaJ2003}. Por lo general, el mosquito, no sobrepasa los 50 a 100 metros durante su
vida \cite{cabezas2005dengue}. En caso de no contar con sitios adecuados de ovipostura y
disponibilidad de alimento tienden a dispersarme una mayor distancia, hasta 3 kilómetros, en
busca de mejores condiciones \cite{ThironIzcazaJ2003}. Los mosquitos tienen la particularidad de
volar en sentido contrario a la dirección al viento \cite{ThironIzcazaJ2003,web-site:speedAnimals}
y a una velocidad máxima de 2 kilómetros por hora \cite{web-site:speedAnimals,kaufmann2004flight}.

Partiendo de las hipótesis realizadas anteriormente, podemos considerar que la dispersión se
encuentra influenciada por el valor de $u(x,y)$, de ese modo a medida que $u(x,y)$ varíe, la
dispersión debe ajustarse al tipo de zona. De forma simplificada se define que la dispersión de un
mosquito adulto que se encuentre en zonas del tipo \textit{Regular}, \textit{Buena} u
\textit{Óptima} se encuentra entre 0 y 100 metros de vuelo. Para las hembras adultas, que
pertenezcan a zonas del tipo \textit{Mala} o \textit{Pésima} se tiene una dispersión entre
100 a 3.000 metros de vuelo, de este modo, las hembras adultas que se encuentren en zonas menos
aptas tenderán a desplazarse en busca de mejores condiciones.
