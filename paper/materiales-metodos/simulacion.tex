
\section{Simulación del proceso evolutivo}
La simulación del proceso evolutivo es el encargado de simular los efectos de las variaciones
climáticas y del entorno en el ciclo de vida del Aedes aegypti.

El Algoritmo \ref{alg:simulador-evolutivo}, describe al simulador como un proceso iterativo cuyo
objetivo es simular los efectos de cada $k_{i}$ para cada individuo $m_{j}$ que pertenezca a la
población.

La población de estudio, $\vec{m}$, se encuentra compuesta por un conjunto $m_{j}(x, y)$ que
representan a los individuos, $h_{j}(x, y)$, $l_{j}(x, y)$, $p_{j}(x, y)$, $am_{j}(x, y)$,
$an_{j}(x, y)$ y $ap_{j}(x, y)$ que corresponden a las poblaciones de $H(x,y)$, $L(x,y)$, $P(x,y)$,
$AM(x, y)$, $AN(x, y)$ y $AP(x, y)$ respectivamente. Todos los individuos,$m_{j}(x, y)$, que se
encuentren en una misma unidad geográfica $(x, y)$ pertenecen a una misma colonia. La población
inicial es obtenida mediante la cantidad de larvas observadas en los puntos de control que
corresponden a la muestra utilizada para el estudio. Por cada larva observada, en un punto de
control ubicado en las coordenadas geográficas, $(x, y)$, se inicializa un individuo, $m_{i}$ con
las mismas coordenadas, $(x, y)$, del punto de control de origen.

El periodo de estudio, $\vec{k}$, se encuentra conformado por un conjunto de temperaturas $k_{i}$
que representan las temperaturas diarias.

\begin{algorithm}
\caption{Simulación del proceso evolutivo \label{alg:simulador-evolutivo}}
\begin{algorithmic}[1]
    \Require $\vec{k}\neq \emptyset \land \vec{m} \neq \emptyset$
    \Ensure $\vec{m'}$
    \ForAll{$k{i} \in \vec{k}$ }
        \State $\vec{huevos} \Leftarrow \emptyset$
        \ForAll{$m_{j}(x, y) \in \vec{m}$}
            \State $desarrollar(m_{j}(x, y), k_{i})$
            \If{$regular(m_{j}(x, y), k_{i})$}
                \State \Comment{Se elimina $m_{j}$ si es un candidato.}
                \State $m_{j}(x, y) \Leftarrow \varnothing $
            \ElsIf{$esta\_maduro(m_{j}(x, y), k_{i})$}
                \State $ cambiar\_estado(m_{j}(x, y)) $
            \ElsIf{$se\_reproduce(m_{j}(x, y), k_{i})$}
                \State $\vec{huevos} \Leftarrow \vec{huevos} + oviponer(m_{j}(x, y))$
            \EndIf
        \EndFor

        \If{$\vec{huevos} \neq \emptyset$}
            \State \Comment{Si ovipone se extiende la población}
            \State $\vec{m} \Leftarrow  \vec{m} + \vec{huevos}$
        \EndIf
    \EndFor \\
    \Return $\vec{m}$
\end{algorithmic}
\end{algorithm}


El proceso $desarrollar(m_{j}, k_{i})$, es el proceso por el cual se calculan las tasas de
desarrollo correspondientes para cada $m_{j}$ . Estas son obtenidas mediante el modelo de
\cite{sharpe1977reaction} presentado en la \secref{subsec:cap4-tasas de desarrollo}.

El desarrollo en las etapas inmaduras ($H(x, y)$, $L(x, y)$ y $P(x, y)$) es realizado con el fin de
simular los efectos de $k_{i}$ en la maduración de $m_{j}$. Sin embargo para los adultos ($AN(x, y)$
y $AP(x, y)$), es realizado con el fin de simular los efectos de $k_{i}$ en la duración del ciclo
gonotrófico de $m_{j}$ y la dispersión del vector, presentada en la
\secref{subsec:cap4-vuelo-dispersion}.

El proceso $regular(m_{j}, k_{i})$ como principal objetivo, determinar la cantidad de individuos
que deben ser eliminados de la población, debido a la mortalidad diaria y seleccionar los
candidatos para dicha eliminación. Como se mencionó anteriormente, en la
\secref{subsec:cap4-mortalidad}, la tasa de mortalidad diaria depende del estado en el que
encuentre el individuo. La selección de candidatos para la eliminación se realiza de forma
aleatoria para los individuos que provenientes de una misma colonia $(x, y)$.

El proceso $esta\_maduro(m_{j}, k_{i})$ se encarga de validar que se cumpla el individuo finalice
su etapa de desarrollo. Sea $R(k_{i})$ la tasa de desarrollo de un individuo para una temperatura
de $k_{i}$ Kelvin en un instante $i$, se considera que ha alcanzado su máximo nivel de madurez y
se encuentra listo para pasar al siguiente estado de su etapa de desarrollo cuando se cumple :

\begin{equation}
\label{eq:condicion-cambio-estado}
    \sum_{i=0}^{N} R(k_{i}) \geq 1
\end{equation}

Donde $N$ es la cantidad de días que le toma al individuo pasar de un estado a otro. Una vez
completo su desarrollo se procede a realizar el cambio de estado mediante el proceso
$cambiar\_estado(m_{j})$. El cambio de estado consiste en modificar las características un
individuo, $m_{i}(x, y)$ para que inicie la nueva etapa de su ciclo de desarrollo. De forma
extendida tenemos:

\begin{itemize}
\item Para que un huevo, $h(x,y)$, perteneciente a $H(x,y)$ pase a ser una larva, $l(x, y)$, que forma parte de $L(x, y)$ debe cumplirse \eqref{eq:condicion-cambio-estado}.

\item Para que una larva, $l(x,y)$, perteneciente a $L(x,y)$ pase a ser una pupa, $p(x, y)$, que forma parte de $P(x, y)$ debe cumplirse \eqref{eq:condicion-cambio-estado}.

\item Para que una pupa, $p(x,y)$, perteneciente a $P(x,y)$ pase a ser un adulto, $a(x, y)$, que forma parte de $A(x, y)$ debe cumplirse \eqref{eq:condicion-cambio-estado}.
\end{itemize}


El proceso $se\_reproduce(m_{j}, k_{i})$, que solo es válido para hembras adultas se encarga de
validar $m_{j}$ haya completado su ciclo gonotrófico, una vez finalizado debe generar una tanda de
huevos mediante $oviponer(m_{j})$. La duración del ciclo gonotrófico y la oviposutra fue presentado
anteriormente en la \secref{subsec:cap4-ciclo-gontrofico-ovipostura}.

