\subsection{Tasas de desarrollo}
Se consideran 5 tasas de desarrollo correspondiente a : la eclosión de huevos,
emergencia a pupas, emergencia a adultos, el ciclo gonotrófico de hembras nulíperas y el ciclo gonotrófico de hembras paridas. Las tasas de desarrollo son calculadas mediante la versión simplificada del modelo de Sharpe y DeMichele, presentado en \cite{sharpe1977reaction},
con inhibición de altas temperaturas de Schoolfield.

\begin{equation} \label{eq:schoolfield}
   R(k)  = R(298K) *\cfrac{ \cfrac{k}{298K} *
    exp \Bigg[
            \cfrac{\Delta H_{A}}{R} \bigg(\cfrac{1}{298K} - \cfrac{1}{k}\bigg)
        \Bigg]}
    {1 + exp\Bigg[\cfrac{\Delta H_{H}}{R} \bigg(\cfrac{1}{T_{1/2}}- \cfrac{1}{k}\bigg)\Bigg] }
\end{equation}

Donde $R(k)$ representa la tasa de desarrollo media ($dias^{-1}$) para una temperatura $K$,en la
escala de Kelvin; $T_{1/2}$ es la temperatura cuando la mitad de la enzima se desactiva, debido a
la alta temperatura, mientras que $\Delta H_{A}$ y $\Delta H_{H}$  son entalpías termodinámicas características del organismo, y $R$ es la constante universal de los gases, igual a
$1,987202$ $cal/K.mol$. Los parámetros $R(298K)$, $\Delta H_{A}$, $T_{1/2}$, y $\Delta H_{H}$ son estimados mediante la de regresión no lineal de Wagner, presentado en \cite{wagner1984modeling}.
Según \cite{otero2006stochastic}, el modelo simplificado de Schoolfield, es lo suficientemente
flexible para el ajuste de los datos biológicos disponibles. Los parámetros deben calcularse para
cada etapa de desarrollo, una vez determinados, la ecuación puede utilizarse para calcular tasas
de desarrollo a cualquier temperatura \cite{rueda1990temperature}.

