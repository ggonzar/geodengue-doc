\subsection{Ciclo gonotrófico y Ovipostura}
El ciclo gonotrófico de los mosquitos es el período de tiempo que existe, entre una alimentación
sanguínea y la ovipostura. En \cite{edman1987host} se observó que hembras nulíparas de Aedes
aegypti poseen un proceso de digestión más lento en las hembras paridas y por ende el ciclo
gonotrófico de las mismas tiende a ser más largo. Como se mencionó anteriormente las tasa de
desarrollo del ciclo gonotrófico para hembras núiperas y paridas puden estimarse mediante la
versión simplificada del modelo de Sharpe y DeMichele.

La cantidad de huevos correspondiente a cada oviposición varía entre 30 y 100 unidades, según lo
observado en \cite{luevano1993ciclo, beltran2001bionomia, cabezas2005dengue}.
