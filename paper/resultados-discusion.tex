\section{Descripción general del entorno de pruebas}
En esta sección se presentan las parámetros adoptados para la configuración del entorno de
pruebas, que se encuentra dividido en : las características de la población inicial, periodo
simulación y los datos climatológicos, y por ultimo los parámetros de simulador del proceso
evolutivo.

Para las pruebas se generaron aleatoriamente 25 puntos de control que fueron distribuidos
geográficamente de forma aleatoria y no uniforme en un área de total de $3,028 km^{2}$
(Ver \figref{fig:distribucion-puntos}). En los 25 puntos de control fueron distribuidos un total de
1.146 individuos en un estado inicial de larvas.

\begin{figure}[!htpb]
    \centering
    \includegraphics[width=0.45\textwidth]{../book/capitulo-6/graphics/extension-poblacion.png}
    \caption{\label{fig:distribucion-puntos}Distribución geográfica de los 25 puntos de control.}
\end{figure}

El periodo de simulación utilizado fue igual a 50 días con 10 temperaturas constantes :
15\textcelsius , 18\textcelsius , 20\textcelsius , 22\textcelsius , 24\textcelsius , 25\textcelsius
, 26\textcelsius , 27\textcelsius , 30\textcelsius , 34\textcelsius. La dirección al igual que la
temperatura fue establecida como una constante para las pruebas, cuyo valor fue la dirección
suroeste, que genera un ángulo que varía entre $202,5^{\circ}$ a $247,5^{\circ}$.

Los parámetros del simulador del proceso evolutivo, en su mayoría son calculados con datos
biológicos correspondientes al área de estudio. Obtener dichos datos requieren minuciosos estudios
de campo que escapan del alcance de este trabajo.

En cuanto a los sitios de reproducción, los parámetros $bs_{min}$ y $bs_{max}$ fueron
configurados según lo observado en \cite{otero2006stochastic, otero2008stochastic}, siendo $15$ y
$50$ los valores adoptados respectivamente.  El valor de $bs_{med}$ fue establecido, en $32,5$,
realizando un promedio entre $bs_{min}$ y $bs_{max}$.

Los coeficientes para el modelo simplificado de Sharpe y DeMichele, con inhibición de altas
temperaturas de Schoolfield, para el calculo de las tasas de desarrollo media en $dias^{-1}$,
fueron tomados de :  \cite{rueda1990temperature} para el desarrollo larvario y el desarrollo
pupal, y de  \cite{otero2006stochastic} para la eclosión de huevos, ciclo gonotrófico para hembras
nulíperas y paridas.

\section{Resultados y discusión}
En esta sección se presentan los resultados experimentales obtenidos mediante la simulación del
proceso evolutivo de ciclo de vida del Aedes aegypti. En general, para todos los casos, la
población inicial sufre un decrecimiento causada por la mortalidad diaria de los individuos a
temperaturas entre 15 y 34 \textcelsius, y por la emergencia de adultos a temperaturas entre 18 y
34 \textcelsius. La aparición de adultos implica que la población de individuos en etapas
inmaduras (huevos, larvas y pupas), llegaron a completar su ciclo de desarrollo para dar lugar a
mosquitos adultos, por lo tanto la población de individuos en etapas inmaduras tiende a disminuir
y mientras que la población de mosquitos adultos tiende a aumentar. El crecimiento de la población
se debe a que las hembras adultas pertenecientes a la población de mosquitos, culminaron su ciclo
gonotrófico y dieron lugar la ovipostura.

\begin{figure}[!htpb]
    \centering
    \begin{subfigure}[b]{0.45\textwidth}
        \includegraphics[width=\textwidth]{./graphics/evolucion-poblacion-all.png}
        \caption{ Población de mosquitos en etapas inmaduras.}
    \end{subfigure}
    ~~~~
    \begin{subfigure}[b]{0.45\textwidth}
        \includegraphics[width=\textwidth]{./graphics/evolucion-poblacion-adultos.png}
        \caption{ Población mosquitos adultos.}
    \end{subfigure}

\caption{\label{fig:poblacion-all}Análisis del comportamiento de la población de mosquitos en relación al tiempo a 10 temperaturas constantes (15-34 \textcelsius)}
\end{figure}

En la \figref{fig:poblacion-all} se pueden apreciar el crecimiento y decrecimiento de la población
a diferentes temperaturas, en donde se pudo observar que a medida que la temperatura aumenta, las
tasas de desarrollo son menores, motivo por el cual las poblaciones de individuos en etapas
inmaduras, sometidos a temperaturas más elevadas, tienden a disminuir su tamaño rápidamente debido
a que se desarrollan con mayor rapidez, dando lugar a su etapa de adulto. Del mismo modo el ciclo
gonotrófico, para las hembras adultas tienden a disminuir su duración, causando que el intervalo
entre oviposturas disminuya, en consecuencia el tamaño de la población de individuos en etapas
inmaduras aumenta rápidamente.

\begin{figure}[!htpb]
    \centering
    \begin{subfigure}[b]{0.225\textwidth}
        \includegraphics[width=\textwidth]{../book/capitulo-6/graphics/raster/temp-24-0.png}
        \caption{\label{fig:poblacion-mapas-a}Población población inicial.}
    \end{subfigure}
    ~~~~
    \begin{subfigure}[b]{0.225\textwidth}
        \includegraphics[width=\textwidth]{./graphics/temp-20-final.png}
        \caption{ Población final a 20 \textcelsius.}
    \end{subfigure}

    \begin{subfigure}[b]{0.225\textwidth}
        \includegraphics[width=\textwidth]{./graphics/temp-24-final.png}
        \caption{ Población final a 24 \textcelsius.}
    \end{subfigure}
    ~~~~
    \begin{subfigure}[b]{0.225\textwidth}
        \includegraphics[width=\textwidth]{../book/capitulo-6/graphics/raster/temp-27-49.png}
        \caption{ Población final a 27 \textcelsius.}
    \end{subfigure}

    \begin{subfigure}[b]{0.225\textwidth}
        \includegraphics[width=\textwidth]{../book/capitulo-6/graphics/raster/temp-30-35.png}
        \caption{ Población final a 30 \textcelsius.}
    \end{subfigure}
    ~~~~
    \begin{subfigure}[b]{0.225\textwidth}
        \includegraphics[width=\textwidth]{../book/capitulo-6/graphics/raster/temp-34-42.png}
        \caption{ Población final a 34 \textcelsius.}
    \end{subfigure}
\caption{\label{fig:poblacion-mapas-all} Mapas de interpolación de la población de mosquitos y la distribución de las hembras adultas (puntos en azul).}
\end{figure}

En la \figref{fig:poblacion-mapas-all} se puede apreciar los mapas de interpolación de las
poblaciones en un su estado inicial y final del periodo de simulación. El estado inicial
(\figref{fig:poblacion-mapas-a}) es el mismo para todas las temperaturas. Comparando el estado
inicial y los estados finales se puede observar que existe una dispersión de los focos en
dirección al noreste debido a que la dirección del viento utilizada, para las pruebas, era del
suroeste. Este patrón de dispersión se podrá observar en todas las poblaciones que fueron
sometidas a temperaturas que permitan la generación de hembras adultas. La dispersión de los focos
de infestación es consecuencia de la dispersión y ovipostura de las hembras adultas emergentes de
la población.

\subsection{Análisis de la población a temperaturas variables}
Los resultados presentados anteriormente corresponden, a pruebas realizadas a 10 temperaturas
constantes(15-34 \textcelsius) durante un periodo de simulación igual a 50 días. Las variaciones
en la temperatura repercuten en la duración del ciclo de vida del vector, disminuyendo y
aumentando su duración para aquellas temperaturas que resulten más o menos favorables. A medida
que las temperaturas resulten menos favorables, el desarrollo de los individuos se tornará más
lento y a aumentará la mortalidad de los individuos.

Con la finalidad de analizar el efecto de múltiples temperaturas en un mismo periodo de
simulación, la población inicial es sometida a un periodo simulación de 90 días con temperaturas
variables. En la \figref{fig:var-temperatura} se puede observar la variación de la temperatura
correspondiente al periodo de simulación de 90 días. La temperatura promedio observada es de 22,56
\textcelsius, con una mínima de 14,5 \textcelsius y una máxima de 33,5 \textcelsius.

\begin{figure}[!htpb]
    \centering
    \includegraphics[width=0.45\textwidth]{../book/capitulo-6/graphics/temperatura-variable-90.png}
    \caption{\label{fig:var-temperatura}Variación de la temperatura durante un periodo de 90 días.}
\end{figure}

En general se pudo observar una duración de 4,07 días para su fase de huevo, 10,13 días para la
fase larval y 2,74 días para fase pupal. La duración del ciclo gonotrófico correspondiente a
hembras nulíperas y paridas fue de 4,75 y 3,43 días respectivamente.

En la \figref{fig:temp-var-poblacion} se puede observar el comportamiento creciente y decreciente
de la población de mosquitos en relación al tiempo. El decrecimiento de la población de individuos
en etapas inmaduras se encuentra influenciado por la mortalidad diaria y emergencia a adultos,
donde esta última causa el crecimiento de la población de adultos. La oviposición de las hembras
adultas, pertenecientes a la población de adultos, es la causante del crecimiento de la población
de individuos en etapas inmaduras. El decrecimiento de la población de adultos es causada por la
mortalidad diaria.

\begin{figure}[!htbp]
    \centering
    \includegraphics[width=0.45\textwidth]{../book/capitulo-6/graphics/temp-var-90-poblacion.png}
    \caption{\label{fig:temp-var-poblacion} Análisis del comportamiento de la población de mosquitos durante un periodo de 90 días a temperaturas variables.}
\end{figure}

En la \figref{fig:temp-var-generacion} se puede observar comportamiento correspondiente de las
generaciones de mosquitos pertenecientes a la población. En general se pudieron observar 4
generaciones de individuos en etapas inmaduras y 3 generaciones de adultos. La primera generación
corresponde a los individuos pertenecientes a la población inicial, la segunda generación
corresponde a los descendientes la primera generación, producto de la oviposición de las hembras
adultas emergentes de la población inicial. La tercera generación corresponde a los descendientes
de la segunda generación. La cuarta y última generación se encuentra compuesta únicamente por
individuos en etapas inmaduras, ya que ninguno de sus individuos emergió para convertirse en adulto.

\begin{figure}[!htbp]
    \centering
    \begin{subfigure}[b]{0.45\textwidth}
        \includegraphics[width=\textwidth]{../book/capitulo-6/graphics/temp-var-90-generacion-inmaduras.png}
        \caption{\label{fig:temp-var-inmaduras-generacion}Crecimiento generacional de la población de individuos en etapas inmaduras a temperatura variable.}
    \end{subfigure}
    ~~~~
    \begin{subfigure}[b]{0.45\textwidth}
        \includegraphics[width=\textwidth]{../book/capitulo-6/graphics/temp-var-90-generacion-adultos.png}
        \caption{\label{fig:temp-var-adultos-generacion}Crecimiento generacional de la población de adultos a temperatura variable.}
    \end{subfigure}

    \caption{\label{fig:temp-var-generacion} Análisis generacional de la población de mosquitos durante un periodo de 90 días a temperatura variable.}
\end{figure}
