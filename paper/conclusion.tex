\section{Conclusión}
El diseño e implementación del modelo como un simulador del proceso evolutivo del vector en el
contexto de un  sistema de información geográfica, permite realizar análisis complejos de la
realidad espacial rápidamente, generando información regionalizada para determinar los niveles de
infestación correspondientes al área de estudio. Contar con esta información regionalizada,
asociada con niveles de infestación permitirá a las autoridades sanitarias definir y planificar
las medidas de control, prevención y logística a realizar con con el fin de disminuir los niveles
de infestación en una región.

El simulador de proceso evolutivo se encuentra compuesto por modelos, ampliamente respaldados por
por el material bibliográfico en \cite{sharpe1977reaction, focks1993dynamic, schoolfield1981non, otero2006stochastic, rueda1990temperature}, utilizados para el cálculo de las tasas de desarrollo
y mortalidad de las distintas etapas de desarrollo del ciclo de vida del vector. Se considera al
modelo resultante como genérico, debido a que sus parámetros pueden ser ajustados para aplicarlos
en cualquier región o área de estudio.

La configuración del simulador del proceso evolutivo requiere de parámetros asociados con las
características biológicas y datos ecológicos correspondientes al área de estudio, por lo que para
su aplicación, estos parámetros de configuración deben ser validados por expertos en el área
mediante trabajos de campo. No obstante, utilizando valores tomados del material bibliográfico de
apoyo, se pudo observar un buen comportamiento de los resultados obtenidos mediante el simulador
del proceso evolutivo, solo presentan pequeñas variaciones en comparación con los valores
observados por expertos en laboratorio en en condiciones controladas. Las variaciones observadas
pueden ser causadas por los distintos rasgos característicos de las cepas de mosquitos, utilizadas
en los estudios de referencia, que permiten una mayor o menor tolerancia a ciertas condiciones.
