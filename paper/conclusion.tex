\section{Conclusión}
%(1) Diseñar un modelo que permita analizar la extensión del vector del dengue y estudiar su posible relación con un potencial foco de riesgo, de forma a realizar una predicción de posibles focos de riesgo.
En este trabajo presentó el diseño e implementación de un modelo predictivo para identificar focos
de infestación del, Aedes aegypti principal vector del dengue, sustentado en métodos de muestreo
para la determinación de la abundancia poblacional, modelos matemáticos para simular el ciclo de
vida del vector en un sistema de información geográfica, con el fin de apoyar a la lucha
preventiva de esta enfermedad. Para el diseño y desarrollo del simulador del proceso evolutivo, se
realizaron ciertas consideraciones que requieren la validación y revisión por parte de expertos en
el área, de forma que se puedan realizar los ajustes correspondientes para su aplicabilidad.El diseño e implementación del modelo como un simulador del proceso evolutivo del vector en el contexto de un sistema de información geográfica, permite realizaranálisis complejos de la realidad espacial rápidamente, generando información regionalizada para determinar los niveles de infestación correspondientes al área de estudio. Se considera al modelo resultante como genérico, debido a que sus parámetros pueden ser ajustados para aplicarlos en cualquier región o área de estudio, y extensible, teniendo en cuenta que puede ser modificado para incluir nuevas variables y procesos.

%(3)Diseñar el modelo de forma paramétrica y escalable, para que sea aplicable y extensible a cualquier región o área de estudio.
El simulador de proceso evolutivo se encuentra compuesto por modelos, ampliamente respaldados por
el material bibliográfico, en \cite{sharpe1977reaction, focks1993dynamic, schoolfield1981non, otero2006stochastic, rueda1990temperature}, que son utilizados para el cálculo de las tasas de
desarrollo y mortalidad de las distintas etapas de desarrollo del ciclo de vida del vector. La
configuración del simulador del proceso evolutivo requiere de parámetros asociados con las
características biológicas y datos ecológicos correspondientes al área de estudio, por lo que para
su aplicación, estos parámetros de configuración deben ser validados por expertos en el área
mediante trabajos de campo. No obstante, utilizando valores tomados del material bibliográfico de
apoyo, se pudo observar un buen comportamiento de los resultados obtenidos mediante el simulador
del proceso evolutivo. Solo presentan pequeñas variaciones en comparación con los valores
observados por expertos en laboratorio en condiciones controladas. Las variaciones observadas
pueden ser causadas por los distintos rasgos característicos de las cepas de mosquitos, utilizadas
en los estudios de referencia, que permiten una mayor o menor tolerancia a ciertas condiciones.

%(4)Generar información relevante que pueda ayudar a las autoridades pertinentes para toma de decisiones en la lucha contra el dengue.
En un futuro, con los ajustes y validaciones correspondientes a ser realizadas por expertos en el
área, la información generada por el simulador del proceso evolutivo del Aedes aegypti, asociada
con niveles de infestación y los mapas de interpolación podrían, permitir a las autoridades
sanitarias, del Paraguay, definir y planificar, de forma más efectiva, las medidas de control,
prevención y logística a realizar con con el fin de disminuir los niveles de infestación en una
región.

