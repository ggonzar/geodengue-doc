\documentclass[final,fmstyle]{./util/ucathesis}
% La opcion 'final' muestra los graficos, para generar una version sin los graficos utiliza la opcion 'draft'

% paquetes recomendados
%\usepackage[chapter]{theorems}
%\usepackage{symbols}
%\usepackage{url}
\usepackage{amsmath,amsthm}

\usepackage[T1]{fontenc}
\usepackage[spanish]{babel}
\usepackage[utf8]{inputenc}
\usepackage{csquotes}
\usepackage[style=numeric,sorting=none,backend=biber]{biblatex}
\usepackage{listings}
\addbibresource{referencias.bib}


% custom commands
\newcommand{\foreign}[1]{{\it #1}}
\DeclareMathOperator*{\argmax}{arg\,max}
\algsetup{indent=2em}

% \setcounter{tocdepth}{3}

% datos de la tesis
\title{Modelo predictivo de focos de dengue aplicado a Sistemas de informaci\'on geogr\'afica}
\author{Maximiliano B\'{a}ez Gonz\'{a}lez y Roberto Ba\~{n}uelos}
\degree{Inform\'{a}tica}

\advisor{Prof. MSc.}{Guillermo Gonz\'{a}lez}

%\newtheorem{definicion}{Definicin}
%\numberwithin{algorithm}{chapter}

\logosource{./graphics/logo.jpg}
\institution{Universidad Nacional de Asunci\'{o}n}
\faculty{Facultad Polit\'{e}cnica}
\address{San Lorenzo - Paraguay}

\begin{document}
\lstset{
	language=java,
	basicstyle=\small\sffamily,
	numbers=left,
	numberstyle=\tiny,
	frame=tb,
	columns=fullflexible,
	showstringspaces=false
}
\maketitle     % esto hace las portadas

% Agradecimientos
%\newpage

\chapter*{\centering Agradecimientos}

A mis padres por haberme brindado la oportunidad de estudiar una carrera Universitaria, por su esfuerzo y entera confianza.

A mis hermanos Alejandra, Mabel, Marcos y Matías gracias la paciencia, por acompañarme siempre en los buenos y malos momentos, por brindarme su apoyo incondicionalmente.

A mi hermano Marcos, gracias por todo el apoyo, la orientación, por iluminar mi camino y darme una pauta para poder realizarme en mis estudios y en la vida, gracias por siempre estar cerca a pesar de la distancia.

A mi tutor Prof. MSc. Guillermo González por encaminar y acompañar el desarrollo este trabajo hasta su culminación, gracias profesor por haber compartido sus conocimientos conmigo y su gran ayuda para lograr esta meta tan importante.

A Lindsay, gracias por siempre estar a mi lado, por acompañarme incondicionalmente, por todos los consejos y por ayudarme a enfrentar y superar los momentos difíciles.

A mis jefes, Ing. Joaquin Lima y Ing. Juan Talavera por brindarme tiempo y espacio para el desarrollo de este proyecto siempre que lo necesité.

A mis amigos y compañeros de clases, por permitirme compartir toda esta etapa de formación académica, humana y profesional, gracias por todos los buenos momentos dentro y fuera de las aulas.




% los siguientes comandos producen 'indices.

% Tabla de contenidos
\tableofcontents
% Lista de figuras
\listoffigures
% Lista de tablas
\listoftables
% Lista de algoritmos
\listofalgorithms
%\include{acronimos}
%\newpage
\chapter*{Lista de Símbolos\hfill}
\addcontentsline{toc}{chapter}{Lista de Símbolos}
\begin{tabbing}
% YOU NEED TO ADD THE FIRST ONE MANUALLY TO ADJUST THE TABBING AND SPACES
$n$~~~~~~~~~~\=\parbox{5in}{Vector size\dotfill \pageref{symbol:nml}}\\
%ADD THE REST OF SYMBOLS WITH THE HELP OF MACRO

%% se añaden nuevos simbolos con el macro \newsymbol y se hace referecnia
% al simbolo utilizando \addsymbol{symbol:LABEL}

%\newsymbol (x_i, y_i): {coordenadas que representan un punto}{symbol:xy_i}

\end{tabbing}


\mainmatter  % inician los capitulos de la tesis

% incluye aqui los capitulos (un archivo .tex por capitulo)

La introducci\'on es lo ultimo que se escribe

%!TEX root = ../tesis.tex
\chapter{Antecedentes Hist\'{o}ricos}
\label{sec:antecedentes}

hablar de antecendetes historicos

%!TEX root = ../tesis.tex
\chapter{Tecnolog\'ias y Herramientas}
\label{sec:tecnologias}



%figura

% introduccion

%~ aplicaciones

%!TEX root = ../tesis.tex
\chapter{Definici\'on del Problema}
\label{sec:problema}

deficion general del problema

% introduccion

%~ problema general
%~ problema especifico
%~ problema motivacion

%!TEX root = ../tesis.tex
\chapter{Soluci\'on Propuesta}
\label{sec:solucion}


% introduccion
soluci\'on propuesta...




\appendix   % inician los apendices de tu tesis

% los cap'itulos que incluyas a partir de aqu'i aparecen
% como ap'endices

% estos comandos generan la bilbiograf'ia
\printbibliography

\end{document}
