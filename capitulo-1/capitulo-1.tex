\chapter{Introducción}
En este capítulo se presentan los antecedentes y la importancia de este trabajo, así como las
propuestas que se realizan con el fin de apoyar la lucha preventiva contra el dengue. Conceptos de
la ecología del vector, metodo de prevención y sistemas de información geográficos son puntos
viltales de este trabajo por lo que son introducidos brevemente en este capítulo. No obstante, en
capítulos siguientes, se tratarán todos estos temas serán desglozados detalladamente.

\section{Justificación y Antecedentes}

El Dengue, una enfermedad viral transmitida por el mosquito Aedes aegypti, es uno de los graves
problemas de salud pública a nivel mundial y una de las de más rápida propagación en el mundo
. En los últimos 50 años, su incidencia ha aumentado 30 veces con la creciente expansión
geográfica hacia nuevos países y, en la actual década, de áreas urbanas a rurales.

Los sistemas de información geográfica (SIG ) constituyen un área de rápido desarrollo dentro
de la informática y ofrecen métodos sumamente innovadores para hacer frente a algunas demandas
técnicas que constituyen un reto. Un sistema de información geográfica es una integración
organizada de hardware, software, datos geográficos y personal, diseñado para capturar, almacenar,
manipular, analizar y visualizar en todas sus formas la información geográficamente referenciada
con el fin de resolver problemas complejos de planificación y gestión \cite{lopezMarcos2007}.

Las autoridades sanitarias, en sus tareas de vigilancia en Salud Pública, tienen en los SIG una
herramienta fundamental para conocer cómo se extiende una enfermedad, estudiar su posible relación
con un potencial foco de riesgo, o localizar un brote epidémico \cite{vgomesAegis2001}.

Con la aparición de herramientas geográfica-sanitarias como AEGIS presentada en
\cite{vgomesAegis2001}, y \cite{NINO2011}, exponen el gran potencial de los SIG en el área de
análisis epidemiológicos.

Actualmente el dengue es catalogado como ejemplo de una enfermedad que puede constituir una
emergencia de salud pública de interés internacional con implicaciones para la seguridad
sanitaria, debido a la necesidad de interrumpir la infección y la rápida propagación de la
epidemia más allá de las fronteras nacionales.

Debido a una lucha no efectiva en contra el dengue, en  el Paraguay, cada día se reportan más
nuevos casos de infectados por el dengue. La detección de focos de infección en base a casos
reportados y el combate correctivo realizado actualmente, no resultan efectivos ante la lucha
contra el dengue. Se podrían obtener mejores resultados, realizando un cambio de enfoque, en el combate  contra el dengue, de uno correctivo a uno preventivo.

\section{Propuesta de esta tesis y Objetivos}
En este trabajo se propone la construcción de una herramienta que permita realizar estudios
epidemiológicos de forma cartográfica, especializada para el particular caso del dengue. Se
pretende constituir un modelo que permita predecir los focos de riesgo del dengue, con la ayuda de
un sistema de información geográfica. Para determinar las posibles zonas de riesgo, se debe
construir un algoritmo que permita simular el comportamiento de un mosquito o de un conjunto de
mosquitos de acuerdo a las variables de la región y tener en cuenta sus patrones migratorios.

El objetivo principal es diseñar y construir una herramienta que permita analizar la extensión del
dengue y estudiar su posible relación con un potencial foco de riesgo, de forma a realizar una
predicción de posibles focos de dengue, que ayude a realizar una lucha preventiva contra la
enfermedad.

Proveer una herramienta que permita georeferenciar los casos posibles y confirmados de dengue, e identificar zonas de riesgo.
Identificar los patrones migratorios del mosquito del dengue.
Ofrecer un modelo aplicable en distintas regiones que cuente con el mismo problema.
Apoyar la lucha contra el dengue de forma preventiva.

\section{Organización del trabajo}
El trabajo está organizado como sigue: el capítulo 2 introduce a los sitemas de inforamción
geográfica, la representación de datos geoespaciales y los metodos de interpolación como
herramienta para análisis espacial. En el capítulo 3 se presenta al aedes aegypti el principal
transmisor del dengue, sus caracteristicas biológicas y los metodos de muestreo de la abundacia
pobacional del vector. Estos capítulos representan el estado del arte de este trabajo.

El capítulo 4 se presenta el modelo matemático propuesto para la identificación de focos de
infestación y la simulación del proceso evolutivo del ciclo de vida del vector. En el capítulo 5
se presenta la implementación computacional denominada GeoDengue, sus requerimientos, diseño y
arquitectura y las herramientas y tecnologias utilizadas para su implementacion.

En el capítulo 6 se presenta los resultados experimentales obtenidos. Finalmente el capítulo 7
presenta las conclusiones de este trabajo y los  posibles trabajos futuros derivados de esta tesis.
