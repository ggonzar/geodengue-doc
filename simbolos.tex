\newpage
\chapter*{Lista de Simbolos\hfill}
\addcontentsline{toc}{chapter}{Lista de Simbolos}
\begin{tabbing}
% YOU NEED TO ADD THE FIRST ONE MANUALLY TO ADJUST THE TABBING AND SPACES
$n$~~~~~~~~~~\=\parbox{5in}{Vector size\dotfill \pageref{symbol:nml}}\\
%ADD THE REST OF SYMBOLS WITH THE HELP OF MACRO

%% se añaden nuevos simbolos con el macro \newsymbol y se hace referecnia
% al simbolo utilizando \addsymbol{symbol:LABEL}

\newsymbol (x_i, y_i): {coordenadas que representan un punto}{symbol:xy_i}
\newsymbol d_{i} : {El i-esimo punto de control}{symbol:d_i}
\newsymbol z_{i} : {Cantidad de individuos de $d_{i}$.}{symbol:z_i}
\newsymbol D: {conjunto de muestras de estudio}{symbol:D}
\newsymbol I(D, d'_j) : {Método interpolador}{symbol:Id_i}

\newsymbol T : {Periodo de tiempo}{symbol:T}
\newsymbol t_{i} : {datos climáticos en el instante i}{symbol:ti}
\newsymbol P : {Población}{symbol:P}
\newsymbol p_i : {El i-esimo individuo de la población}{symbol:Ii}

%atributos de p_i
\newsymbol \eta(p_i): {Madurez del individuo $p_{i}$}{symbol:ma-pj}
\newsymbol \xi(p_i): {Expectativa de vida de $p_{j}$}{symbol:ex-pj}
\newsymbol \tau(p_i): {Estado de vida de $p_{j}$}{symbol:estado-pj}
\newsymbol S(p_i): {Sexo de $p_{j}$}{symbol:sexo-pj}

%operadores básicos
\newsymbol \eta(t_i, p_i): {función de madurez}{symbol:madurez}
\newsymbol \xi(t_i, p_i): {función de expectativa de vida}{symbol:espectativa-vida}
\newsymbol \theta (t_{i}, p_{j}): {función de reducción}{symbol:esta-muerto}

\newsymbol N: {tamaño del periodo}{symbol:N}
\newsymbol M: {tamaño de la población}{symbol:M}

\newsymbol \mathbf{\omega}: {matriz de madurez}{symbol:mat-madurez}
\newsymbol \mathbf{\upsilon}: {matriz de espectativa de vida}{symbol:mat-espectativa}

% ALWAYS KEEP THE FOLLOWING LINE
\end{tabbing}
