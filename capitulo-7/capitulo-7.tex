\chapter{Conclusiones y Trabajos Futuros}
Este trabajo presentó la implementación de un modelo para predecir e identificar focos de riesgo
del vector del dengue, con el fin de apoyar a la lucha preventiva de esta enfermedad. Donde el
modelo se sustenta en metodos de muestreo para la determinación de la abundacia poblacional. El
modelado e implementación de la herramienta como un sistema de información geográfica permite
realizar análisis complejos de la realidad espacial rápidamente.

El modelo cuenta con una buena báse matemática de respado gracias a la utilización de los modelos
\citet{sharpe1977reaction}, \citet{schoolfield1981non} y \citet{otero2006stochastic} para el
calculo de las tasas de desarrollo y mortalidad de las distintas etapas de desarrollo del ciclo de
vida del vector. Se considera que el modelo resultante es genérico, donde los parámetos pueden ser
ajustados para que sean aplicables en cualquier lugar.

En general se pudo observar un buen comportamiento, los resultados obtenidos solo presentan
pequeñas variaciones en comparación con los observados por expertos en laboratorio en en
condiciones controladas. Estas variaciones pueden deberse a los rasgos caracterisitcos de las
distintas cepas, utilizadas en los estudios de referencia, que permiten una mayor o menor
tolerancia a ciertas condiciones.

Como trabajo futuro podría considerarse la extensión del modelo para incluir otras variables
con el fin de analizar su relación y el impacto de las mismas con las zonas de riesgo. Se podrían
incluir variables como: casos reportados de dengue, posibles zonas de riesgo(cementerios, patios
baldios, etc)y la densidad poblacional del área de estudio.

Un interesante trabajo futuro sería incluir el impacto de las lluvias en la generación de sitios
de reproducción, asi como el efecto de la fumigación en el desarrollo de ciclo de vida del vector.
Se pude aprobechar la información generada, por el simulador del proceso evolutivo, para realizar
una optimizacíon de las rutas de fumigación.

Dado que este trabajo se propone realizar el conteo de lavas mediante el procesamiento digital de
imágnes, sería interesante análizar otros métodos, como el de \cite{gonzalez2008segmentacion},
con el fín de mejorar la confianza y la presición de conteo de larvas.
