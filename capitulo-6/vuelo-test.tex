\section{Vuelo y dispersión}
Para el análisis de la distancia recorrida, en metros, del adulto de Aedes aegypti, se realizaron
pruebas a 9 temperaturas constantes (15-34\textcelsius), solo las hembras que han ovipuesto al
menos una vez fueron incluidas. En la tabla \tabref{tab:pomedio-vuelo-test} se presentan los
resultados obtenidos para la disperción de las hembras adultas del aedes aegypit agrupadas por el
tipo su tipo de zona, en general se obtuvo un promedio de $65,74$, $63,97$ $1308,19$ metros para las zonas buena, normal y malas respectivamente. Existen diversos estudios, que han reportado que
la disperción del aedes aegypti en relación a las caracteristicas de su ambiente. A continuación
se mencionaran algunos estudios realizados, con el fin de realizar una comparación con los resultados obtenidos.

En \cite{cabezas2005dengue} señalan que por lo general mosquito no sobrepasa los 50 a 100 metros
durante su vida, ya que tiende a permanecer en el lugar donde emergió.

Según \cite{ThironIzcazaJ2003} por lo general, la hembra de Ae. aegypti, permanece físicamente en
donde emergió, siempre y cuando no halla algún factor que la perturbe o no disponga de huéspedes,
sitios de reposo y de postura. En caso de no haber recipientes adecuados, la hembra grávida es capaz de volar hasta tres kilómetros en busca de este sitio.

Los autores de \cite{dengueUruguayCap8} señalan que, para las estrategias de control de Aedes
aegypti en zonas urbanas donde existen brotes de dengue y fiebre amarilla se asume que los
mosquitos tienen un rango de vuelo durante su vida de 50 a 100 metros.

En \cite{luevano1993ciclo} se reporta, que el aedes aegypti es un mosquito doméstico que
generalmente esta confinado a las casas donde se cria, tiene un rango de vuelo corto, entre 23 a 50 metros, y raramente se dispersa a largas distancias.

\cite{mcdonald1977population} en Kenia, liberó poblaciones de Aedes aegypti a las distancias de,
200, 400 y 800 metros, y observó que aproximadamente el 50 \% de los mosquitos marcados se
dispersaron a 200 metros del punto en el cual fueron liberados, un 10 \% a 400 metros y solamente
el 1 \% se dispersó a 800 metros.

En \cite{trpis1986dispersal}, los autores reportaron que en Kenia, que la tasa media de dispersión
de las hembras fue de 57 metros. La distancia máxima de las hembras durante 24 horas fue de 154
metros.

La mayoría de los estudios coinciden que, los adultos del aedes aegypti, en condiciones óptimas de
disponibilidad de alimento y sitios adecuados de ovipostura, tienden a permanecer en el lugar
donde emergieron, con una dispersión media estimada entre 50 y a 100 metros, su presencia es
prácticamente un indicio certero de la proximidad de los criaderos. En caso de no contar con
sitios adecuados de ovipostura y disponibilidad de alimento tienden a dispersarme una mayor
distancia en busca de mejores condiciones.

\begin{table}
    \begin{minipage}{\textwidth}
        \caption{ \label{tab:pomedio-vuelo-test} Análisis de la dispersión, por zona, del adulto
        de Aedes aegypti diez temperaturas constantes (15-34 \textcelsius).}
        \begin{tabular}{p{4cm} *{4}{c}  }
          \hline \\
          Temperatura (\textcelsius)& Buena & Normal & Mala & Media Obtenida\\
          \hline
          \hline \\
          18 & 89,06 & 66,61 & 1274,97 & 476,88\\
          20 & 66,52 & 73,53 & 1516,43 & 552,16\\
          22 & 77,89 & 63,78 & 1003,9 & 381,86\\
          24 & 60,28 & 62,31 & 1077,6 & 400,06\\
          25 & 54,52 & 63,66 & 1397,31 & 505,16\\
          26 & 61,61 & 60,79 & 1509,67 & 544,02\\
          27 & 65,24 & 63,72 & 1433,58 & 520,85\\
          30 & 62,61 & 59,56 & 1213,1 & 445,09\\
          34 & 53,92 & 61,75 & 1347,13 & 487,6\\
          Media General & 65,74 & 63,97 & 1308,19 & 479,3\\
        \end{tabular}
    \end{minipage}
\end{table}
