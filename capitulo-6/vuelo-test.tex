\section{Promedio de vuelo}
Para el análisis de la distancia recorrida, en metros, del adulto de Aedes aegypti, se realizaron 
pruebas a 10 temperaturas constantes (15-34\textcelsius). En la tabla \ref{tab:pomedio-vuelo-test} se
presentan los resultados obtenidos, en general se obtuvo un promedio de 355,15 metros para distancia
recorrida, y unos 67,821 metros de dispersión, los autores \cite{cabezas2005dengue} señalan que por lo
general mosquito no sobrepasa los 50 a 100 metros durante su vida, ya que tiende a permanecer en el lugar
donde emergió.

\begin{table}
    \begin{center}
    
        \caption{ \label{tab:pomedio-vuelo-test} Análisis de la  la distancia recorrida, durante 
         el vuelo, del adulto de Aedes aegypti diez temperaturas constantes (15-34 \textcelsius).}
    
        \begin{tabular}{p{3cm} c  c }
           \hline \\
            Temperatura \textcelsius   & Distancia recorrida& Desplazamiento máximo\\
            \hline
            \hline \\
               15    &  99,09    &  42,39 \\
               18    &  392,35   &  60,17 \\
               20    &  326      &  85,62 \\
               22    &  417,82   &  77,87 \\
               24    &  367,39   &  74,71 \\
               25    &  380,41   &  67,75 \\
               26    &  427,33   &  66,45 \\
               27    &  362,21   &  65,79 \\
               30    &  389,03   &  68,51 \\
               34    &  389,92   &  68,95 \\
            General  &  355,155  &  67,821 \\
        \end{tabular}

    \end{center}
\end{table}
