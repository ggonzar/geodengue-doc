\section{Tasas de desarrollo y mortalidad}

Como se mencionó anteriormente en la sección \ref{sec:cap-3-cambio-climatico}, las tasas de desarrollo,
supervivencia de las distintas etapas del Aedes Aegypti, son susceptibles a los efectos de la 
temperatura. 

En esta sección se presentan los resultados obtenidos, mediante el proceso el proceso evolutivo, y 
pruebas realizadas para validar el desarrollo y la mortalidad de las distintas etapas del Aedes Aegypti.

La validación de los resultados se realiza mediante la comparación de los resultados obtenidos contra los valores definidos y presentados en distintos trabajos que se utilizaron como referencia. En cada prueba se especifican las condiciones de cada prueba y el error encontrado.

En algunos casos se encontradon diferencias significativa entre los resultados obtenidos en nuestro modelo y los resultados presentados en los trabajos que se utilizaron como referencia debido a diversos factores.
Uno de los principales factores es el tipo de cepa del Aedes Aegypti utilizado, ya que existen estudios que demuestran la diferencia de desarrollo entre las diferentes cepas del mosquito(si bien el trabajo que realizamos no utiliza pruebas de laboratorio los valores utilizados para definir el tipo de mosquito corresponden a una cepa en específica). Otros factores corresponden a variables ambientales y biológicas como la alimentación, humedad, iluminación, contaminación del agua a la cual son sometidos los individuos en los laboratorios.

Por último otro factor importante es el estudio que se realiza del mosquito en su ambiente natural, ya que por ejemplo, un mosquito adulto puede vivir meses en un laboratorio pero sin en un habitat natural puede vivir desde 1 día a pocas semanas. 

\subsection{Desarrollo de la etapa de huevo}
Se realizó un análisis para determinar la tasa de desarrollo, en días, de los huevos de Aedes aegypti, 
de acuerdo a los resultados obtenidos (Tabla \ref{tab:desarrollo-huevo-test}), se obtuvo un promedio
general de 3.6 días para la población que fue sometida a diferentes 5 temperaturas. Los resultados
obtenidos son comparados con los resultados de la duración en días descriptos en \cite{BESERRA2006}.

\begin{table}
    \begin{center}
        \begin{tabular}{p{5cm} c c c c c c }
            Población    &18 \textcelsius & 22 \textcelsius & 26 \textcelsius & 30 \textcelsius & 34 \textcelsius & Media General\\

            \hline \\
            Boqueirão    & 9,3  & 6,5  & 3,9  & 3,3  & 3,5  & 4,3  \\
            B. dos Santos& --   & 5,9  & 4,3  & 3,7  & 2,9  & 4,2  \\
            C. Grande    & --   & 5,5  & 3,4  & 4,4  & 3    & 4,07 \\
            Itaporanga   & 9,2  & 6,2  & 4,7  & 3,1  & 3,5  & 4,38 \\
            Remígio      & --   & 6    & 4,5  & 2    & 2,5  & 3.75 \\
            Esperados    & 6.61 & 5,1  & 3.9  & 3.03 & 2.37 & 3.6  \\
            Resultados   & 6.08 & 5.06 & 3.03 & 3.03 & 2.02 & 3.29 \\
            
        \end{tabular}
        \caption{ \label{tab:desarrollo-huevo-test} Análisis de la tasa de desarrollo, en días, de los
         huevos de Aedes aegypti a cinco temperaturas constantes (18-34 \textcelsius).}
    \end{center}
\end{table}


\subsection{Desarrollo de las etapas inmaduras}
Para el análisis de la tasa de desarrollo, en días, de las etapas inmaduras,larva y pupa, del Aedes
aegypti, se utilizó como referencia los resultados obtenidos por \cite{rueda1990temperature} a 6
temperaturas constantes (15-34\textcelsius).

En la tabla \ref{tab:desarrollo-inmaduras-test} se puede apreciar los resultados obtenidos mediante el
proceso evolutivo para la población de prueba sometida a las 6 temperaturas constantes. Estos datos son
comparados con los resultados obtenidos en \cite{rueda1990temperature}. En general se obtuvo un 
desarrollo de $7,91$ días para las etapas inmaduras, se puede observar una diferencia de $0,22$ en
comparación a los $8,13$ días para la media predicha y $0,18$ para los $8.09$ días de la media observada. Los
autores de \cite{rueda1990temperature} definen como la media predicha, a la tasa de desarrollo obtenida
mediante el modelo de Sharpe \& DeMichele, cabe resaltar que los parámetros para dicho modelo son los
mismos utilizados durante el proceso evolutivo.

\begin{table}
    \begin{minipage}{\textwidth}   
        \caption{ \label{tab:desarrollo-inmaduras-test} Análisis de la tasa de desarrollo del Aedes 
         Aegypti desde eclosión de los huevos a la emergencia de adultos a seis temperaturas constantes 
        (15-34 \textcelsius).}
        
        \begin{tabular}{p{3cm} c  c c c r }
            \hline \\
            Estadio de & Temperatura    & Media$^{a}$& Mediana$^{a}$& Media$^{b}$& Media$^{c}$\\
            desarrollo & \textcelsius   & observada  & observada    & predicha   & obtenida\\
            \hline
            \hline \\
            Larva        & 15           & 46,83 & 45,82 & 42.33 & 43.47\\ 
                         & 20           & 9,31  & 9,06  & 14.89 & 14.29\\ 
                         & 25           & 8,61  & 8,7   & 6.48  & 6.10\\ 
                         & 27           & 4,47  & 4,46  & 5.28  & 5.07\\ 
                         & 30           & 4,99  & 4,86  & 4.72  & 4.05\\ 
                         & 34           & 5,06  & 5,04  & 5.61  & 5.07\\ 
            \\
            Pupa         & 15           & 8.49  & 8.46  & 6.9  & 6.68\\ 
                         & 20           & 3.11  & 3.04  & 4.10 & 4.12\\ 
                         & 25           & 3.03  & 2.98  & 2.5  & 2.05\\ 
                         & 27           & 1.79  & 1.81  & 2.06 & 2.03\\  
                         & 30           & 1.82  & 1.79  & 1.56 & 1.02\\ 
                         & 34           & 1.09  & 1.05  & 1.08 & 1.02\\ 
        \end{tabular}
        \footnotetext[1]{Valores observados por \cite{rueda1990temperature}.}    
        \footnotetext[2]{Resultados obtenidos por el modelo de Sharpe \& DeMichele.}    
        \footnotetext[3]{Resultados obtenidos mediante el proceso evolutivo.}
    \end{minipage}
\end{table}

\subsection{Mortalidad del adulto}
La tasa de la mortalidad fue definida por \cite{otero2006stochastic} como una constante de valor 0.09
independiente de la temperatura. En \cite{ThironIzcazaJ2003} a la supervivencia como constante con
mortalidad diaria de 10\%. En la tabla \ref{tab:mortalidad-adulto-test} se presentan los resultados
obtenidos de la tasa de mortalidad del Aedes aegypti a diez temperaturas constantes (18-34 \textcelsius).

\begin{table}
    \begin{center}
   
        \caption{ \label{tab:mortalidad-adulto-test} Análisis de la tasa de mortalidad del adulto del
         Aedes Aegypti a diez temperaturas constantes (15-34 \textcelsius).}
        
        \begin{tabular}{p{3cm} c c c c c c }
                    \hline \\
                    Temperatura & Adultos Muertos & Total de Adultos & \% Mortalidad\\
                    \hline
                    \hline \\
                    
                    15 \textcelsius & 270    & 2216    & 12,18\\
                    18 \textcelsius & 7099   & 68092   & 10,43\\
                    20 \textcelsius & 11050  & 99858   & 11,07\\
                    22 \textcelsius & 14250  & 117775  & 12,1\\
                    24 \textcelsius & 17708  & 157624  & 11,23\\
                    25 \textcelsius & 19500  & 175071  & 11,14\\
                    26 \textcelsius & 30804  & 268608  & 11,47\\
                    27 \textcelsius & 30957  & 278450  & 11,12\\
                    30 \textcelsius & 221381 & 2033769 & 10,89\\
                    34 \textcelsius & 27716  & 235984  & 11,74\\
                     \hline 
                    Total           & 380735 & 3437447 & 11,08\\

        \end{tabular}
    \end{center}
\end{table}

