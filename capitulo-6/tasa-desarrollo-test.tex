\section{Desarollo y mortalidad}
En esta sección se realizarán pruebas cuyo objetivo principal es validar el desarrollo y la mortalidad de
individuos de la población. Se validará si la duración del ciclo de vida y la mortalidad se adecua a los
promedios definidos por expertos.

\subsection{Duración del ciclo vida}
Se realiza una comparativa entre, la duración del ciclo de vida, obtenida y la definida por
\cite{dengueUruguayCap2}, además se calculará el error en días. Con esta prueba se pretende
validar que la duración de ciclo de vida se encuentre entre los rangos determinados por los 
expertos mediante observación y experimentación en laboratorios (ver \ref{tab:desarrollo-ciclo-test}).

\begin{table}
\begin{tabular}{c c c c c }
Temperatura & Duración del ciclo& Duración promedio & Resultados & Error\\
\hline
20  & 10-47  & 26.83 & 0 & 0   \\
25  & 9-29  & 17.59 & 0 & 0   \\
30  & 5-16  & 9.75 & 0 & 0   \\
\end{tabular}
\caption{ \label{tab:desarrollo-ciclo-test} Comparativa de los resultados obtenidos con los 
promedios de duración del ciclo indicados en \cite{dengueUruguayCap2}}
\end{table}

\subsection{Duración de las etapas inmaduras como porcentaje del tiempo ocupado para llegar a adulto}
En \cite{dengueUruguayCap2} se define la duración promedio de los cuatro estados larvales sucesivos y la
pupa, expresada como porcentaje del tiempo ocupado por la larva hasta que llega a adulto, son 14.6, 13.9,
17.5, 33.3 y 20.6, respectivamente. Durante el proceso evolutivo se maneja a la larva como una entidad
única y no se tiene en cuenta la transición entre sus estadios larvales, por lo que la duración promedio
de los estados larva y la pupa, expresados como porcentaje del tiempo ocupado por la larva a llegar a
adulto quedan resumidos como 79.3 y 20.6 (Ver \ref{tab:desarrollo-ciclo-distribucion-test}).

\begin{table}
\begin{tabular}{c c c c c }
Cantidad de individuos & Duración del ciclo & Larva & Pupa  & Error\\
\hline
-1   & -1 & -1 & -1 & -1   \\
-1   & -1 & -1 & -1 & -1   \\
-1   & -1 & -1 & -1 & -1   \\
\end{tabular}
\caption{ \label{tab:desarrollo-ciclo-distribucion-test} Duración promedio, en días, de los estados larva
y la pupa, expresados como porcentaje del tiempo ocupado por la larva a llegar a adulto}
\end{table}

\subsection{Desarrollo de las etapas inmaduras en distintos contenedores}
Se realiza una comparativa de la duración promedio del desarrollo de las etapas inmaduras de Ae. aegypti 
en distintos contenedores de \cite{manrique1998desarrollo} (Ver 
\ref{tab:desarrollo-ciclo-contenedores-test}).

\begin{table}
\begin{tabular}{p{5cm} c c c c }
Contenedor           & Larva & Pupa & Total \\
\hline
Neumáticos           & 8.1   & 3.05 & 11.15\\
Florero              & 12.79 & 1.28 & 14.07\\
Jarrón               & 14.92 & 2.2  & 17.12\\
Trampa para hormigas & 15.57 & 1.95 & 17.52\\
Laboratorio          & 4.74  & 2    & 6.74\\
Resultados           & -1    & -1   & -1\\
\end{tabular}

\caption{ \label{tab:desarrollo-ciclo-contenedores-test} Duración promedio en días de las  larvas y pupas
de Ae. aegypti en neumáticos comparados con mediciones en el laboratorio y en otros contenedores.}
\end{table}
