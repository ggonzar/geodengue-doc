\section{Desarollo y mortalidad}
En esta sección se realizarán pruebas cuyo objetivo principal es validar el desarrollo y la mortalidad de
individuos de la población. Se validará si la duración del ciclo de vida y la mortalidad se adecua a los
promedios definidos por expertos.

\subsection{Duración del ciclo vida}
Las tasas de desarrollo, supervivencia y en general la duración del ciclo de vida de las distintas
etapas del Aedes Aegypti, son susceptibles a los efectos de la temperatura. 

Se recolectaron los datos correspondientes a la duración del ciclo de vida de las etapas inmaduras, desde
la eclosión de huevos  hasta alcanzar el estado adulto, a partir de estos datos se calcula la media y la
mediana predicha del número de días y se realiza una comparación, a seis temperaturas constantes 
(15-34 \textcelsius), con los resultados obtenidos en  \cite{rueda1990temperature} bajo condiciones de
laboratorio.

\begin{table}
\begin{center}
\begin{tabular}{p{2cm} c c c c c c }
Estadio de & Temperatura    &   & Media     & Mediana  & Media    & Mediana\\
desarrollo & \textcelsius   & n & observada & observada& predicha & predicha\\

\hline
First instar & 15 & 318 & 7.67 (0.14)a  & 7.68  & -1 & -1\\ 
             & 20 & 337 & 2.67 (0.I1)b  & 2.55  & -1 & -1\\ 
             & 25 & 284 & 2.74 (0.12)b  & 2.69  & -1 & -1\\ 
             & 27 & 344 & 0.96 (0.03)c  & 0.98  & -1 & -1\\ 
             & 30 & 284 & 1.18 (1.07)cd & 1.16  & -1 & -1\\ 
             & 34 & 254 & 1.42 (0.03)d  & 1.40  & -1 & -1\\ 
Second instar& 15 & 190 & 8.88 (0.40)a  & 8.19  & -1 & -1\\ 
             & 20 & 301 & 1.43 (0.I1)b  & 1.45  & -1 & -1\\ 
             & 25 & 242 & 1.35 (0.05)b  & 1.27  & -1 & -1\\ 
             & 27 & 291 & 0.77 (0.03)b  & 0.71  & -1 & -1\\ 
             & 30 & 240 & 0.89 (0.05)b  & 0.86  & -1 & -1\\ 
             & 34 & 211 & 1.31 (0.04)b  & 1.31  & -1 & -1\\ 
Thrid  instar& 15 & 93  & 14.97 (0.83)a & 13.92 & -1 & -1\\ 
             & 20 & 262 & 1.62 (0.05)b  & 1.61  & -1 & -1\\ 
             & 25 & 172 & 1.37 (0.06)b  & 1.48  & -1 & -1\\ 
             & 27 & 251 & 0.96 (0.03)b  & 0.99  & -1 & -1\\ 
             & 30 & 201 & 0.98 (0.03)b  & 0.93  & -1 & -1\\ 
             & 34 & 167 & 0.84 (0.04)b  & 0.85  & -1 & -1\\ 
Fourd instar & 15 & 30  & 15.31 (1.22)a & 16.03 & -1 & -1\\ 
             & 20 & 221 & 3.59 (0.07)b  & 3.45  & -1 & -1\\ 
             & 25 & 139 & 3.15 (0.06)bc & 3.26  & -1 & -1\\ 
             & 27 & 207 & 1.78 (0.03)d  & 1.78  & -1 & -1\\ 
             & 30 & 161 & 1.94 (0.04)cd & 1.91  & -1 & -1\\ 
             & 34 & 130 & 1.49 (0.05)d  & 1.48  & -1 & -1\\ 
Pupa         & 15 & 4   & 8.49 (1.13)a  & 8.46  & -1 & -1\\ 
             & 20 & 162 & 3.11 (0.07)b  & 3.04  & -1 & -1\\ 
             & 25 & 75  & 3.03 (0.04)b  & 2.98  & -1 & -1\\ 
             & 27 & 142 & 1.79 (0.03)c  & 1.81  & -1 & -1\\  
             & 30 & 127 & 1.82 (0.04)c  & 1.79  & -1 & -1\\ 
             & 34 & 97  & 1.09 (0.03)d  & 1.05  & -1 & -1\\ 
Total c      & 15 & 4   & 55.33 (3.55)a & 58.31 & -1 & -1\\ 
             & 20 & 162 & 12.43 (0.19)b & 12.49 & -1 & -1\\ 
             & 25 & 75  & 11.72 (0.11)b & 11.57 & -1 & -1\\ 
             & 27 & 142 & 6.36 (0.04)c  & 6.49  & -1 & -1\\ 
             & 30 & 127 & 6.86 (0.04)c  & 6.89  & -1 & -1\\ 
             & 34 & 97  & 6.15 (0.05)c  & 6.29  & -1 & -1\\ 
\end{tabular}
\caption{ \label{tab:desarrollo-ciclo-temp-test} Media, la mediana, mediana predicha y media predicha
del número de días para el desarrollo de Ae. aegypti desde eclosión de los huevos a la emergencia de
adultos a seis temperaturas constantes (15-34 \textcelsius) \cite{rueda1990temperature}}
\end{center}
\end{table}


\subsection{Duración de las etapas inmaduras como porcentaje del tiempo ocupado para llegar a adulto}

Según \cite{manrique1998desarrollo}, se ha reportado que el intervalo de temperatura óptimo para el
desarrollo de las formas juveniles de Ae. aegypti es entre 25 y 30 C, con una duración aproximada de
diez días (17) 

En \cite{dengueUruguayCap2} se define la duración promedio de los cuatro estados larvales sucesivos y la
pupa, expresada como porcentaje del tiempo ocupado por la larva hasta que llega a adulto, son 14.6, 13.9,
17.5, 33.3 y 20.6, respectivamente. Durante el proceso evolutivo se maneja a la larva como una entidad
única y no se tiene en cuenta la transición entre sus estadios larvales, por lo que la duración promedio
de los estados larva y la pupa, expresados como porcentaje del tiempo ocupado por la larva a llegar a
adulto quedan resumidos como 79.3 y 20.6. (Ver tabla \ref{tab:desarrollo-ciclo-distribucion-test}).

\begin{table}
\begin{center}
\begin{tabular}{c c c c c }
Cantidad de individuos & Duración del ciclo & Larva & Pupa  & Error\\
\hline
-1   & -1 & -1 & -1 & -1   \\
-1   & -1 & -1 & -1 & -1   \\
-1   & -1 & -1 & -1 & -1   \\
\end{tabular}
\caption{ \label{tab:desarrollo-ciclo-distribucion-test} Duración promedio, en días, de los estados larva
y la pupa, expresados como porcentaje del tiempo ocupado por la larva a llegar a adulto}
\end{center}
\end{table}

\subsection{Desarrollo de las etapas inmaduras en distintos contenedores}
Se realiza una comparativa de la duración promedio del desarrollo de las etapas inmaduras de Ae. aegypti 
en distintos contenedores de \cite{manrique1998desarrollo} (Ver tabla
\ref{tab:desarrollo-ciclo-contenedores-test}).

\begin{table}
\begin{center}
\begin{tabular}{p{5cm} c c c c }
Contenedor           & Larva & Pupa & Total \\
\hline
Neumáticos           & 8.1   & 3.05 & 11.15\\
Florero              & 12.79 & 1.28 & 14.07\\
Jarrón               & 14.92 & 2.2  & 17.12\\
Trampa para hormigas & 15.57 & 1.95 & 17.52\\
Laboratorio          & 4.74  & 2    & 6.74\\
Resultados           & -1    & -1   & -1\\
\end{tabular}

\caption{ \label{tab:desarrollo-ciclo-contenedores-test} Duración promedio en días de las  larvas y pupas
de Ae. aegypti en neumáticos comparados con mediciones en el laboratorio y en otros contenedores.}
\end{center}
\end{table}
