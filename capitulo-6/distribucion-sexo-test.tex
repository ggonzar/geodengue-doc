\section{Distribución de Sexo}
Se realizó un análisis para determinar la distribución del sexo del Aedes Aegypti. Según
\cite{otero2006stochastic}, alrededor de la mitad de los adultos emergentes son hembras,
y se define una proporción de 1.02:1 macho: hembra. Los autores de \cite{manrique1998desarrollo}
la proporción sexual promedio de adultos emergidos es de 3.00 machos por 2.75 hembras, lo cual
no representa una diferencia significativa de una relación 1:1 en las proporciones sexuales.

En general se realizaron pruebas variando la cantidad de individuos de la población, los resultados se
pueden apreciar en la tabla \ref{tab:distribucion-sexo-test}, en se observa que existe una relación 1:1
para la distribución del sexo de los mosquitos.

\begin{table}
    \begin{center}
        \caption{ \label{tab:distribucion-sexo-test} Análisis de la distribución del sexo de Aedes
        aegypti.}
        \begin{tabular}{p{3cm} c c c }
            \hline \\
            Total de & Adultos & Adultos & Relación \\
            adultos  & machos  & hembras & (macho:hembra) \\
            \hline
            \hline \\
            912    &  461    &  451    &  0,99 : 1,01 \\
            1581   &  812    &  769    &  0,97 : 1,03 \\
            4154   &  2084   &  2070   &  1    : 1 \\
            9722   &  4940   &  4782   &  0,98 : 1,02 \\
            9045   &  4472   &  4573   &  1,01 : 0,99 \\
            16248  &  8104   &  8144   &  1    : 1 \\
            30693  &  15418  &  15275  &  1    : 1 \\
            28411  &  14224  &  14187  &  1    : 1 \\
        \end{tabular}

    \end{center}
\end{table}
