
\section{Ciclo gonotrófico}

Para el análisis de la tasa de desarrollo, en días, del ciclo gonotrófico de las hembras del Aedes aegypti, se
tuvo en cuenta 2 tipos de poblaciones $A1$ que corresponden a todas las hembras nulíparas que no han ovipuesto, 
$A2$ corresponde a las hembras que ya han ovipuesto. En la tabla \ref{tab:ciclo-gonotrofico-test}
se puede observar una media de 5,03 días para $A1$ y 2,92 días para $A_{2}$.


La duración del ciclo gonotrófico varía de 3 a 5 días dependiendo de la temperatura ambiental y las 
carateristicas del medio. En \cite{sivanathan2006ecology} la duración de se encuentra acotada entre 
2.73 a 3 días. Las variaciones en la tempeartura influyen en tiempo de digestión de la sangre y el 
desarrollo de los ovarios, a medida que la temepratura deciende, la digestión y por ende el ciclo 
gonotrófico tomará más tiempo. En la figura \ref{} se pude observar la duración del ciclo gonotrófico 
en relación a temperatura.

\begin{table}
    \begin{center}

        \caption{ \label{tab:ciclo-gonotrofico-test} Análisis de duración del cliclo gonotrófico
        de la hembra de Aedes Aegypti nueve temperaturas constantes  (18-34 \textcelsius).}
        \begin{tabular}{p{3cm} c*10 }
            \hline \\
            Población & 18\textcelsius & 20 \textcelsius & 22 \textcelsius & 24 \textcelsius 
                      & 25 \textcelsius & 26\textcelsius  & 27 \textcelsius & 30 \textcelsius 
                      & 34\textcelsius & Media General\\
            \hline
            \hline \\
            A1            & 8,97 & 7,4  & 6,13  & 5,08  & 4,63 & 4,22  & 3,85 & 2,94 & 2,06 & 5,03\\
            A2            & 5,21 & 4,3  & 3,56  & 2,95  & 2,69 & 2,45  & 2,24 & 1,71 & 1,2 & 2,92\\
            Media         & 7,09 & 5,85 & 4,85  & 4,02  & 3,66 & 3,34  & 3,05 & 2,33 & 1,63 & 3,98\\
        \end{tabular}
    \end{center}
\end{table}
