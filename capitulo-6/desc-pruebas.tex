\section{Pruebas realizadas}
Conjunto de pruebas a realizadas utilizadas para la validación del modelo y los datos.

\subsection{Tasa de desarollo}
Verificar las tasas de desarrollo de los individuos de la población de forma a validar
si el incremento de la madurez del individuo es correcta. Se debe contar con el tiempo
promedio de de la duración de cada estado, en días, para compararlos con los promedios
generales.

\begin{itemize}
    \item Temperatura constante
    \item Duración en promedio en cada estado a temperatura constante
    \item Desviación estándar
    \item Calcular el error
\end{itemize}

\section{Distribución de Sexo}
Se realizó un análisis para determinar la distribución del sexo del Aedes aegypti. Según
\cite{otero2006stochastic}, alrededor de la mitad de los adultos emergentes son hembras, y se
define una proporción de 1.02:1 macho: hembra. Los autores de \cite{manrique1998desarrollo} la
proporción sexual promedio de adultos emergidos es de $3$ machos por $2,75$ hembras, lo cual no representa una diferencia significativa de una relación 1:1 en las proporciones sexuales.

En general se realizaron pruebas variando la cantidad de individuos de la población, los
resultados se pueden apreciar en la tabla \tabref{tab:distribucion-sexo-test}, donde se observa que
existe una relación 1:1 para la distribución del sexo de los mosquitos.

\begin{table}
    \centering
        \caption{ \label{tab:distribucion-sexo-test} Análisis de la distribución del sexo de Aedes
        aegypti.}
        \begin{tabular}{l c c c }
            \hline \\
            Total de & Adultos & Adultos & Relación \\
            adultos  & machos  & hembras & (macho:hembra) \\
            \hline
            \hline \\
            912    &  461    &  451    &  0,99 : 1,01 \\
            1581   &  812    &  769    &  0,97 : 1,03 \\
            4154   &  2084   &  2070   &  1    : 1 \\
            9722   &  4940   &  4782   &  0,98 : 1,02 \\
            9045   &  4472   &  4573   &  1,01 : 0,99 \\
            16248  &  8104   &  8144   &  1    : 1 \\
            30693  &  15418  &  15275  &  1    : 1 \\
            28411  &  14224  &  14187  &  1    : 1 \\
        \end{tabular}
\end{table}


\subsection{Tasa de mortalidad}
Verificar las tasas de mortalidad de los individuos de la población. Se debe contar la cantidad de muertes de individuos, en días y por estado, para compararlos con los promedios generales.

\begin{itemize}
    \item Temperatura constante 
    \item Total de individuos muertos por dia y estado
    \item Desviación estándar
    \item Calcular el error
    \item Validar umbrales de mortalidad.
\end{itemize}

\subsection{Promedio de vuelo}
Verificar el comportamiento de los individuos como poblacion respecto al 
la cantidad de vuelos realizados y la cantidad metros recorridos en promedio por cada individuo

En promedio no se desplaza más de 500 metros en toda su vida

\begin{itemize}
    \item Temperatura constante 
    \item Identificacion de vuelos y la distancia recorrida
    \item Desviación estándar
    \item Calcular el error
    \item Validar umbrales de vuelo,correspondiente a la temperatura.
\end{itemize}

\subsection{Ovipostura}
Verificar el total de los individuos que realizan ovipostura. Obtener el total de
nuevos individuos (cantidad total de huevos generedados por la poblacion) dentro 
del periodo de estudio. Identificar la cantidad promedio de huevos generados en cada ovipostura

El promedio ronda entre los 68 a 100 huevos

\begin{itemize}
    \item Temperatura constante 
    \item Identificacion de ovipostura y la cantidad de huevos total
    \item Desviación estándar
    \item Calcular el error
\end{itemize}


\subsection{Inseminación}
El promedio de Hembras que son inseminadas antes de su primera alimentación y después de su primera
alimentación, y la cantidad de días promedio de inseminación
 
\subsection{Duración del ciclo gonotrófico}
Verificar la duración en promedio,en días, del ciclo gonotrófico. La duración promedio tiene que ser 3
días para temperaturas en promedio 30 C

\begin{itemize}
    \item Temperatura constante 
    \item Cantidad de días entre oviposturas.
    \item Promedio de duración para hembras nuliperas.
    \item Desviación estándar
    \item Calcular el error
\end{itemize}
