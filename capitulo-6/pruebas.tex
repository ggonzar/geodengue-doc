\section{ Descripción de las pruebas}

\subsection{Introducción}

Como fundamentacion y validacion de los algoritmos y procesos implementados
hemos realizado un conjunto de pruebas para determinar la veracidad y presición del 
trabajo que hemos desarrollado\\

El conjunto de pruebas abarca toda las combinaciones mas representativas
de valores de variables de entrada para realizar un analisis del comportamiento 
del sistema implementado.\\

\subsection{Prueba de madurez y mortalidad de etapas inmaduras}

La prueba consiste en someter a un proceso evolutivo a las larvas 
recolectadas de las larvitrampas y determinar los valores de porcentajes 
de madurez y mortalidad de los individuos.\\

\begin{itemize}
\item Variables de entrada \\
- variable 1 : cantidad de larvas por dispositivo de ovipostura (larvitrampa) \\
- variable 2 : valor en grados celcius de la temperatura por hora (dentro del periodo de prueba)\\
- variable 3: velocidad del viento en m/h y la direccion en grados (angulos respecto
a la linea del ecuador)\\

\item Dominio de las variables de entrada\\
- variable 1 (larvitrampa) : [ 0, 150 ]\\
La cantidad de larvas para el estudio va de 0 (ninguna larva reside en el 
dispositivo) hasta 150 individuos como maximo dentro de un solo dispositivo
de ovipostura.\\

- variable 2 (temperatura) :  [ 10º, 40º ]\\
Los valores de la temperatura fueron seleccionados de acuerdo a las pruebas realizadas en
[Temperature-related duration of aquatic stages of the Afrotropical 
malaria vector mosquito Anopheles gambiae in the laboratory M. N. BAYOH 
and S. W. LINDSAY]\\

- variable 3 : En la prueba realizada para las etapas inmaduras del 
mosquito el valor de la variable viento es despreciable. Los valores 
del viento no afectan en el analisis de la madurez y mortalidad 
de las etapas acuaticas\\

El periodo de estudio esta constituido por 30 dias. 30 días es el total
de días sobre los cuales se realiza la simulación \\

\item Definicion del conjunto de pruebas
Cada larvitrampa pertenece a uno de los siguientes tipos de zonas que 
indican el grado de calidad favorable de la ubicacion para la supervivencia
del aedes aegypti :\\

- Zona Optima : Abundancia de larvas recolectadas en el mismo dispositivo
y en los dispositivos aledaños \\
- Zona Buena : Buena cantidad de larvas recolectadas en el mismo dispositivo
y en los dispositivos aledaños \\
- Zona Normal : Numero medio de larvas recolectadas en el mismo dispositivo
y en los dispositivos aledaños \\
- Zona Mala : Numero por debajo de la media de larvas recolectadas en 
el mismo dispositivo y en los dispositivos aledaños \\
- Zona Pesima : Numero muy bajo de larvas recolectadas en el mismo dispositivo
y en dispositivos aledaños \\


La temperatura es discretizada en valores aumentados en 2 unidades por 
prueba (10º, 12º, 14º, ... , 38º, 40º) dentro del rango de dominio definido
anteriormente \\

Dado los valores posibles discretos de las zonas y la temperatura definimos
un conjunto de pruebas que abarcan las posibles combinaciones de tipo 
de zona y temperatura : \\

Por ej: \\

Prueba x : \\
	- Temperatura Constante : 10º\\
	- Tipo de Zona Constante : Zona Optima\\
	- Cantidad de larvas por dispositivo : 100\\

Prueba y :\\
	- Temperatura Constante : 18º\\
	- Tipo de Zona Constante : Zona Normal\\
	- Cantidad de larvas por dispositivo : 100 \\

Prueba z : \\
	- Temperatura Constante : 32º \\
	- Tipo de Zona Constante : Zona Pesima\\
	- Cantidad de larvas por dispositivo : 100\\


\item Resultados Esperados 
El conjunto de salidas esperadas para la mortalidad y la madurez de cada
etapa viene determinado segun varios trabajos de pruebas en laboratorios. \\ 

Los valores esperados de tiempo de madurez en las etapas inmaduras son :\\
larva : 9 a 13 dias para pasar al estado de pupa\\
pupa : de 2 a 4 dias para pasar al estado adulto\\

Los valores esperados de mortalidad esperados en etapas inmaduras del
aedes aegypti es :\\
larva : 72\% a 74\% de mortalidad (solo el 10\% o el 5\% sobrevive a la siquiente
etapa)\\
pupa : 17\% a 19\% de mortalidad  \\

\item Resultados Obtenidos

La siguiente tabla representa el porcentaje de mortalidad (mor) y el termino
promedio de tiempo de madurez del estado de larva (mad).\\

\begin{tabular}{c*{5}{c}r}
Temperatura/Tipo Zona              &optima&buena &normal & mala & pesima  \\
\hline
10 & 0 & 0 & 0 & 0 & 0   \\
12 & 0 & 0 & 0 & 0 & 0   \\
14 & 0 & 0 & 0 & 0 & 0   \\
16 & 0 & 0 & 0 & 0 & 0   \\
18 & 0 & 0 & 0 & 0 & 0   \\
20 & 0 & 0 & 0 & 0 & 0   \\
22 & 0 & 0 & 0 & 0 & 0   \\
24 & 0 & 0 & 0 & 0 & 0   \\
26 & 0 & 0 & 0 & 0 & 0   \\
28 & 0 & 0 & 0 & 0 & 0   \\
30 & 0 & 0 & 0 & 0 & 0   \\
32 & 0 & 0 & 0 & 0 & 0   \\
34 & 0 & 0 & 0 & 0 & 0   \\
36 & 0 & 0 & 0 & 0 & 0   \\
38 & 0 & 0 & 0 & 0 & 0   \\
40 & 0 & 0 & 0 & 0 & 0   \\
\end{tabular}

Tabla de madurez y mortalidad de las larvas. Cantidad promedio de larvas
por dispositivo es "x"

La tabla representa el porcentaje de mortalidad (mor) y el termino
promedio de tiempo de madurez del estado de pupa (mad)

\begin{tabular}{c*{5}{c}r}
Temperatura/Tipo Zona              &optima&buena &normal & mala & pesima  \\
\hline
10 & 0 & 0 & 0 & 0 & 0   \\
12 & 0 & 0 & 0 & 0 & 0   \\
14 & 0 & 0 & 0 & 0 & 0   \\
16 & 0 & 0 & 0 & 0 & 0   \\
18 & 0 & 0 & 0 & 0 & 0   \\
20 & 0 & 0 & 0 & 0 & 0   \\
22 & 0 & 0 & 0 & 0 & 0   \\
24 & 0 & 0 & 0 & 0 & 0   \\
26 & 0 & 0 & 0 & 0 & 0   \\
28 & 0 & 0 & 0 & 0 & 0   \\
30 & 0 & 0 & 0 & 0 & 0   \\
32 & 0 & 0 & 0 & 0 & 0   \\
34 & 0 & 0 & 0 & 0 & 0   \\
36 & 0 & 0 & 0 & 0 & 0   \\
38 & 0 & 0 & 0 & 0 & 0   \\
40 & 0 & 0 & 0 & 0 & 0   \\
\end{tabular}

16
18
20
22
24
26
28
30
32
34
36
38
40
Tabla de madurez y mortalidad de las pupas. Cantidad promedio de larvas
por dispositivo es "y"

\end{itemize}
