\section{Número de huevos}
Para el análisis del número de huevos generados por las hembras de Aedes
aegypti, se utilizó como referencia los resultados obtenidos por
\cite{osoriopontificia} a once temperaturas constantes (15-34\textcelsius),
agrupados por la cantidad de alimentación realizada( 1 -5).

De acuerdo a los resultados obtenidos referentes al número de huevos, la tabla
\ref{tab:ovipostura-cantidad-test} muestra que el número de huevos
aumenta cuando a medida que aumenta el número de ingestas de sangre, lo que coincide con la tendencia de \cite{osoriopontificia}.

\begin{table}
    \begin{center}

        \caption{ \label{tab:ovipostura-cantidad-test} Análisis de la
        cantidad de huevos de Aedes Aegypti generados a once temperaturas
        constantes  (15-34 \textcelsius).}
        \begin{tabular}{p{3cm} c c  }
            \hline \\
            Tomas de sangre &Media observada & Media Obtenida \\
            \hline
            \hline \\
            1               & 66,12  & 64,78\\
            2               & 106,5  & 148,11\\
            3               & 136,04 & 192,22\\
            4               & 267,86 & 228,33\\
            5               & 271,4  & 260\\
            Media General   & 129,86 & 109\\
        \end{tabular}
    \end{center}
\end{table}

