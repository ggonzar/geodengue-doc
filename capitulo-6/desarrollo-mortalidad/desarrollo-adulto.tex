\subsection{Longevidad del adulto}
Se realizó un análisis para determinar la longevidad, en días, de las hembras Aedes aegypti, de 
acuerdo a los resultados obtenidos (Tabla \ref{tab:desarrollo-adulto-hembra-test}), se obtuvo un 
promedio general de -1 días para la población que fue sometida a diferentes temperaturas. Los 
resultados obtenidos son comparados con los resultados de la duración en días descriptos en 
\cite{BESERRA2006}.

\begin{table}
    \begin{center}
        \begin{tabular}{p{5cm} c c c c c c }
            Población    &18 \textcelsius & 22 \textcelsius & 26 \textcelsius & 30 \textcelsius & 34 \textcelsius & Media General\\
            \hline \\
            Boqueirão    & 36,9 & 32,6 & 44,8 & 19,3 & 15,4 & 29,8\\
            B.           & 34   & 30,8 & 36,2 & 23,3 & 17,6 & 28,4\\
            C.           & 29,2 & 39   & 48,5 & 18   & 9,9  & 28,9\\
            Itaporanga   & 41,1 & 28   & 33,6 & 18,4 & 27,3 & 29,7\\
            Remígio      & 53,8 & 29,8 & 42,8 & 27,8 & 14,6 & 33,7\\
            Predichos    & -1   & -1   & -1   & -1   & -1   & -1  \\
        \end{tabular}
    \caption{ \label{tab:desarrollo-adulto-hembra-test} Análisis de la longevidad de las hembras
     Aedes aegypti a cinco temperaturas constantes (18-34 \textcelsius).}
    \end{center}
\end{table}
