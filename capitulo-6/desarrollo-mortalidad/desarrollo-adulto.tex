\subsection{Mortalidad del adulto}
La tasa de la mortalidad fue definida por \cite{otero2006stochastic} como una constante de valor 0.09
independiente de la temperatura. En \cite{ThironIzcazaJ2003} a la supervivencia como constante con
mortalidad diaria de 10\%. En la tabla \ref{tab:mortalidad-adulto-test} se presentan los resultados
obtenidos de la tasa de mortalidad del Aedes aegypti a nueve temperaturas constantes (18-34 \textcelsius).

\begin{table}
    \begin{center}
   
        \caption{ \label{tab:mortalidad-adulto-test} Análisis de la tasa de mortalidad del adulto del
         Aedes Aegypti a nueve temperaturas constantes (18-34 \textcelsius).}
        
        \begin{tabular}{p{3cm} c c c c c c }
                    \hline \\
                    Temperatura & Adultos Muertos & Total de Adultos & \% Mortalidad\\
                    \hline
                    \hline \\
                    
                    18 \textcelsius & 2     & 84     & 2,38\\
                    20 \textcelsius & 3180  & 29170  & 10,9\\
                    22 \textcelsius & 6930  & 63105  & 10,98\\
                    24 \textcelsius & 11647 & 105260 & 11,06\\
                    25 \textcelsius & 10070 & 92473  & 10,89\\
                    26 \textcelsius & 10101 & 93471  & 10,81\\
                    27 \textcelsius & 8652  & 78813  & 10,98\\
                    30 \textcelsius & 20355 & 171442 & 11,87\\
                    34 \textcelsius & 1298  & 12582  & 10,32\\ 
                    \hline 
                    Total           & 72235 & 646400 & 11,17
        \end{tabular}
    \end{center}
\end{table}
