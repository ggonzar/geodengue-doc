\subsection{Mortalidad del adulto}
La tasa de la mortalidad fue definida por \cite{otero2006stochastic} como una constante de valor 0.09
independiente de la temperatura. En \cite{ThironIzcazaJ2003} a la supervivencia como constante con
mortalidad diaria de 10\%. En la tabla \ref{tab:mortalidad-adulto-test} se presentan los resultados
obtenidos de la tasa de mortalidad del Aedes aegypti a diez temperaturas constantes (18-34 \textcelsius).

\begin{table}
    \begin{center}

        \caption{ \label{tab:mortalidad-adulto-test} Análisis de la tasa de mortalidad del adulto del
         Aedes Aegypti a diez temperaturas constantes (15-34 \textcelsius).}

        \begin{tabular}{p{3cm} c c c c c c }
                    \hline \\
                    Temperatura & Adultos Muertos & Total de Adultos & \% Mortalidad\\
                    \hline
                    \hline \\

                    15 \textcelsius & 270    & 2216    & 12,18\\
                    18 \textcelsius & 7099   & 68092   & 10,43\\
                    20 \textcelsius & 11050  & 99858   & 11,07\\
                    22 \textcelsius & 14250  & 117775  & 12,1\\
                    24 \textcelsius & 17708  & 157624  & 11,23\\
                    25 \textcelsius & 19500  & 175071  & 11,14\\
                    26 \textcelsius & 30804  & 268608  & 11,47\\
                    27 \textcelsius & 30957  & 278450  & 11,12\\
                    30 \textcelsius & 221381 & 2033769 & 10,89\\
                    34 \textcelsius & 27716  & 235984  & 11,74\\
                    \hline
                    Total           & 380735 & 3437447 & 11,08\\

        \end{tabular}
    \end{center}
\end{table}
