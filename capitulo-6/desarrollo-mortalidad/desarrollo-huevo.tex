\subsection{Desarrollo de la etapa de huevo}
Se realizó un análisis para determinar la tasa de desarrollo, en días, de los huevos de Aedes aegypti, 
de acuerdo a los resultados obtenidos (Tabla \ref{tab:desarrollo-huevo-test}), se obtuvo un promedio
general de 3.6 días para la población que fue sometida a diferentes 5 temperaturas. Los resultados
obtenidos son comparados con los resultados de la duración en días descriptos en \cite{BESERRA2006}.

\begin{table}
    \begin{center}
        \begin{tabular}{p{5cm} c c c c c c }
            Población    &18 \textcelsius & 22 \textcelsius & 26 \textcelsius & 30 \textcelsius & 34 \textcelsius & Media General\\

            \hline \\
            Boqueirão    & 9,3  & 6,5  & 3,9  & 3,3  & 3,5  & 4,3  \\
            B. dos Santos& --   & 5,9  & 4,3  & 3,7  & 2,9  & 4,2  \\
            C. Grande    & --   & 5,5  & 3,4  & 4,4  & 3    & 4,07 \\
            Itaporanga   & 9,2  & 6,2  & 4,7  & 3,1  & 3,5  & 4,38 \\
            Remígio      & --   & 6    & 4,5  & 2    & 2,5  & 3.75 \\
            Esperados    & 6.61 & 5,1  & 3.9  & 3.03 & 2.37 & 3.6  \\
            Resultados   & 6.08 & 5.06 & 3.03 & 3.03 & 2.02 & 3.29 \\
            
        \end{tabular}
        \caption{ \label{tab:desarrollo-huevo-test} Análisis de la tasa de desarrollo, en días, de los
         huevos de Aedes aegypti a cinco temperaturas constantes (18-34 \textcelsius).}
    \end{center}
\end{table}
