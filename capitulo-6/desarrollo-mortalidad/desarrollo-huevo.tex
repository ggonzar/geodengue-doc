\subsection{Desarrollo de la etapa de huevo}
Para el análisis de la tasa de desarrollo, en días, del huevo de Aedes aegypti, se utilizó como
referencia los resultados obtenidos por \cite{BESERRA2006} para 5 cepas de Aedes Aegyti, provenientes de
diferentes poblaciones, a 5 temperaturas constantes (18-34\textcelsius).

En la tabla \ref{tab:desarrollo-huevo-test} se puede apreciar los resultados obtenidos mediante el
proceso evolutivo para la población de prueba sometida a las 5 temperaturas constantes. Estos datos son
comparados con los resultados obtenidos en \cite{BESERRA2006} y la tasa de desarrollo esperada obtenida
mediante el modelo de Sharpe \& DeMichele, definida en la sección \ref{subsec:operadores-basicos}
mediante la ecuación \eqref{eq:sharpe-demichele}. Se obtuvo un desarrollo de 3.29 días con 
una diferencia de 0.46 días con la población de Remígio y de 1.09 días con Itaporanga. Los autores de
\cite{BESERRA2006}, señalan que a 18 \textcelsius, no existe desarrollo embrionario para las poblaciones
de B. dos Santos, C. Grande y Remígio, por lo que estos resultados no fueron incluidos en el calculo de
la media general. 

\begin{table}
    \begin{minipage}{\textwidth}
        
        \caption{\label{tab:desarrollo-huevo-test} Análisis de la tasa de desarrollo, en días, de los
         huevos de Aedes aegypti a cinco temperaturas constantes (18-34 \textcelsius).}
         
        \begin{tabular}{p{5cm} c c c c c c }
            \hline \\
            Población    &18 \textcelsius & 22 \textcelsius & 26 \textcelsius & 30 \textcelsius & 34 \textcelsius & Media General\\

            \hline
            \hline \\
            Boqueirão        & 9,3 $^{b}$  & 6,5  & 3,9  & 3,3  & 3,5  & 4,3  \\
            B. dos Santos    & -- $^{a}$   & 5,9  & 4,3  & 3,7  & 2,9  & 4,2  \\
            C. Grande        & -- $^{a}$   & 5,5  & 3,4  & 4,4  & 3    & 4,07 \\
            Itaporanga       & 9,2 $^{b}$  & 6,2  & 4,7  & 3,1  & 3,5  & 4,38 \\
            Remígio          & -- $^{a}$   & 6    & 4,5  & 2    & 2,5  & 3.75 \\
            Esperados$^{c}$  & 6.61 $^{b}$ & 5,1  & 3.9  & 3.03 & 2.37 & 3.6  \\
            Resultados$^{d}$ & 6.08 $^{b}$ & 5.06 & 3.03 & 3.03 & 2.02 & 3.29 \\
            
        \end{tabular}
        \footnotetext[1]{No hubo desarrollo embrionario a 18\textcelsius, \cite{BESERRA2006}.}   
        \footnotetext[2]{Como no hubo desarrollo embrionario en las demas poblaciones a 18\textcelsius,
        se decidió no incluir estos datos en la media general, sólo representa el promedio para
        estas poblaciones \cite{BESERRA2006}.}  
        \footnotetext[3]{Resultados obtenidos por el modelo de Sharpe \& DeMichele.}    
        \footnotetext[4]{Resultados obtenidos mediante el proceso evolutivo.}
    \end{minipage}
\end{table}
