\subsection{Huevo}
Para el análisis de la tasa de desarrollo, en días, del huevo de Aedes aegypti, se incluyeron
los huevos que se desarrollaron completamente para pasar a ser una larva. Con el fin de realizar
una comparativa con los resultados obtenidos, en la \tabref{tab:desarrollo-huevo-test} se
utiliza como referencia los resultados obtenidos por \cite{BESERRA2006}, donde se presentan los
requerimientos térmicos para el desarrollo del aedes aegypti en condiciones naturales, para 5
cepas de Aedes Aegyti, provenientes de diferentes  poblaciones, a 5 temperaturas constantes
(18-34\textcelsius).


\begin{table}[H]
    \begin{minipage}{\textwidth}
        \caption{\label{tab:desarrollo-huevo-test} Análisis de la tasa de desarrollo, en días, de
        los huevos de Aedes aegypti a cinco temperaturas constantes (18-34 \textcelsius).}

        \begin{tabular}{p{5cm} c c c c c c }
            \hline \\
            Población    &18 \textcelsius & 22 \textcelsius & 26 \textcelsius & 30 \textcelsius & 34 \textcelsius & Media General\\

            \hline
            \hline \\
            Boqueirão            & 9,3 $^{b}$  & 6,5  & 3,9  & 3,3  & 3,5  & 4,3  \\
            B. dos Santos        & -- $^{a}$   & 5,9  & 4,3  & 3,7  & 2,9  & 4,2  \\
            C. Grande            & -- $^{a}$   & 5,5  & 3,4  & 4,4  & 3    & 4,07 \\
            Itaporanga           & 9,2 $^{b}$  & 6,2  & 4,7  & 3,1  & 3,5  & 4,38 \\
            Remígio              & -- $^{a}$   & 6    & 4,5  & 2    & 2,5  & 3.75 \\
            Media Predicha$^{c}$ & 6.61 $^{b}$ & 5,1  & 3.9  & 3.03 & 2.37 & 3.6  \\
            Media Obtenida$^{d}$ & 6.08 $^{b}$ & 5.06 & 3.03 & 3.03 & 2.02 & 3.29 \\

        \end{tabular}
        \footnotetext[1]{No hubo desarrollo embrionario a 18\textcelsius, \cite{BESERRA2006}.}
        \footnotetext[2]{Como no hubo desarrollo embrionario en las demas poblaciones a 18\textcelsius,
        se decidió no incluir estos datos en la media general, sólo representa el promedio para
        estas poblaciones \cite{BESERRA2006}.}
        \footnotetext[3]{Resultados obtenidos por el modelo de Sharpe \& DeMichele.}
        \footnotetext[4]{Resultados obtenidos mediante el proceso evolutivo.}
    \end{minipage}
\end{table}

En la \figref{fig:desarrollo-huevo-baserra2006}, se puede observar una compartiva con los
valores observados en \cite{BESERRA2006}, la media predicha y la media obtenida. En general se
obtuvo un desarrollo de $3.29$ días a 4 temperaturas constantes (22, 26, 30, 34 \textcelsius), con
una diferencia de, $0,97$, $0,87$, $0,87$, $1,15$ y $0,57$ días con la media general de los
valores observados en \cite{BESERRA2006} para las poblaciones de Boqueirão, B. dos Santos, C.
Grande, Itaporanga y Remígio respectivamente. Los autores de \cite{BESERRA2006}, señalan que a
18 \textcelsius, no existe desarrollo embrionario para las poblaciones de B. dos Santos, C. Grande
y Remígio, por lo que estos resultados no fueron incluidos en el calculo de la media general. En
relación a la media predicha, que fue obtenida mediante el modelo de \cite{sharpe1977reaction}, se
obtuvo una diferencia de $0,32$ días con la media obtenida.

\begin{figure}[H]
\begin{minipage}{\textwidth}
    \begin{tabular}{c c }
        \initbox
        \num\putindeepbox[7pt]{\includegraphics[width=0.4\textwidth]{capitulo-6/graphics/desarrollo-huevos-1.png}} &
        \num\putindeepbox[7pt]{\includegraphics[width=0.4\textwidth]{capitulo-6/graphics/desarrollo-huevos-2.png}} \\

        \num\putindeepbox[7pt]{\includegraphics[width=0.4\textwidth]{capitulo-6/graphics/desarrollo-huevos-3.png}} &
        \num\putindeepbox[7pt]{\includegraphics[width=0.4\textwidth]{capitulo-6/graphics/desarrollo-huevos-4.png}} \\

        \num\putindeepbox[7pt]{\includegraphics[width=0.4\textwidth]{capitulo-6/graphics/desarrollo-huevos-5.png}} &
        \num\putindeepbox[7pt]{\includegraphics[width=0.4\textwidth]{capitulo-6/graphics/desarrollo-huevos-6.png}} \\
    \end{tabular}

    \caption{\label{fig:desarrollo-huevo-baserra2006}
    Comparativa entre ,la media obtenida, la media predicha y las medias observadas por \cite{
    BESERRA2006} en distintas pobalciones a 4 temperaturas contantes (18-34 \textcelsius) para
    las tasas de desarrollo de los huevos del Aedes Aegypti.}

    \footnotetext[1]{Tasa de desarrollo de Boqueirão en comparación con la media predicha y obtenida.}
    \footnotetext[2]{Tasa de desarrollo de B. dos Santos en comparación con la media predicha y obtenida.}
    \footnotetext[3]{Tasa de desarrollo de C. Grande en comparación con la media predicha y obtenida.}
    \footnotetext[4]{Tasa de desarrollo de Itaporanga en comparación con la media predicha y obtenida.}
    \footnotetext[5]{Tasa de desarrollo de Remígio en comparación con la media predicha y obtenida.}
    \footnotetext[6]{Media observada en los poblados de Boqueirão, B. dos Santos, C. Grande, Itaporanga, Remígio en comparación con la media predicha y obtenida.}

\end{minipage}
\end{figure}

%%Analisis de la Motalidad
En cuanto a la tasa de mortalidad diaria, \cite{otero2006stochastic} la define como una constante
independiente a la temperatura del 1\%, en la \tabref{tab:mortalidad-huevo-test} se presentan
los porcentajes de mortalidad obtenidos a 9 temperaturas constantes. En genral se obtuvo, un $1,64$
\% de mortalidad diaria, $98,36$ \% de supervivencia y un error del $0,64$ \% en comparación al 1
\% esperado.

\begin{table}[H]
    \begin{minipage}{\textwidth}
        \centering
        \caption{ \label{tab:mortalidad-huevo-test} Análisis de la tasa de mortalidad del huevo del
         Aedes Aegypti a nueve temperaturas constantes (18-34 \textcelsius).}

        \begin{tabular}{c c c c c c}
                    \hline \\
                    Temperatura&Huevos$^{a}$&Total$^{b}$&Mortalidad$^{c}$&Supervivencia$^{d}$\\
                    \textcelsius& Muertos   & de Huevos & (\%)           & (\%)\\
                    \hline
                    \hline \\
                    18            & 2618    & 156387   & 1,67 & 98,33\\
                    20            & 10962   & 659232   & 1,66 & 98,34\\
                    22            & 19089   & 1148391  & 1,66 & 98,34\\
                    24            & 53216   & 3228224  & 1,65 & 98,35\\
                    25            & 108324  & 6570126  & 1,65 & 98,35\\
                    26            & 126936  & 7753716  & 1,64 & 98,36\\
                    27            & 157990  & 9649119  & 1,64 & 98,36\\
                    30            & 611360  & 37347840 & 1,64 & 98,36\\
                    34            & 185972  & 11433440 & 1,63 & 98,37\\
                    Media General & 1276467 & 77946475 & 1,64 & 98,36\\
        \end{tabular}
        \footnotetext[1]{Total de huevos muertos en cada tempratura.}
        \footnotetext[2]{Total de huevos generados en cada temperatura}
        \footnotetext[3]{Tasa de mortalidad porcetual obtenida.}
        \footnotetext[4]{Tasa de supervivencia porcetual obtenida.}
    \end{minipage}
\end{table}

Los errores obtenidos se originan duarante el redondeo realizado al aplicar la tasa de
mortalidad diaria en cada colonia, debido a que el modelo, a la hora de calcular la cantidad de
huevos a ser eliminados de la colonia a la que pertenecen, solo permite eliminar a un número entero
de huevos. Si contamos con un grupo 10 de colonias cada una con 63 huevos, un total de 630 huevos.
Aplicando la tasa de mortalidad $0,01$ en cada colonia se obtine un total de $0,63$, aplicando el
operador de redondeo,definido por la ecuación \eqref{eq:operador-redondeo}, es $1$ huevo por
colonia. Al eliminar un huevo por colonia, en total se eliminarán 10 huevos en comparación al $6,3$
que se obtendría al eliminar $0,63$ huevos por colonia. En la tabla
\tabref{tab:mortalidad-huevo-error} se presentan los resultados del análisis realizado.

\begin{table}
    \begin{minipage}{\textwidth}
        \caption{ \label{tab:mortalidad-huevo-error} Análisis de la aplicación de la tasa de mortalidad entera del huevo del Aedes Aegypti.}

        \begin{tabular}{p{4cm} p{4cm} p{3cm} l }
                    \hline \\
                    Colonia & Cantidad de Huevos & Mortalidad$^{a}$ & Mortalidad Entera$^{b}$\\
                    \hline
                    \hline \\

                    1       & 63  & 0,63 & 1\\
                    2       & 63  & 0,63 & 1\\
                    3       & 63  & 0,63 & 1\\
                    4       & 63  & 0,63 & 1\\
                    5       & 63  & 0,63 & 1\\
                    6       & 63  & 0,63 & 1\\
                    7       & 63  & 0,63 & 1\\
                    8       & 63  & 0,63 & 1\\
                    9       & 63  & 0,63 & 1\\
                    10      & 63  & 0,63 & 1\\
                    Media General  & 630 & 6,3 & 10\\

        \end{tabular}
    \footnotetext[1]{Cantidad de huevos a eliminar al aplicar la de mortalidad de 0,01 definida en
    \cite{otero2006stochastic}.}
    \footnotetext[2]{Cantidad de huevos a eliminar aplicando el operador de redondeo definido por la ecuación \eqref{eq:operador-redondeo}.}
    \end{minipage}
\end{table}
