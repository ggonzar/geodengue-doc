
\subsection{Desarrollo de las etapas inmaduras}
Para el análisis de la tasa de desarrollo, en días, de las etapas inmaduras,larva y pupa, del Aedes
aegypti, se utilizó como referencia los resultados obtenidos por \cite{rueda1990temperature} a 6
temperaturas constantes (15-34\textcelsius).

En la tabla \ref{tab:desarrollo-inmaduras-test} se puede apreciar los resultados obtenidos mediante el
proceso evolutivo para la población de prueba sometida a las 6 temperaturas constantes. Estos datos son
comparados con los resultados obtenidos en \cite{rueda1990temperature}. En general se obtuvo un 
desarrollo de $7,91$ días para las etapas inmaduras, se puede observar una diferencia de $0,22$ en
comparación a los $8,13$ días para la media predicha y $0,18$ para los $8.09$ días de la media observada. Los
autores de \cite{rueda1990temperature} definen como la media predicha, a la tasa de desarrollo obtenida
mediante el modelo de Sharpe \& DeMichele, cabe resaltar que los parámetros para dicho modelo son los
mismos utilizados durante el proceso evolutivo.

\begin{table}
    \begin{minipage}{\textwidth}   
        \caption{ \label{tab:desarrollo-inmaduras-test} Análisis de la tasa de desarrollo del Aedes 
         Aegypti desde eclosión de los huevos a la emergencia de adultos a seis temperaturas constantes 
        (15-34 \textcelsius).}
        
        \begin{tabular}{p{3cm} c  c c c r }
            \hline \\
            Estadio de & Temperatura    & Media$^{a}$& Mediana$^{a}$& Media$^{b}$& Media$^{c}$\\
            desarrollo & \textcelsius   & observada  & observada    & predicha   & obtenida\\
            \hline
            \hline \\
            Larva        & 15           & 46,83 & 45,82 & 42.33 & 43.47\\ 
                         & 20           & 9,31  & 9,06  & 14.89 & 14.29\\ 
                         & 25           & 8,61  & 8,7   & 6.48  & 6.10\\ 
                         & 27           & 4,47  & 4,46  & 5.28  & 5.07\\ 
                         & 30           & 4,99  & 4,86  & 4.72  & 4.05\\ 
                         & 34           & 5,06  & 5,04  & 5.61  & 5.07\\ 
            \\
            Pupa         & 15           & 8.49  & 8.46  & 6.9  & 6.68\\ 
                         & 20           & 3.11  & 3.04  & 4.10 & 4.12\\ 
                         & 25           & 3.03  & 2.98  & 2.5  & 2.05\\ 
                         & 27           & 1.79  & 1.81  & 2.06 & 2.03\\  
                         & 30           & 1.82  & 1.79  & 1.56 & 1.02\\ 
                         & 34           & 1.09  & 1.05  & 1.08 & 1.02\\ 
        \end{tabular}
        \footnotetext[1]{Valores observados por \cite{rueda1990temperature}.}    
        \footnotetext[2]{Resultados obtenidos por el modelo de Sharpe \& DeMichele.}    
        \footnotetext[3]{Resultados obtenidos mediante el proceso evolutivo.}
    \end{minipage}
\end{table}
