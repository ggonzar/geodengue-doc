
\subsection{Desarrollo de las etapas inmaduras}
Se realizó un análisis para determinar la tasa de desarrollo, en días, de los las etapas inmaduras del
Aedes aegypti en sus fases de larva y pupa, de acuerdo a los resultados obtenidos (Tabla 
\ref{tab:desarrollo-inmaduras-test}), se obtuvo un promedio general de -1 días para la población que 
fue sometida a diferentes temperaturas. Los resultados obtenidos son comparados con los resultados de 
la duración en días descriptos en \cite{BESERRA2006} y en \cite{rueda1990temperature}.

\begin{table}
    \begin{center}
        \begin{tabular}{p{2cm} c  c c c c }
            Estadio de & Temperatura    & Media     & Mediana  & Media    & Media\\
            desarrollo & \textcelsius   & observada & observada& esperada & predicha\\
            \hline
            \hline \\
            Larva        & 15           & 46,83 & 45,82 & 42.33 & 43.47\\ 
                         & 20           & 9,31  & 9,06  & 14.89 & 14.29\\ 
                         & 25           & 8,61  & 8,7   & 6.48  & 6.10\\ 
                         & 27           & 4,47  & 4,46  & 5.28  & 5.07\\ 
                         & 30           & 4,99  & 4,86  & 4.72  & 4.05\\ 
                         & 34           & 5,06  & 5,04  & 5.61  & 5.07\\ 
            \\
            Pupa         & 15           & 8.49  & 8.46  & 6.9  & 6.68\\ 
                         & 20           & 3.11  & 3.04  & 4.10 & 4.12\\ 
                         & 25           & 3.03  & 2.98  & 2.5  & 2.05\\ 
                         & 27           & 1.79  & 1.81  & 2.06 & 2.03\\  
                         & 30           & 1.82  & 1.79  & 1.56 & 1.02\\ 
                         & 34           & 1.09  & 1.05  & 1.08 & 1.02\\ 
        \end{tabular}
    \caption{ \label{tab:desarrollo-inmaduras-test} Análisis de la tasa de de desarrollo del Aedes 
    Aegypti desde eclosión de los huevos a la emergencia de adultos a seis temperaturas constantes 
    (15-34 \textcelsius)}
    \end{center}
\end{table}
