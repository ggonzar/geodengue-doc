
\subsection{Desarrollo de las etapas inmaduras}
Se recolectaron los datos correspondientes a la duración del ciclo de vida de las etapas inmaduras, desde
la eclosión de huevos  hasta alcanzar el estado adulto, a partir de estos datos se calcula la media y la
mediana predicha del número de días y se realiza una comparación, a seis temperaturas constantes 
(15-34 \textcelsius), con los resultados obtenidos en  \cite{rueda1990temperature} bajo condiciones de
laboratorio.

\begin{table}
\begin{center}
\begin{tabular}{p{2cm} c c c c c c }
Estadio de & Temperatura    &   & Media     & Mediana  & Media    & Mediana\\
desarrollo & \textcelsius   & n & observada & observada& predicha & predicha\\
\hline \\
Larva        & 15 & 631 & 46,83 & 45,82 & -1 & -1\\ 
             & 20 & 1121& 9,31  & 9,06  & -1 & -1\\ 
             & 25 & 837 & 8,61  & 8,7   & -1 & -1\\ 
             & 27 & 1093& 4,47  & 4,46  & -1 & -1\\ 
             & 30 & 886 & 4,99  & 4,86  & -1 & -1\\ 
             & 34 & 762 & 5,06  & 5,04  & -1 & -1\\ 
\\
Pupa         & 15 & 4   & 8.49  & 8.46  & -1 & -1\\ 
             & 20 & 162 & 3.11  & 3.04  & -1 & -1\\ 
             & 25 & 75  & 3.03  & 2.98  & -1 & -1\\ 
             & 27 & 142 & 1.79  & 1.81  & -1 & -1\\  
             & 30 & 127 & 1.82  & 1.79  & -1 & -1\\ 
             & 34 & 97  & 1.09  & 1.05  & -1 & -1\\ 
\\
Total c      & 15 & 4   & 55.33 & 58.31 & -1 & -1\\ 
             & 20 & 162 & 12.43 & 12.49 & -1 & -1\\ 
             & 25 & 75  & 11.72 & 11.57 & -1 & -1\\ 
             & 27 & 142 & 6.36  & 6.49  & -1 & -1\\ 
             & 30 & 127 & 6.86  & 6.89  & -1 & -1\\ 
             & 34 & 97  & 6.15  & 6.29  & -1 & -1\\ 
\end{tabular}
\caption{ \label{tab:desarrollo-ciclo-temp-test} Análisis de la tasa de de desarrollo del Aedes aegypti
desde eclosión de los huevos a la emergencia de adultos a seis temperaturas constantes 
(15-34 \textcelsius)}
\end{center}
\end{table}
