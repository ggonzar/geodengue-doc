
\subsection{Desarrollo de las etapas inmaduras}
Se realizó un análisis para determinar la tasa de desarrollo, en días, de los las etapas inmaduras del
Aedes aegypti en sus fases de larva y pupa, de acuerdo a los resultados obtenidos (Tabla 
\ref{tab:desarrollo-inmaduras-test}), se obtuvo un promedio general de -1 días para la población que 
fue sometida a diferentes temperaturas. Los resultados obtenidos son comparados con los resultados de 
la duración en días descriptos en \cite{BESERRA2006} y en \cite{rueda1990temperature}.

\begin{table}
\begin{center}
\begin{tabular}{p{2cm} c  c c c c }
Estadio de & Temperatura    & Media     & Mediana  & Media    & Mediana\\
desarrollo & \textcelsius   & observada & observada& predicha & predicha\\
\hline \\
Larva        & 15           & 46,83 & 45,82 & 42.33 & -1\\ 
             & 20           & 9,31  & 9,06  & 14.89 & -1\\ 
             & 25           & 8,61  & 8,7   & 8.07  & -1\\ 
             & 27           & 4,47  & 4,46  & 5.28  & -1\\ 
             & 30           & 4,99  & 4,86  & 4.72  & -1\\ 
             & 34           & 5,06  & 5,04  & 5.61  & -1\\ 
\\
Pupa         & 15           & 8.49  & 8.46  & 6.9  & -1\\ 
             & 20           & 3.11  & 3.04  & 4.10 & -1\\ 
             & 25           & 3.03  & 2.98  & 2.5  & -1\\ 
             & 27           & 1.79  & 1.81  & 2.06 & -1\\  
             & 30           & 1.82  & 1.79  & 1.56 & -1\\ 
             & 34           & 1.09  & 1.05  & 1.08 & -1\\ 
\end{tabular}
\caption{ \label{tab:desarrollo-inmaduras-test} Análisis de la tasa de de desarrollo del Aedes aegypti
desde eclosión de los huevos a la emergencia de adultos a seis temperaturas constantes 
(15-34 \textcelsius)}
\end{center}
\end{table}
