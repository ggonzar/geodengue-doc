\subsection{Duración de las etapas inmaduras como porcentaje del tiempo ocupado para llegar a adulto}

Según \cite{manrique1998desarrollo}, se ha reportado que el intervalo de temperatura óptimo para el
desarrollo de las formas juveniles de Ae. aegypti es entre 25 y 30 C, con una duración aproximada de
diez días (17) 

En \cite{dengueUruguayCap2} se define la duración promedio de los cuatro estados larvales sucesivos y la
pupa, expresada como porcentaje del tiempo ocupado por la larva hasta que llega a adulto, son 14.6, 13.9,
17.5, 33.3 y 20.6, respectivamente. Durante el proceso evolutivo se maneja a la larva como una entidad
única y no se tiene en cuenta la transición entre sus estadios larvales, por lo que la duración promedio
de los estados larva y la pupa, expresados como porcentaje del tiempo ocupado por la larva a llegar a
adulto quedan resumidos como 79.3 y 20.6. (Ver tabla \ref{tab:desarrollo-ciclo-distribucion-test}).

\begin{table}
\begin{center}
\begin{tabular}{c c c c c c }
Temp         & Media & Media      & Media     & \% error& \% error\\
\textcelsius & total & larva (\%) & pupa (\%) & larva   & pupa \\
\hline \\
15           & -1    & -1         & -1        & -1      & -1 \\
20           & -1    & -1         & -1        & -1      & -1 \\
25           & -1    & -1         & -1        & -1      & -1 \\
27           & -1    & -1         & -1        & -1      & -1 \\
30           & -1    & -1         & -1        & -1      & -1 \\
34           & -1    & -1         & -1        & -1      & -1 \\

\end{tabular}
\caption{ \label{tab:desarrollo-ciclo-distribucion-test} Duración promedio, en días, de los estados larva
y la pupa, expresados como porcentaje del tiempo ocupado por la larva a llegar a adulto}
\end{center}
\end{table}
