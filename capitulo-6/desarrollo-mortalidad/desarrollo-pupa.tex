\subsection{Desarrollo de la pupa}
Para el análisis de la tasa de desarrollo, en días, de las pupas, del Aedes aegypti, incluyeron
las pupas que se desarrollaron completamente para pasar a ser un adulto. Existen diversos
estudios, que han reportado que tasa de desarrollo de las pupas, se encuentra influenciada por la
temperaturura y varia de acuerdo a las localidades y subespecies. A continuación se mencionaran
algunos estudios realizados correspondientes a la tasa de desarrollo en días de la pupa, con el
fin de realizar una comparación con los resultados obtenidos.

En \cite{rueda1990temperature}, se reporta el efecto de  temperaturas constantes sobre las tasas
de desarrollo, el crecimiento y la supervivencia de los estados inmaduros de Aeedes aegypti,
determinadas en condiciones de laboratorio, y el modelo dependiente de la temperatura de
\cite{sharpe1977reaction}. En la tabla \ref{tab:desarrollo-pupa-rueda1990temperature-test} se
presentan los resultados obtenidos a seis temperaturas constantes(15-34 \textcelsius) en
comparación a los obtenidos en \cite{rueda1990temperature}.


\begin{table}
    \begin{minipage}{\textwidth}
        \caption{ \label{tab:desarrollo-pupa-rueda1990temperature-test} Análisis de la tasa de desarrollo de las pupas del Aedes Aegypti a seis temperaturas constantes
        (15-34 \textcelsius).}
        \begin{tabular}{p{5cm} c c c c c c c}
            \hline\\
            Resultados & 15\textcelsius & 20\textcelsius & 25\textcelsius & 27\textcelsius
            & 30\textcelsius & 34\textcelsius &  Media General\\
            \hline
            \hline \\
            Media Observada$^{a}$   & 46,83 & 9,31  & 8,61 & 4,47 & 4,99 & 5,06 & 13,21\\
            Mediana Observada$^{a}$ & 45,82 & 9,06  & 8,7  & 4,46 & 4,86 & 5,04 & 12,99\\
            Media Predicha$^{b}$    & 42,33 & 14,89 & 6,48 & 5,28 & 4,72 & 5,61 & 13,22\\
            Media obtenida$^{c}$    & 44,34 & 14,99 & 6,62 & 5,53 & 4,47 & 5,6 & 13,59\\

        \end{tabular}
        \footnotetext[1]{Valores observados por \cite{rueda1990temperature}.}
        \footnotetext[2]{Resultados obtenidos por el modelo de \cite{sharpe1977reaction}.}
        \footnotetext[3]{Resultados obtenidos mediante el proceso evolutivo.}
    \end{minipage}
\end{table}


En \cite{BESERRA2006}, se presentan los requerimientos térmicos para el desarrollo del aedes
aegypti en condiciones naturales, para 5 cepas de Aedes Aegyti, provenientes de diferentes
poblaciones, a 5 temperaturas constantes (18-34\textcelsius). En la tabla
\ref{tab:desarrollo-pupa-baserra2006-test} se presentan los resultados obtenidos a cinco
temperaturas constantes(15-34 \textcelsius) en comparación a las tasas de desarrollo obtenidas en
\cite{BESERRA2006} para las pupas del Aedes Aegypti.

\begin{table}
    \begin{minipage}{\textwidth}
        \caption{\label{tab:desarrollo-pupa-baserra2006-test} Análisis de la tasa de desarrollo de
        las pupas del Aedes Aegypti a cinco temperaturas constantes (15-34 \textcelsius).}
        \begin{tabular}{p{5cm} c c c c c c }
            \hline\\
            Población    &18 \textcelsius & 22 \textcelsius & 26 \textcelsius & 30 \textcelsius
            & 34 \textcelsius & Media General\\
            \hline
            \hline \\
            Boqueirão$^{a}$    & 19,6  & 13,4  & 8,7  & 6,4  & 6,9 & 11\\
            B, dos Santos$^{a}$& 18,9  & 13,4  & 10,2 & 7,9  & 7,5 & 11,6\\
            C, Grande$^{a}$    & 18,3  & 11,1  & 6,6  & 5,8  & 6   & 9,6\\
            Itaporanga$^{a}$   & 21    & 12,3  & 7,3  & 5,9  & 9,7 & 11,2\\
            Remígio$^{a}$      & 18,9  & 15,8  & 8,4  & 5,4  & 6   & 10,9\\
            Resultados$^{b}$   & 23,37 & 10,85 & 5,51 & 4,47 & 5,6 & 9,96\\
        \end{tabular}
        \footnotetext[1]{Los valores presentados para estas poblaciones fueron tomados de
         \cite{BESERRA2006}.}
        \footnotetext[2]{Resultados obtenidos mediante el proceso evolutivo.}
    \end{minipage}
\end{table}
