
\subsection{Mortalidad Huevos}
Para el análisis de la mortalidad de los huevos del Aedes Aegypti, se obtuvo una mortalidad diaria de 1,24 \% 
en comparación con el 1\%,  definida en \cite{otero2006stochastic}. El error es causado por el operador de 
redondeo definido por la ecuación  \eqref{eq:operador-redondeo} utilizada para calcular la cantidad, entera, de candidatos para la mortalidad, de huevos pertenecientes a una colonia.

\begin{table}
    \begin{center}
   
        \caption{ \label{tab:mortalidad-huevo-test} Análisis de la tasa de mortalidad del huevo del
         Aedes Aegypti a nueve temperaturas constantes (18-34 \textcelsius).}
        
        \begin{tabular}{p{3cm} c c c c c c }
                    \hline \\
                    Temperatura & Huevos Muertos & Total de Huevos & \% Mortalidad\\
                    \hline
                    \hline \\
                    15 \textcelsius & 8448   & 681604  & 1,24\\
                    20 \textcelsius & 17164  & 1392228  & 1,23\\
                    22 \textcelsius & 21080  & 1517152  & 1,39\\
                    24 \textcelsius & 32928  & 2529476  & 1,3\\
                    25 \textcelsius & 48792  & 3791908  & 1,29\\
                    26 \textcelsius & 46874  & 3808616  & 1,23\\
                    27 \textcelsius & 60254  & 4998074  & 1,21\\
                    30 \textcelsius & 489900 & 39788450 & 1,23\\
                    34 \textcelsius & 44660  & 3456096  & 1,29\\
                    \hline 
                    Total           & 761652 & 61282000 & 1,24\\

        \end{tabular}
    \end{center}
\end{table}
