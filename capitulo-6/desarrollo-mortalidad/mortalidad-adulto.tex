\subsection{Adulto}
Como se menciona en la la sección \ref{subsec:mortalidad} la tasa de la mortalidad utilizada fue
la definida en \cite{otero2006stochastic}, como una constante de valor 0.09 independiente de la
temperatura. En la tabla \ref{tab:mortalidad-diaria-adulto-test} se presentan los resultados
obtenidos de la tasa de mortalidad del Aedes aegypti a diez temperaturas constantes (18-34
\textcelsius). En general se  se obtuvo una mortalidad, diaria de $11,08$\% en coparación al 9\%
definido en \cite{otero2006stochastic} y el 10\% señalado en \cite{ThironIzcazaJ2003}.


\begin{table}
    \begin{center}

        \caption{ \label{tab:mortalidad-diaria-adulto-test} Análisis de la tasa de mortalidad del adulto del
         Aedes Aegypti a diez temperaturas constantes (15-34 \textcelsius).}

        \begin{tabular}{p{3cm} c c c c c c }
                    \hline \\
                    Temperatura & Adultos Muertos & Total de Adultos & \% Mortalidad\\
                    \hline
                    \hline \\

                    15 \textcelsius & 270    & 2216    & 12,18\\
                    18 \textcelsius & 7099   & 68092   & 10,43\\
                    20 \textcelsius & 11050  & 99858   & 11,07\\
                    22 \textcelsius & 14250  & 117775  & 12,1\\
                    24 \textcelsius & 17708  & 157624  & 11,23\\
                    25 \textcelsius & 19500  & 175071  & 11,14\\
                    26 \textcelsius & 30804  & 268608  & 11,47\\
                    27 \textcelsius & 30957  & 278450  & 11,12\\
                    30 \textcelsius & 221381 & 2033769 & 10,89\\
                    34 \textcelsius & 27716  & 235984  & 11,74\\
                    \hline
                    Total           & 380735 & 3437447 & 11,08\\

        \end{tabular}
    \end{center}
\end{table}

Los autores \cite{ThironIzcazaJ2003} señalan que la mortalidad de una población de adultos, se encuentra entre el 50\% a los 7 días y un 95\% a los 30 días. En la tabla
\ref{tab:mortalidad-periodo-adulto-test} se presenta los resultados correspondientes a la mortalidad de los adultos a 7 y 30 días. En general se obtuvo un $54,96$\% a los 7 días y un
$96,66$\% a los 30 días, en comparación al 50\% y el 95\% señaldados por señalado por los autores
de \cite{ThironIzcazaJ2003}.


\begin{table}
    \begin{center}
        \caption{ \label{tab:mortalidad-periodo-adulto-test} Análisis de la tasa de mortalidad del
        adulto del Aedes Aegypti a nueve temperaturas constantes (15-34 \textcelsius) para los
        periodos de 7 y 30 días.}

        \begin{tabular}{p{3cm} c c c c c c }
                    \hline \\
                    Temperatura & Total de & Muertos a  & Muertos a   & Mortalidad & Mortalidad\\
                    Temperatura & adultos  & los 7 días & los 30 días & 7 días(\%) & 30 días(\%)\\
                    \hline
                    \hline \\
                    18\textcelsius &370  & 177 & 305 & 56,08 & 82,43\\
                    20\textcelsius &526  & 295 & 508 & 57,89 & 96,58\\
                    22\textcelsius &672  & 389 & 659 & 54,03 & 98,07\\
                    24\textcelsius &818  & 442 & 797 & 54,55 & 97,43\\
                    25\textcelsius &869  & 474 & 846 & 54,67 & 97,35\\
                    26\textcelsius &900  & 492 & 878 & 55,1 & 97,56\\
                    27\textcelsius &882  & 486 & 861 & 54,99 & 97,62\\
                    30\textcelsius &991  & 545 & 962 & 56,72 & 97,07\\
                    34\textcelsius &744  & 422 & 730 & 56,72 & 98,12\\
                    Media general   & 6772 & 3722 & 6546 & 54,96 & 96,66\\

        \end{tabular}
    \end{center}
\end{table}


Los errores obtenidos se originan duarante el redondeo realizado al aplicar la tasa de
mortalidad diaria en cada colonia, debido a que el modelo, a la hora de calcular la cantidad de
adultos a ser eliminados de la colonia a la que pertenecen, solo permite eliminar a un número
entero de adultos. Si contamos con un grupo 10 de colonias cada una con 6 adultos cada una,
tenemos un total de 60 adultos.Aplicando la tasa de mortalidad $0,09$ en cada colonia se obtine un
total de $0,54$, aplicando el operador de redondeo,definido por la ecuación
\eqref{eq:operador-redondeo}, es de $1$ adulto por colonia. Al eliminar un adulto por colonia, en
total se eliminarán 10 adultos en comparación al $5,4$ que se obtendría al eliminar $0,54$ adultos
por colonia. En la  tabla \ref{tab:mortalidad-adulto-error} se presentan los resultados del
análisis realizado.

\begin{table}
    \begin{minipage}{\textwidth}
        \caption{ \label{tab:mortalidad-adulto-error} Análisis de la aplicación de la tasa de
        mortalidad entera del adulto del Aedes Aegypti.}

        \begin{tabular}{p{4cm} p{4cm} p{3cm} l }
                    \hline \\
                    Colonia & Cantidad de Adultos & Mortalidad$^{a}$ & Mortalidad Entera$^{b}$\\
                    \hline
                    \hline \\
                    1       & 6  & 0,54 & 1\\
                    2       & 6  & 0,54 & 1\\
                    3       & 6  & 0,54 & 1\\
                    4       & 6  & 0,54 & 1\\
                    5       & 6  & 0,54 & 1\\
                    6       & 6  & 0,54 & 1\\
                    7       & 6  & 0,54 & 1\\
                    8       & 6  & 0,54 & 1\\
                    9       & 6  & 0,54 & 1\\
                    10      & 6  & 0,54 & 1\\
                    Total   & 60 & 5,4  & 10\\
        \end{tabular}
    \footnotetext[1]{Cantidad de adultos a eliminar al aplicar la de mortalidad de 0,09 definida en
    \cite{otero2006stochastic}.}
    \footnotetext[2]{Cantidad de audltos a eliminar aplicando el operador de redondeo definido por la ecuación \eqref{eq:operador-redondeo}.}
    \end{minipage}
\end{table}
