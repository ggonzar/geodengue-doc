
\subsection{Índices de Stegomia}
\label{sec:densidad-vectorial-indices-stegomia}
La Organización mundial de la salud (OMS) recomienda la utlización de tres indicadores
entomológicos, generalmente conocidos como índices de Stegomia, para estimar la densidad del
vector, Índice de casas (I.C.), Índice de Recipientes (I.R.) e Índice de Breteau (I.B.).Estos
indices son calculados a partir de muestreos de larvas y recipientes.

\subsubsection{Índice de Casas}
El índice de Casas (I.C.) es el valor numérico que especifíca el porcentaje de viviendas
infestadas con larvas, pupas o ambos estados de desarrollo del Aedes aegypti o Aedes albopictus
\cite{ibanez1995vectores, world2009dengue}, y se encuentra expresado como :

\begin{equation}
I.C. = \frac{\textit{número de casas infestadas} * 100}{\textit{número de casas inspeccionadas}}
\end{equation}

Donde :
\begin{itemize}
\item \textit{número de casas infestadas} : cantidad de casas que cuentan con al menos un contenedor que alberga a larvas o pupas de Aedes aegypti.
\item \textit{número de casas inspeccionadas} : El total de casas analizadas para el estudio.
\end{itemize}

Para este indice, se analizan los contenedores de las viviendas y alrededores, donde, se registra
como positiva cuando al menos un contenedor presenta larvas o pupas \cite{ibanez1995vectores}. Es
el de uso más generalizado para la distribución de medición de la población, pero no tiene en
cuenta el número de recipientes positivos ni la productividad de esos recipientes
\cite{world2009dengue}. Puede ser utilizado para proporcionar una indicación rápida de la
distribución del mosquito en una área determinada.

\subsubsection{Índice de Recipientes}
El índice de Recipientes es un valor numérico que consiste en el porcentaje de recipientes que
contienen agua y están infestados con las larvas o pupas del Aedes aegypti o Aedes albopictus
\cite{ibanez1995vectores, world2009dengue}, y se encuentra expresado como :

\begin{equation}
I.R. = \frac{\textit{número de contenedores positivos} * 100}{\textit{número de contenedores inspeccionadas}}
\end{equation}

Donde :
\begin{itemize}
\item \textit{número de contenedores positivos}: La cantidad de contenedores en los cuales se observan larvas o pupas de Aedes aegypti.
\item \textit{número de contenedores inspeccionadas} : El total de contenedores analizadas para el estudio.
\end{itemize}

Este índice sólo ofrece información sobre la proporción de recipientes que mantienen agua y que
son positivos \cite{world2009dengue}. Las encuestas que utilizan el índice del envase son mucho
más lentas a realizar que las encuestas sobre el índice de la casa, porque requieren que se
examinen todos los recipientes para detectar la presencia de etapas no maduras y registrar los
detalles en envases positivos y negativos para determinar su especie mediante análisis
laboratoriales.

\subsubsection{Índice de Breteau}
El índice de Breteau (I.B.) es un valor numérico que define el número recipientes con larvas,
pupas o ambos estados de desarrollo del Aedes aegypti o Aedes albopictus, que se encuentran número
de recipientes positivos por cada 100 casas inspeccionadas
\cite{ibanez1995vectores, MARQUES1993,world2009dengue}, expresada como :

\begin{equation}
I.B. = \frac{\textit{número de contenedores positivos} * 100}{\textit{número de casas inspeccionadas}}
\end{equation}

Donde :
\begin{itemize}
\item \textit{número de contenedores positivos} : La cantidad de contenedores en los cuales se observan larvas o pupas de Aedes aegypti.
\item \textit{número de casas inspeccionadas} : El total de casas analizadas para el estudio.
\end{itemize}

El índice de Breteau establece una relación entre recipientes positivos y casas, y se considera el
índice más informativo \cite{world2009dengue}.

\subsubsection{Problemática de las técnicas tradicionales}
Las técnicas tradicionales de vigilancia de Aedes aegypti utilizan índices stegomia para
determinar el grado de infestación, dispersión y densidad del mosquito en una zona y tiempo
determinados \cite{NINO2011}. La Organización Mundial de la Salud (OMS), para estimar la densidad
del vector, ha recomendado los siguientes indicadores entomológicos : Índice de Casa, Índice de
Recipiente e Índice de Breteau \cite{cenaprece2013}.

Estos índices se fundamentan en la detección de la presencia de formas inmaduras (larvas de
mosquito, pupas y restos de larvas y pupas) del vector dentro de recipientes domésticos. Un
problema es que es que los recipientes más comunes con frecuencia, no resultan ser los más
productivos \cite{world2009dengue}. Hay que tener en cuenta que la tasa de aparición de mosquitos
adultos en recipientes de gran acumulación de agua (tanques, baldes, piletas, etc), puede diferir
considerablemente de la tasa en recipientes pequeños (latas, botellas, plantas, etc), pero como
la inspección solo las registra como positivas o negativas. En consecuencia, pueden ser
clasificados como iguales, sitios con índices larvarios iguales, pero con diferentes perfiles de
recipientes, por lo tanto, las capacidades de transmisión, pueden ser bastante diferentes
\cite{world2009dengue}.

Son considerados una pobre indicación de la producción de mosquitos adultos
\cite{world2009dengue, cenaprece2013}, debido a que no reflejan la asociación que existe entre las
densidades de mosquitos y tipo de recipientes presentes, con los riesgos de transmisión de dengue
\cite{cenaprece2013}, además no se puede medir la productividad del recipiente
\cite{world2009dengue} y en consecuencia se proporciona poca o nula información de aquellas
viviendas en las que existe un mayor riesgo de presencia de mosquitos \cite{cenaprece2013}.

Actualmente, sólo son recomendados para detectar la calidad de las acciones (control de calidad)
realizadas por el personal de control larvario \cite{cenaprece2013}. Existen numerosos métodos e
indicadores más prácticos, eficientes y económicos para determinar las poblaciones de Aedes
aegypti \cite{cenaprece2013}, como larvitrampas, ovitrampas y mosquitericas.
