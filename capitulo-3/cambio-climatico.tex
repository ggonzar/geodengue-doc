
\section{Aedes aegypti y el cambio climático}
Los cambios temporales y espaciales en la temperatura, precipitaciones y humedad que se prevé que ocurran bajo
diferentes escenarios de cambio climático afectarán la biología y ecología de Ae. aegypti y, en consecuencia, 
los riesgos de transmisión de la enfermedad del dengue\cite{dengueUruguayCap2}.

%Reescribir esta parte%
Los mayores efectos del cambio del clima sobre la transmisión de la enfermedad se observan probablemente en los
extremos del rango de temperaturas en el cual ocurre la misma (14-18\textcelsius como límite inferior y 
35-40\textcelsius como límite superior)\cite{dengueUruguayCap2}. Por debajo del rango inferior existe un impacto
no lineal sobre el período de incubación extrínseca, y, en consecuencia, sobre la transmisión de la enfermedad,
mientras que por encima del rango superior de temperatura la transmisión se interrumpe\cite{dengueUruguayCap2}.

%Reescribir esta parte%
Si la temperatura aumenta, las larvas de Ae. aegypti necesitan menos tiempo para madurar\cite{dengueUruguayCap2} y,
en consecuencia, hay una mayor capacidad para producir más descendientes durante el período detransmisión. Por 
su parte, los mosquitos-hembra adultas digieren más rápidamente la sangre y se alimentan más frecuentemente
(Gillies, 1953), lo cual incrementa la intensidad de la transmisión. Por encima de 34oC generalmente se produce un
impacto negativo sobre la sobrevivencia del vector\cite{dengueUruguayCap2}. 
%Reescribir esta parte%

El incremento de la temperatura en algunas regiones del mundo podría permitir una mayor tasa de sobrevivencia del
vector en invierno y ayudar a extender su distribución a regiones previamente libres de la enfermedad, o a
aumentar la transmisión de la enfermedad en regiones endémicas, y también a cambiar las estaciones de transmisión.
Las temperaturas mínimas parecen ser las más críticas para el mosquito en muchas regiones por el umbral de
sobrevivencia y de desarrollo. Es también más baja la tasa de alimentación, lo cual reduce las posibilidades de
contacto con sus hospederos y eventualmente afecta la tasa de transmisión viral\cite{dengueUruguayCap2}. Las
condiciones del tiempo en los dos meses previos podrían ser críticas para la trasmisión del dengue en el mes en
curso\cite{dengueUruguayCap2}.
