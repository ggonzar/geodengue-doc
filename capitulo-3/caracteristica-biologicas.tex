
\section{Aedes Aegypti. Mosquito transmisor del dengue}
\label{sec:caracteristicas-biologicas}
El Aedes Aegypti  es el principal transmisor de la enfermedad del dengue. Es un mosquito que tiene como principales caracteristicas su contextura oscura y pequeña y sus manchas blancas en las extremidades. Mide apróximadamente 5mm. Al igual, evidentemente, que la enfermedad del dengue el mosquito se encuentra en zonas urbanas de regiones y paises tropicales.\\

Se estima que el mosquito causa mas de 50 millones de infecciones y mas de 26.000 muertes por año
Se recomiendan como método preventivo la utilización de repelentes que contengan DEET ya que son considerados como el mejor repelente contra el mosquito\\

\subsection{Biología}

El ciclo de vida del Aedes Aegypti es comun como otros mosquitos de especies similares. Basicamente consta de 4 estados:\\
\begin{itemize}
\item Huevo
\item Larva
\item Pupa
\item Adulto\\
\end{itemize}

Los huevos son depositados en las paredes de los recipientes por encima del nivel del agua. Los mismos son resitentes a temporadas de sequías largas. El contacto del agua con los huevos el paso a la siguiente etapa del mostiquito.\\

Los huevos eclosionan y pasan al estado de larvas. Las larvas son individuos móviles que se alimentan de residuos orgánicos encontrados en las paredes y fondos de los recipientes que los contienen. El paso a la siguiente etapa depende de la cantidad de alimentos, la temperatura y la densidad de larvas en el recipiente. Si se dan las condiciones se produce la metamorfosis y pasa al estado de pupa.\\

Las pupas no se alimentan, flotan en la superficie hasta alcanzar el estado adulto en aproximadamente 2 a 4 dias. El tiempo total de huevo a adulto es entre 7 a 14 dias. El tiempo es variable segun la duracion de cada cambio de estado.\\

El adulto se alimenta del nectar y aceites de plantas, pero la hembra del aedes aegipty necesita alimentarse de sangre para la formacion de los huevos y lo hace atravez de cualquier vertebrado teniendo preferencia por el humano.\\

Sus hábitos alimenticios pueden describirse por franjas horarias ya que se alimentan al amanecer y al atardecer. Buscan lugares oscuros de donde emerger y atacan a la victima en la piel cualquier parte del cuerpo  descubierta. Son muy habituales las picaduras en las manos, brazos, cuello, piernas. El contagio se produce cuando el mosquito hembra se alimenta de sangre humana infectada con el virus y luego pica a otro humano. No puede transmitirse de forma directa entre humano y humano\\

Su habitat para reproduccion y ovipostura son los lugares con agua estancada preferentemente limpia, lugares oscuros y quietos tales como latas, botellas vacías, neumáticos usados, baldes, etc.\\

\subsection{Cambios Climáticos}

Uno de los aspectos más importantes del mosquito Aedes Aegypti es su dependencia a la temperatura. Un país con temperatura tropical (promedio 25º C.) es un país ideal para la supervivencia del Aedes Aegypti no así un país con extremo calor o un clima más frío. Cada etapa de su desarrollo está ligado a condiciones climáticas; no solo temperatura sino también, lluvias y humedad. La lluvia permite que el agua se acumule en distintos recipientes; barriles, llantas y cubiertas, planteras, canaletas, etc. Se realizaron varios estudios analizando la influencia de la temperatura en el desarrollo del mosquito Aedes Aegypti. De los resultados de las pruebas se pueden obtener datos como el promedio de días en el que se pasa del estado larva a pupa ver Cuadro 1.\\

Esta información es muy valiosa en el estudio del mosquito ya que con el pronóstico del tiempo uno puede estimar el tiempo de desarrollo del Aedes Aegypti y determinar el crecimiento de la población actual (por ej. En 15 días aumentará la población actual del Aedes Aegypti en un 20\% dada las condiciones del clima previsto en esta zona)

\subsection{Ciclo gonadotrófico}
Después de cada alimentación se desarrolla un lote de huevos. Si la hembra completa su alimentación sanguínea (2-3 mg) desarrollará y pondrá 100-200 huevos, el intervalo dura de dos a tres días. La hembra grávida buscará recipientes oscuros o sombreados para depositar sus huevos, prefiriendo aguas limpias y claras.\\

\subsection{Rango de vuelo}
La hembra no sobrepasa los 50-100 m durante su vida (puede permanecer en la misma casa donde emergió). Si no hay recipientes, una hembra grávida puede volar tres kilómetros para poner sus huevos. Los machos se dispersan menos que las hembras.\\

\subsection{Conducta de reposo}
Descansan en lugares sombreados como alcobas, baños, patios o cocinas. Se les captura sobre ropas colgadas, debajo de muebles, toallas, cortinas y mosquiteros.\\

\subsection{Longevidad}
Los adultos pueden permanecer vivos en el laboratorio durante meses y en la naturaleza pocas semanas. Con una mortalidad diaria de 10\%, la mitad de los mosquitos morirán durante la primera semana y 95\% en el primer mes.\\


\begin{table}
\centering
\begin{tabular}{l|r}
Temperatura & Tiempo en estado larval y pupa \\\hline
13 & 0 \\
15-20 & 10 a 17.4 \\
20-25 & 9 a 13 \\
25-36 & 5 a 7 \\
36+ & 0
\end{tabular}
\caption{\label{tab:widgets}Tiempo promedio de duración en días del estado larval y pupa a diferentes temperaturas.}
\end{table}
