
\section{Aedes Aegypti. Mosquito transmisor del dengue}
\label{sec:caracteristicas-biologicas}

De todas las especies de mosquitos conocidos, con importancia en salud pública, \textit{Aedes Aegypti},
es considerada la más peligrosa por tener la capacidad de transmitir el mayor número de enfermedades 
arbovirales\footnote{Las infecciones arbovirales son los virus que se transmiten por los mosquitos. 
“Arbo” es una abreviatura que significa transmitida por los artrópodos, los cuales son insectos.}, al 
hombre.\cite{ThironIzcazaJ2003}

Por sus hábitos, el \textit{Aedes Aegypti}, es considerado como doméstico ya que se encuentra radicado en
criaderos naturales y artificiales en, viviendas humanas o en sus alrededores. Para establecer sus criaderos
necesita prácticamente de cualquier objeto que retenga agua.

\begin{itemize}
    \item  Recipientes artificiales :jarrones, floreros, tambos, pilas, tanques, cubetas, son los lugares más
    comunes para su cría, así como también aquellos que tienen la capacidad de retener agua de lluvia
    principalmente, tales como llantas, envases desechados y canales de techo, entre otros.
    \item Recipientes naturales : conchas de moluscos, cáscaras de frutos, huecos en los árboles, axilas de
    plantas y otras cavidades naturales.
\end{itemize}

Prefieren agua limpia, con bajo tenor orgánico y de sales disueltas. La puesta de huevos la realizan en la
superficie del recipiente. Algunos recipientes le son más atractivos que otros, en especial los de color oscuro,
de boca ancha, que están al nivel del suelo y se encuentran en la sombra\cite{ThironIzcazaJ2003}.

\subsection{Ciclo biológico del Aedes Aegypti}
Son insectos de metamorfosis completa. Durante su desarrollo ontogénico pasan por los estados de huevo, larva,
pupa y adulto\cite{web-site:gMonteroBiologia}.

\subsubsection{Huevo}
\label{subsec:ciclo-biologico-huevo}
Los huevos miden aproximadamente un milímetro de longitud, son depositados uno a uno al ras del agua quedando
adheridos a las paredes del recipiente\cite{ThironIzcazaJ2003}. Los mismos desarrollan una gran resistencia una
vez que han completado el desarrollo embrionario, el embrión dentro del huevo es capaz de resistir largos 
períodos de desecación por meses o hasta por más de un año. El contacto del agua con los huevos el paso a la
siguiente etapa del mosquito.

\subsubsection{Larva}
\label{subsec:ciclo-biologico-larva}
Las larvas que emergen inician un ciclo de 4 estadios larvales, son exclusivamente acuáticas la fase larval es el
período de mayor alimentación y crecimiento. Pasan la mayor parte del tiempo alimentándose de material orgánico
sumergido o acumulado en las paredes y el fondo del recipiente\cite{web-site:gMonteroBiologia}. Según 
\cite{ThironIzcazaJ2003} la duración del desarrollo larval está en función de la temperatura, la
disponibilidad de alimento y la densidad de larvas en el criadero. En condiciones óptimas, el período larval
desde la eclosión hasta la pupación puede ser de cinco días, pero por lo regular ocurre de siete a catorce 
días. Tanto \cite{ThironIzcazaJ2003} y \cite{web-site:gMonteroBiologia} afirman que los primeros tres estadios se
desarrollan  rápidamente, el cuarto se toma más tiempo aumentando considerablemente  su tamaño y peso, en
condiciones de baja temperatura o escasez de alimento el cuarto estadio puede prolongarse por varias semanas. 

En cuanto la mortalidad,\cite{ThironIzcazaJ2003} señala que la mortalidad, más elevada ocurre frecuentemente en
los primeros estadios larvales. La mayoría de los hábitats se encuentran en condiciones inestables y es posible
que esta sea las causa de mayor mortalidad de larvas y pupas.

\subsubsection{Pupa}
\label{subsec:ciclo-biologico-pupa}
Las pupas no se alimentan, su función es la metamorfosis del estadio larval al adulto. El estadio de pupa dura
aproximadamente dos o tres días, emergiendo alrededor del 88\% de los adultos en cuestión de 48 
horas\cite{ThironIzcazaJ2003}. Según \cite{web-site:gMonteroBiologia}, el período pupal dura de 1 a 3 días en
condiciones favorables, con temperaturas entre 28 y 32 \textcelsius. Las variaciones extremas de temperatura
pueden dilatar este período.

\subsubsection{Adulto}
\label{subsec:ciclo-biologico-adulto}

La función más importante del adulto de \textit{Aedes Aegypti} es la reproducción. El ciclo completo de 
huevo a adulto, se completa en óptimas condiciones de temperatura y alimentación en 10 
días\cite{web-site:gMonteroBiologia}. Pueden permanecer vivos en el laboratorio durante meses y en la naturaleza
pocas semanas. Con una mortalidad diaria de 10\%, la mitad de los mosquitos morirán durante la primera semana y 
95\% en el primer mes\cite{ThironIzcazaJ2003}. 

Las hembras se alimentan de sangre de cualquier vertebrado teniendo una marcada predilección por la del hombre.
Necesitan alimentarse de sangre para obtener las proteínas necesarias para la formación de los huevos. Las partes
bucales del macho no están adaptadas para chupar sangre, se alimentan de carbohidratos de cualquier fuente
accesible como frutos o néctar de flores que satisface sus requerimientos energéticos, las hembras también se
alimentan de esta misma fuente como complemento indispensable\cite{ThironIzcazaJ2003}.

Su hábitat para reproducción y ovipostura son los lugares con agua estancada preferentemente limpia, lugares
oscuros y quietos tales como latas, botellas vacías, neumáticos usados, baldes, etc.

Según \cite{ThironIzcazaJ2003}, antes de 24 horas ambos sexos están listos para el apareamiento, alrededor del 
58\% de las hembras nulíparas son inseminadas antes de su primera alimentación sanguínea, un 17\% durante y el 
25\% es inseminada entre la segunda alimentación y la primera oviposición; los machos rondan como voladores
solitarios aunque es más común que lo hagan en grupos pequeños atraídos por los mismos huéspedes vertebrados 
que las hembras. La alimentación y la postura ocurren principalmente durante el día registrando mayor actividad
en las primeras horas de haber amanecido, a media mañana, a media tarde o al anochecer. El apareamiento que por
lo general se efectúa durante el vuelo debido a que el macho es atraído por el sonido emitido por las alas de la
hembra, una alimentada de sangre ocurren pocos apareamientos. Una inseminación es suficiente para fecundar todos
los huevos que la hembra produzca en toda su vida\cite{ThironIzcazaJ2003}. 

Es común que después de cada alimentación sanguínea la hembra desarrolle un lote de huevos, la cantidad de 
huevos depende de la alimentación según \cite{cabezas2005dengue} puede variar entre 100 a 200 huevos. El
intervalo de tiempo que transcurre entre la alimentación sanguínea y la postura, denominado ciclo gonotrófico, 
es de 48 horas en los trópicos bajo condiciones óptimas de temperatura\cite{ThironIzcazaJ2003}. 

Los mosquitos tienen la particularidad de volar en sentido contrario al la dirección al viento a una
velocidad máxima de $2 km/h$ según \cite{web-site:speedAnimals}. En cuanto al desplazamiento, según 
\cite{cabezas2005dengue} la hembra no sobrepasa los 50-100 metros durante su vida, tiende a permanecer en el
mismo lugar donde emergió. Sin embargo si no hay recipientes aptos, una hembra grávida puede volar tres
kilómetros para poner sus huevos. Los machos se dispersan menos que las hembras.

\subsection{Cambios Climáticos}
Uno de los aspectos más importantes del mosquito Aedes Aegypti es su dependencia a la temperatura. Un país con temperatura tropical (promedio 25\textcelsius.) es un país ideal para la supervivencia del Aedes Aegypti no así un país con extremo calor o un clima más frío. Cada etapa de su desarrollo está ligado a condiciones climáticas; no solo temperatura sino también, lluvias y humedad. La lluvia permite que el agua se acumule en distintos recipientes; barriles, llantas y cubiertas, planteras, canaletas, etc. Se realizaron varios estudios analizando la influencia de la temperatura en el desarrollo del mosquito Aedes Aegypti. De los resultados de las pruebas se pueden obtener datos como el promedio de días en el que se pasa del estado larva a pupa ver Cuadro 1.\\

Esta información es muy valiosa en el estudio del mosquito ya que con el pronóstico del tiempo uno puede estimar el tiempo de desarrollo del Aedes Aegypti y determinar el crecimiento de la población actual (por ej. En 15 días aumentará la población actual del Aedes Aegypti en un 20\% dada las condiciones del clima previsto en esta zona)

\begin{table}
\centering
\begin{tabular}{l|r}
Temperatura & Tiempo en estado larval y pupa \\\hline
13 & 0 \\
15-20 & 10 a 17.4 \\
20-25 & 9 a 13 \\
25-36 & 5 a 7 \\
36+ & 0
\end{tabular}
\caption{\label{tab:widgets}Tiempo promedio de duración en días del estado larval y pupa a diferentes temperaturas.}
\end{table}
