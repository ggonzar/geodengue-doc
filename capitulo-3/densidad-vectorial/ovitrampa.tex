\section{Ovitrampa}
\label{sec:densidad-vectorial-ovitrampa}
Son recipientes que ofrecen a las hembras de Aedes aegypti un lugar colocar
los huevos. Detecta la presencia de huevos y por lo tanto actividad de
ovipostura. Las ovitrampas consisten en frascos de plástico o pequeñas
macetas plásticas de unos 500 ml de color oscuro preferentemente, en cuyo
interior, se coloca una pieza plana de madera (baja-lengua o similar).
Asimismo, también pueden construirse con un pote de vidrio de boca ancha,
de aprox. medio litro, pintado de negro por fuera y equipado con una paleta
de cartón o madera (baja-lengua) sujeta verticalmente al interior, con su
lado áspero mirando hacia adentro. Las dimensiones del recipiente no son
críticas pero todos los frascos a usar en un estudio particular deben ser
idénticos. Al frasco se le deberá agregar 250 ml de agua limpia.
