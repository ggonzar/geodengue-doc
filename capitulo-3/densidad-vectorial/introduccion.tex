\section{Métodos de muestreo de la abundancia poblacional}
\label{sec:densidad-vectorial-introduccion}

Está ampliamente aceptado que la vigilancia de Ae. aegypti es un aspecto
muy importante en la lucha contra el dengue. Esta afirmación se basa en
la asunción de que existe una correlación positiva entre la densidad del
vector y la enfermedad humana\cite{dengueUruguayCap2}.

Muchos programas de control del dengue se basan el la utilización de
índices larvarios o índices de Stegomia, como indicadores de las densidades
poblaciones de Aedes aegypti, para dirigir y focalizar espacial y temporalmente
las acciones de control del vector.
