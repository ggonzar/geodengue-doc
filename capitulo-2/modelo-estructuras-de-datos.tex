\section{Modelo y estructura de datos}
\label{sec:cap2-modelo-estructura-datos}

A lo largo del tiempo, el hombre ha tratado de representar la superficie terrestre y los objetos que esta alberga.
Los primeros mapas tenían como objetivo, servir de apoyo para la navegación, indicando los rumbos a seguir para
desplazarse de un puerto a otro. Se caracterizaban por brindar exactitud en los rumbos y las distancias existente 
entre los puertos, no así por su exactitud en la representación de las tierras.

Con el trascurrir del tiempo ya no era suficiente con llegar de un puerto a otro, surgiendo así la necesidad de mejorar
la medición de las distancias y superficies sobre los nuevos territorios para conseguir un mejor dominio 
sobre estos. Además se fueron incluyendo diversos elementos como recursos y factores ambientales de la superficie
terrestre para mejorar la visión de la distribución de los fenómenos naturales y asentamientos humanos sobre 
la superficie terrestre.

Durante el siglo XVII, cartógrafos especializados como Mercator demostraron que no sólo el uso de un sistema de
proyección matemático y un ajustado sistema de coordenadas mejoraba la fiabilidad de las medidas y la localización de
las áreas de tierra, sino que el registro de fenómenos espaciales a través de un modelo convenido de distribución de
fenómenos naturales y asentamientos humanos era de un valor incalculable para la navegación, para la búsqueda de rutas
y en la estrategia militar \cite{llopis2006sistemas}.

\subsection{Coordenadas geográficas}
Las coordenadas geográficas son un sistema de referencia que utilizan coordenadas angulares (latitud y longitud) 
cuyo fin es el de determinar los ángulos laterales de la superficie terrestre. La latitud es el ángulo que existe
entre un punto cualquiera y el Ecuador, medida sobre el meridiano que pasa por dicho punto. La longitud mide el ángulo 
a lo largo del ecuador desde cualquier punto de la Tierra.

Para la representación de objetos puntuales en una superficie, se utilizan las coordenadas X e Y que caracterizan la 
planimetría y una coordenada Z que representa la altimertría del objeto en cuestión. 

\subsection{Proyecciones}
Se denomina proyecciones cartográficas o proyecciones geográficas al conjunto de métodos utilizados para establecer
una correspondencia matemática entre los puntos de la superficie curva de la tierra y sus transformaciones en una
superficie plana. 

El problema principal a la hora de realizar una proyección es que no existe forma de representar un una superficie
plana toda la superficie curva, de la tierra, sin deformarla. Teniendo en cuenta que la curvatura de la superficie terrestre
es proporcional al tamaño del área representada, esta deformación solo es relevante para zonas muy amplias. Si el área
representada es pequeña, como ciudad, la deformación o distorsión es despreciable por lo que comúnmente son utilizadas
coordenadas planas, relativas a un origen de coordenadas arbitrario y medidas sobre el terreno.

Con la aparición y difusión de los SIG, el conjunto de herramientas que ofrece dio paso a posibilidad
de combinar información de diferentes mapas con diferentes proyecciones, esto ha incrementado la relevancia de la cartografía 
más allá de la simple confección de mapas.

\subsection{Elementos de representación cartográfica}
Para representar un objeto cualquiera o un fenómeno geográfico en un mapa es fundamental conocer las características 
de este dato que contempla los tres aspectos siguientes: dimensiones, nivel de medida y distribución. El análisis de las 
características de los elementos permite elegir la simbología más adecuada para representar los fenómenos geográficos.

A cada entidad espacial se puede asociar diversas variables han desarrollado un amplio conjunto de técnicas para cartografiar
los hechos de la superficie
terrestre.

\subsubsection{Dimensiones}
Por su extensión, los fenómenos que se representan en un mapa pueden clasificarse como puntuales, lineales, poligonales 
o espacio-temporales.

\begin{itemize}
    \item Fenómenos puntuales : indican la presencia de entidades de un modo puntual. Pueden representarse utilizando 
    diferentes símbolos o colores para una variable cualitativa, o diferentes tamaños para variables cuantitativas.
    
    \item Fenómenos lineales : simbolizan entidades, naturales o artificiales, de forma lineal conformadas a partir de
    la unión de varios puntos. Pueden utilizarse diferentes anchuras de linea, diferentes colores o diferentes tipos de
    linea para representar propiedades como la anchura de los ríos o categorías de vías de comunicación.
    
    \item Fenómenos Poligonales : La información puede ser bidimensional o tridimensional, que, por su tamaño, pueden 
    ser representados como polígonos o porciones homogéneas del terreno en relación a una variable cualitativa. 
    Pueden utilizarse diferentes colores o tramas para representar variables cualitativas o cuantitativas.

    \item Fenómenos espacio-temporales : La información depende del movimiento del fenómeno con respecto al paso del
    tiempo (migraciones de aves, expansión de una civilización,etc.).
\end{itemize}

\subsubsection{Nivel de medida}
Los elementos de la naturaleza se miden con el fin de clasificarlos y compararlos; lo que no siempre implica una
magnitud cuantitativa, ya que puede ser cualitativa u jerárquica. En orden creciente de precisión tenemos:
\begin{itemize}
    \item Escala nominal : asigna una característica no numérica a un fenómeno, por lo que sólo se pueden hacer 
    comparaciones de tipo cualitativo. Por ejemplo, un mapa de cuencas hidrográficas, un mapa de suelos. 
    Este es el nivel más elemental de medida, pues no informa acerca de la cantidad o el orden.
    
    \item Escala ordinal : establece una cierta jerarquía no mensurable o no cuantificable entre los diferentes
    elementos. Por ejemplo, un mapa en el que aparecen núcleos de población, cuyos símbolos están jerarquizados
    según el número de habitantes sin especificar cantidad.
    
    \item Escala cuantitativa o de intervalo : La escala cuantitativa o de intervalo asigna una característica 
    numérica a un fenómeno geográfico. Por ejemplo, en un mapa de temperaturas medias los intervalos son valores 
    numéricos (expresados en grados Celsius o Fahrenheit). Es necesario emplear algún tipo de unidad convencional.
\end{itemize}

\subsubsection{Distribución}
La ocurrencia de un fenómeno sobre una superficie terrestre puede darse a lo largo y ancho de toda ella, o de forma 
discontinua, dándose el fenómeno en alguna localización del territorio. Los fenómenos en cuestión son :

\begin{itemize}
    \item Fenómenos continuos :  son los que tienen presencia en todos los puntos del territorio objeto de 
    representación, aunque sólo se tengan medidas de algunos puntos significativos. Por ejemplo: la temperatura,
    altitud sobre en nivel del mar, niveles de contaminación atmosférica,densidad poblacional, etc.
    
    \item Fenómenos discretos : son los que tiene presencia en algunos puntos del territorio objeto de 
    representación. Un ejemplo son los datos de población, dado que se localizan en determinadas áreas y 
    no en todos los puntos del territorio. Algunos fenómenos discretos pueden transformarse en continuos mediante 
    la aplicación de una relación. Por ejemplo, el número de habitantes de una provincia (fenómeno discreto) pasa a 
    ser un fenómeno continuo cuando se habla de densidad de población: la relación se aplica dividiendo el número 
    de habitantes por la superficie de la provincia en km^2
\end{itemize}
