\section{Modelo y estructura de datos}
\label{sec:cap2-modelo-estructura-datos}

A lo largo del tiempo, el hombre ha tratado de representar la superficie terrestre y los objetos que esta alberga.
Los primeros mapas tenían como objetivo, servir de apoyo para la navegación, indicando los rumbos a seguir para
desplazarse de un puerto a otro. Se caracterizaban por brindar exactitud en los rumbos y las distancias existente 
entre los puertos, no así por su exactitud en la representación de las tierras.

Con el trascurrir del tiempo ya no era suficiente con llegar de un puerto a otro, surgiendo así la necesidad de mejorar
la medición de las distancias y superficies sobre los nuevos territorios para conseguir un mejor dominio 
sobre estos. Además se fueron incluyendo diversos elementos como recursos y factores ambientales de la superficie
terrestre para mejorar la visión de la distribución de los fenómenos naturales y asentamientos humanos sobre 
la superficie terrestre.

Durante el siglo XVII, cartógrafos especializados como Mercator demostraron que no sólo el uso de un sistema de
proyección matemático y un ajustado sistema de coordenadas mejoraba la fiabilidad de las medidas y la localización de
las áreas de tierra, sino que el registro de fenómenos espaciales a través de un modelo convenido de distribución de
fenómenos naturales y asentamientos humanos era de un valor incalculable para la navegación, para la búsqueda de rutas
y en la estrategia militar \cite{llopis2006sistemas}.

\subsection{Coordenadas geográficas}
Las coordenadas geográficas son un sistema de referencia que utilizan coordenadas angulares (latitud y longitud) 
cuyo fin es el de determinar los ángulos laterales de la superficie terrestre. La latitud es el ángulo que existe
entre un punto cualquiera y el Ecuador, medida sobre el meridiano que pasa por dicho punto. La longitud mide el ángulo 
a lo largo del ecuador desde cualquier punto de la Tierra.
Para la representación de objetos puntuales en una superficie, se utilizan las coordenadas X e Y que caracterizan la 
planimetría y una coordenada Z que representa la altimertría del objeto en cuestión. 

\subsection{Proyecciones}
Se denomina proyecciones cartográficas o proyecciones geográficas al conjunto de métodos utilizados para establecer
una correspondencia matemática entre los puntos de la superficie curva de la tierra y sus transformaciones en una
superficie plana. 

El problema principal a la hora de realizar una proyección es que no existe forma de representar un una superficie
plana toda la superficie curva, de la tierra, sin deformarla. Teniendo en cuenta que la curvatura de la superficie terrestre
es proporcional al tamaño del área representada, esta deformación solo es relevante para zonas muy amplias. Si el área
representada es pequeña, como ciudad, la deformación o distorsión es despreciable por lo que comúnmente son utilizadas
coordenadas planas, relativas a un origen de coordenadas arbitrario y medidas sobre el terreno.

Con la aparición y difusión de los SIG, el conjunto de herramientas que ofrece dio paso a posibilidad
de combinar información de diferentes mapas con diferentes proyecciones, esto ha incrementado la relevancia de la cartografía 
más allá de la simple confección de mapas.

\subsection{Elementos de representación cartográfica}
A cada entidad espacial se puede asociar diversas variables (binomiales, cualitativas, ordinales o cuantitativas).
Por ejemplo, a una carretera se puede asociar su anchura, categoría o flujo de vehículos; a un municipio población, 
renta, etc.; a un pozo la cantidad de agua extraída al año, el nivel del agua o su composición. Normalmente
al representar una entidad se representará también alguna de las variables asociadas a ella.
El conjunto de ciencias involucradas en la producción de mapas (Geodesia, Cartografía, Geografía, Geología,
Ecología, etc.) han desarrollado un amplio conjunto de técnicas para cartografiar los hechos de la superficie
terrestre.
\begin{itemize}

\item Isolineas. Son lineas que unen puntos con igual valor, sirven por tanto para cartografiar variables cuan-
titativas. Un buen ejemplo son las curvas de nivel del mapa topográfico o las isobaras de los mapas del
tiempo.

\item Coropletas. Areas con valor comprendido entre dos umbrales y pintadas con un color homogeneo. Per-
miten representar variables cuantitativas de un modo más simplificado.

\item Símbolos para indicar la presencia de entidades de un modo puntual. Pueden repesentarse utilizando di-
ferentes símbolos o colores para representar una variable cualitativa (por ejemplo el partido gobernante),
o diferentes tamaños para representar variables cuantitativas (por ejemplo el número de habitantes).

\item Lineas que simbolizan entidades, naturales o artificiales, de forma lineal (carreteras, ríos). Pueden utili-
zarse diferentes anchuras de linea, diferentes colores o diferentes tipos de linea para representar propie-
dades como la anchura de los ríos o categorías de vías de comunicación.

\item Poligonos representan objetos poligonales que, por su tamaño, pueden ser representados como tales
(siempre dependiendo de la escala del mapa) o porciones homogeneas del terreno en relación a una
variable cualitativa (tipo de roca). Pueden utilizarse diferentes colores o tramas para representar variables
cualitativas o cuantitativas, por ejemplo en un mapa de municipios se puede representar la población
municipal mediante sombreados.
\end{itemize}

En catografía, suele distinguirse entre mapas topográficos, considerados de propósito general, y mapas temá-
ticos (geológicos, vegetación, etc.) que reflejan un sólo aspecto de la realidad. Los mapas, especialmente los
topográficos (figura ??), tratan de reflejar el máximo número de elementos potencialmente interesantes para el
usuario, evitando llegar a confundirle por exceso de información. Una de las estrategias empleadas para ello
es eliminar parte de la información (por ejemplo una curva de nivel que cruza una población) confiando en
que la capacidad de nuestro cerebro para reconstruir objetos a partir de información parcial. Esta estrategia se
denomina generalización.
De este modo un mapa deja en ocasiones de ser un modelo de la superficie terrestre para ser una representación
visual que incluye información variada y no totalmente estructurada.
