\section{Definición de Sistema de información geográfica}
\label{sec:cap2-definicion-sig}

Un Sistema de Información geográfica (SIG) es la integración organizada de hardware, software y datos geográficos
diseñada para almacenar, manejar, capturar, analizar y desplegar la información geográficamente de múltiples
formas, con el fin de resolver problemas de planificación y gestión geográfica. También puede definirse como un
modelo de una parte de la realidad referido a un sistema de coordenadas terrestre y construido para satisfacer 
unas necesidades concretas de información \cite{lopezMarcos2007}. Su fundamentación se basa en principios formales
de matemáticas discretas, modelos de datos y geometría computacional; su desarrollo,en nuevas tecnologías de la
información: estándares e ingeniería de software, almacenes de datos, Web-SIG, metadatos, ambientes y lenguajes
visuales, graficación entre muchas otras \cite{lunaPaulina2010}.

La característica principal de los SIG es el manejo de datos complejos basados en datos geométricos (coordenadas e
información topológica) y datos de atributos (información nominal) la cual describe las propiedades de los objetos
geométricos tales como punto, lineas y polígonos.
