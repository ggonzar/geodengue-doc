\section{Definición de Sistema de información geográfica}
\label{sec:cap2-definicion-sig}

Un Sistema de Información geográfica (SIG) es la integración organizada de hardware, software y datos geográficos
diseñada para almacenar, manejar, capturar, analizar y desplegar la información geográficamente de múltiples
formas, con el fin de resolver problemas de planificación y gestión geográfica. También puede definirse como un
modelo de una parte de la realidad referido a un sistema de coordenadas terrestre y construido para satisfacer 
unas necesidades concretas de información \cite{lopezMarcos2007}. Su fundamentación se basa en principios formales
de matemáticas discretas, modelos de datos y geometría computacional; su desarrollo,en nuevas tecnologías de la
información: estándares e ingeniería de software, almacenes de datos, Web-SIG, metadatos, ambientes y lenguajes
visuales, graficación entre muchas otras \cite{lunaPaulina2010}.

La característica principal de los SIG es el manejo de datos complejos basados en datos geométricos (coordenadas e
información topológica) y datos de atributos (información nominal) la cual describe las propiedades de los objetos
geométricos tales como punto, lineas y polígonos.

En la actualidad las funciones básicas, y más habitualmente utilizadas, de un SIG son el almacenamiento,
visualización, consulta y análisis de datos espaciales. Un uso algo más avanzado sería la utilización 
de un SIG para la toma de decisiones en ordenación territorial o para la modelización de procesos ambientales.

\begin{itemize}
    \item  Almacenamiento : el almacenamiento de datos espaciales implica modelizar la realidad y codificar de
    forma cuantitativa este modelo.

    \item Visualización : La información se presenta en un espacio de cuatro dimensiones (3 espaciales y el tiempo) 
    pero debido al peso que la tradición cartográfica tiene sobres los SIG, una de las formas prioritarias de 
    presentación de los datos es en su proyección sobre el espacio bidimensional definido mediante coordenadas cartesianas.
    
    \item Consultas  : en una base de datos, las consulta se basan en propiedades temáticas,mientras que en un SIG 
    las consultas se basan tanto en atributos temáticos como en propiedades espaciales.

    \item Análisis :  el uso de herramientas de análisis espacial y álgebra de mapas para el desarrollo y
    verificación de hipótesis acerca de la distribución espacial de las variables y objetos.

    \item Toma de decisiones : la utilización de un SIG para resolver problemas de toma de decisión en
    planificación física, ordenación territorial, estudios de impacto ambiental, etc.

    \item Modelización : las aplicaciones más elaboradas de los SIG son aquellas relacionadas con la integración 
    de modelos matemáticos de procesos naturales, dinámicos y espacialmente distribuidos.
\end{itemize}
