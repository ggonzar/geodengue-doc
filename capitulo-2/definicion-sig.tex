\section{Definición de Sistema de información geográfica}
\label{sec:cap2-definicion-sig}

Un Sistema de Información geográfica (SIG) es la integración organizada de hardware, software y datos geográficos
diseñada para almacenar, manejar, capturar, analizar y desplegar la información geográficamente en múltiples
formas, con el fin de resolver problemas de planificación y gestión geográfica. También puede definirse como un
modelo de una parte de la realidad referido a un sistema de coordenadas terrestre y construido para satisfacer 
unas necesidades concretas de información \cite{lopezMarcos2007}.

El potencial de aplicaciones se diversifica en: biología, energía e infraestructura, planeación urbana y regional,
monitoreo ambiental y geografía física, transportación y logística.
Su fundamentación se basa en principios formales de matemáticas discretas, modelos de datos y geometría
computacional; su desarrollo,en nuevas tecnologías de la información: estandares e ingeniería de software, bodegas
de datos, Web-SIG, metadatos, ambientes y lenguajes visuales, graficación entre muchas otras \cite{lunaPaulina2010}.
