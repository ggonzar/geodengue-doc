A lo largo del tiempo, el hombre ha tratado de representar la superficie terrestre y los objetos que esta alberga.
Los primeros mapas tenían como principal objetivo, ser una herramienta de apoyo para la navegación, indicando así
los rumbos a seguir permitiendo al hombre desplazarse de un puerto a otro. Se caracterizaban por brindar exactitud
en los rumbos y las distancias existente entre los puertos, no así por su exactitud en la representación de las
tierras. 

Con el trascurrir del tiempo ya no era suficiente con llegar de un puerto a otro, surgiendo así la
necesidad de mejorar precisión en la medición de las distancias y superficies sobre los nuevos territorios para
conseguir un mejor dominio sobre estos. De forma adicional se fueron incluyendo diversos elementos como recursos y
factores ambientales de la superficie terrestre para mejorar la visión de la distribución de los fenómenos
naturales y asentamientos humanos sobre la superficie terrestre.


APARICION DE LOS GIS.



El comienzo del siglo XX vio el desarrollo de la "foto litografía" donde los mapas eran separados en capas. El avance del hardware impulsado por la investigación en armamento nuclear daría lugar, a comienzos de los años 60, al desarrollo de aplicaciones cartográficas para computadores de propósito general.6
El año 1962 vio la primera utilización real de los SIG en el mundo, concretamente en Ottawa (Ontario, Canadá) y a cargo del Departamento Federal de Silvicultura y Desarrollo Rural. Desarrollado por Roger Tomlinson, el llamado Sistema de información geográfica de Canadá (Canadian Geographic Information System, CGIS) fue utilizado para almacenar, analizar y manipular datos recogidos para el Inventario de Tierras Canadá (Canada Land Inventory, CLI) - una iniciativa orientada a la gestión de los vastos recursos naturales del país con información cartográfica relativa a tipos y usos del suelo, agricultura, espacios de recreo, vida silvestre, aves acuáticas y silvicultura, todo ello escala de 1:50.000. Se añadió, así mismo, un factor de clasificación para permitir el análisis de la información.
El Sistema de información geográfica de Canadá fue el primer SIG en el mundo similar a tal y como los conocemos hoy en día, y un considerable avance con respecto a las aplicaciones cartográficas existentes hasta entonces, puesto que permitía superponer capas de información, realizar mediciones y llevar a cabo digitalizaciones y escaneos de datos. Asimismo, soportaba un sistema nacional de coordenadas que abarcaba todo el continente, una codificación de líneas en "arcos" que poseían una verdadera topológica integrada y que almacenaba los atributos de cada elemento y la información sobre su localización en archivos separados. Como consecuencia de esto, Tomlinson está considerado como "el padre de los SIG", en particular por el empleo de información geográfica convergente estructurada en capas, lo que facilita su análisis espacial.7 El CGIS estuvo operativo hasta la década de los 90 llegando a ser la base de datos sobre recursos del territorio más grande de Canadá. Fue desarrollado como un sistema basado en una computadora central y su fortaleza radicaba en que permitía realizar análisis complejos de conjuntos de datos que abarcaban todo el continente. El software, decano de los sistemas de información geográfica, nunca estuvo disponible de manera comercial.
En 1964, Howard T. Fisher formó en la Universidad de Harvard el Laboratorio de Computación Gráfica y Análisis Espacial en la Harvard Graduate School of Design (LCGSA 1965-1991), donde se desarrollaron una serie de importantes conceptos teóricos en el manejo de datos espaciales, y en la década de 1970 había difundido código de software y sistemas germinales, tales como SYMAP, GRID y ODYSSEY - los cuales sirvieron como fuentes de inspiración conceptual para su posterior desarrollos comerciales - a universidades, centros de investigación y empresas de todo el mundo.8
En la década de los 80, M&S Computing (más tarde Intergraph), Environmental Systems Research Institute (ESRI) y CARIS (Computer Aided Resource Information System) emergerían como proveedores comerciales de software SIG. Incorporaron con éxito muchas de las características de CGIS, combinando el enfoque de primera generación de sistemas de información geográfica relativo a la separación de la información espacial y los atributos de los elementos geográficos representados con un enfoque de segunda generación que organiza y estructura estos atributos en bases de datos.
En la década de los años 70 y principios de los 80 se inició en paralelo el desarrollo de dos sistemas de dominio público. El proyecto Map Overlay and Statistical System (MOSS) se inició en 1977 en Fort Collins (Colorado, EE. UU.) bajo los auspicios de la Western Energy and Land Use Team (WELUT) y el Servicio de Pesca y Vida Silvestre de Estados Unidos (US Fish and Wildlife Service). En 1982 el Cuerpo de Ingenieros del Laboratorio de Investigación de Ingeniería de la Construcción del Ejército de los Estados Unidos (USA-CERL) desarrolla GRASS como herramienta para la supervisión y gestión medioambiental de los territorios bajo administración del Departamento de Defensa.
Esta etapa de desarrollo está caracterizada, en general, por la disminución de la importancia de las iniciativas individuales y un aumento de los intereses a nivel corporativo, especialmente por parte de las instancias gubernamentales y de la administración.
Los 80 y 90 fueron años de fuerte aumento de las empresas que comercializaban estos sistemas, debido el crecimiento de los SIG en estaciones de trabajo UNIX y ordenadores personales. Es el periodo en el que se ha venido a conocer en los SIG como la fase comercial. El interés de las distintas grandes industrias relacionadas directa o indirectamente con los SIG crece en sobremanera debido a la gran avalancha de productos en el mercado informático internacional que hicieron generalizarse a esta tecnología.
En la década de los noventa se inicia una etapa comercial para profesionales, donde los sistemas de información geográfica empezaron a difundirse al nivel del usuario doméstico debido a la generalización de los ordenadores personales o microordenadores.
A finales del siglo XX principio del XXI el rápido crecimiento en los diferentes sistemas se ha consolidado, restringiéndose a un número relativamente reducido de plataformas. Los usuarios están comenzando a exportar el concepto de visualización de datos SIG a Internet, lo que requiere una estandarización de formato de los datos y de normas de transferencia. Más recientemente, ha habido una expansión en el número de desarrollos de software SIG de código libre, los cuales, a diferencia del software comercial, suelen abarcar una gama más amplia de sistemas operativos, permitiendo ser modificados para llevar a cabo tareas específicas.
