A lo largo del tiempo, el hombre ha tratado de representar la superficie terrestre y los objetos que esta alberga.
Los primeros mapas tenían como principal objetivo, ser una herramienta de apoyo para la navegación, indicando así
los rumbos a seguir permitiendo al hombre desplazarse de un puerto a otro. Se caracterizaban por brindar exactitud
en los rumbos y las distancias existente entre los puertos, no así por su exactitud en la representación de las
tierras. Con el trascurrir del tiempo ya no era suficiente con llegar de un puerto a otro, surgiendo así la
necesidad de mejorar precisión en la medición de las distancias y superficies sobre los nuevos territorios para
conseguir un mejor dominio sobre estos. De forma adicional se fueron incluyendo diversos elementos como recursos y
factores ambientales de la superficie terrestre para mejorar la visión de la distribución de los fenómenos
naturales y asentamientos humanos sobre la superficie terrestre.
APARICION DE LOS GIS.
