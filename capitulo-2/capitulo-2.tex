\chapter{Sistemas de Información Geográfica}
Con el transcurrir del tiempo se ha tratado de representar la superficie terrestre y los elementos
que esta alberga. Los primeros mapas, que no presentaban un alto grado de exactitud, eran
utilizados como herramienta para la navegación. Con el paso del tiempo los requerimientos fueron
creciendo, surgiendo así la necesidad de mejorar la precisión en las mediciones y la inclusión de
elementos adicionales con el fin de modelar, recursos, fenómenos naturales y asentamientos humanos
en la superficie.

Los sistemas de información geográfica o SIG, surgen ante la necesidad de registrar, procesar y
analizar la información geográfica de forma más eficiente. Se encuentra principalmente compuesto
por la información geográfica y herramientas correspondientes para su gestión y análisis. En este
capítulo se introduce a los sistemas de información geográfica, la representación de datos
geoespaciales y los métodos de interpolación como herramienta para análisis espacial.

%##Capítulo 2. Sistemas de Información Geográfica
%* Introducción
%* Estructuras de datos de los Sistemas de Información Geográfica
\section{Definición de Sistema de información geográfica}
\label{sec:cap2-definicion-sig}

Un Sistema de Información geográfica (SIG) es la integración organizada de hardware, software y datos geográficos
diseñada para almacenar, manejar, capturar, analizar y desplegar la información geográficamente de múltiples
formas, con el fin de resolver problemas de planificación y gestión geográfica. También puede definirse como un
modelo de una parte de la realidad referido a un sistema de coordenadas terrestre y construido para satisfacer 
unas necesidades concretas de información \cite{lopezMarcos2007}. Su fundamentación se basa en principios formales
de matemáticas discretas, modelos de datos y geometría computacional; su desarrollo,en nuevas tecnologías de la
información: estándares e ingeniería de software, almacenes de datos, Web-SIG, metadatos, ambientes y lenguajes
visuales, graficación entre muchas otras \cite{lunaPaulina2010}.

La característica principal de los SIG es el manejo de datos complejos basados en datos geométricos (coordenadas e
información topológica) y datos de atributos (información nominal) la cual describe las propiedades de los objetos
geométricos tales como punto, lineas y polígonos.

%* Sistemas de proyección
\section{Sistemas de proyección}
\label{sec:cap2-sistemas-de-proyeccion}
Durante el siglo XVII, cartógrafos especializados, como Mercator, demostraron que no sólo el uso de
un sistema de proyección matemático y un ajustado sistema de coordenadas mejoraba la fiabilidad de
las medidas y la localización de las áreas de tierra, sino que el registro de fenómenos espaciales
a través de un modelo convenido de distribución de fenómenos naturales y asentamientos humanos era
de un valor incalculable para la navegación, para la búsqueda de rutas y en la estrategia militar
\citep{llopis2006sistemas}.

\subsection{Coordenadas geográficas}
Las coordenadas geográficas proveen un sistema de referencia, que se basan en la utilización de
coordenadas angulares como latitud y longitud (\figref{fig:sig-plano}). Su objetivo es el de
determinar los ángulos laterales de la superficie terrestre. Se denomina latitud al ángulo que
existe entre un punto cualquiera y el Ecuador, medida sobre el meridiano que pasa por dicho punto
\citep{fAlonsoSig2006}. La longitud mide el ángulo a lo largo de la línea del Ecuador desde
cualquier punto de la Tierra\citep{fAlonsoSig2006}.

\begin{figure}[!htbp]
\centering
\includegraphics[width=0.7\textwidth]{capitulo-2/graphics/ejes-tierra.png}
\caption{\label{fig:sig-plano} Representación de los meridianos, paralelos, longitud y latitud en la superfice (Tomado de \cite{website:emcLonLat2014}).}
\end{figure}


Para la representación de objetos puntuales en una superficie, se utilizan las coordenadas $X$ e
$Y$ que caracterizan la planimetría y una coordenada $Z$ representa la altimertría del objeto
puntual en cuestión. En la \figref{fig:sig-xyz} se puede apreciar la representación de un objeto
puntual en una sistema tridimensional.

\begin{figure}[!htbp]
\centering
\includegraphics[width=0.5\textwidth]{capitulo-2/graphics/coordenadas-xyz.jpg}
\caption{\label{fig:sig-xyz} Representación de un objeto puntual en un sistema tridimensional
 $(x,y,z)$.}
\end{figure}

\subsection{Proyecciones}
Las proyecciones cartográficas o proyecciones geográficas se encuentran descritas por el conjunto
de métodos utilizados para establecer una correspondencia matemática entre los puntos pertenecientes a la superficie curva de la tierra y sus transformaciones en una superficie plana.

El problema principal a la hora de realizar una proyección es que no existe forma de representar
una superficie plana toda la superficie curva, de la tierra, sin deformarla siendo el objetivo
principal minimizar las deformaciones en la medida que sea posible \citep{fAlonsoSig2006}.
Teniendo en cuenta que la curvatura de la superficie terrestre es proporcional al tamaño del área
representada \citep{llopis2006sistemas}, si el área o región que se desea representar es pequeña,
entonces la deformación o distorsión resultante es despreciable, por lo que puede ser modelada con
coordenadas planas.

Con la aparición y difusión de los SIG, el conjunto de herramientas que ofrece se dio paso a
posibilidad de combinar información de diferentes mapas con diferentes proyecciones, esto ha
incrementado la relevancia de la cartografía más allá de la simple confección de mapas
\citep{llopis2006sistemas}.

\subsection{Elementos de representación cartográfica}
La representación de los objetos, recursos o fenómenos geográficos en una superficie o mapa, se
debe identificar los rasgos característicos teniendo en cuenta, principalmente, tres aspectos:
sus dimensiones, el nivel de medida y la distribución \citep{fomentoConceptos2010}. El análisis de
las características de estos elementos permiten seleccionar, adecuadamente, los símbolos a
utilizar para representar los fenómenos geográficos correspondientes \citep{fomentoConceptos2010}.
A cada entidad espacial puede ser asociada a diversas variables, para representar estas entidades
como hechos de la superficie terrestre, se han desarrollado un amplio conjunto de técnicas para
cartografiarlos.

\subsubsection{Dimensiones}
Teniendo en cuenta sus dimensiones y extensión, los fenómenos geográficos a ser representados en el
mapa pueden clasificarse en : puntuales, lineales, poligonales y espacio temporales
\citep{fAlonsoSig2006}.

\begin{itemize}
    \item \textit{Fenómenos puntuales} : indican la presencia de entidades de un modo puntual. Normalmente se utilizan se utilizan símbolos o colores para una variable cualitativa y distintos tamaños para las variables cuantitativas.

    \item \textit{Fenómenos lineales} : representan entidades naturales o artificiales de forma lineal, se encuentran conformadas a partir de dos o más fenómenos puntuales. Se pueden variar el ancho de de las líneas para describir la anchura de los elementos, también se utilizan distintos tipos de líneas (continuas y discontinuas) para caracterizar los fenómenos lineales.

    \item \textit{Fenómenos Poligonales} : Este fenómeno puede ser descrito por dos tipos de información bidimensional o tridimensional. Teniendo en cuenta el tamaño de los mismos pueden ser representados como polígonos o porciones homogéneas del terreno relacionadas a una variable cualitativa.

    \item \textit{Fenómenos espacio-temporales}: Existe una dependencia del fenómeno con respecto al paso del tiempo.
\end{itemize}

\subsubsection{Nivel de medida}

Los elementos de la naturaleza se miden con el fin de clarificarlos y, posteriormente, compararlos.
Esto no necesariamente implica a una magnitud cuantitativa, ya que pueden ser de utilizadas las
cualitativas u jerárquicas\citep{fomentoConceptos2010}. Teniendo en cuenta su orden de precisión
tenemos las siguientes escalas:

\begin{itemize}
    \item \textit{Nominal} : Se encarga de asignar una característica, no numérica, al fenómeno de forma que solo se pueden realizar comparaciones cualitativas. Este es el nivel más elemental de medida, pues no informa acerca de la cantidad o el orden.

    \item \textit{Ordinal} : Se establece una jerarquía no cuantificable entre los diferentes elementos. Por ejemplo, un mapa en el que aparecen núcleos de población, cuyos símbolos están jerarquizados según el número de habitantes sin especificar cantidad.

    \item \textit{Cuantitativa} :Asigna una característica numérica a un fenómeno geográfico, normalmente es necesario emplear algún tipo de unidad convencional.
\end{itemize}

\subsubsection{Distribución}
La distribución de los fenómenos, o su ocurrencia, puede darse a lo largo y ancho de la superficie
terrestre que los alberga ya sea de forma continua en una ubicación de la superficie o de forma
discontinua \citep{fomentoConceptos2010}. Estos fenómenos pueden dividirse, de acuerdo a su distribución, en :

\begin{itemize}
    \item \textit{Continuos} : Son aquellos que tienen presencia en todos los puntos de la superficie, aunque sólo se cuenten con las medidas de algunos puntos significativos de la superficie. En esta categoría tenemos a la temperatura y la densidad poblacional.

    \item \textit{Discretos} : son aquellos fenómenos que tiene presencia sólo en algunos puntos de la superficie. Algunos de estos fenómenos discretos pueden transformarse en continuos mediante la aplicación de una relación. Por ejemplo tenemos el fenómeno discreto, número de habitantes de una provincia, para que se transforme en un fenómeno continuo se debe dividir el número de habitantes con el tamaño de la superficie en $km^2$, de esta forma podemos obtener la densidad poblacional que es un fenómeno continuo.
\end{itemize}

%* Técnicas gráficas de representación
\section{Representación de los datos }
\label{sec:cap2-tecnicas-graficas-representacion}

Los objetos del mundo real se pueden describir mediante los fenómenos discretos y continuos. Las variables y
objetos se muestrean y organizan para lograr una representación adecuada. En un SIG existen básicamente dos
modelos lógicos que se conocen como formato raster y formato vectorial y que dan lugar a los dos grandes tipos
de capas de información espacial.

\subsection{Formato raster}
El formato raster o de retícula se centra en las propiedades del espacio más que en la precisión de la localización. Divide
el espacio en un conjunto regular de celdillas, cada una de estas celdillas contiene un número que puede ser el identificador
de un objeto o del valor de una variable.Se trata de un modelo de datos muy adecuado para la representación de variables
continuas en el espacio.

Los datos raster se compone de filas y columnas de celdas, cada celda almacena un valor único. Los datos raster pueden ser imágenes
con un valor de color en cada celda (o píxel de la imagen). Otros valores registrados para cada celda puede ser un valor discreto o
un valor nulo si no se dispone de datos. Si bien una trama de celdas almacena un valor único, estas pueden ampliarse mediante el
uso de las bandas del raster para representar los colores RGB (rojo, verde, azul), o una tabla extendida de atributos con una
fila para cada valor único de celdas. La resolución del conjunto de datos raster es el ancho de la celda en unidades sobre el
terreno.

\subsection{Formato vectorial}
Los datos vectoriales, se caracterizan por la precisión de localización de los elementos geográficos en el espacio, donde
los fenómenos a representar son discretos, con límites bien definidos. Generalmente se considera que el formato vectorial
es más adecuado para la representación de entidades o variables cualitativas y el formato raster para representar superficies.



\begin{figure}
\centering
\includegraphics[width=0.6\textwidth]{capitulo-2/graphics/dimensiones-datos.jpg}
\caption{\label{fig:sig-xyz} Elementos geométricos utilizados para modelar digitalmente las entidades en un SIG.}
\end{figure}

Los diferentes objetos se encuentran representados como puntos, lineas o polígonos, donde cada una de estos elementos
geométricos se encuentra vinculado a una fila en una base de datos que describe sus atributos. De tal forma que para modelar
digitalmente las entidades del mundo real se utilizan estos tres elementos geométricos.
\begin{itemize}
    \item Puntos : se utilizan para las entidades geográficas que pueden ser descriptas por un único punto de referencia. Los
    puntos transmiten la menor cantidad de información, en los elementos puntuales no puede medirse la distancia. También
    se pueden utilizar para representar zonas a una escala pequeña.

    \item Líneas o polilíneas : las líneas unidimensionales o polilíneas son usadas para rasgos lineales como ríos,
    caminos, ferrocarriles, rastros, líneas topográficas o curvas de nivel. De igual forma que en las entidades
    puntuales, en pequeñas escalas pueden ser utilizados para representar polígonos. En los elementos lineales puede
    medirse la distancia.

    \item Polígonos : se utilizan para representar elementos geográficos que cubren un área particular de la superficie de la tierra.
    Los polígonos transmiten la mayor cantidad de información en archivos con datos vectoriales y en ellos se pueden medir el
    perímetro y el área.

\end{itemize}

\subsection{Ventajas y desventajas de los formatos raster y vectorial}
El debate acerca de la conveniencia de uno u otro modelo debe basarse en el tipo de estudio o enfoque que se quiera
hacer, pero también del software y fuentes de datos disponibles.

Está claro que las superficies se representan más eficientemente en formato raster y sólo pueden representarse
en formato vectorial mediante los modelos híbridos que no resultan adecuados para la realización de posteriores
análisis ya que todas las operaciones que permite el modelo ráster resultaran mucho más lentas con el modelo
vectorial.

Tradicionalmente se ha considerado que para la representación de los objetos resulta más eficiente la utilización
de un formato vectorial ya que La estructura de los datos es compacta y almacena los datos sólo de los elementos
digitalizados por lo que requiere menos memoria para su almacenamiento y tratamiento. Los elementos son representados
como gráficos vectoriales que no pierden definición si se amplía la escala de visualización. Sin emabargo el formato
vectorial es más lento, debido a su compleja estructura, que el raster para la utilización de herramientas de análisis
espacial y consultas acerca de posiciones geográficas concretas.

En el caso de las variables cualitativas estaríamos en un caso intermedio entre los dos anteriores.
Las ventajas del modelo ráster incluyen la simplicidad, la velocidad en la ejecución de los operadores y que es
el modelo de datos que utilizan las imágenes de satélite o los modelos digitales de terreno. Entre las desventajas
del modelo ráster destaca su inexactitud que depende de la resolución de los datos y la gran cantidad de espacio
que requiere para el almacenamiento de los datos. Este último problema puede compensarse mediante diversos
sistemas de compresión. Además en muchos casos se confunde precisión y exactitud cuando se trabaja con
datos vectoriales de modo que la exactitud en las coordenadas del modelo vectorial es más teórica que real.

Hoy en día se pueden codificar las formas en un modelo vectorial y los procesos con un modelo ráster,
para ello se requieren herramientas eficaces de paso de un formato al otro. Resulta sencillo, finalmente, la
visualización simultánea de datos en los dos formatos gracias a la capacidad gráfica actual.

%* Análisis espacial con SIG
\section{Análisis espacial en un SIG}
\label{sec:cap2-analisis-espacial-sig}
Dada la amplia gama de técnicas de análisis espacial que se han desarrollado durante
el último medio siglo, cualquier resumen o revisión sólo puede cubrir el tema a una
profundidad limitada \citep{rojas2012ejecucion}.

El análisis en un SIG puede definirse como el proceso de transformación de los datos geográficos
en información útil para un problema determinado, teniendo como finalidad modelar fenómenos
geográficos, las asociaciones y relaciones existentes en la información geográfica almacenada. Se
encuentra determinada por la existencia de relaciones topológicas entre los elementos y permite
realizar cálculos entre variables y obtener así nuevos datos \citep{bravo2000breve}.


%* Métodos de interpolación
\section{Métodos de interpolación}
\label{sec:cap2-metodos-interpolacion}
Todos los métodos de interpolación se basan en la presunción lógica de que
cuanto más cercanos están dos puntos sobre la superficie terrestre, los
valores de cualquier variable cuantitativa que midamos en ellos serán más
parecidos, para expresarlo más técnicamente, las variables espaciales
muestran autocorrelación espacial[2].

La interpolación espacial, consiste en usar puntos con valores conocidos,
también llamados puntos de control, para estimar una variable en lugares
donde se desconoce; también se considera una forma de transformar información
puntual en información de superficie, con el objetivo de combinarla con
otros datos para facilitar el análisis y la modelación espacial.
El resultado de la interpolación espacial depende de un algoritmo
computacional o una ecuación matemática en la cual se emplean los datos
de los puntos de control[1].

\section{Métodos de interpolación locales}
Los método locales, utilizan la interpolación utilizando la información
de los puntos más cercanos. Asumen autocorrelación espacial y estiman los
valores de Z como una media ponderada de los valores de un conjunto de
puntos de muestreo cercanos[2].

\subsection{Polígonos de Voronoi}
Es uno de los métodos de interpolación más simples, simples basado en la
distancia euclidiana. Se crean al unir los puntos entre sí, trazando las
mediatrices de los segmento de unión. Las intersecciones de estas mediatrices
determinan una serie de polígonos en un espacio bidimensional alrededor de
los puntos de control, de manera que el perímetro de los polígonos generados
sea equidistante a los puntos vecinos y designando su área de influencia, como :
\begin{itemize}
    \item Centros hospitalarios.
    \item Estaciones de bomberos.
    \item Bocas de metro.
    \item Centros comerciales.
    \item Control del tráfico aéreo.
    \item Telefonía móvil.
\end{itemize}

\subsection{Ponderación de la inversa de la distancia (IDW)}
Estima los puntos del modelo realizando una asignación de pesos a los datos
del entorno en función inversa a la distancia que los separa del punto en
cuestión. De esta forma, se acepta  que  los puntos más próximos al centroide
intervienen de manera más relevante en la obtención del valor definitivo
de Z para ese punto.

La elección del exponente de ponderación(p) determina la contribución de
los puntos circundantes al punto problema, cuanto mayor es p, más contribuyen
los puntos próximos. Es necesario contar con muchos puntos para la interpolación.

Uno de los problemas más importantes de los métodos basados en medias
ponderadas es que, interpolan basándose en el valor medio de un conjunto
de puntos situados en las proximidades, por tanto nunca se van a obtener
valores mayores o menores que los de los puntos utilizados para hacer la
interpolación[2]. En consecuencia no se van a interpolar correctamente
máximos o mínimos locales y además los puntos de muestreo aparecen en el
mapa final como máximos y mínimos locales erróneos.

\subsection{Kriging}
El Kriging es un método geoestadístico de interpolación espacial de carácter
global, exacto y estocástico. La idea básica de este método corresponde a
la noción de dependencia espacial, según la cual las muestras cercanas
tienen mayor similitud entre sí que las más apartadas[1].

Se presenta con un método de interpolación con una expresión general
similar a la anterior (IDW). La diferencia básica es que asume que la
altitud puede definirse como una variable regionalizada. Supone que la
variación espacial de la variable a representar puede ser explicada al
menos parcialmente mediante funciones de correlación espacial(la variación
espacial de los valores de z puede deducirse de los valores circundantes
de acuerdo con unas funciones homogéneas en toda el área) [4].

\subsection{Tipos de Kriging}
Kriging simples
Asume que las medias locales son relativamente constantes y de valor muy
semejante a la media de la población que es conocida. La media de la
población es utilizada para cada estimación local, en conjunto con los
puntos vecinos establecidos como necesarios para la estimación.
Kriging ordinario
Las medias locales no son necesariamente próximas de la media de la población,
usándose apenas los puntos vecinos para la estimación. Es el método más
ampliamente utilizado en los problemas ambientales.

\subsection{Semivarianza y semivariograma}
El método de Kriging utiliza diversas teorías explayadas en la estadística.
Una semivarianza es la medida del grado de dependencia espacial entre dos
muestras. La magnitud de la semivarianza entre dos puntos depende de la
distancia entre ellos. Efecto pepita,  es el valor del semivariograma en
el origen. Resulta del componente aleatorio, no correlacionado espacialmente,
que experimenta cualquier variable espacial. Se denomina así por las pepitas
de oro que representan un brusco incremento en la variable concentración de
oro para distancias muy cortas.
Meseta, es el valor máximo que adopta el semivariograma para distancias
elevadas más allá de las cuales no hay autocorrelación espacial.
Rango, es la distancia a la que se alcanza la meseta. Puede asimilarse a
la distancia más allá de la cual dos medidas pueden considerarse independientes.

%* Aplicaciones de SIG y análisis epidemiológicos
\section{Aplicaciones y análisis epidemiologico}
\label{sec:cap2-aplicaciones-analisis-epidemiologico}

Con el correr del tiempo su el potencial ha incrementado rápidamente, desde su concepción en los años setenta,
actualmente las áreas en las que se aplica se ha diversificado, entre las más resaltantes podemos nombrar:
biología, energía e infraestructura, planificación urbana y regional, monitoreo ambiental y geografía física,
transportación y logística.

\section{Vigilancia entomológica del dengue}
\label{sec:gis-vigilancia-entomologica-dengue}
La vigilancia entomológica del vector, el Aedes aegypti, se utiliza con propósitos operativos, y de
investigación, para determinar los cambios en la distribución geográfica del vector, la vigilancia
y evaluación de los programas de control, obtener medidas relativas de la población del vector en
el tiempo, y facilitar la toma de decisiones \cite{world2009dengue}, con el fin de disminuir
población del vector \cite{world2009dengue,dengueUruguayCap1, cenaprece2013,NINO2011}.

El análisis de la distribución espacial y temporal de las poblaciones del vector, puede llegar a
jugar un papel importante en la planificación y evaluación de medidas orientadas a la disminución
de las poblaciones del vector y en consecuencia, reducir los casos de dengue
\cite{world2009dengue,dengueUruguayCap1, cenaprece2013,nino2008uso}. Los SIG constituyen una
herramienta esencial para el análisis de la distribución espacial de las poblaciones
\cite{vgomesAegis2001,petric2012surveillance}, permitiendo obtener mejores resultados en
combinación con las metodologías de vigilancia entomológica y médicas\cite{petric2012surveillance}.

La utilización de los GIS permite hacer un análisis rápido para determinar anticipadamente las
intervenciones mas adecuadas que eviten o disminuyan el desarrollo de
epidemias\cite{bottinelli2002estratificacion}. Si bien el estudio es preliminar, permite
apreciar perfectamente las áreas de mayor riesgo de transmisión del virus del dengue\cite{bottinelli2002estratificacion, NINO2011}.

Los métodos de muestreo, como larvitrampas y ovitrampas resultan eficientes y económicos para
determinar determinar la distribución espacial y temporal del Aedes aegypti y otros mosquitos
\cite{dengueUruguayCap1, cenaprece2013}.


%!TEX root = ../tesis.tex
\subsection{Identificación de focos de infestación}
\label{sec:cap5-identificacion-focos}

En \citet{NINO2011} se ha diseñado una metodología de vigilancia entomológica que localiza focos
de infestación de A. aegypti, mediante el uso de larvitrampas, para generar información
regionalizada sobre la abundancia larval y técnicas de interpolación espacial que permiten
visualizar de forma continua el nivel de infestación vectorial.

Las larvitrampas, o puntos de control, distribuidas geográficamente permiten, mediante técnicas de
interpolación espacial, obtener mapas de interpolación donde se puede apreciar los niveles de
infestación del vector del dengue y el riesgo correspondiente a la abundacia de mosqutios
observada en el área de estudio. El hecho de contar con esta información regionalizada permitirá
a las autoridades pertinentes definir y planificar mejor las medidas de prevención y control a
realizarce para reducir los niveles de infestación en las zonas criticas. Según \citet{NINO2011},
los mapas de superficie generados con esta metodología indican, con mayor detalle que los índices
aédicos tradicionales, los lugares específicos donde sería necesario tomar medidas de prevención y
control de acuerdo al grado de infestación, permitiendo así una mayor racionalización de tiempo y
recursos.

La selección del método de interpolación se realizó teniendo en cuenta el factor humano en la
distribución de los puntos de control. En \citet{villatoro2007comparacion} se realiza una
comparación de los interpoladores IDW y Kriging, donde los autores señalan que el método Kriging
fue más preciso y eficiente que el IDW, aunque la diferencia entre ambos métodos no fue muy amplia.
Sin embargo, cuando el distanciamiento, es muy grande, los variogramas no son posibles
de obtener, entonces el Kriging deja de ser una opción y comparativamente el IDW se perfila como
el mejor \citep{villatoro2007comparacion}. El método seleccionado finalmente fue el IDW, debido
que la distribución del los puntos de control no será perfecta, inclusive, en algunas localidades
la distribución no será de forma uniforme.


