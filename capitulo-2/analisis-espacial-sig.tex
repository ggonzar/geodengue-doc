\section{Análisis espacial en un SIG}
\label{sec:cap2-analisis-espacial-sig}

Un SIG puede reconocer y analizar las relaciones espaciales que existen en la información geográfica almacenada.
Estas relaciones topológicas permiten realizar modelizaciones y análisis espaciales complejos. Así, por ejemplo, 
el SIG puede discernir la parcela o parcelas catastrales que son atravesadas por una línea de alta tensión, o bien
saber qué agrupación de líneas forman una determinada carretera.

En suma podemos decir que en el ámbito de los sistemas de información geográfica se entiende como topología a las 
relaciones espaciales entre los diferentes elementos gráficos (topología de nodo/punto, topología de red/arco/línea, 
topología de polígono) y su posición en el mapa (proximidad, inclusión, conectividad y vecindad). Estas relaciones, 
que para el ser humano pueden ser obvias a simple vista, el software las debe establecer mediante un lenguaje y unas
reglas de geometría matemática.

Para llevar a cabo análisis en los que es necesario que exista consistencia topológica de los elementos de la base de 
datos suele ser necesario realizar previamente una validación y corrección topológica de la información gráfica. 
Para ello existen herramientas en los SIG que facilitan la rectificación de errores comunes de manera automática o 
semiautomática.

La geoestadística analiza patrones espaciales con el fin de conseguir predicciones a partir de datos espaciales concretos.
Es una forma de ver las propiedades estadísticas de los datos espaciales. A diferencia de las aplicaciones estadísticas 
comunes, en la geoestadística se emplea el uso de la teoría de grafos y de matrices algebraicas para reducir el número de 
parámetros en los datos. Tras ello, el análisis de los datos asociados a entidad geográfica se llevaría a cabo en segundo 
lugar.

Cuando se miden los fenómenos, los métodos de observación dictan la exactitud de cualquier análisis posterior. Debido a 
la naturaleza de los datos (por ejemplo, los patrones de tráfico en un entorno urbano, las pautas meteorológicas en el 
océano, etc.), grado de precisión constante o dinámico se pierde siempre en la medición. Esta pérdida de precisión se 
determina a partir de la escala y la distribución de los datos recogidos. Los SIG disponen de herramientas que ayudan a 
realizar estos análisis, destacando la generación de modelos de interpolación espacial.

