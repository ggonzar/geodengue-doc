\section{Análisis espacial en un SIG}
\label{sec:cap2-analisis-espacial-sig}

El análisis en un SIG puede definirse como el proceso de transformación de los datos geográficos en información
útil para un problema determinado, teniendo como finalidad modelar fenómenos geográficos, las asociaciones
y relaciones existentes en la información geográfica almacenada.

En el proceso de análisis espacial, la definición de los métodos de análisis requieren que las tareas y
transformaciones sean realizadas por funciones con conocimiento de la naturaleza del problema, tal que estas
interactúen con datos estructurados. Los resultados deben reflejar la naturaleza y la calidad de los datos, así
como la pertinencia de los métodos y funciones aplicadas.

Para que el análisis sea posible debe existir una consistencia topológica de los elementos de la base de datos,
normalmente es necesario realizar una validación y corrección previa de la información.

La estadística espacial o geoestadística es la reunión de un conjunto de metodologías apropiadas para el análisis
de datos que corresponden a la medición de variables aleatorias en diversos sitios (puntos del espacio o
agregaciones espaciales) de una región\cite{rgeraldoGeoestasistica}. Analiza los patrones espaciales con el fin de
conseguir predicciones a partir de a partir de datos espaciales concretos. A diferencia de las aplicaciones
estadísticas comunes, sólo se dispone de valores medidos en algunos puntos de muestreo en un espacio infinito, 
por tanto las estimaciones de las medias, desviaciones típicas y covarianzas, no son fiables.
