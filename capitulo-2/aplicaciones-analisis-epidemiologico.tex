\section{Aplicaciones y análisis epidemiologico}
\label{sec:cap2-aplicaciones-analisis-epidemiologico}
Desde sus inicios las autoridades sanitarias, en sus tareas de vigilancia en Salud Pública, han
encontrado en los SIG una herramienta fundamental para determinar cómo se extiende una enfermedad,
estudiar su posible relación con un potencial foco de riesgo, o localizar un brote epidémico
\citep{vgomesAegis2001}.

El año 1854 marcó un hito para la epidemiología moderna, en \citet{jCerdaJohnSnow2007} se describe
como el Dr. John Snow, pionero de la epidemiología, cartografió la incidencia de los casos de
cólera en el distrito de Soho en Londres, permitiéndole así localizar con precisión un pozo de
agua contaminado como la fuente causante del brote de cólera. Actualmente, el Dr. John Snow, es
considerado como el padre de la epidemiología moderna por introducir el uso de  metodologías de
investigación epidemiológica como, la implementación de encuestas y la epidemiología espacial
\citep{jCerdaJohnSnow2007}.

Con el transcurrir del tiempo el potencial de los SIG a incrementado considerablemente,  incrementando y diversificando las áreas de aplicación. Entre las áreas más resaltantes podemos
mencionar: biología, energía e infraestructura, planificación urbana y regional, monitoreo
ambiental y geografía física, transportación y logística \citep{fAlonsoSig2006}. En cuanto a las
aplicaciones, entre las más usuales destacan las del campo científico, gestión y empresarial
\citep{fAlonsoSig2006}.

La capacidad de superponer la localización de los casos de una enfermedad como puntos con
información espacial relacionada es una herramienta de considerable poder
\citep{iMolinaSigEpidemiologia}.

Los SIG son capaces de simplificar grandes tareas como la localización de eventos en espacio y
tiempo, el monitoreo de eventos de salud y el comportamiento de factores de riesgo en un período
de tiempo dado, la identificación de áreas geográficas y grupos de población con grandes
necesidades de salud y contribuye a la solución de tales necesidades mediante el análisis de
múltiples variables y la evaluación del impacto de intervenciones en salud
\citep{iMolinaSigEpidemiologia}. Los análisis más utilizados, por las entidades sanitarias son:

\begin{itemize}
\item \textit{La cartografía de enfermedades}, cuyo fin es representar la distribución espacial de la enfermedad.

\item \textit{el análisis de focos de riesgo}, que establece bandas alrededor de un punto o una  zona geográfica representando una potencial  fuente contaminante para comparar el riesgo de una enfermedad en cada una de esas bandas.
\end{itemize}

El potencial analítico y la capacidad combinatoria de la información proveniente de, registros
de mortalidad, hospitales, bases de datos oficiales de las entidades de salud, observatorios
meteorológicos y proyectos específicos orientados a la salud, permiten implementar infinitos tipos
de análisis.
