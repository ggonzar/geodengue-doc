\section{Aplicaciones y análisis epidemiologico}
\label{sec:cap2-aplicaciones-analisis-epidemiologico}

Con el correr del tiempo su el potencial ha incrementado rápidamente, desde su concepción en los años setenta,
actualmente las áreas en las que se aplica se ha diversificado, entre las más resaltantes podemos nombrar:
biología, energía e infraestructura, planificación urbana y regional, monitoreo ambiental y geografía física,
transportación y logística. Entre las aplicaciones más usuales destacan las del campo científico, gestión 
y empresarial.

Las autoridades sanitarias, en sus tareas de vigilancia en Salud Pública, tienen en los GIS una 
herramienta fundamental para conocer cómo se extiende una enfermedad, estudiar su posible relación 
con un potencial foco de riesgo, o localizar un brote epidémico\cite{vgomesAegis2001}.

La información necesaria para realizar este tipo de estudios proviene de muy diversas fuentes: 
registros de mortalidad, hospitales, facultativos, bases de datos oficiales, observatorios 
medioambientales o meteorológicos, proyectos específicos. Por tanto, es muy importante 
recopilar y tratar de forma unificada toda esta información para facilitar su acceso y análisis.
