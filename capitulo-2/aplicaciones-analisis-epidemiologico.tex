\section{Aplicaciones y análisis epidemiologico}
\label{sec:cap2-aplicaciones-analisis-epidemiologico}

Desde sus inicios las autoridades sanitarias, en sus tareas de vigilancia en Salud Pública, han encontrado en los
GIS una  herramienta fundamental para conocer cómo se extiende una enfermedad, estudiar su posible relación con un
potencial foco de riesgo, o localizar un brote epidémico\cite{vgomesAegis2001}. 

El año 1854 marcó un hito para la epidemiología moderna, en \cite{jCerdaJohnSnow2007} se describe como el Dr.
John Snow, pionero de la epidemiología, cartografió la incidencia de los casos de cólera en el distrito de Soho en
Londres, permitiéndole así localizar con precisión un pozo de agua contaminado como la fuente causante del brote de
colera. El Dr. John Snow, actualmente, es considerado como el padre de la epidemiología moderna, introduciendo el
uso de metodologías de investigación epidemiológica moderna, por ejemplo, la implementación de encuestas y la
epidemiología espacial.

Con el correr del tiempo su el potencial ha incrementado rápidamente, desde su concepción, actualmente las áreas 
en las que se aplica se ha diversificado, entre las más resaltantes podemos nombrar: biología, energía e
infraestructura, planificación urbana y regional, monitoreo ambiental y geografía física, transportación y
logística. Entre las aplicaciones más usuales destacan las del campo científico, gestión y empresarial. 

Los análisis más utilizados por las entidades sanitarias son la cartografía de enfermedades, cuyo fin es
representar la distribución espacial de la enfermedad,  y el análisis de focos de riesgo, que establece 
bandas alrededor de un punto o una zona geográfica representando una potencial fuente contaminante para 
comparar el riesgo de una enfermedad en cada una de esas bandas. La información necesaria para realizar 
este tipo de estudios proviene de muy diversas fuentes: registros de mortalidad, hospitales, facultativos, 
bases de datos oficiales, observatorios medioambientales o meteorológicos, proyectos específicos. Por 
tanto, es muy importante recopilar y tratar de forma unificada toda esta información para facilitar su 
acceso y análisis.

Los SIG son capaces de simplificar grandes tareas como la localización de eventos en espacio y tiempo, 
el monitoreo de eventos de salud y el comportamiento de factores de riesgo en un período de tiempo dado, la
identificación de áreas geográficas y grupos de población con grandes necesidades de salud y contribuye a la
solución de tales necesidades mediante el análisis de múltiples variables y la evaluación del impacto de
intervenciones en salud\cite{iMolinaSigEpidemiologia}.
