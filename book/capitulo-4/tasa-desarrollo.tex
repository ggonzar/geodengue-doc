\subsection{Tasas de desarrollo}
\label{subsec:cap4-tasas de desarrollo}
En el modelo se cuenta con 4 tasas de desarrollos correspondientes a : la eclosión de huevos,
emergencia a pupas, emergencia a adultos y el ciclo gonotrófico. Estos valores son obtenidos
mediante el modelo no lineal de Sharpe y DeMichele, presentado en \cite{sharpe1977reaction}, para procesos poiquilotermos\footnote{ La poiquilotermia o ectotermia es un término aplicado a ciertos
animales con temperatura corporal variable}, donde el proceso de maduración es controlado por
una enzima que actúa en un rango de temperatura determinado, la enzima se desactiva a las bajas temperaturas, $TL$, y altas $TH$.

\begin{equation} \label{eq:sharpe-demichele}
   R(k)  = R(298K) *\cfrac{ \cfrac{k}{298K} *
    exp \Bigg[
            \cfrac{\Delta H_{A}}{R} \bigg(\cfrac{1}{298K} - \cfrac{1}{k}\bigg)
        \Bigg]}
    {1 + exp\Bigg[\cfrac{\Delta H_{H}}{R} \bigg(\cfrac{1}{T_{H}}- \cfrac{1}{k}\bigg)\Bigg] +  exp\Bigg[\cfrac{\Delta H_{L}}{R} \bigg(\cfrac{1}{T_{L}}- \cfrac{1}{k}\bigg)\Bigg]}
\end{equation}

Donde $R(k)$ representa la tasa de desarrollo media ($dias^{-1}$) para una temperatura $K$,en la
escala de Kelvin; $T_{H}$ , $T_{L}$ son temperaturas absolutas, en la escala de Kelvin, mientras
que $\Delta H_{A}$, $\Delta H_{H}$ y $\Delta H_{L}$ son entalpías termodinámicas características
del organismo, y $R$, igual $1,987202$ $cal/K.mol$, es la constante universal de los gases.

Schoolfield presentó, en \cite{schoolfield1981non}, un modelo simplificado con inhibición de altas
temperaturas, con una única alta temperatura de desactivación. El modelo se encuentra definido
como :

\begin{equation} \label{eq:schoolfield}
   R(k)  = R(298K) *\cfrac{ \cfrac{k}{298K} *
    exp \Bigg[
            \cfrac{\Delta H_{A}}{R} \bigg(\cfrac{1}{298K} - \cfrac{1}{k}\bigg)
        \Bigg]}
    {1 + exp\Bigg[\cfrac{\Delta H_{H}}{R} \bigg(\cfrac{1}{T_{1/2}}- \cfrac{1}{k}\bigg)\Bigg] }
\end{equation}

Donde $T_{1/2}$ es la temperatura cuando la mitad de la enzima se desactiva, debido a la alta
temperatura. Los parámetros $R(298K)$, $\Delta H_{A}$, $T_{1/2}$, y $\Delta H_{H}$ son estimados
mediante la de regresión no lineal de Wagner, presentado en \cite{wagner1984modeling}. Para este
trabajo, al igual que \cite{rueda1990temperature, otero2006stochastic}, se adopta el modelo de
Schoolfield, ya que, según \cite{otero2006stochastic}, es lo suficientemente flexible para el
ajuste de los datos biológicos disponibles. Los parámetros deben calcularse para cada etapa de
desarrollo, una vez determinados, la ecuación puede utilizarse para calcular tasas de desarrollo a
cualquier temperatura \cite{rueda1990temperature}.
