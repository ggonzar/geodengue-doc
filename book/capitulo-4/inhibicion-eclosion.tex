Probabilidades de transición dependientes de la densidad reflejan los procesos regulatorios que afectan a las poblaciones. Hemos introducido dos procesos regulatorios: la mortalidad dependiente de la densidad de las larvas y la inhibición de la eclosión de huevos por las larvas.


Trama de la inhibición por larvas. La posibilidad de un proceso de regulación complejo, en el que la alta densidad de larvas inhibe la eclosión de los huevos, la inducción de los huevos para entrar en diapausa, fue descubierto por Livdahl et al. (1984). Hemos introducido este efecto a través de un factor de reducción de la tasa de eclosión cuando las larvas supera un destino predeterminado. La tasa de eclosión se convierte luego

%%formula

Según Livdahl et al. (1984) la fracción de la eclosión cambia en algún lugar entre 10 y 70 larvas por litro. La región entre estos valores no se ha explorado, por lo tanto, hemos considerado que el efecto de inhibición tiene lugar para las densidades por encima de un valor determinado llamado la densidad crítica. Los valores de densidad crítica entre 10 y 70 larvas por litro se han considerado, así como un tamaño medio del sitio de cría de medio l.



Para bs Se refleja no sólo una característica de la especie, sino también una característica del medio ambiente. Como tal, se espera que tomar diferentes valores en diferentes lugares.
