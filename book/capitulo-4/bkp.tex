

Para la identificación de focos de infestación se propone, utilizar la metodología de vigilancia
entomológica basada en el uso de larvitrampas propuesta en \cite{NINO2011}. En donde las
larvitrampas, o puntos de control, deben ser distribuidas geográficamente (\figref{fig:sig-distribucion-puntos-control}) para generar información
regionalizada que, mediante técnicas de interpolación espacial, permiten obtener mapas de
interpolación donde se puede apreciar los niveles de infestación del vector, y el riesgo
correspondiente a la abundancia de mosquitos observada en el área de estudio
\cite{NINO2011, nino2008uso, journal.pone.0054167, albierispatial}. El hecho de contar con esta
información regionalizada permitirá a las autoridades pertinentes definir y planificar mejor las
medidas de prevención y control a realizarse para reducir los niveles de infestación en las zonas
criticas \cite{NINO2011, nino2008uso, petric2012surveillance}.


le asocia un valor denominado densidad relativa de larvas. El valor de  es estimado, a partir de los valores conocidos de los puntos de
control, mediante el método de interpolación definido por ecuación \eqref{eq:interpolacion-idw} en
un radio $r$.
