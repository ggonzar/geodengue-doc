\chapter{Introducción}

El dengue es una enfermedad viral transmitida por el mosquito del Aedes aegypti, en donde la
prevención y el control de esta enfermedad requiere de nuevas técnicas de vigilancia. El análisis
de la distribución espacial del Aedes aegypti cumple un papel importante a la hora de definir,
planificar y evaluar las acciones a realizar con el fin de disminuir las poblaciones de mosquitos.

En este capítulo se presentan los antecedentes y la importancia de este trabajo, así como las
propuestas que se realizan con el fin de apoyar la lucha preventiva contra el dengue. Conceptos de
la ecología del vector\footnote{Se le llama vector a un mecanismo, generalmente un organismo, que
transmite un agente infeccioso o infestante desde los individuos afectados a otros que aún no
portan ese agente}, métodos de prevención y sistemas de información geográficos son puntos
vitales de este trabajo por lo que son introducidos brevemente en este capítulo. No obstante, en
capítulos siguientes, se tratarán todos estos temas detalladamente.

\section{Justificación y Antecedentes}
El Aedes aegypti es el principal vector del dengue en América, y afecta a más de 2,500 millones
de personas que viven en zonas de riesgo \cite{world2009dengue, gustavo2006dengue}. Actualmente la enfermedad es considerada un problema grave para la salud pública a nivel mundial
\cite{dengueUruguayCap1, world2009dengue, DIBO2005}, siendo catalogada como ejemplo de una
enfermedad que puede constituir una emergencia de salud pública de interés internacional con
implicaciones para la seguridad sanitaria \cite{dengueUruguayCap1, world2009dengue}, debido a la
necesidad de interrumpir la infección y la rápida propagación de la epidemia más allá de las
fronteras \cite{world2009dengue}.

El Paraguay desde el año 2009 es considerado un país endémico\cite{planControlMspbs2014}, cuyo
clima, subtropical, favorece la aparición y desarrollo del dengue. Las autoridades sanitarias
nacionales llevan a cabo acciones para la vigilancia entomológica y de control vectorial para la
disminución de las poblaciones de mosquitos, mediante control químico y el control mecánico, donde
la vigilancia entomológica es realizada con el fin monitorear la densidad vectorial en zonas
endémicas y no endémicas, mediante técnicas basadas en utilización de índices tradicionales.
Actualmente existen numerosos métodos e indicadores más prácticos, eficientes y económicos para
determinar las poblaciones de Aedes aegypti \cite{cenaprece2013}, como larvitrampas y ovitrampas.

Las autoridades sanitarias del Paraguay no cuentan con datos computables, geográficamente,
relacionados con a casos reportados, confirmados, sospechosos y fatales de dengue que permitan
realizar análisis estadísticos y espaciales, como regresiones geográficamente ponderadas para
determinar la relación existente entre los casos de dengue y variables como : datos
climatológicos, criaderos de mosquitos, e índices de infestación larvaria. Para realizar este
trabajo de análisis, primero se deben diseñar y desarrollar las plataformas para el registro de la
información, definir las políticas y metodologías de actualización y migración de los
datos, para posteriormente realizar un análisis. Esto implica un trabajo institucional con un
alto costo y cuya ejecución podría llevar un tiempo considerable. Teniendo en cuenta dicho
requerimiento previo, se deben optar por nuevas metodologías que permitan generar información para
el análisis sin la necesidad de requerimientos institucionales para realizar estimaciones válidas.

Las metodologías de vigilancia entomológica basadas en el uso de larvitrampas y ovitrampas
permiten generar información regionalizada sobre el estado y la distribución de la población del
vector. El análisis de la distribución espacial del Aedes aegypti, en los sistemas de información
geográfica, permite a las autoridades sanitarias una mejor definición, planificación y evaluación
de las acciones a realizar para disminuir las poblaciones del vector en las regiones con alta
densidad vectorial.

Los sistemas de información geográfica se encuentran definidos como una integración organizada de
hardware, software, datos geográficos y personal, diseñado para capturar, almacenar, manipular,
analizar y visualizar en todas sus formas la información geográficamente referenciada con el fin
de resolver problemas complejos de planificación y gestión \citep{lopezMarcos2007}.

Las autoridades sanitarias, en sus tareas de vigilancia en salud pública, tienen en los sistemas
de información geográfica una herramienta fundamental para conocer cómo se extiende una
enfermedad, estudiar su posible relación con un potencial foco de riesgo, o localizar un brote
epidémico \cite{vgomesAegis2001}.

En Paraguay anualmente se reportan miles de casos de infección por dengue, algunos de ellos con
derivación fatal. La detección de focos de infección basados en índices tradicionales y casos
reportados son considerados como correctivos. Es imperativo, por parte de las autoridades
sanitarias, el uso de nuevas técnicas para la vigilancia entomológica, con un enfoque preventivo.

\section{Propuesta y Objetivos}
En este trabajo se propone un modelo y se presenta una herramienta que permita realizar estudios
epidemiológicos de forma cartográfica, especializada para el particular caso del dengue. Para ello
se utilizan métodos de abundancia poblacional distribuidos geográficamente y técnicas de
interpolación espacial para la determinación de focos de riesgo. Además se provee de un simulador
del comportamiento del Aedes aegypti, donde este es sometido a las variaciones climáticas de
acuerdo a las características de la región, con el fin de generar información que pueda ser útil a
la hora de aplicar y definir las medidas preventivas correspondientes.

\subsection{Objetivo General}
Diseñar un modelo que permita analizar la extensión del vector del dengue y estudiar su posible
relación con un potencial foco de riesgo, de forma a realizar una predicción de posibles focos de
riesgo.

\subsection{Objetivos Específicos}

\begin{itemize}

\item Analizar nuevos métodos de muestreo de la abundancia poblacional del vector, con el fin de apoyar la lucha preventiva contra la enfermedad.

\item Diseñar un modelo e implementar un sistema computacional mediante el cual se puedan procesar y presentar los resultados obtenidos, en un sistema de información geográfica.

\item Diseñar el modelo de forma paramétrica, extensible y escalable, para que sea aplicable y extensible a cualquier región o área de estudio.

\item Generar información relevante que pueda ayudar a las autoridades pertinentes para toma de decisiones en la lucha contra el dengue.

\end{itemize}

\section{Organización del trabajo}
El trabajo está organizado como sigue: en el Capítulo 2 se presenta al Aedes aegypti el principal
transmisor del dengue, sus características biológicas y los métodos de muestreo de la abundancia
poblacional del vector. El Capítulo 3 introduce a los sistemas de información geográfica, la
representación de datos geoespaciales y los métodos de interpolación como herramienta para
análisis espacial. Estos capítulos representan el estado del arte de este trabajo.

El Capítulo 4 se presenta el modelo matemático propuesto para la identificación de focos de
infestación y la simulación del proceso evolutivo del ciclo de vida del vector. En el Capítulo 5
se presenta la implementación computacional denominada GeoDengue, sus requerimientos, diseño y
arquitectura, las herramientas y tecnologías utilizadas para su desarrollo.

En el Capítulo 6 se presentan los resultados experimentales obtenidos. En el Capítulo 7
presentan las conclusiones de este trabajo y finalmente en el Capítulo 8 se exponen los posibles
trabajos futuros derivados de esta trabajo final de grado.
