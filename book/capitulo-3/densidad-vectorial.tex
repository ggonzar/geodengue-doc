\section{Métodos de muestreo de la abundancia poblacional}
\label{sec:densidad-vectorial-introduccion}

Está ampliamente aceptado que la vigilancia de Aedes aegypti es un aspecto muy importante en la
lucha contra el dengue, esta afirmación se basa en la asunción de que existe una correlación
positiva entre la densidad del vector y la enfermedad humana\cite{world2009dengue, dengueUruguayCap2}.

Muchos programas de control del dengue se basan el la utilización de índices larvarios o índices
de Stegomia, como indicadores de las densidades poblaciones de Aedes aegypti, para dirigir y
focalizar espacial y temporalmente las acciones de control del vector.

%~ * Índices de Stegomia

\subsection{Índices de Stegomia}
\label{sec:densidad-vectorial-indices-stegomia}
La Organización mundial de la salud (OMS) recomienda la utlización de tres indicadores
entomológicos, generalmente conocidos como índices de Stegomia, para estimar la densidad del
vector, Índice de casas (I.C.), Índice de Recipientes (I.R.) e Índice de Breteau (I.B.).Estos
indices son calculados a partir de muestreos de larvas y recipientes.

\subsubsection{Índice de Casas}
El índice de Casas (I.C.) es el valor numérico que especifíca el porcentaje de viviendas
infestadas con larvas, pupas o ambos estados de desarrollo del Aedes aegypti o Aedes albopictus
\cite{ibanez1995vectores, world2009dengue}, y se encuentra expresado como :

\begin{equation}
I.C. = \frac{\text{número de casas infestadas} * 100}{\text{número de casas inspecionadas}}
\end{equation}

Donde :
\begin{itemize}
\item El número de casas infestadas : cantidad de casas que cuentan con al menos un contenedor que alberga a larvas o pupas de Aedes aegypti.
\item número de casas inspeccionadas : El total de casas analizadas para el estudio.
\end{itemize}

Para este indice, se analizan los contenedores de las viviendas y alrededores, donde, se registra
como positiva cuando al menos un contenedor presenta larvas o pupas \cite{ibanez1995vectores}. Es
el de uso más generalizado para la distribución de medición de la población, pero no tiene en
cuenta el número de recipientes positivos ni la productividad de esos recipientes
\cite{world2009dengue}. Puede ser utilizado para proporcionar una indicación rápida de la
distribución del mosquito en una área determinada.

\subsubsection{Índice de Recipientes}
El índice de Recipientes es un valor numérico que consiste en el porcentaje de recipientes que
contienen agua y están infestados con las larvas o pupas del Aedes aegypti o Aedes albopictus
\cite{ibanez1995vectores, world2009dengue}, y se encuentra expresado como :

\begin{equation}
I.R. = \frac{\text{número de contenedores positivos} * 100}{\text{número de contenedores inspecionadas}}
\end{equation}

Donde :
\begin{itemize}
\item número de contenedores positivos : La cantidad de contenedores en los cuales se observan larvas o pupas de Aedes aegypti.
\item número de contenedores  inspeccionadas : El total de contenedores analizadas para el estudio.
\end{itemize}

Este índice sólo ofrece información sobre la proporción de recipientes que mantienen agua y que
son positivos \cite{world2009dengue}. Las encuestas que utilizan el índice del envase son mucho
más lentas a realizar que las encuestas sobre el índice de la casa, porque requieren que se
examinen todos los recipientes para detectar la presencia de etapas no maduras y registrar los
detalles en envases positivos y negativos para determinar su especie mediante análisis
laboratoriales.

\subsubsection{Índice de Breteau}
\cite{ibanez1995vectores} \cite{world2009dengue} \cite{}
El índice de Breteau (I.B.) es un valor numérico que define el número recipientes con larvas,
pupas o ambos estados de desarrollo del Aedes aegypti o Aedes albopictus, que se encuentran número
de recipientes positivos por cada 100 casas inspeccionadas
\cite{ibanez1995vectores, MARQUES1993,world2009dengue}, expresada como :

\begin{equation}
I.B. = \frac{\text{número de contenedores positivos} * 100}{\text{número de casas inspecionadas}}
\end{equation}

Donde :
\begin{itemize}
\item número de contenedores positivos : La cantidad de contenedores en los cuales se observan larvas o pupas de Aedes aegypti.
\item número de casas inspeccionadas : El total de casas analizadas para el estudio.
\end{itemize}

El índice de Breteau establece una relación entre recipientes positivos y casas, y se considera el
índice más informativo, pero nuevamente, no se puede medir la productividad del recipiente
\cite{world2009dengue}.

\subsubsection{Probelmatica}
Las técnicas tradicionales de vigilancia de A. aegypti usan los índices
stegomia para determinar el grado de infestación, dispersión y densidad
del mosquito en una zona y tiempo determinados.

Estos índices se fundamentan en la detección visual de formas
inmaduras del vector dentro de recipientes domésticos, técnica considerada
poco sensible por la habilidad de las larvas para escapar y su capacidad de
permanecer sumergidas por largos períodos de tiempo. Asi mismo, la
proporción de viviendas y recipientes infestados con A. aegypti no provee
información fehaciente sobre la densidad poblacional al registrar como
positivo un recipiente o casa sin tener en cuenta la cantidad de formas
inmaduras presentes, lo cual quiere decir que para el índice es igual
si hay una o cientos de ellas.

\begin{itemize}
    \item Se basan en búsqueda de fases inmaduras por lo que representan
        una estimación indirecta de las poblaciones de mosquitos adultos.
    \item No reflejan la asociación que existe entre las densidades de
        mosquitos o cantidad y/o tipo de recipientes presentes, con los
        riesgos de transmisión de dengue.
    \item Proporcionan poca o nula información de aquellas viviendas en
        las que existe un mayor riesgo de presencia de mosquitos.
\end{itemize}

Además estos indicadores no reflejan las poblaciones de adultos ni estiman
riesgo entomológico, lo cual es muy importante en la transmisión del dengue.
Por lo cual, a la fecha sólo son recomendados para detectar la calidad de
las acciones (control de calidad) realizadas por el personal de control
larvario. Existen numerosos métodos e indicadores para determinar las
poblaciones de Aedes aegypti en la etapa de huevo, larva, pupa o adulto;
uno de los métodos más prácticos, eficientes y económicos es el monitoreo
de poblaciones de este vector por medio de ovitrampas. Las ovitrampas han
sido usadas desde 1965 en la vigilancia del Aedes aegypti (L), como un
instrumento para determinar la distribución del mosquito, medir la fluctuación
estacional de las poblaciones y para evaluar la eficacia de la aplicación
de insecticidas; además, como una estrategia de muestreo presencia-ausencia,
lo cual permite una estimación de la densidad mediante la proporción de
muestras positivas y son especialmente útiles para la detección temprana
de reinfestaciones. \cite{cenaprece2013}

\subsection{Larvitrampas}
\label{sec:densidad-vectorial-larvitrampas}
Las larvitrampas son, dispositivos artificiales creados con el fin de simular el habitad del
vector de forma controlada. El diseño más simple (\figref{fig:cap3-larvitrampas}) es una sección
radial de una llanta llena de agua \cite{world2009dengue}.

\begin{figure}[H]
\centering
\includegraphics[width=0.4\textwidth]{capitulo-3/graphics/larvitrampa.png}
\caption{\label{fig:cap3-larvitrampas} Diseño de una larvitrampa(Tomado de
\cite{manualControlArg2009}).}
\end{figure}

Se basan en la detección del vector en su etapa, inmadura, larval
\cite{manualControlArg2009, MARQUES1993}, que brinda información sobre los patrones de actividad
espacial y estacional de ovipostura, y además, permiten reconocer las condiciones climáticas
favorables para la eclosión y desarrollo larvario \cite{manualControlArg2009}.

En las áreas infestadas, o con alto riesgo de infección con Aedes aegypti, la inspección debe
realizarse de forma periódica, según \cite{manualControlArg2009}, con el fin de :

\begin{itemize}
    \item Conocer la distribución del vector y el grado de infestación para establecer el nivel de riesgo de transmisión de dengue en las áreas geográficas infestadas.
    \item Detectar oportunamente la infestación en las áreas no infestadas.
    \item Detectar la introducción de Aedes albopictus en áreas no infestadas.
    \item Evaluación de acciones realizadas.
\end{itemize}

Las llantas o neumáticos son considerados como criaderos de alto riesgo
\cite{bisset2008distribucion, manrique1998desarrollo}, donde, la supervivencia y la duración del
ciclo de vida en neumáticos son menores a los reportados para otros tipos de contenedores
\cite{manrique1998desarrollo}. Esto resalta la importancia de los neumáticos como criaderos y
blanco para el control del vector del dengue \cite{manrique1998desarrollo}.

Las larvitrmpas, construidas de neumáticos, permiten transformar estos criaderos de alto
riesgo en una herramienta para el control del vector del dengue mediante el reciclaje.

\subsection{Ovitrampa}
\label{sec:densidad-vectorial-ovitrampa}
Las ovitrampas constituyen un método sensible y económico para el monitoreo del Aedes y útiles
para determinar el comportamiento poblacional del vector y conocer las áreas de riesgo
entomológico \cite{cenaprece2013}. Su diseño estándar se encuentra compuesto por una jarra de
vidrio pintada de negro \cite{dengueUruguayCap1, world2009dengue}, equipada con una chapa de
madera o paleta de madera \cite{dengueUruguayCap1, world2009dengue, website:TimothyOvitrap2014,
manualControlArg2009}. Así mismo pueden utilizarse recipientes de plástico
\cite{website:TimothyOvitrap2014, cenaprece2013, manualControlArg2009, MARQUES1993} en reemplazo
de las jarras de vidrio por su bajo costo.


\begin{figure}[H]
\centering
\includegraphics[width=0.3\textwidth]{capitulo-3/graphics/ovitrampa.jpg}
\caption{\label{fig:cap3-larvitrampas} Diseño de una ovitrampa (Tomado de
\cite{website:TimothyOvitrap2014}).}
\end{figure}

Las ovitrampas constituyen un método sensible y económico para el monitoreo del Aedes aegypti
\cite{cenaprece2013, world2009dengue}, y son útiles para determinar el comportamiento poblacional
del vector y conocer las áreas de riesgo entomológico \cite{cenaprece2013}. La información
proporcionada permite para determinar la distribución espacial y temporal de Ae. aegypti y otros
mosquitos \cite{dengueUruguayCap1}. Se pueden instalar y preparar en forma relativamente rápida en
grandes áreas \cite{world2009dengue}.



%~ * Distribución de dispositivos de ovipostura
%\subsection{Larvitrampas}
\label{sec:densidad-vectorial-larvitrampas}
Antes de la utilización de la larvitrampa, ésta debe cepillarse y flamearse,
luego mantenerla sumergida en agua durante no menos de tres días, para
asegurarse que el agua no contenga residuos de sustancias que puedan actuar
como larvicida. De esta manera, además, se garantiza la destrucción de
algún huevo del mosquito que estuviese previamente en el neumático o en
larvitrampas ya utilizadas.

%~ * Seguimiento y control de dispositivos de ovipostura
%\subsection{Ovitrampa}
\label{sec:densidad-vectorial-ovitrampa}
Son recipientes que ofrecen a las hembras de Aedes aegypti un lugar colocar
los huevos. Detecta la presencia de huevos y por lo tanto actividad de
ovipostura. Las ovitrampas consisten en frascos de plástico o pequeñas
macetas plásticas de unos 500 ml de color oscuro preferentemente, en cuyo
interior, se coloca una pieza plana de madera (baja-lengua o similar).
Asimismo, también pueden construirse con un pote de vidrio de boca ancha,
de aprox. medio litro, pintado de negro por fuera y equipado con una paleta
de cartón o madera (baja-lengua) sujeta verticalmente al interior, con su
lado áspero mirando hacia adentro. Las dimensiones del recipiente no son
críticas pero todos los frascos a usar en un estudio particular deben ser
idénticos. Al frasco se le deberá agregar 250 ml de agua limpia.

%~ * Recolección de resultados
%\section{Mosquitérica genérica}
Es un dispositivos de oviposturas más recientes, cuenta con un sencillo
diseño y de fácil fabricación debido a que los materiales necesarios para
su construcción son fácilmente accesibles.

