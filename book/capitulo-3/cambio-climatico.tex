\section{Aedes aegypti y el cambio climático}
\label{sec:cap-3-cambio-climatico}
Los cambios temporales y espaciales en la temperatura, precipitaciones y humedad que se prevé que
ocurran bajo diferentes escenarios de cambio climático afectarán la biología y ecología del vector,
y en consecuencia los riesgos de transmisión de la enfermedad del dengue \cite{dengueUruguayCap2}
.  En las regiones templadas los factores que limitan la distribución y densidad de las poblaciones
de mosquitos son la temperatura, la frecuencia de las lluvias y la duración y severidad del
invierno \cite{ThironIzcazaJ2003}.

Los adultos no resisten temperaturas elevadas ni frío intenso, para los insectos no ectoparásitos\footnote{Un ectoparásito es un organismo que vive en el exterior de otro organismo (el huésped) y se beneficia de la relación a expensas de éste.}
en general, a temperaturas elevadas la humedad en el aire tiene gran efecto sobre su resistencia,
así como la evaporación, el tiempo de exposición y la superficie corporal también
\cite{ThironIzcazaJ2003}. Los mayores efectos del cambio del clima sobre la transmisión de la
enfermedad se observan probablemente en los extremos del rango de temperaturas (como límite
inferior 14-18 \textcelsius, y como límite superior 35-40 \textcelsius ) en el cual ocurre la misma
\cite{dengueUruguayCap2}.

La supervivencia y longevidad, del vector, depende de la temperatura, humedad y disponibilidad de
alimento \cite{ThironIzcazaJ2003}, de modo que un incremento en la temperatura podría permitir una
mayor tasa de supervivencia y una menor tasa de mortalidad \cite{dengueUruguayCap2}, favoreciendo
la dispersión y extensión del la enfermedad. Las temperaturas mínimas parecen ser las más críticas
para el mosquito en muchas regiones por el umbral de supervivencia y de desarrollo
\cite{dengueUruguayCap2}.
