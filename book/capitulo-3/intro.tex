
El Dengue, es una enfermedad viral transmitida por el mosquito Aedes aegypti es considerado como
uno de los graves problemas de salud pública a nivel mundial
\citep{dengueUruguayCap1, world2009dengue, DIBO2005}. En los últimos 50 años, su incidencia ha
aumentado 30 veces con la creciente expansión geográfica hacia nuevos países y, en la actual
década, de áreas urbanas a rurales \cite{world2009dengue}. Aproximadamente, 2.500 millones de
personas viven en países con dengue endémico \cite{world2009dengue, gustavo2006dengue}, y más de
100 países que han informado de la presencia de esta enfermedad en su territorio
\cite{gustavo2006dengue}.

Fue descrita por primera vez en 1780 por Benjamin Rush, en Filadelfia, Pensilvania, Estados Unidos
de América \citep{gustavo2006dengue}. Es un mosquito considerado como doméstico
\cite{luevano1993ciclo}, que afecta principalmente a zonas urbanas de países con climas cálidos.
Existen distintos tipos de serotipos de este virus que circulan principalmente en países del
sudeste asiático, del Pacifico occidental y de América Latina y el Caribe, por lo que la
enfermedad se considera tropical\citep{gustavo2006dengue}. En América Latina se ha producido la
re-emergencia de esta enfermedad en los últimos años, con hiperendemias\footnote{ Cuando una
endemia aumenta su incidencia.} en varios países que derivaron en centenares de miles de casos
\cite{dengueUruguayCap1}. En Paraguay las epidemias del dengue se han agravado, desde la
reintroducción del virus en el año 1988, que según \cite{planControlMspbs2014}, fueron favorecidas
principalmente por: el aumento de la infestación con el vector en varias ciudades del país, la
alta circulación del virus dengue en la región, la falta de participación de la población para la
eliminación de criaderos del mosquito Aedes aegypti, la gran movilidad de la población, dentro y
fuera del país y las condiciones climatológicas favorecen el desarrollo del vector.

El Paraguay es un país con un clima predominantemente subtropical, lo que favorece la aparición y
desarrollo del dengue, que es considerada una enfermedad tropical\cite{gustavo2006dengue,DIBO2005}.
Los factores ambientales y sociales actúan con mucha fuerza sobre el problema del dengue en
Paraguay \cite{website:mspbsHistoria2014}. En consecuencia, en las últimas décadas se han
observado un crecimiento considerable de las notificaciones de posibles casos de dengue, algunas
con derivaciones fatales, afectando a una gran parte de la población y del territorio Paraguayo
(Ver \tabref{tab:cap3-historico-casos-dengue}).

\begin{table}
    \begin{minipage}{\textwidth}
        \begin{center}
        \caption{\label{tab:cap3-historico-casos-dengue} Histórico de casos de dengue notificados, confirmados y con derivación fatal en Paraguay (tomados de \citep{website:mspbsHistoria2014}).}
        \begin{tabular}{p{3cm} c c c c}
            \hline\\
            Año & Periodo (inicio / fin) & Notificados & Confirmados & Muertes\\
            \hline
            \hline\\
            2014 & 29-12-13 / 31-05-14 & 10541 & 1052 & 2\\
            2013 & 30-12-12 / 21-12-13 & 153793 & 131306 & 70\\
            2012 & 01-01-12 / 22-12-12 & 37815 & 30588 & 11\\
            2011 & 03-01-11 / 29-12-11 & 53397 & 42264 & 62\\
            2010 & 11-10-09 / 25-12-10 & 21951 & 13760 & --$^a$\\
        \end{tabular}
        \footnotetext[1]{No se encontraron datos sobre muertes en el periodo.}
        \end{center}
    \end{minipage}
\end{table}


Desde el año 2009 el Paraguay se ha convertido en un país endémico de dengue, lo que implica que
todo el año se registran casos y existe población susceptible que puede enfermar y presentar
formas graves de la enfermedad \cite{planControlMspbs2014}. Con el objetivo de prevenir y mitigar
el impacto, en términos de morbilidad\footnote{Morbilidad es la cantidad de individuos que son
considerados enfermos o que son víctimas de enfermedad en un espacio y tiempo determinado.} y
mortalidad, del dengue, en \cite{planControlMspbs2014}, se ha presentado un plan de acción que
contempla :

\begin{itemize}
    \item \textit{Coordinación y Planificación}: fomentar el trabajo mediante la elaboración de un plan de acción y asegurar un alcance a nivel nacional.

    \item \textit{Vigilancia Entomológica} : monitorear el comportamiento el vector.

    \item \textit{Control Vectorial} : disminuir la población de mosquitos en etapas epidémicas, mediante el control químico y el control mecánico.

    \item \textit{Vigilancia Laboratorial} :brindar un diagnostico rápido y proveer de estándares para los procedimientos laboratoriales para el diagnostico.

    \item \textit{Atención integral de las personas con dengue} :asegurar la atención adecuada y oportuna a las personas con síndrome febril agudo con sospecha de dengue; y  mantener en funcionamiento un laboratorio de apoyo en servicios que permita asegurar cantidad suficiente de pruebas de apoyo clínico para casos de dengue.

    \item \textit{Vigilancia epidemiológica} : aumentar la sensibilidad y oportunidad del sistema de vigilancia; y desarrollar la capacidad necesaria para el control de brotes en zonas de riesgo.

    \item \textit{Componente ambiental} : promover intervenciones sobre determinantes que favorecen la presencia del vector, mediante la elaboración de una propuesta de ley que regule la gestión de neumáticos usados en Paraguay y la promoción de la reducción de residuos sólidos en municipios de alto riesgo.

    \item \textit{Promoción y participación social y comunitaria} : fomentar la movilización social y comunitaria para acciones de control y prevención.

    \item \textit{Comunicación} : informar a diferentes sectores de la población sobre la problemática del dengue.
\end{itemize}

El apoyo y la participación de la ciudadanía es clave en la lucha contra el dengue ya que sin su
apoyo no es posible controlar el vector ni realizar una vigilancia efectiva. Además las
autoridades sanitarias nacionales deben contar con planes preventivos y de contingencia para
evitar la dispersión de la enfermedad, y reducir la cantidad de ciudadanos infectados.
