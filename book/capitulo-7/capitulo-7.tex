\chapter{Conclusiones y Trabajos Futuros}
En este trabajo presentó el diseño e implementación de un modelo predictivo para identificar focos
de infestación del vector del dengue, sustentado en métodos de muestreo para la determinación de la
abundancia poblacional, con el fin de apoyar a la lucha preventiva de esta enfermedad.

El diseño e implementación del modelo como un simulador del proceso evolutivo del vector en el
contexto de un sistema de información geográfica, permite realizar análisis complejos de la
realidad espacial rápidamente, generando información regionalizada para determinar los niveles de
infestación correspondientes al área de estudio. Contar con esta información regionalizada,
asociada con niveles de infestación permitirá a las autoridades sanitarias definir y planificar
las medidas de control, prevención y logística a realizar con con el fin de disminuir los niveles
de infestación en una región.

El simulador de proceso evolutivo se encuentra compuesto por modelos, ampliamente respaldados por
por el material bibliográfico en
\cite{sharpe1977reaction, focks1993dynamic, schoolfield1981non, otero2006stochastic, rueda1990temperature}, utilizados para el cálculo de las tasas de desarrollo y mortalidad de las
distintas etapas de desarrollo del ciclo de vida del vector. Se considera al modelo resultante
como genérico, debido a que sus parámetros pueden ser ajustados para aplicarlos en cualquier
región o área de estudio.

La configuración del simulador del proceso evolutivo requiere de parámetros asociados con las
características biológicas y datos ecológicos correspondientes al área de estudio, por lo que para
su aplicación, estos parámetros de configuración deben ser validados por expertos en el área
mediante trabajos de campo. No obstante, utilizando valores tomados del material bibliográfico de
apoyo, se pudo observar un buen comportamiento de los resultados obtenidos mediante el simulador
del proceso evolutivo, solo presentan pequeñas variaciones en comparación con los valores
observados por expertos en laboratorio en condiciones controladas. Las variaciones observadas
pueden ser causadas por los distintos rasgos característicos de las cepas de mosquitos, utilizadas
en los estudios de referencia, que permiten una mayor o menor tolerancia a ciertas condiciones.

Como trabajo futuro podría considerarse la extensión del modelo para incluir otras variables
con el fin de analizar su relación y el impacto de las mismas con las zonas de riesgo. Se podrían
incluir variables como: casos reportados de dengue, posibles zonas de riesgo(cementerios, patios
baldíos, etc), y la densidad poblacional correspondiente al área de estudio. También se podrían
incluir variables climatologícas como : las lluvias con el fin de analizar su impacto en el
desarrollo, la dispersión del vector del Aedes aegypti, y en la generación de sitios de
reproducción, así como tener en cuenta las limitaciones geográficas a la hora del vuelo y la
ovipostura.

Un interesante trabajo futuro sería aprovechar la información generada, por el simulador del
proceso evolutivo, para realizar una optimizacíon de las rutas de fumigación y analizar el efecto
de la fumigación en el desarrollo de ciclo de vida del vector.

Teniendo en cuenta que las tareas de geoprocesamiento requieren gran capacidad computacional y que
los parámetros de entrada como : muestras muy grandes, una alta densidad de larvas, y un
clima muy favorable, pueden hacer crecer el tamaño de la población de forma considerable, causando
que el tiempo de respuesta aumente. Se podría paralelizar el simulador del proceso evolutivo para
optimizar el tiempo de respuesta y distribuir mejor las tareas de procesamiento.

Dado que este trabajo se propone realizar el conteo de lavas mediante el procesamiento digital de
imágenes, sería interesante analizar otros métodos, como el presentado en
\cite{gonzalez2008segmentacion}, con el fin de mejorar la confianza y la precisión de conteo de
larvas.
