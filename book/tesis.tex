\documentclass[final,fmstyle]{./util/ucathesis}

% paquetes recomendados
\usepackage{amsmath,amsthm}
\usepackage{textcomp}
\usepackage[T1]{fontenc}
\usepackage[spanish]{babel}
\usepackage[utf8]{inputenc}
\usepackage{csquotes}
\usepackage{enumerate}
\usepackage{enumitem}
\usepackage{caption}
\usepackage{subcaption}
%\usepackage[style=authoryear, uniquename=full, sorting=none,backend=biber, natbib=true]{biblatex}
\usepackage[style=alphabetic, uniquename=full, sorting=none,backend=biber, natbib=true]{biblatex}
%\usepackage[style=authoryear,maxnames=1,backend=biber]{biblatex}

\usepackage{listings}
% para la lista de simbolos
\usepackage{array} %for vertical thick lines in tables
\usepackage{multirow} %multirow tables
\usepackage{nicefrac} %for fractions like 1/4
\usepackage[table]{xcolor} %para colorear las tablas
% para las tablas

%referencias
\addbibresource{referencias.bib}
%se importan las configuraciones custmizdas realizadas.
\usepackage{tablefootnote}
\renewcommand{\thefootnote}{\arabic{footnote}}
% Macro for 'List of Symbols', 'List of Notations' etc...
\def\listofsymbols{
    \newpage
\chapter*{Lista de Simbolos\hfill}
\addcontentsline{toc}{chapter}{Lista de Simbolos}
\begin{tabbing}
% YOU NEED TO ADD THE FIRST ONE MANUALLY TO ADJUST THE TABBING AND SPACES
$n$~~~~~~~~~~\=\parbox{5in}{Vector size\dotfill \pageref{symbol:nml}}\\
%ADD THE REST OF SYMBOLS WITH THE HELP OF MACRO

%% se añaden nuevos simbolos con el macro \newsymbol y se hace referecnia
% al simbolo utilizando \addsymbol{symbol:LABEL}

\newsymbol T : {Periodo de tiempo}{symbol:T}
\newsymbol t_{i} : {datos climáticos en el instante i}{symbol:ti}
\newsymbol P : {Población}{symbol:P}
\newsymbol p_i : {El i-esimo individuo de la población}{symbol:Ii}
%atributos de p_i
\newsymbol \eta(p_i): {Madurez del individuo $p_{i}$}{symbol:ma-pj}
\newsymbol \xi(p_i): {Expectativa de vida de $p_{j}$}{symbol:ex-pj}
\newsymbol \tau(p_i): {Estado de vida de $p_{j}$}{symbol:estado-pj}
\newsymbol S(p_i): {Sexo de $p_{j}$}{symbol:sexo-pj}
%operadores básicos
\newsymbol \eta(t_i, p_i): {función de madurez}{symbol:madurez}
\newsymbol \xi(t_i, p_i): {función de expectativa de vida}{symbol:espectativa-vida}
\newsymbol \theta (t_{i}, p_{j}): {función de reducción}{symbol:esta-muerto}

\newsymbol N: {tamaño del periodo}{symbol:N}
\newsymbol M: {tamaño de la población}{symbol:M}

\newsymbol \mathbf{\omega}: {Matriz de madurez}{symbol:mat-madurez}
\newsymbol \mathbf{\upsilon}: {Matriz de espectativa de vida}{symbol:mat-espectativa}

% ALWAYS KEEP THE FOLLOWING LINE
\end{tabbing}

    \clearpage{}
}
\def\newsymbol #1: #2#3{$#1$ \> \parbox{5in}{#2 \dotfill \pageref{#3}}\\}
\def\addsymbol#1{\label{#1}}
% Para las imagenes en grilla
% custom commands
\newcommand{\foreign}[1]{{\it #1}}
\DeclareMathOperator*{\argmax}{arg\,max}
\algsetup{indent=2em}

\usepackage{amsmath}

\newcounter{eqn}
\renewcommand*{\theeqn}{\alph{eqn})}
\newcommand{\num}{\refstepcounter{eqn}\text{\theeqn}\;}
\newcommand{\initbox}{\setcounter{eqn}{0}}

\newcommand{\figref}[1]{Figura \ref{#1}}
\newcommand{\tabref}[1]{Tabla \ref{#1}}
\newcommand{\secref}[1]{sección \ref{#1}}


\makeatletter
\newcommand{\putindeepbox}[2][0.7\baselineskip]{{%
    \setbox0=\hbox{#2}%
    \setbox0=\vbox{\noindent\hsize=\wd0\unhbox0}
    \@tempdima=\dp0
    \advance\@tempdima by \ht0
    \advance\@tempdima by -#1\relax
    \dp0=\@tempdima
    \ht0=#1\relax
    \box0
}}
\makeatother


% datos de la tesis
\title{Modelo predictivo de focos de dengue aplicado a Sistemas de informaci\'on geogr\'afica}
\author{Maximiliano B\'{a}ez Gonz\'{a}lez}
\degree{Inform\'{a}tica}

\advisor{Prof. MSc.}{Guillermo Gonz\'{a}lez}

%\newtheorem{definicion}{Definicin}

\logosource{./graphics/logo.png}
\institution{Universidad Nacional de Asunci\'{o}n}
\faculty{Facultad Polit\'{e}cnica}
\address{San Lorenzo - Paraguay}

\begin{document}

\maketitle     % esto hace las portadas

% Agradecimientos
\newpage

\chapter*{\centering Agradecimientos}

A mis padres por haberme brindado la oportunidad de estudiar una carrera Universitaria, por su esfuerzo y entera confianza.

A mis hermanos Ale, Mabel, Marcos y Matías gracias la paciencia, por acompañarme siempre en los buenos y malos momentos, por brindarme su apoyo incondicionalmente.

A mi hermano Marcos, gracias por todo el apoyo, la orientación, por iluminar mi camino y darme una pauta para poder realizarme en mis estudios y en la vida, gracias por siempre estar cerca a pesar de la distancia.

A mi tutor Prof. MSc. Guillermo González por encaminar y acompañar el desarrollo este trabajo hasta su culminación, gracias profesor por haber compartido sus conocimientos conmigo y su gran ayuda para lograr esta meta tan importante.

A Lindsay, gracias por siempre estar a mi lado, por acompañarme incondicionalmente, por todos los consejos y por ayudarme a enfrentar y superar los momentos difíciles.

A mis jefes, Ing. Joaquin Lima y Ing. Juan Talavera por brindarme tiempo y espacio para el desarrollo de este proyecto siempre que lo necesité.

A mis amigos y compañeros de clases, por permitirme compartir toda esta etapa de formación académica, humana y profesional, gracias por todos los buenos momentos dentro y fuera de las aulas.




% los siguientes comandos producen 'indices.

% Tabla de contenidos
\tableofcontents
% Lista de figuras
\listoffigures
% Lista de tablas
\listoftables
% Lista de algoritmos
\listofalgorithms
%\include{acronimos}
%\listofsymbols


\mainmatter  % inician los capitulos de la tesis


% incluye aqui los capitulos (un archivo .tex por capitulo)
\chapter{Introducción}
En este capítulo se presentan los antecedentes y la importancia de este trabajo, así como las
propuestas que se realizan con el fin de apoyar la lucha preventiva contra el dengue. Conceptos de
la ecología del vector, método de prevención y sistemas de información geográficos son puntos
vitales de este trabajo por lo que son introducidos brevemente en este capítulo. No obstante, en
capítulos siguientes, se tratarán todos estos temas serán desglosados detalladamente.

\section{Justificación y Antecedentes}

El Dengue es una enfermedad viral transmitida por el mosquito Aedes aegypti y representa un
problema grave para la salud pública a nivel mundial \citep{dengueUruguayCap1, world2009dengue, DIBO2005}. Más de 2 500 millones de personas viven en
zonas en riesgo de dengue y más de 100 países han informado de la presencia de esta enfermedad en
su territorio \cite{world2009dengue, gustavo2006dengue}.

Los sistemas de información geográfica (SIG) constituyen un área de rápido desarrollo dentro
de la informática y ofrecen métodos sumamente innovadores para hacer frente a algunas demandas
técnicas que constituyen un reto. Un sistema de información geográfica es una integración
organizada de hardware, software, datos geográficos y personal, diseñado para capturar, almacenar,
manipular, analizar y visualizar en todas sus formas la información geográficamente referenciada
con el fin de resolver problemas complejos de planificación y gestión \citep{lopezMarcos2007}.

Las autoridades sanitarias, en sus tareas de vigilancia en Salud Pública, tienen en los SIG una
herramienta fundamental para conocer cómo se extiende una enfermedad, estudiar su posible relación
con un potencial foco de riesgo, o localizar un brote epidémico \citep{vgomesAegis2001}.

Actualmente el dengue es catalogado como ejemplo de una enfermedad que puede constituir una
emergencia de salud pública de interés internacional con implicaciones para la seguridad
sanitaria \citep{dengueUruguayCap1, world2009dengue}, debido a la necesidad de interrumpir la infección y la rápida propagación de la
epidemia más allá de las fronteras.

Debido a una lucha no efectiva en contra el dengue, en  el Paraguay, cada día se reportan más
nuevos casos de infectados por el dengue. La detección de focos de infección en base a casos
reportados y el combate correctivo realizado actualmente, no resultan efectivos ante la lucha
contra el dengue. Se podrían obtener mejores resultados, realizando un cambio de enfoque, en el
combate contra el dengue, de uno correctivo a uno preventivo.

\section{Propuesta de esta tesis y Objetivos}
En este trabajo se propone la construcción de una herramienta que permita realizar estudios
epidemiológicos de forma cartográfica, especializada para el particular caso del dengue. Se
pretende constituir un modelo que permita predecir los focos de riesgo del dengue, con la ayuda de
un sistema de información geográfica. Para determinar las posibles zonas de riesgo, se debe
construir un algoritmo que permita simular el comportamiento de un mosquito o de un conjunto de
mosquitos de acuerdo a las variables de la región y tener en cuenta sus patrones migratorios.

El objetivo principal es diseñar un modelo y construir una herramienta que permita analizar la
extensión del vector del dengue y estudiar su posible relación con un potencial foco de riesgo, de
forma a realizar una predicción de posibles focos y ayudar a realizar una lucha preventiva contra
la enfermedad. Donde el modelo resultante se pueda aplicar a cualquier región o área de estudio y
que ayude a las autoridades pertinentes para toma de decisiones en la lucha contra el dengue.

\section{Organización del trabajo}
El trabajo está organizado como sigue: el capítulo 2 introduce a los sistemas de información
geográfica, la representación de datos geoespaciales y los métodos de interpolación como
herramienta para análisis espacial. En el capítulo 3 se presenta al aedes aegypti el principal
transmisor del dengue, sus características biológicas y los métodos de muestreo de la abundancia
pobacional del vector. Estos capítulos representan el estado del arte de este trabajo.

El capítulo 4 se presenta el modelo matemático propuesto para la identificación de focos de
infestación y la simulación del proceso evolutivo del ciclo de vida del vector. En el capítulo 5
se presenta la implementación computacional denominada GeoDengue, sus requerimientos, diseño y
arquitectura y las herramientas y tecnologías utilizadas para su implementación.

En el capítulo 6 se presenta los resultados experimentales obtenidos. Finalmente el capítulo 7
presenta las conclusiones de este trabajo y los  posibles trabajos futuros derivados de esta tesis.

\chapter{Aedes aegypti, principal transmisor del dengue}
%* Historia de la enfermedad
El Dengue, es una enfermedad viral transmitida por el mosquito Aedes Aegypti es considerado como
uno de los graves problemas de salud pública a nivel mundial \citep{dengueUruguayCap1}. Fue
descrita por primera vez en 1780 por Benjamin Rush, en Filadelfia, Pensilvania, Estados Unidos de
América \citep{gustavo2006dengue}.

El Aedes Aegypti es un mosquito considerado como doméstico \citep{luevano1993ciclo}, que,
principalmente a zonas urbanas de paises con clímas cálidos. Existen distintos tipos de serotipos
de este virus que circulan principalmente en países del sudeste asiático, del Pacifico occidental
y de América Latina y el Caribe, por lo que la enfermedad se considera tropical
\citep{gustavo2006dengue}. En América Latina se ha producido la re-emergencia de esta enfermedad
en los últimos años, con hiperendemias\footnote{ Cuando una endemia aumenta su incidencia.} en
varios países que llevaron a centenares de miles de casos \citep{dengueUruguayCap1}. En Paraguay
las epidemias del dengue se han agravado, desde la reintroducción del virus en el año 1988, que
según \cite{planControlMspbs2014}, fueron favorecidas principalmente por, el aumento de la
infestación con el vector en varias ciudades del país, la alta circulación del virus dengue en la
región, la falta de participación de la población para la eliminación de criaderos de mosquitos de
Aedes Aegypti, la gran movilidad de la población, dentro y fuera del país y las condiciones
climatológicas favorecen el desarrollo del vector.

El paraguay es un país con un clima predominantemente subtropical, lo que favorece la aparición y
desarrollo del dengue, que es considerada una enfermedad tropical \cite{gustavo2006dengue}. Los
factores ambientales y sociales actúan con mucha fuerza sobre el problema del dengue en Paraguay
\cite{website:mspbsHistoria2014}. En consecuencia, en las últimas décadas se han observado un
crecimiento considerable de las notificaciones de posibles casos de dengue, algunas con
derivaciones fatales, afectando a una gran parte de la población y del territorio Paraguayo
(Ver \tabref{tab:cap3-historico-casos-dengue}).

\begin{table}[H]
    \begin{minipage}{\textwidth}
        \begin{center}
        \caption{\label{tab:cap3-historico-casos-dengue} Histórico de casos de dengue notificados, confirmados y con derivación fatal en Paraguay (tomados de \citep{website:mspbsHistoria2014}).}
        \begin{tabular}{p{3cm} c c c c}
            \hline\\
            Año & Periodo (inicio/fin) & Notificados & Confirmados & Muertes\\
            \hline
            \hline\\
            2014 & 29-12-13 / 31-05-14 & 10541 & 1052 & 2\\
            2013 & 30-12-12 / 21-12-13 & 153793 & 131306 & 70\\
            2012 & 01-01-12 / 22-12-12 & 37815 & 30588 & 11\\
            2011 & 03-01-11 / 29-12-11 & 53397 & 42264 & 62\\
            2010 & 11-10-09 / 25-12-10 & 21951 & 13760 & --$^a$\\
        \end{tabular}
        \footnotetext[1]{No se encontraron datos sobre muertes en el periodo.}
        \end{center}
    \end{minipage}
\end{table}


Desde el año 2009 el Paraguay se ha convertido en un país endémico de dengue, lo que implica que
todo el año se registran casos y existe población susceptible que puede enfermar y presentar formas graves de la enfermedad \cite{planControlMspbs2014}. Con el objetivo de prevenir y mitigar
el impacto, en términos de morbilidad y mortalidad, del dengue, en \cite{planControlMspbs2014}, se
ha presentado un Plan de Acción que contempla :

\begin{itemize}
    \item \textit{Coordinación y Planificación}: fomentar el trabajo mediante la elaboración de un plan de acción y asegurar un alcance a nivel nacional.

    \item \textit{Vigilancia Entomológica} : monitorear el comportamiento el vector.

    \item \textit{Control Vectorial} : disminuir la población de mosquitos en etapas epidémicas, mediante el control químico y el control mecánico.

    \item \textit{Vigilancia Laboratorial} :brindar un diagnostico rápido y proveer de estándares para los procedimientos laboratoriales para el diagnostico.

    \item \textit{Atención integral de las personas con dengue} :asegurar la atención adecuada y oportuna a las personas con síndrome febril agudo con sospecha de dengue; y  mantener en funcionamiento un laboratorio de apoyo en servicios que permita asegurar cantidad suficiente de pruebas de apoyo clínico para casos de dengue.

    \item \textit{Vigilancia epidemiológica} : aumentar la sensibilidad y oportunidad del sistema de vigilancia; y desarrollar la capacidad necesaria para el control de brotes en zonas de riesgo.

    \item \textit{Componente ambiental} : promover intervenciones sobre determinantes que favorecen la presencia del vector, mediante la elaboración de una propuesta de ley que regule la gestión de neumáticos usados en Paraguay y la promoción de la reducción de residuos sólidos en municipios de alto riesgo.

    \item \textit{Promoción y participación social y comunitaria} : fomentar la movilización social y comunitaria para acciones de control y prevención.

    \item \textit{Comunicación} : informar a diferentes sectores de la población sobre la problemática del dengue.
\end{itemize}

El apoyo y la participación de la ciudadanía es clave en la lucha contra el dengue ya que sin su
apoyo no es posible controlar el vector ni realizar una vigilancia efectiva. Cada ciudadano debe
conocer la enfermedad, los efectos de la misma y el vector que la transmite. Además se debe sumar
la predisposición de ayudar con la eliminación de los criaderos.

Las autoridades sanitarias nacionales deben reaccionar de inmediato ante el riesgo de epidemias de
dengue, sin esperar a que ocurran muertes y antes de que se sienta la presión de la población y de
los medios masivos de comunicación \citep{gustavo2006dengue}. Es necesario tomar conciencia de que
si las grandes sumas de dinero que se gastan cuando ocurren epidemias se invirtieran en eliminar o
atenuar los macrofactores determinantes correspondientes, se podrían evitar o atenuar muchas
epidemias \citep{gustavo2006dengue}.

%* Dengue en Paraguay
%* Distribución geográfica
%* Salud pública

\subsection{Epidemiología}

Durante la decada 2000 se ha visto un notable crecimiento de afectados de la enfermedad en la zona de america latina en paises como Paraguay, Brasil, Venezuela Peru. Principalmente en nuestro pais, Paraguay, existen epidemias  de distintos grados de gravedad en cada temporada de verano. Estas epidemias son ocasionadas principalmente por la falta de medidas de prevención y control contra el mosquito transmisor Aedes Aegypti.\\

Si bien es cierto que se realizan campañas contra la enfermedad, los recorridos de fumigacion y limpieza terminan siendo obsoletos ante el no despertar ciudadano frente a la enfermedad. La existencia de grandes cantidades de patios baldíos y terrenos sucios en zonas urbanas dificulta la eliminación del agente transmisor y como consecuencia la enfermedad prevalece y se propaga.\\

La cantidad de afectados por la enfermedad en los ultimos años en todo Paraguay son:
\begin{itemize}
\item 13.766 casos en el año 2010
\item 42.945 casos en el año 2011
\item 2.347 casos en el año 2012
\item 130.155 casos en el año 2013\\
\end{itemize}

El aumento de la población del vector en zonas urbanas del pais, la masiva presencia del virus dengue en los países vecinos, la no colaboración de la poblacion para la eliminacion de criaderos de mosquitos de Aedes aegypti, además de la gran circulación de la poblacion en entre zonas habitadas son alguno de los factores que permiten que la enfermedad siga provocando olas de alerta y riesgo. A estos factores hay qye sumar el factor de la temperatura de nuestro país que es ideal para el desarrollo del agente transmisor.\\


Uno de los objetivos criticos es proveer un sistema de informacion que permita accionar preventivo en zonas de posibles focos de la enfermedad. Cada año en las ultimas 2 decadas se revive la misma situacion, hospitales abarrotados, escases de medicos y medicamentos, situacion de alerta, colapso social ante la epidemia.\\

\subsection{Distribucion Geografica}

El dengue es una enfermedad presente en todo el territorio del Paraguay con fuerte presencia desde hace una decada. Pero la distribucion de la enfermedad no es homogenea sino que se mapea a nivel macro a la distribucion de la poblacion. De ahí que el departamento central es uno de los departamentos con mayor indice de infestación de la enfermedad. En la figura 1 se obserba el resumen de los primeros meses del 2014. Departamentos clasificados segun sean zonas endemicas o no. Los departamentos con mayor densidad poblacional y movimiento de personas son siempre zonas endemicas.\\

Las zonas más afectadas en nuestro país son los departamentos de Central, Amambay e Itapúa. Seguidos de Coordillera, Paraguari y Alto Parana. Otras zonas y departamentos también presentan casos de la enfermedad solo que no son consideradas zonas endemicas o de riesgo. Esta distribución que denota las zonas de mayor riesgo se repite en cada época de flagelo de la enfermedad.\\

A nivel intrinseco en los objetivos de este trabajo está realizar resumenes en mapas similares a los que se muestran en las figuras 1, 2 y 3 pero conteniendo informacion a priori sobre la enfermedad. No señalar las zonas endemicas o no endemicas sino señalar las posibles zonas endémicas o no endémicas según información recolectada sobre la distribución larvaria. Esto permitiría realizar planes de control y prevención de la enfermedad.\\

\begin{figure}
\centering
\includegraphics[width=0.8\textwidth]{./graphics/Diapositiva03.JPG}
\caption{\label{fig:mapa1}Zonas endemicas en todo el pais. Anho 2014}
\end{figure}

\begin{figure}
\centering
\includegraphics[width=0.8\textwidth]{./graphics/Diapositiva04.JPG}
\caption{\label{fig:mapa2}Mapa de riesgo de la semana epidermiologica 05. Anho 2013 }
\end{figure}

\begin{figure}
\centering
\includegraphics[width=0.8\textwidth]{./graphics/Diapositiva05.JPG}
\caption{\label{fig:mapa3}Mapa del area metropolitana de Asuncion.}
\end{figure}

\subsection{Salud publica}
Cada año el ministerio de salud pública y bienestar social (MSPyBS) en conjunto con otras organizaciones estatales y privadas realiza un plan de de acción en contra de la enfermedad. Un documento detallado presenta el proyecto y los responsables de llevarlo a cabo. El problema de los planes de acción es la dependencia del accionar ciudadano.
Del documento Dengue Plan 12 11 13 del MSPyBS se menciona los objetivos del plan de acción: \\

\textit{"...Los objetivos específicos para cada pilar son los siguientes:\\
\begin{enumerate}
\item Coordinación y Planificación: Fomentar el trabajo intersectorial, el monitoreo y la difusión del plan de acción
\item Vigilancia Entomológica: Vigilar, informar y alertar sobre riesgos ambientales
\item Control Vectorial: Disminuir la población de mosquitos
\item Vigilancia Laboratorial: Garantizar la representatividad de la vigilancia
\item Laboratorio de apoyo en Servicios: Garantizar el acceso a pruebas laboratoriales de diagnóstico
\item Vigilancia Epidemiológica: Aumentar la sensibilidad y oportunidad de la vigilancia
\item Componente Ambiental: Promover intervenciones sobre determinantes de la presencia del mosquito
\item Promoción y Participación Social: Fomentar la movilización social en acciones de control y prevención
\item Comunicación: Informar a la población sobre la problemática del dengue
\item Atención Integral: Asegurar la atención adecuada y oportuna a las personas con síndrome febril agudo con sospecha de dengue..."
\end{enumerate}}

Se concluye de los objetivos la dependencia del accionar de la población. Sin el apoyo de la ciudadanía no es posible controlar el vector ni realizar una vigilancia efectiva . La participación ciudadana tiene que empezar desde el conocimiento de la enfermedad y del vector transmisor. A eso debe sumarse la predisposición para realizar la limpieza del hogar/es propios mediante la eliminación de recipientes que acumulen agua principalmente. El problema es la enorme cantidad de patios baldíos y terrenos desabitados que luego de lluvias terminan siendo sitios propensos para alojar al mosquito.\\


La propuesta realizada en este trabajo tambíén requiere de la participación de la ciudadanía, pero desde un punto de vista distinto ya que el plan de acción será aplicado a las zonas de alto índice de población larval en primer lugar, permitiendo optimizar recursos y fuerza. Requerir de la acción ciudadana luego de que se presenten casos de la enfermedad se torna imposible, el estado de alerta desata preocupación y poco interés en luchar contra la enfermedad. La población en caso de riesgo busca refugiarse del la ola de la enfermedad y aunque existe colaboración para realizar un control sobre el vector el hecho de que ya exista afectados impide lograr una lucha eficiente.

%* Características biológicas de Aedes aegypti
%* Factores ecológicos y productividad de Aedes aegypti

\section{Aedes Aegypti. Mosquito transmisor del dengue}
\label{sec:caracteristicas-biologicas}

De todas las especies de mosquitos conocidos, con importancia en salud pública, \textit{Aedes Aegypti},
es considerada la más peligrosa por tener la capacidad de transmitir el mayor número de enfermedades 
arbovirales\footnote{Las infecciones arbovirales son los virus que se transmiten por los mosquitos. 
“Arbo” es una abreviatura que significa transmitida por los artrópodos, los cuales son insectos.}, al 
hombre.\cite{ThironIzcazaJ2003}

Por sus hábitos, el \textit{Aedes Aegypti}, es considerado como doméstico ya que se encuentra radicado en
criaderos naturales y artificiales en, viviendas humanas o en sus alrededores. Para establecer sus criaderos
necesita prácticamente de cualquier objeto que retenga agua.

\begin{itemize}
    \item  Recipientes artificiales :jarrones, floreros, tambos, pilas, tanques, cubetas, son los lugares más
    comunes para su cría, así como también aquellos que tienen la capacidad de retener agua de lluvia
    principalmente, tales como llantas, envases desechados y canales de techo, entre otros.
    \item Recipientes naturales : conchas de moluscos, cáscaras de frutos, huecos en los árboles, axilas de
    plantas y otras cavidades naturales.
\end{itemize}

Prefieren agua limpia, con bajo tenor orgánico y de sales disueltas. La puesta de huevos la realizan en la
superficie del recipiente. Algunos recipientes le son más atractivos que otros, en especial los de color oscuro,
de boca ancha, que están al nivel del suelo y se encuentran en la sombra\cite{ThironIzcazaJ2003}.

\subsection{Ciclo biológico del Aedes Aegypti}
Son insectos de metamorfosis completa. Durante su desarrollo ontogénico pasan por los estados de huevo, larva,
pupa y adulto\cite{web-site:gMonteroBiologia}.

\subsubsection{Huevo}
\label{subsec:ciclo-biologico-huevo}
Los huevos miden aproximadamente un milímetro de longitud, son depositados uno a uno al ras del agua quedando
adheridos a las paredes del recipiente\cite{ThironIzcazaJ2003}. Los mismos desarrollan una gran resistencia una
vez que han completado el desarrollo embrionario, el embrión dentro del huevo es capaz de resistir largos 
períodos de desecación por meses o hasta por más de un año. El contacto del agua con los huevos el paso a la
siguiente etapa del mosquito.

\subsubsection{Larva}
\label{subsec:ciclo-biologico-larva}
Las larvas que emergen inician un ciclo de 4 estadios larvales, son exclusivamente acuáticas la fase larval es el
período de mayor alimentación y crecimiento. Pasan la mayor parte del tiempo alimentándose de material orgánico
sumergido o acumulado en las paredes y el fondo del recipiente\cite{web-site:gMonteroBiologia}. Según 
\cite{ThironIzcazaJ2003} la duración del desarrollo larval está en función de la temperatura, la
disponibilidad de alimento y la densidad de larvas en el criadero. En condiciones óptimas, el período larval
desde la eclosión hasta la pupación puede ser de cinco días, pero por lo regular ocurre de siete a catorce 
días. Tanto \cite{ThironIzcazaJ2003} y \cite{web-site:gMonteroBiologia} afirman que los primeros tres estadios se
desarrollan  rápidamente, el cuarto se toma más tiempo aumentando considerablemente  su tamaño y peso, en
condiciones de baja temperatura o escasez de alimento el cuarto estadio puede prolongarse por varias semanas. 

En cuanto la mortalidad,\cite{ThironIzcazaJ2003} señala que la mortalidad, más elevada ocurre frecuentemente en
los primeros estadios larvales. La mayoría de los hábitats se encuentran en condiciones inestables y es posible
que esta sea las causa de mayor mortalidad de larvas y pupas.

\subsubsection{Pupa}
\label{subsec:ciclo-biologico-pupa}
Las pupas no se alimentan, su función es la metamorfosis del estadio larval al adulto. El estadio de pupa dura
aproximadamente dos o tres días, emergiendo alrededor del 88\% de los adultos en cuestión de 48 
horas\cite{ThironIzcazaJ2003}. Según \cite{web-site:gMonteroBiologia}, el período pupal dura de 1 a 3 días en
condiciones favorables, con temperaturas entre 28 y 32 \textcelsius. Las variaciones extremas de temperatura
pueden dilatar este período.

\subsubsection{Adulto}
\label{subsec:ciclo-biologico-adulto}

La función más importante del adulto de \textit{Aedes Aegypti} es la reproducción. El ciclo completo de 
huevo a adulto, se completa en óptimas condiciones de temperatura y alimentación en 10 
días\cite{web-site:gMonteroBiologia}. Pueden permanecer vivos en el laboratorio durante meses y en la naturaleza
pocas semanas. Con una mortalidad diaria de 10\%, la mitad de los mosquitos morirán durante la primera semana y 
95\% en el primer mes\cite{ThironIzcazaJ2003}. 

Las hembras se alimentan de sangre de cualquier vertebrado teniendo una marcada predilección por la del hombre.
Necesitan alimentarse de sangre para obtener las proteínas necesarias para la formación de los huevos. Las partes
bucales del macho no están adaptadas para chupar sangre, se alimentan de carbohidratos de cualquier fuente
accesible como frutos o néctar de flores que satisface sus requerimientos energéticos, las hembras también se
alimentan de esta misma fuente como complemento indispensable\cite{ThironIzcazaJ2003}.

Su hábitat para reproducción y ovipostura son los lugares con agua estancada preferentemente limpia, lugares
oscuros y quietos tales como latas, botellas vacías, neumáticos usados, baldes, etc.

Según \cite{ThironIzcazaJ2003}, antes de 24 horas ambos sexos están listos para el apareamiento, alrededor del 
58\% de las hembras nulíparas son inseminadas antes de su primera alimentación sanguínea, un 17\% durante y el 
25\% es inseminada entre la segunda alimentación y la primera oviposición; los machos rondan como voladores
solitarios aunque es más común que lo hagan en grupos pequeños atraídos por los mismos huéspedes vertebrados 
que las hembras. La alimentación y la postura ocurren principalmente durante el día registrando mayor actividad
en las primeras horas de haber amanecido, a media mañana, a media tarde o al anochecer. El apareamiento que por
lo general se efectúa durante el vuelo debido a que el macho es atraído por el sonido emitido por las alas de la
hembra, una alimentada de sangre ocurren pocos apareamientos. Una inseminación es suficiente para fecundar todos
los huevos que la hembra produzca en toda su vida\cite{ThironIzcazaJ2003}. 

Es común que después de cada alimentación sanguínea la hembra desarrolle un lote de huevos, la cantidad de 
huevos depende de la alimentación según \cite{cabezas2005dengue} puede variar entre 100 a 200 huevos. El
intervalo de tiempo que transcurre entre la alimentación sanguínea y la postura, denominado ciclo gonotrófico, 
es de 48 horas en los trópicos bajo condiciones óptimas de temperatura\cite{ThironIzcazaJ2003}. 

Los mosquitos tienen la particularidad de volar en sentido contrario al la dirección al viento a una
velocidad máxima de $2 km/h$ según \cite{web-site:speedAnimals}. En cuanto al desplazamiento, según 
\cite{cabezas2005dengue} la hembra no sobrepasa los 50-100 metros durante su vida, tiende a permanecer en el
mismo lugar donde emergió. Sin embargo si no hay recipientes aptos, una hembra grávida puede volar tres
kilómetros para poner sus huevos. Los machos se dispersan menos que las hembras.

\subsection{Cambios Climáticos}
Uno de los aspectos más importantes del mosquito Aedes Aegypti es su dependencia a la temperatura. Un país con temperatura tropical (promedio 25\textcelsius.) es un país ideal para la supervivencia del Aedes Aegypti no así un país con extremo calor o un clima más frío. Cada etapa de su desarrollo está ligado a condiciones climáticas; no solo temperatura sino también, lluvias y humedad. La lluvia permite que el agua se acumule en distintos recipientes; barriles, llantas y cubiertas, planteras, canaletas, etc. Se realizaron varios estudios analizando la influencia de la temperatura en el desarrollo del mosquito Aedes Aegypti. De los resultados de las pruebas se pueden obtener datos como el promedio de días en el que se pasa del estado larva a pupa ver Cuadro 1.\\

Esta información es muy valiosa en el estudio del mosquito ya que con el pronóstico del tiempo uno puede estimar el tiempo de desarrollo del Aedes Aegypti y determinar el crecimiento de la población actual (por ej. En 15 días aumentará la población actual del Aedes Aegypti en un 20\% dada las condiciones del clima previsto en esta zona)

\begin{table}
\centering
\begin{tabular}{l|r}
Temperatura & Tiempo en estado larval y pupa \\\hline
13 & 0 \\
15-20 & 10 a 17.4 \\
20-25 & 9 a 13 \\
25-36 & 5 a 7 \\
36+ & 0
\end{tabular}
\caption{\label{tab:widgets}Tiempo promedio de duración en días del estado larval y pupa a diferentes temperaturas.}
\end{table}


\section{Aedes aegypti y el cambio climático}
Los cambios temporales y espaciales en la temperatura, precipitaciones y humedad que se prevé que ocurran bajo
diferentes escenarios de cambio climático afectarán la biología y ecología de Ae. aegypti y, en consecuencia, 
los riesgos de transmisión de la enfermedad del dengue\cite{dengueUruguayCap2}.

%Reescribir esta parte%
Los mayores efectos del cambio del clima sobre la transmisión de la enfermedad se observan probablemente en los
extremos del rango de temperaturas en el cual ocurre la misma (14-18\textcelsius como límite inferior y 
35-40\textcelsius como límite superior)\cite{dengueUruguayCap2}. Por debajo del rango inferior existe un impacto
no lineal sobre el período de incubación extrínseca, y, en consecuencia, sobre la transmisión de la enfermedad,
mientras que por encima del rango superior de temperatura la transmisión se interrumpe\cite{dengueUruguayCap2}.

%Reescribir esta parte%
Si la temperatura aumenta, las larvas de Ae. aegypti necesitan menos tiempo para madurar\cite{dengueUruguayCap2} y,
en consecuencia, hay una mayor capacidad para producir más descendientes durante el período detransmisión. Por 
su parte, los mosquitos-hembra adultas digieren más rápidamente la sangre y se alimentan más frecuentemente
(Gillies, 1953), lo cual incrementa la intensidad de la transmisión. Por encima de 34oC generalmente se produce un
impacto negativo sobre la sobrevivencia del vector\cite{dengueUruguayCap2}. 
%Reescribir esta parte%

El incremento de la temperatura en algunas regiones del mundo podría permitir una mayor tasa de sobrevivencia del
vector en invierno y ayudar a extender su distribución a regiones previamente libres de la enfermedad, o a
aumentar la transmisión de la enfermedad en regiones endémicas, y también a cambiar las estaciones de transmisión.
Las temperaturas mínimas parecen ser las más críticas para el mosquito en muchas regiones por el umbral de
sobrevivencia y de desarrollo. Es también más baja la tasa de alimentación, lo cual reduce las posibilidades de
contacto con sus hospederos y eventualmente afecta la tasa de transmisión viral\cite{dengueUruguayCap2}. Las
condiciones del tiempo en los dos meses previos podrían ser críticas para la trasmisión del dengue en el mes en
curso\cite{dengueUruguayCap2}.
%revisar%

Uno de los aspectos más importantes del mosquito Aedes Aegypti es su dependencia a la temperatura. Un país con temperatura tropical (promedio 25\textcelsius.) es un país ideal para la supervivencia del Aedes Aegypti no así un país con extremo calor o un clima más frío. Cada etapa de su desarrollo está ligado a condiciones climáticas; no solo temperatura sino también, lluvias y humedad. La lluvia permite que el agua se acumule en distintos recipientes; barriles, llantas y cubiertas, planteras, canaletas, etc. Se realizaron varios estudios analizando la influencia de la temperatura en el desarrollo del mosquito Aedes Aegypti. De los resultados de las pruebas se pueden obtener datos como el promedio de días en el que se pasa del estado larva a pupa ver Cuadro 1.\\

Esta información es muy valiosa en el estudio del mosquito ya que con el pronóstico del tiempo uno puede estimar el tiempo de desarrollo del Aedes Aegypti y determinar el crecimiento de la población actual (por ej. En 15 días aumentará la población actual del Aedes Aegypti en un 20\% dada las condiciones del clima previsto en esta zona)

\begin{table}
\centering
\begin{tabular}{l|r}
Temperatura & Tiempo en estado larval y pupa \\\hline
13 & 0 \\
15-20 & 10 a 17.4 \\
20-25 & 9 a 13 \\
25-36 & 5 a 7 \\
36+ & 0
\end{tabular}
\caption{\label{tab:widgets}Tiempo promedio de duración en días del estado larval y pupa a diferentes temperaturas.}
\end{table}

%* Métodos de muestreo de la abundancia poblacional
%~ * Dispositivos de ovipostura
\section{Introducción}
\label{sec:densidad-vectorial-introduccion}

Tres indicadores entomológicos son recomendados por la OMS (Organización
mundial de la salud) para estimar la densidad del vector del dengue: El
Índice de Casa, Índice de Recipiente e Índice de Breteau. Muchos
programas de control del dengue usan los índices larvarios como indicadores
de las densidades poblaciones de Aedes aegypti, para dirigir y focalizar
espacial y temporalmente las acciones de control del vector. Sin embargo:

Los indicadores tradicionales son poco confiables porque
\begin{itemize}
    \item Se basan en búsqueda de fases inmaduras por lo que representan
        una estimación indirecta de las poblaciones de mosquitos adultos.
    \item No reflejan la asociación que existe entre las densidades de
        mosquitos o cantidad y/o tipo de recipientes presentes, con los
        riesgos de transmisión de dengue.
    \item Proporcionan poca o nula información de aquellas viviendas en
        las que existe un mayor riesgo de presencia de mosquitos.
\end{itemize}

Además estos indicadores no reflejan las poblaciones de adultos ni estiman
riesgo entomológico, lo cual es muy importante en la transmisión del dengue.
Por lo cual, a la fecha sólo son recomendados para detectar la calidad de
las acciones (control de calidad) realizadas por el personal de control
larvario. Existen numerosos métodos e indicadores para determinar las
poblaciones de Aedes aegypti en la etapa de huevo, larva, pupa o adulto;
uno de los métodos más prácticos, eficientes y económicos es el monitoreo
de poblaciones de este vector por medio de ovitrampas. Las ovitrampas han
sido usadas desde 1965 en la vigilancia del Aedes aegypti (L), como un
instrumento para determinar la distribución del mosquito, medir la fluctuación
estacional de las poblaciones y para evaluar la eficacia de la aplicación
de insecticidas; además, como una estrategia de muestreo presencia-ausencia,
lo cual permite una estimación de la densidad mediante la proporción de
muestras positivas y son especialmente útiles para la detección temprana
de reinfestaciones. \cite{cenaprece2013}

Las técnicas tradicionales de vigilancia de A. aegypti usan los índices
aédicos de recipientes, de viviendas y de Breteau para determinar el grado
de infestación, dispersión y densidad del mosquito en una zona y tiempo
determinados. Estos índices se fundamentan en la detección visual de formas
inmaduras del vector dentro de recipientes domésticos, técnica considerada
poco sensible por la habilidad de las larvas para escapar y su capacidad de
permanecer sumergidas por largos períodos de tiempo (3, 5). Asi mismo, la
proporción de viviendas y recipientes infestados con A. aegypti no provee
información fehaciente sobre la densidad poblacional al registrar como
positivo un recipiente o casa sin tener en cuenta la cantidad de formas
inmaduras presentes, lo cual quiere decir que para el índice es igual
si hay una o cientos de ellas.


%~ * Índices de Stegomia
\section{Índices de Stegomia}
\label{sec:densidad-vectorial-indices-stegomia}

La Organización mundial de la salud (OMS) recomienda la utlización de tres
indicadores entomológicos, generalmente conocidos como índices de Stegomia,
para estimar la densidad del vector, Índice de casas (I.C.), Índice de
Recipientes (I.R.) e Índice de Breteau (I.B.).Estos indices son calculados a
partir de muestreos de larvas y recipientes.

\subsubsection{Índice de Casas}
El índice de Casas (I.C.) es el valor numérico que especifíca el porcentaje
de viviendas infestadas con larvas, pupas o ambos estados de desarrollo
del mosquito transmisor del dengue. Para este indice, se analizan los
contenedores de las viviendas y alrededores.

\begin{equation}
I.C. = \frac{\text{número de casas infestadas} * 100}{\text{número de casas inspecionadas}}
\end{equation}

Donde :
\begin{itemize}
\item El número de casas infestadas : cantidad de casas que
cuentan con al menos un contenedor que alberga a larvas o pupas de Aedes
aegypti.
\item número de casas inspeccionadas : El total de casas analizadas para
el estudio.
\end{itemize}

Es el de uso más generalizado para la distribución de medición de
la población larvaria. Es el índice más rápido y simple para examinar la
población larval. Puede ser utilizado para proporcionar una indicación
rápida de la distribución del mosquito en una área determinada. Sus
defectos son que no tiene en cuenta el número de envases positivos por
yarda ni la productividad de esos envases.

\subsubsection{Índice de Recipientes}
El índice de Recipientes es un valor numérico que consiste en el
porcentaje de recipientes que contienen agua y están infestados con las
larvas y/ó crisálidas del mosquito transmisor del virus del dengue.
El índice del envase proporciona una indicación más detallada de la
abundancia de la población larvaria.

\begin{equation}
I.R. = \frac{\text{número de contenedores positivos} * 100}{\text{número de contenedores inspecionadas}}
\end{equation}

Donde :
\begin{itemize}
\item número de contenedores positivos : La cantidad de contenedores en los
cuales se observan larvas o pupas de Aedes aegypti.
\item número de contenedores  inspeccionadas : El total de contenedores
analizadas para el estudio.
\end{itemize}

Las encuestas larvales que utilizan el índice del envase son mucho más
lentas a realizar que las encuestas sobre el índice de la casa, pues
requieren generalmente que todos los envases en una premisa puedan ser
examinadas para las etapas no maduras y los detalles guardados de envases
positivos y negativos. El índice del envase no proporciona ninguna información
en la productividad de diversos envases.

\subsubsection{Índice de Breteau}
El índice de Breteau (I.B.) es un valor numérico que define el número de
insectos en desarrollo que se encuentran en las viviendas humanas por
la cantidad del total inspeccionado.

\begin{equation}
I.B. = \frac{\text{número de contenedores positivos} * 100}{\text{número de casas inspecionadas}}
\end{equation}
Donde :
\begin{itemize}
\item número de contenedores positivos : La cantidad de contenedores en los
cuales se observan larvas o pupas de Aedes aegypti.
\item número de casas inspecionadas : El total de casas analizadas para
el estudio.
\end{itemize}

La determinación correcta requiere de una encuesta completa de todos los
envases en una premisa que pueda hacer este tipo de ennumeración. Los
datos se utilizan para determinar el índice de la casa. Usando la
combinación del índice de Breteau y el índice de la casa, es fácil
determinar si el problema es extenso dentro de un área ó se enfoca a
unas viviendas.

\subsubsection{Probelmatica}
Las técnicas tradicionales de vigilancia de A. aegypti usan los índices
stegomia para determinar el grado de infestación, dispersión y densidad
del mosquito en una zona y tiempo determinados.

Estos índices se fundamentan en la detección visual de formas
inmaduras del vector dentro de recipientes domésticos, técnica considerada
poco sensible por la habilidad de las larvas para escapar y su capacidad de
permanecer sumergidas por largos períodos de tiempo. Asi mismo, la
proporción de viviendas y recipientes infestados con A. aegypti no provee
información fehaciente sobre la densidad poblacional al registrar como
positivo un recipiente o casa sin tener en cuenta la cantidad de formas
inmaduras presentes, lo cual quiere decir que para el índice es igual
si hay una o cientos de ellas.

\begin{itemize}
    \item Se basan en búsqueda de fases inmaduras por lo que representan
        una estimación indirecta de las poblaciones de mosquitos adultos.
    \item No reflejan la asociación que existe entre las densidades de
        mosquitos o cantidad y/o tipo de recipientes presentes, con los
        riesgos de transmisión de dengue.
    \item Proporcionan poca o nula información de aquellas viviendas en
        las que existe un mayor riesgo de presencia de mosquitos.
\end{itemize}

Además estos indicadores no reflejan las poblaciones de adultos ni estiman
riesgo entomológico, lo cual es muy importante en la transmisión del dengue.
Por lo cual, a la fecha sólo son recomendados para detectar la calidad de
las acciones (control de calidad) realizadas por el personal de control
larvario. Existen numerosos métodos e indicadores para determinar las
poblaciones de Aedes aegypti en la etapa de huevo, larva, pupa o adulto;
uno de los métodos más prácticos, eficientes y económicos es el monitoreo
de poblaciones de este vector por medio de ovitrampas. Las ovitrampas han
sido usadas desde 1965 en la vigilancia del Aedes aegypti (L), como un
instrumento para determinar la distribución del mosquito, medir la fluctuación
estacional de las poblaciones y para evaluar la eficacia de la aplicación
de insecticidas; además, como una estrategia de muestreo presencia-ausencia,
lo cual permite una estimación de la densidad mediante la proporción de
muestras positivas y son especialmente útiles para la detección temprana
de reinfestaciones. \cite{cenaprece2013}




%~ * Distribución de dispositivos de ovipostura
\section{Larvitrampas}
\label{sec:densidad-vectorial-larvitrampas}
Antes de la utilización de la larvitrampa, ésta debe cepillarse y flamearse,
luego mantenerla sumergida en agua durante no menos de tres días, para
asegurarse que el agua no contenga residuos de sustancias que puedan actuar
como larvicida. De esta manera, además, se garantiza la destrucción de
algún huevo del mosquito que estuviese previamente en el neumático o en
larvitrampas ya utilizadas.

\subsection{Especificaciones para la colocación e inspección}
Instalarla a una altura de 50 cm (del suelo a la base de la larvitrampa).
Protegerla de la luz directa del sol, el aire, la lluvia, en lugares a
media luz o completamente a la sombra. No deben ubicarse cercanas a depósitos
de agua. Debe evitarse su colocación en lugares completamente pavimentados,
u otros que tengan mucha refracción de la luz. Debe estar visible para la
hembra del mosquito. Protegerla de niños y animales domésticos (perros,
gatos, roedores, etc.)

\subsection{Forma de revisión}
Se establece una rutina semanal para revisar las larvitrampas, para lo
cual, una vez por semana debe vaciarse todo su contenido cuidadosamente
(para que no quede ninguna larva en sus paredes) en un recipiente adecuado
para realizar la inspección. En caso de ser positivas, se registra como
tal y las larvas serán colectadas en tubos para ser enviadas al laboratorio
para su determinación taxonómica. Luego, el dispositivo se lava y se acondiciona
para ser colocadas nuevamente siguiendo las especificaciones ya descritas.

\subsection{Consideración final}
Tener en cuenta que en verano, con condiciones más favorables para el
desarrollo de esta especie, las larvas pueden alcanzar el estadio de
adulto entre 6 y 7 días desde la ovipostura, por lo que es necesaria la
inspección de todas las larvitrampas en los tiempos indicados a fin de
evitar que alguna de ellas se transformen en criaderos de adultos.


%~ * Seguimiento y control de dispositivos de ovipostura
\section{Ovitrampa}
\label{sec:densidad-vectorial-ovitrampa}
Son recipientes que ofrecen a las hembras de Aedes aegypti un lugar colocar
los huevos. Detecta la presencia de huevos y por lo tanto actividad de
ovipostura. Las ovitrampas consisten en frascos de plástico o pequeñas
macetas plásticas de unos 500 ml de color oscuro preferentemente, en cuyo
interior, se coloca una pieza plana de madera (baja-lengua o similar).
Asimismo, también pueden construirse con un pote de vidrio de boca ancha,
de aprox. medio litro, pintado de negro por fuera y equipado con una paleta
de cartón o madera (baja-lengua) sujeta verticalmente al interior, con su
lado áspero mirando hacia adentro. Las dimensiones del recipiente no son
críticas pero todos los frascos a usar en un estudio particular deben ser
idénticos. Al frasco se le deberá agregar 250 ml de agua limpia.


\subsection{Especificaciones para la colocación e inspección}
La colocación debe realizarse en lugares representativos del municipio,
especialmente en las zonas donde se produjeron casos de dengue autóctonos
o importados. Respecto al número de ovitrampas a colocar, se sugiere no
menor a 10 por localidad. La idea es mantener el mismo circuito (mismos
lugares de colocación), un modelo a “escala ciudad”, para tener la idea de
la "presencia" relacionada con la distribución geográfica del vector, se
basa en el criterio que la información sea independiente. O sea que sea
improbable (más bien imposible) que una hembra pueda poner huevos en dos
ovitrampas contiguas. Además la instalación debería basarse en la capacidad
operativa de trabajo, y para ello se pueden colocar las ovitrampas en una
grilla con puntos más o menos equidistantes de aproximadamente 400 metros
de lado.

Cada ovitrampa se coloca en un lugar accesible, protegido donde predomine
la sombra y haya cierto grado de humedad (ambiente sombreado). Debe asegurarse
la presencia de moradores al retirarla.Sobre un plano de la localidad o
sector a muestrear se seleccionarán los puntos donde se colocarán las
ovitrampas. Una variante sería colocar una por Unidad Sanitaria que el
municipio posea, asumiendo que la ubicación de las mismas brindará una
visión representativa del conjunto. Conviene tener presente que en este
caso, el muestreo puede no ser representativo de viviendas regulares.

Las ovitrampas deben ser inspeccionadas semanalmente y en el caso de detectar
paletas con huevos, cuando no puedan ser leídos en el nivel local, se deberán
remitir para su lectura a los laboratorios de entomología más cercanos,
(Divisiones de Zoonosis Urbanas, División de Zoonosis Rurales, CEPAVE, etc.).
La remisión será en un sobre o bolsita plástica, con los datos para georreferenciar.
La vigilancia entomológica se debe realizar en forma continua anual. Es
importante destacar que una vez detectada la presencia de Aedes Aegypti por
cualquiera de los sistemas de monitoreo (larvitrampas u ovitrampas) se deben
realizar las acciones inmediatas de control focal en la comunidad.

\subsection{Consideración final}
Es importante añadir un identificador a cada ovitrampa que permita la
identificarlas fácilmente. El rótulo se debe colocar sobre la baja-lengua
o paleta de la ovitrampa, debe estar debidamente escrito (con lápiz) el
número y/o código de la ovitrampa. También se rotulará el frasco sobre
su pared con tinta indeleble. Se recomienda numerar cada una de las paletas
o baja-lenguas y agregarle iniciales para identificar el municipio y
detallar en el protocolo común los datos de cada una (lugar físico por.
ej. calle, barrio y zona del municipio como también la fecha del retiro
de las mismas de su lugar para su posterior envío).


%~ * Recolección de resultados
\section{Mosquitérica genérica}
Es un dispositivos de oviposturas más recientes, cuenta con un sencillo
diseño y de fácil fabricación debido a que los materiales necesarios para
su construcción son fácilmente accesibles.



\chapter{Sistemas de Información Geográfica}
%##Capítulo 2. Sistemas de Información Geográfica
%* Introducción
%A lo largo del tiempo, el hombre ha tratado de representar la superficie terrestre y los objetos que esta alberga.
Los primeros mapas tenían como principal objetivo, ser una herramienta de apoyo para la navegación, indicando así
los rumbos a seguir permitiendo al hombre desplazarse de un puerto a otro. Se caracterizaban por brindar exactitud
en los rumbos y las distancias existente entre los puertos, no así por su exactitud en la representación de las
tierras. Con el trascurrir del tiempo ya no era suficiente con llegar de un puerto a otro, surgiendo así la
necesidad de mejorar precisión en la medición de las distancias y superficies sobre los nuevos territorios para
conseguir un mejor dominio sobre estos. De forma adicional se fueron incluyendo diversos elementos como recursos y
factores ambientales de la superficie terrestre para mejorar la visión de la distribución de los fenómenos
naturales y asentamientos humanos sobre la superficie terrestre.
APARICION DE LOS GIS.

%* Estructuras de datos de los Sistemas de Información Geográfica
\section{Definición de Sistema de información geográfica}
\label{sec:cap2-definicion-sig}

Un Sistema de Información geográfica (SIG) es la integración organizada de hardware, software y datos geográficos
diseñada para almacenar, manejar, capturar, analizar y desplegar la información geográficamente de múltiples
formas, con el fin de resolver problemas de planificación y gestión geográfica. También puede definirse como un
modelo de una parte de la realidad referido a un sistema de coordenadas terrestre y construido para satisfacer 
unas necesidades concretas de información \cite{lopezMarcos2007}. Su fundamentación se basa en principios formales
de matemáticas discretas, modelos de datos y geometría computacional; su desarrollo,en nuevas tecnologías de la
información: estándares e ingeniería de software, almacenes de datos, Web-SIG, metadatos, ambientes y lenguajes
visuales, graficación entre muchas otras \cite{lunaPaulina2010}.

La característica principal de los SIG es el manejo de datos complejos basados en datos geométricos (coordenadas e
información topológica) y datos de atributos (información nominal) la cual describe las propiedades de los objetos
geométricos tales como punto, lineas y polígonos.

%* Sistemas de proyección
\section{Sistemas de proyección}
\label{sec:cap2-sistemas-de-proyeccion}
Durante el siglo XVII, cartógrafos especializados, como Mercator, demostraron que no sólo el uso de
un sistema de proyección matemático y un ajustado sistema de coordenadas mejoraba la fiabilidad de
las medidas y la localización de las áreas de tierra, sino que el registro de fenómenos espaciales
a través de un modelo convenido de distribución de fenómenos naturales y asentamientos humanos era
de un valor incalculable para la navegación, para la búsqueda de rutas y en la estrategia militar
\citep{llopis2006sistemas}.

\subsection{Coordenadas geográficas}
Las coordenadas geográficas proveen un sistema de referencia, que se basan en la utilización de
coordenadas angulares como latitud y longitud (\figref{fig:sig-plano}). Su objetivo es el de
determinar los ángulos laterales de la superficie terrestre. Se denomina latitud al ángulo que
existe entre un punto cualquiera y el Ecuador, medida sobre el meridiano que pasa por dicho punto
\citep{fAlonsoSig2006}. La longitud mide el ángulo a lo largo de la línea del Ecuador desde
cualquier punto de la Tierra\citep{fAlonsoSig2006}.

\begin{figure}[!htbp]
\centering
\includegraphics[width=0.7\textwidth]{capitulo-2/graphics/ejes-tierra.png}
\caption{\label{fig:sig-plano} Representación de los meridianos, paralelos, longitud y latitud en la superfice (Tomado de \cite{website:emcLonLat2014}).}
\end{figure}


Para la representación de objetos puntuales en una superficie, se utilizan las coordenadas $X$ e
$Y$ que caracterizan la planimetría y una coordenada $Z$ representa la altimertría del objeto
puntual en cuestión. En la \figref{fig:sig-xyz} se puede apreciar la representación de un objeto
puntual en una sistema tridimensional.

\begin{figure}[!htbp]
\centering
\includegraphics[width=0.5\textwidth]{capitulo-2/graphics/coordenadas-xyz.jpg}
\caption{\label{fig:sig-xyz} Representación de un objeto puntual en un sistema tridimensional
 $(x,y,z)$.}
\end{figure}

\subsection{Proyecciones}
Las proyecciones cartográficas o proyecciones geográficas se encuentran descritas por el conjunto
de métodos utilizados para establecer una correspondencia matemática entre los puntos pertenecientes a la superficie curva de la tierra y sus transformaciones en una superficie plana.

El problema principal a la hora de realizar una proyección es que no existe forma de representar
una superficie plana toda la superficie curva, de la tierra, sin deformarla siendo el objetivo
principal minimizar las deformaciones en la medida que sea posible \citep{fAlonsoSig2006}.
Teniendo en cuenta que la curvatura de la superficie terrestre es proporcional al tamaño del área
representada \citep{llopis2006sistemas}, si el área o región que se desea representar es pequeña,
entonces la deformación o distorsión resultante es despreciable, por lo que puede ser modelada con
coordenadas planas.

Con la aparición y difusión de los SIG, el conjunto de herramientas que ofrece se dio paso a
posibilidad de combinar información de diferentes mapas con diferentes proyecciones, esto ha
incrementado la relevancia de la cartografía más allá de la simple confección de mapas
\citep{llopis2006sistemas}.

\subsection{Elementos de representación cartográfica}
La representación de los objetos, recursos o fenómenos geográficos en una superficie o mapa, se
debe identificar los rasgos característicos teniendo en cuenta, principalmente, tres aspectos:
sus dimensiones, el nivel de medida y la distribución \citep{fomentoConceptos2010}. El análisis de
las características de estos elementos permiten seleccionar, adecuadamente, los símbolos a
utilizar para representar los fenómenos geográficos correspondientes \citep{fomentoConceptos2010}.
A cada entidad espacial puede ser asociada a diversas variables, para representar estas entidades
como hechos de la superficie terrestre, se han desarrollado un amplio conjunto de técnicas para
cartografiarlos.

\subsubsection{Dimensiones}
Teniendo en cuenta sus dimensiones y extensión, los fenómenos geográficos a ser representados en el
mapa pueden clasificarse en : puntuales, lineales, poligonales y espacio temporales
\citep{fAlonsoSig2006}.

\begin{itemize}
    \item \textit{Fenómenos puntuales} : indican la presencia de entidades de un modo puntual. Normalmente se utilizan se utilizan símbolos o colores para una variable cualitativa y distintos tamaños para las variables cuantitativas.

    \item \textit{Fenómenos lineales} : representan entidades naturales o artificiales de forma lineal, se encuentran conformadas a partir de dos o más fenómenos puntuales. Se pueden variar el ancho de de las líneas para describir la anchura de los elementos, también se utilizan distintos tipos de líneas (continuas y discontinuas) para caracterizar los fenómenos lineales.

    \item \textit{Fenómenos Poligonales} : Este fenómeno puede ser descrito por dos tipos de información bidimensional o tridimensional. Teniendo en cuenta el tamaño de los mismos pueden ser representados como polígonos o porciones homogéneas del terreno relacionadas a una variable cualitativa.

    \item \textit{Fenómenos espacio-temporales}: Existe una dependencia del fenómeno con respecto al paso del tiempo.
\end{itemize}

\subsubsection{Nivel de medida}

Los elementos de la naturaleza se miden con el fin de clarificarlos y, posteriormente, compararlos.
Esto no necesariamente implica a una magnitud cuantitativa, ya que pueden ser de utilizadas las
cualitativas u jerárquicas\citep{fomentoConceptos2010}. Teniendo en cuenta su orden de precisión
tenemos las siguientes escalas:

\begin{itemize}
    \item \textit{Nominal} : Se encarga de asignar una característica, no numérica, al fenómeno de forma que solo se pueden realizar comparaciones cualitativas. Este es el nivel más elemental de medida, pues no informa acerca de la cantidad o el orden.

    \item \textit{Ordinal} : Se establece una jerarquía no cuantificable entre los diferentes elementos. Por ejemplo, un mapa en el que aparecen núcleos de población, cuyos símbolos están jerarquizados según el número de habitantes sin especificar cantidad.

    \item \textit{Cuantitativa} :Asigna una característica numérica a un fenómeno geográfico, normalmente es necesario emplear algún tipo de unidad convencional.
\end{itemize}

\subsubsection{Distribución}
La distribución de los fenómenos, o su ocurrencia, puede darse a lo largo y ancho de la superficie
terrestre que los alberga ya sea de forma continua en una ubicación de la superficie o de forma
discontinua \citep{fomentoConceptos2010}. Estos fenómenos pueden dividirse, de acuerdo a su distribución, en :

\begin{itemize}
    \item \textit{Continuos} : Son aquellos que tienen presencia en todos los puntos de la superficie, aunque sólo se cuenten con las medidas de algunos puntos significativos de la superficie. En esta categoría tenemos a la temperatura y la densidad poblacional.

    \item \textit{Discretos} : son aquellos fenómenos que tiene presencia sólo en algunos puntos de la superficie. Algunos de estos fenómenos discretos pueden transformarse en continuos mediante la aplicación de una relación. Por ejemplo tenemos el fenómeno discreto, número de habitantes de una provincia, para que se transforme en un fenómeno continuo se debe dividir el número de habitantes con el tamaño de la superficie en $km^2$, de esta forma podemos obtener la densidad poblacional que es un fenómeno continuo.
\end{itemize}

%* Técnicas gráficas de representación
\section{Representación de los datos }
\label{sec:cap2-tecnicas-graficas-representacion}

Los objetos del mundo real se pueden describir mediante los fenómenos discretos y continuos. Las variables y
objetos se muestrean y organizan para lograr una representación adecuada. En un SIG existen básicamente dos
modelos lógicos que se conocen como formato raster y formato vectorial y que dan lugar a los dos grandes tipos
de capas de información espacial.

\subsection{Formato raster}
El formato raster o de retícula se centra en las propiedades del espacio más que en la precisión de la localización. Divide
el espacio en un conjunto regular de celdillas, cada una de estas celdillas contiene un número que puede ser el identificador
de un objeto o del valor de una variable.Se trata de un modelo de datos muy adecuado para la representación de variables
continuas en el espacio.

Los datos raster se compone de filas y columnas de celdas, cada celda almacena un valor único. Los datos raster pueden ser imágenes
con un valor de color en cada celda (o píxel de la imagen). Otros valores registrados para cada celda puede ser un valor discreto o
un valor nulo si no se dispone de datos. Si bien una trama de celdas almacena un valor único, estas pueden ampliarse mediante el
uso de las bandas del raster para representar los colores RGB (rojo, verde, azul), o una tabla extendida de atributos con una
fila para cada valor único de celdas. La resolución del conjunto de datos raster es el ancho de la celda en unidades sobre el
terreno.

\subsection{Formato vectorial}
Los datos vectoriales, se caracterizan por la precisión de localización de los elementos geográficos en el espacio, donde
los fenómenos a representar son discretos, con límites bien definidos. Generalmente se considera que el formato vectorial
es más adecuado para la representación de entidades o variables cualitativas y el formato raster para representar superficies.



\begin{figure}
\centering
\includegraphics[width=0.6\textwidth]{capitulo-2/graphics/dimensiones-datos.jpg}
\caption{\label{fig:sig-xyz} Elementos geométricos utilizados para modelar digitalmente las entidades en un SIG.}
\end{figure}

Los diferentes objetos se encuentran representados como puntos, lineas o polígonos, donde cada una de estos elementos
geométricos se encuentra vinculado a una fila en una base de datos que describe sus atributos. De tal forma que para modelar
digitalmente las entidades del mundo real se utilizan estos tres elementos geométricos.
\begin{itemize}
    \item Puntos : se utilizan para las entidades geográficas que pueden ser descriptas por un único punto de referencia. Los
    puntos transmiten la menor cantidad de información, en los elementos puntuales no puede medirse la distancia. También
    se pueden utilizar para representar zonas a una escala pequeña.

    \item Líneas o polilíneas : las líneas unidimensionales o polilíneas son usadas para rasgos lineales como ríos,
    caminos, ferrocarriles, rastros, líneas topográficas o curvas de nivel. De igual forma que en las entidades
    puntuales, en pequeñas escalas pueden ser utilizados para representar polígonos. En los elementos lineales puede
    medirse la distancia.

    \item Polígonos : se utilizan para representar elementos geográficos que cubren un área particular de la superficie de la tierra.
    Los polígonos transmiten la mayor cantidad de información en archivos con datos vectoriales y en ellos se pueden medir el
    perímetro y el área.

\end{itemize}

\subsection{Ventajas y desventajas de los formatos raster y vectorial}
El debate acerca de la conveniencia de uno u otro modelo debe basarse en el tipo de estudio o enfoque que se quiera
hacer, pero también del software y fuentes de datos disponibles.

Está claro que las superficies se representan más eficientemente en formato raster y sólo pueden representarse
en formato vectorial mediante los modelos híbridos que no resultan adecuados para la realización de posteriores
análisis ya que todas las operaciones que permite el modelo ráster resultaran mucho más lentas con el modelo
vectorial.

Tradicionalmente se ha considerado que para la representación de los objetos resulta más eficiente la utilización
de un formato vectorial ya que La estructura de los datos es compacta y almacena los datos sólo de los elementos
digitalizados por lo que requiere menos memoria para su almacenamiento y tratamiento. Los elementos son representados
como gráficos vectoriales que no pierden definición si se amplía la escala de visualización. Sin emabargo el formato
vectorial es más lento, debido a su compleja estructura, que el raster para la utilización de herramientas de análisis
espacial y consultas acerca de posiciones geográficas concretas.

En el caso de las variables cualitativas estaríamos en un caso intermedio entre los dos anteriores.
Las ventajas del modelo ráster incluyen la simplicidad, la velocidad en la ejecución de los operadores y que es
el modelo de datos que utilizan las imágenes de satélite o los modelos digitales de terreno. Entre las desventajas
del modelo ráster destaca su inexactitud que depende de la resolución de los datos y la gran cantidad de espacio
que requiere para el almacenamiento de los datos. Este último problema puede compensarse mediante diversos
sistemas de compresión. Además en muchos casos se confunde precisión y exactitud cuando se trabaja con
datos vectoriales de modo que la exactitud en las coordenadas del modelo vectorial es más teórica que real.

Hoy en día se pueden codificar las formas en un modelo vectorial y los procesos con un modelo ráster,
para ello se requieren herramientas eficaces de paso de un formato al otro. Resulta sencillo, finalmente, la
visualización simultánea de datos en los dos formatos gracias a la capacidad gráfica actual.

%* Análisis espacial con SIG
\section{Análisis espacial en un SIG}
\label{sec:cap2-analisis-espacial-sig}
Dada la amplia gama de técnicas de análisis espacial que se han desarrollado durante
el último medio siglo, cualquier resumen o revisión sólo puede cubrir el tema a una
profundidad limitada \citep{rojas2012ejecucion}.

El análisis en un SIG puede definirse como el proceso de transformación de los datos geográficos
en información útil para un problema determinado, teniendo como finalidad modelar fenómenos
geográficos, las asociaciones y relaciones existentes en la información geográfica almacenada. Se
encuentra determinada por la existencia de relaciones topológicas entre los elementos y permite
realizar cálculos entre variables y obtener así nuevos datos \citep{bravo2000breve}.


%* Métodos de interpolación
\section{Métodos de interpolación}
\label{sec:cap2-metodos-interpolacion}
Todos los métodos de interpolación se basan en la presunción lógica de que
cuanto más cercanos están dos puntos sobre la superficie terrestre, los
valores de cualquier variable cuantitativa que midamos en ellos serán más
parecidos, para expresarlo más técnicamente, las variables espaciales
muestran autocorrelación espacial[2].

La interpolación espacial, consiste en usar puntos con valores conocidos,
también llamados puntos de control, para estimar una variable en lugares
donde se desconoce; también se considera una forma de transformar información
puntual en información de superficie, con el objetivo de combinarla con
otros datos para facilitar el análisis y la modelación espacial.
El resultado de la interpolación espacial depende de un algoritmo
computacional o una ecuación matemática en la cual se emplean los datos
de los puntos de control[1].

\section{Métodos de interpolación locales}
Los método locales, utilizan la interpolación utilizando la información
de los puntos más cercanos. Asumen autocorrelación espacial y estiman los
valores de Z como una media ponderada de los valores de un conjunto de
puntos de muestreo cercanos[2].

\subsection{Polígonos de Voronoi}
Es uno de los métodos de interpolación más simples, simples basado en la
distancia euclidiana. Se crean al unir los puntos entre sí, trazando las
mediatrices de los segmento de unión. Las intersecciones de estas mediatrices
determinan una serie de polígonos en un espacio bidimensional alrededor de
los puntos de control, de manera que el perímetro de los polígonos generados
sea equidistante a los puntos vecinos y designando su área de influencia, como :
\begin{itemize}
    \item Centros hospitalarios.
    \item Estaciones de bomberos.
    \item Bocas de metro.
    \item Centros comerciales.
    \item Control del tráfico aéreo.
    \item Telefonía móvil.
\end{itemize}

\subsection{Ponderación de la inversa de la distancia (IDW)}
Estima los puntos del modelo realizando una asignación de pesos a los datos
del entorno en función inversa a la distancia que los separa del punto en
cuestión. De esta forma, se acepta  que  los puntos más próximos al centroide
intervienen de manera más relevante en la obtención del valor definitivo
de Z para ese punto.

La elección del exponente de ponderación(p) determina la contribución de
los puntos circundantes al punto problema, cuanto mayor es p, más contribuyen
los puntos próximos. Es necesario contar con muchos puntos para la interpolación.

Uno de los problemas más importantes de los métodos basados en medias
ponderadas es que, interpolan basándose en el valor medio de un conjunto
de puntos situados en las proximidades, por tanto nunca se van a obtener
valores mayores o menores que los de los puntos utilizados para hacer la
interpolación[2]. En consecuencia no se van a interpolar correctamente
máximos o mínimos locales y además los puntos de muestreo aparecen en el
mapa final como máximos y mínimos locales erróneos.

\subsection{Kriging}
El Kriging es un método geoestadístico de interpolación espacial de carácter
global, exacto y estocástico. La idea básica de este método corresponde a
la noción de dependencia espacial, según la cual las muestras cercanas
tienen mayor similitud entre sí que las más apartadas[1].

Se presenta con un método de interpolación con una expresión general
similar a la anterior (IDW). La diferencia básica es que asume que la
altitud puede definirse como una variable regionalizada. Supone que la
variación espacial de la variable a representar puede ser explicada al
menos parcialmente mediante funciones de correlación espacial(la variación
espacial de los valores de z puede deducirse de los valores circundantes
de acuerdo con unas funciones homogéneas en toda el área) [4].

\subsection{Tipos de Kriging}
Kriging simples
Asume que las medias locales son relativamente constantes y de valor muy
semejante a la media de la población que es conocida. La media de la
población es utilizada para cada estimación local, en conjunto con los
puntos vecinos establecidos como necesarios para la estimación.
Kriging ordinario
Las medias locales no son necesariamente próximas de la media de la población,
usándose apenas los puntos vecinos para la estimación. Es el método más
ampliamente utilizado en los problemas ambientales.

\subsection{Semivarianza y semivariograma}
El método de Kriging utiliza diversas teorías explayadas en la estadística.
Una semivarianza es la medida del grado de dependencia espacial entre dos
muestras. La magnitud de la semivarianza entre dos puntos depende de la
distancia entre ellos. Efecto pepita,  es el valor del semivariograma en
el origen. Resulta del componente aleatorio, no correlacionado espacialmente,
que experimenta cualquier variable espacial. Se denomina así por las pepitas
de oro que representan un brusco incremento en la variable concentración de
oro para distancias muy cortas.
Meseta, es el valor máximo que adopta el semivariograma para distancias
elevadas más allá de las cuales no hay autocorrelación espacial.
Rango, es la distancia a la que se alcanza la meseta. Puede asimilarse a
la distancia más allá de la cual dos medidas pueden considerarse independientes.

%* Aplicaciones de SIG y análisis epidemiológicos
\section{Aplicaciones y análisis epidemiologico}
\label{sec:cap2-aplicaciones-analisis-epidemiologico}

Con el correr del tiempo su el potencial ha incrementado rápidamente, desde su concepción en los años setenta,
actualmente las áreas en las que se aplica se ha diversificado, entre las más resaltantes podemos nombrar:
biología, energía e infraestructura, planificación urbana y regional, monitoreo ambiental y geografía física,
transportación y logística.


\chapter{Solución propuesta}
%~ * Modelo Propuesto
%---------------------------0-----------------------------------

\begin{frame}[c]{Geodengue}
\begin{center}
    \includegraphics[width=\textwidth]{./graphics/propuesta.png}
\end{center}
\end{frame}


%----------------------------2----------------------------------
\begin{frame}[c]{Preparación de datos de entrada}
  \begin{center}
    \begin{itemize}
      \item Selección del área de estudio.
      \item Distribución geográfica de los puntos de control.
      \item Revisión de los puntos de control.
      \item Generación de la población inicial a partir de los puntos de control.
      \item Obtener datos climáticos para el periodo de simulación.
    \end{itemize}
  \end{center}
\end{frame}
%-----------------------------3---------------------------------

\begin{frame}[c]{Preparación de datos de entrada. Conteo de larvas mediante PDI.}
Utilizar procesamiento digital de imágenes para agilizar el conteo de larvas obtenidas de las larvitrampas.
    \begin{figure}
      \begin{subfigure}[b]{0.4\textwidth}
          \includegraphics[width=\textwidth]{../book/capitulo-5/graphics/larvas-original.png}
          \caption{Imagen original antes de la transformación.}
      \end{subfigure}
      ~~~~
      \begin{subfigure}[b]{0.4\textwidth}
          \includegraphics[width=\textwidth]{../book/capitulo-5/graphics/larvas-otsu.png}
          \caption{Imagen luego de la umbralización de Otsu.}
      \end{subfigure}
    \end{figure}
\end{frame}

%---------------------------4-----------------------------------

\begin{frame}[c]{Simulación del proceso evolutivo. Ciclo de vida del vector.}
\begin{center}
    \includegraphics[width=7.8cm]{./graphics/ciclo-de-vida.jpg}
\end{center}
\end{frame}


\begin{frame}[c]{Eventos que influyen en el ciclo de vida del vector.}
  \begin{itemize}
    \item Tasas de desarrollo para la eclosión de huevos, larvas, pupas y el ciclo gonotrófico de las hembras adultas.
    \item Mortalidad de huevos, larvas, pupas y mosquitos adultos.
    \item Dispersión de adultos.
    \item Ovipostura de hembras adultas nulíperas y paridas.
  \end{itemize}
\end{frame}

\begin{frame}[c]{Simulación del proceso evolutivo. Tasas de desarrollo}
  \begin{itemize}
    \item Las tasas dependen no sólo de valores de la población, sino también de la temperatura lo que introduce una dependencia del tiempo.
    \item Se consideran 4 tasas de desarrollos : la eclosión de huevos, emergencia a pupas, emergencia a adultos y el ciclo gonotrófico.
    \item El cálculo de las tasas de desarrollo se realiza mediante el modelo no lineal de Sharpe y DeMichele.
    \item El modelo de Sharpe y DeMichele debe ajustarse con los datos biológicos disponibles.
    \item El modelo de Sharpe y DeMichele puede utilizarse para calcular tasas de desarrollo a cualquier temperatura.
  \end{itemize}
\end{frame}


\begin{frame}[c]{Simulación del proceso evolutivo. Mortalidad}

\end{frame}


\begin{frame}[c]{Simulación del proceso evolutivo. Ciclo gonotrófico y Ovipostura.}
\begin{center}
    \includegraphics[width=\textwidth]{./graphics/cliclo-gonotrofico-tiempo-2.jpg}
\end{center}
\end{frame}

\begin{frame}[c]{Simulación del proceso evolutivo. Dispersión.}
\begin{center}
  \includegraphics[width=8.5cm]{./graphics/dispersion.jpg}
\end{center}
\end{frame}


\begin{frame}[c]{Simulación del proceso evolutivo}
\begin{center}
  \includegraphics[height=7.5cm]{./graphics/algoritmo-propuesto.png}
\end{center}
\end{frame}

%--------------------------5------------------------------------


\begin{frame}[c]{Generación y publicación de Mapas de interpolación}
\begin{center}
    \includegraphics[width=\textwidth]{./graphics/identificacion-focos.png}
\end{center}
\end{frame}

%--------------------------6------------------------------------

\begin{frame}[t]{Análisis de los resultados}
\begin{center}
    \begin{itemize}
      \item Tasa promedio de desarrollo.
      \item Tasa promedio de mortalidad diaria.
      \item Cantidad promedio de huevos.
      \item Dispersión media.
      \item Cartografía del vector.
      \item Mapas de interpolación.
    \end{itemize}
\end{center}
\end{frame}

%--------------------------7------------------------------------

\begin{frame}[t]{Presentación de los resultados}
  \begin{itemize}
    \item Gráficos explicativos.
    \item Tablas detalladas.
    \item Mapas de interpolación como capas raster.
    \item Objetos puntuales como capas vectoriales.
  \end{itemize}
\end{frame}

  %~ * Introducción
  %~ * Definición formal
%~ * Conteo de larvas
%~ * Dispositivos de ovipostura
\section{Introducción}
\label{sec:densidad-vectorial-introduccion}

Tres indicadores entomológicos son recomendados por la OMS (Organización
mundial de la salud) para estimar la densidad del vector del dengue: El
Índice de Casa, Índice de Recipiente e Índice de Breteau. Muchos
programas de control del dengue usan los índices larvarios como indicadores
de las densidades poblaciones de Aedes aegypti, para dirigir y focalizar
espacial y temporalmente las acciones de control del vector. Sin embargo:

Los indicadores tradicionales son poco confiables porque
\begin{itemize}
    \item Se basan en búsqueda de fases inmaduras por lo que representan
        una estimación indirecta de las poblaciones de mosquitos adultos.
    \item No reflejan la asociación que existe entre las densidades de
        mosquitos o cantidad y/o tipo de recipientes presentes, con los
        riesgos de transmisión de dengue.
    \item Proporcionan poca o nula información de aquellas viviendas en
        las que existe un mayor riesgo de presencia de mosquitos.
\end{itemize}

Además estos indicadores no reflejan las poblaciones de adultos ni estiman
riesgo entomológico, lo cual es muy importante en la transmisión del dengue.
Por lo cual, a la fecha sólo son recomendados para detectar la calidad de
las acciones (control de calidad) realizadas por el personal de control
larvario. Existen numerosos métodos e indicadores para determinar las
poblaciones de Aedes aegypti en la etapa de huevo, larva, pupa o adulto;
uno de los métodos más prácticos, eficientes y económicos es el monitoreo
de poblaciones de este vector por medio de ovitrampas. Las ovitrampas han
sido usadas desde 1965 en la vigilancia del Aedes aegypti (L), como un
instrumento para determinar la distribución del mosquito, medir la fluctuación
estacional de las poblaciones y para evaluar la eficacia de la aplicación
de insecticidas; además, como una estrategia de muestreo presencia-ausencia,
lo cual permite una estimación de la densidad mediante la proporción de
muestras positivas y son especialmente útiles para la detección temprana
de reinfestaciones. \cite{cenaprece2013}

Las técnicas tradicionales de vigilancia de A. aegypti usan los índices
aédicos de recipientes, de viviendas y de Breteau para determinar el grado
de infestación, dispersión y densidad del mosquito en una zona y tiempo
determinados. Estos índices se fundamentan en la detección visual de formas
inmaduras del vector dentro de recipientes domésticos, técnica considerada
poco sensible por la habilidad de las larvas para escapar y su capacidad de
permanecer sumergidas por largos períodos de tiempo (3, 5). Asi mismo, la
proporción de viviendas y recipientes infestados con A. aegypti no provee
información fehaciente sobre la densidad poblacional al registrar como
positivo un recipiente o casa sin tener en cuenta la cantidad de formas
inmaduras presentes, lo cual quiere decir que para el índice es igual
si hay una o cientos de ellas.


%~ * Índices de Stegomia
\section{Índices de Stegomia}
\label{sec:densidad-vectorial-indices-stegomia}

La Organización mundial de la salud (OMS) recomienda la utlización de tres
indicadores entomológicos, generalmente conocidos como índices de Stegomia,
para estimar la densidad del vector, Índice de casas (I.C.), Índice de
Recipientes (I.R.) e Índice de Breteau (I.B.).Estos indices son calculados a
partir de muestreos de larvas y recipientes.

\subsubsection{Índice de Casas}
El índice de Casas (I.C.) es el valor numérico que especifíca el porcentaje
de viviendas infestadas con larvas, pupas o ambos estados de desarrollo
del mosquito transmisor del dengue. Para este indice, se analizan los
contenedores de las viviendas y alrededores.

\begin{equation}
I.C. = \frac{\text{número de casas infestadas} * 100}{\text{número de casas inspecionadas}}
\end{equation}

Donde :
\begin{itemize}
\item El número de casas infestadas : cantidad de casas que
cuentan con al menos un contenedor que alberga a larvas o pupas de Aedes
aegypti.
\item número de casas inspeccionadas : El total de casas analizadas para
el estudio.
\end{itemize}

Es el de uso más generalizado para la distribución de medición de
la población larvaria. Es el índice más rápido y simple para examinar la
población larval. Puede ser utilizado para proporcionar una indicación
rápida de la distribución del mosquito en una área determinada. Sus
defectos son que no tiene en cuenta el número de envases positivos por
yarda ni la productividad de esos envases.

\subsubsection{Índice de Recipientes}
El índice de Recipientes es un valor numérico que consiste en el
porcentaje de recipientes que contienen agua y están infestados con las
larvas y/ó crisálidas del mosquito transmisor del virus del dengue.
El índice del envase proporciona una indicación más detallada de la
abundancia de la población larvaria.

\begin{equation}
I.R. = \frac{\text{número de contenedores positivos} * 100}{\text{número de contenedores inspecionadas}}
\end{equation}

Donde :
\begin{itemize}
\item número de contenedores positivos : La cantidad de contenedores en los
cuales se observan larvas o pupas de Aedes aegypti.
\item número de contenedores  inspeccionadas : El total de contenedores
analizadas para el estudio.
\end{itemize}

Las encuestas larvales que utilizan el índice del envase son mucho más
lentas a realizar que las encuestas sobre el índice de la casa, pues
requieren generalmente que todos los envases en una premisa puedan ser
examinadas para las etapas no maduras y los detalles guardados de envases
positivos y negativos. El índice del envase no proporciona ninguna información
en la productividad de diversos envases.

\subsubsection{Índice de Breteau}
El índice de Breteau (I.B.) es un valor numérico que define el número de
insectos en desarrollo que se encuentran en las viviendas humanas por
la cantidad del total inspeccionado.

\begin{equation}
I.B. = \frac{\text{número de contenedores positivos} * 100}{\text{número de casas inspecionadas}}
\end{equation}
Donde :
\begin{itemize}
\item número de contenedores positivos : La cantidad de contenedores en los
cuales se observan larvas o pupas de Aedes aegypti.
\item número de casas inspecionadas : El total de casas analizadas para
el estudio.
\end{itemize}

La determinación correcta requiere de una encuesta completa de todos los
envases en una premisa que pueda hacer este tipo de ennumeración. Los
datos se utilizan para determinar el índice de la casa. Usando la
combinación del índice de Breteau y el índice de la casa, es fácil
determinar si el problema es extenso dentro de un área ó se enfoca a
unas viviendas.

\subsubsection{Probelmatica}
Las técnicas tradicionales de vigilancia de A. aegypti usan los índices
stegomia para determinar el grado de infestación, dispersión y densidad
del mosquito en una zona y tiempo determinados.

Estos índices se fundamentan en la detección visual de formas
inmaduras del vector dentro de recipientes domésticos, técnica considerada
poco sensible por la habilidad de las larvas para escapar y su capacidad de
permanecer sumergidas por largos períodos de tiempo. Asi mismo, la
proporción de viviendas y recipientes infestados con A. aegypti no provee
información fehaciente sobre la densidad poblacional al registrar como
positivo un recipiente o casa sin tener en cuenta la cantidad de formas
inmaduras presentes, lo cual quiere decir que para el índice es igual
si hay una o cientos de ellas.

\begin{itemize}
    \item Se basan en búsqueda de fases inmaduras por lo que representan
        una estimación indirecta de las poblaciones de mosquitos adultos.
    \item No reflejan la asociación que existe entre las densidades de
        mosquitos o cantidad y/o tipo de recipientes presentes, con los
        riesgos de transmisión de dengue.
    \item Proporcionan poca o nula información de aquellas viviendas en
        las que existe un mayor riesgo de presencia de mosquitos.
\end{itemize}

Además estos indicadores no reflejan las poblaciones de adultos ni estiman
riesgo entomológico, lo cual es muy importante en la transmisión del dengue.
Por lo cual, a la fecha sólo son recomendados para detectar la calidad de
las acciones (control de calidad) realizadas por el personal de control
larvario. Existen numerosos métodos e indicadores para determinar las
poblaciones de Aedes aegypti en la etapa de huevo, larva, pupa o adulto;
uno de los métodos más prácticos, eficientes y económicos es el monitoreo
de poblaciones de este vector por medio de ovitrampas. Las ovitrampas han
sido usadas desde 1965 en la vigilancia del Aedes aegypti (L), como un
instrumento para determinar la distribución del mosquito, medir la fluctuación
estacional de las poblaciones y para evaluar la eficacia de la aplicación
de insecticidas; además, como una estrategia de muestreo presencia-ausencia,
lo cual permite una estimación de la densidad mediante la proporción de
muestras positivas y son especialmente útiles para la detección temprana
de reinfestaciones. \cite{cenaprece2013}




%~ * Distribución de dispositivos de ovipostura
\section{Larvitrampas}
\label{sec:densidad-vectorial-larvitrampas}
Antes de la utilización de la larvitrampa, ésta debe cepillarse y flamearse,
luego mantenerla sumergida en agua durante no menos de tres días, para
asegurarse que el agua no contenga residuos de sustancias que puedan actuar
como larvicida. De esta manera, además, se garantiza la destrucción de
algún huevo del mosquito que estuviese previamente en el neumático o en
larvitrampas ya utilizadas.

\subsection{Especificaciones para la colocación e inspección}
Instalarla a una altura de 50 cm (del suelo a la base de la larvitrampa).
Protegerla de la luz directa del sol, el aire, la lluvia, en lugares a
media luz o completamente a la sombra. No deben ubicarse cercanas a depósitos
de agua. Debe evitarse su colocación en lugares completamente pavimentados,
u otros que tengan mucha refracción de la luz. Debe estar visible para la
hembra del mosquito. Protegerla de niños y animales domésticos (perros,
gatos, roedores, etc.)

\subsection{Forma de revisión}
Se establece una rutina semanal para revisar las larvitrampas, para lo
cual, una vez por semana debe vaciarse todo su contenido cuidadosamente
(para que no quede ninguna larva en sus paredes) en un recipiente adecuado
para realizar la inspección. En caso de ser positivas, se registra como
tal y las larvas serán colectadas en tubos para ser enviadas al laboratorio
para su determinación taxonómica. Luego, el dispositivo se lava y se acondiciona
para ser colocadas nuevamente siguiendo las especificaciones ya descritas.

\subsection{Consideración final}
Tener en cuenta que en verano, con condiciones más favorables para el
desarrollo de esta especie, las larvas pueden alcanzar el estadio de
adulto entre 6 y 7 días desde la ovipostura, por lo que es necesaria la
inspección de todas las larvitrampas en los tiempos indicados a fin de
evitar que alguna de ellas se transformen en criaderos de adultos.


%~ * Seguimiento y control de dispositivos de ovipostura
\section{Ovitrampa}
\label{sec:densidad-vectorial-ovitrampa}
Son recipientes que ofrecen a las hembras de Aedes aegypti un lugar colocar
los huevos. Detecta la presencia de huevos y por lo tanto actividad de
ovipostura. Las ovitrampas consisten en frascos de plástico o pequeñas
macetas plásticas de unos 500 ml de color oscuro preferentemente, en cuyo
interior, se coloca una pieza plana de madera (baja-lengua o similar).
Asimismo, también pueden construirse con un pote de vidrio de boca ancha,
de aprox. medio litro, pintado de negro por fuera y equipado con una paleta
de cartón o madera (baja-lengua) sujeta verticalmente al interior, con su
lado áspero mirando hacia adentro. Las dimensiones del recipiente no son
críticas pero todos los frascos a usar en un estudio particular deben ser
idénticos. Al frasco se le deberá agregar 250 ml de agua limpia.


\subsection{Especificaciones para la colocación e inspección}
La colocación debe realizarse en lugares representativos del municipio,
especialmente en las zonas donde se produjeron casos de dengue autóctonos
o importados. Respecto al número de ovitrampas a colocar, se sugiere no
menor a 10 por localidad. La idea es mantener el mismo circuito (mismos
lugares de colocación), un modelo a “escala ciudad”, para tener la idea de
la "presencia" relacionada con la distribución geográfica del vector, se
basa en el criterio que la información sea independiente. O sea que sea
improbable (más bien imposible) que una hembra pueda poner huevos en dos
ovitrampas contiguas. Además la instalación debería basarse en la capacidad
operativa de trabajo, y para ello se pueden colocar las ovitrampas en una
grilla con puntos más o menos equidistantes de aproximadamente 400 metros
de lado.

Cada ovitrampa se coloca en un lugar accesible, protegido donde predomine
la sombra y haya cierto grado de humedad (ambiente sombreado). Debe asegurarse
la presencia de moradores al retirarla.Sobre un plano de la localidad o
sector a muestrear se seleccionarán los puntos donde se colocarán las
ovitrampas. Una variante sería colocar una por Unidad Sanitaria que el
municipio posea, asumiendo que la ubicación de las mismas brindará una
visión representativa del conjunto. Conviene tener presente que en este
caso, el muestreo puede no ser representativo de viviendas regulares.

Las ovitrampas deben ser inspeccionadas semanalmente y en el caso de detectar
paletas con huevos, cuando no puedan ser leídos en el nivel local, se deberán
remitir para su lectura a los laboratorios de entomología más cercanos,
(Divisiones de Zoonosis Urbanas, División de Zoonosis Rurales, CEPAVE, etc.).
La remisión será en un sobre o bolsita plástica, con los datos para georreferenciar.
La vigilancia entomológica se debe realizar en forma continua anual. Es
importante destacar que una vez detectada la presencia de Aedes Aegypti por
cualquiera de los sistemas de monitoreo (larvitrampas u ovitrampas) se deben
realizar las acciones inmediatas de control focal en la comunidad.

\subsection{Consideración final}
Es importante añadir un identificador a cada ovitrampa que permita la
identificarlas fácilmente. El rótulo se debe colocar sobre la baja-lengua
o paleta de la ovitrampa, debe estar debidamente escrito (con lápiz) el
número y/o código de la ovitrampa. También se rotulará el frasco sobre
su pared con tinta indeleble. Se recomienda numerar cada una de las paletas
o baja-lenguas y agregarle iniciales para identificar el municipio y
detallar en el protocolo común los datos de cada una (lugar físico por.
ej. calle, barrio y zona del municipio como también la fecha del retiro
de las mismas de su lugar para su posterior envío).


%~ * Recolección de resultados
\section{Mosquitérica genérica}
Es un dispositivos de oviposturas más recientes, cuenta con un sencillo
diseño y de fácil fabricación debido a que los materiales necesarios para
su construcción son fácilmente accesibles.


  %~ * Dispositivos de ovipostura
  %~ * Distribución de dispositivos de ovipostura
  %~ * Seguimiento y control de dispositivos de ovipostura
  %~ * Recolección de resultados
  %~ * Reciclaje y reutilización de dispositivos
%~ * Identificación de focos de Riesgo
\input{capitulo-4/identificacion-focos/identificacion-focos.tex}
  %~ * Resultado del conteo de larvas
  %~ * Datos de origen
  %~ * Interpolación
  %~ * Representación
%~ * Predicción de Focos
%!TEX root = ../tesis.tex
\section{Proceso Evolutivo}
\label{sec:solucion-evolutivo}

El proceso de evolución de las muestras consiste en un proceso, en el cual
las muestras obtenidas mediante los dispositivos de ovipostura  son
expuestas a un conjunto de variaciones en un periodo de tiempo. Las
variaciones que, principalmente, afectan a las muestras son :

\begin{itemize}
    \item \em Las variaciones del clima en dicho periodo \rm: Se someten
    las muestras obtenidas a las distintas variaciones climáticas ocurridas
    en el periodo de tiempo seleccionado para el estudio.
    \item \em La naturaleza del mosquito \rm: Cada elemento de la muestra,
    es sometido a cambios considerando la naturaleza del mosquito. Los
    aspectos que se tienen en cuenta son su ciclo de vida del mosquito,
    ciclo reproductivo y el desplazamiento.
\end{itemize}


\subsection{Consideraciones}

\begin{itemize}
    \item La población contiene a todos los mosquitos (macho, hembra)
    en cualquiera de sus estados(huevo,larvas, pupa, adulto).
    \item La población inicial está compuesta por el conjunto de mediciones
    de los dispositivos de ovipostura.
    \item La salida final a interpolar van a ser larvas, salida complementaria
    es el resto de la población.
    \item Un mosquito de la población tiene los siguientes atributos :
        \begin{itemize}
            \item Sexo : Macho o hembra
            \item Edad : cantidad de días que lleva vivo el mosquito.
            \item Estado : Huevo, Larva, Pupa, Adulto.
            \item Ubicación : coordenadas longitud y latitud
            \item Madurez : el valor que indica el grado de madurez del individuo.
            \item Expectativa de vida : es un valor numérico que varía de
                acuerdo a las condiciones climáticas a las que es sometido
                el mosquito.
        \end{itemize}
   \item Periodo es el intervalo de tiempo al que será sometido la población inicial a evolución.
\end{itemize}


\subsection{Descripción de los pasos del pseudo código (ver tabla)}

1- Se itera sobre cada día contemplado dentro del periodo de estudio.
De dichos días se conocen distintas propiedas como la temperatura máxima,
media, y mínima. Además se conoce el nivel de humedad y si hubo o no
lluvias.

2- Se procesa cada mosquito dentro de la población. La población está
compuesta por el total de mosquitos registrados mediante el conteo de larvas
en los dispositivos de ovipostura.

3- A cada mosquito de la poblacion se le hace desarrollar en el día.
En otras palabras; aplicar las condiciones del día actual al mosquito. Dicho
proceso funciona como modelo de simulación de la vida de un mosquito
dada las condiciones del día actual. Para poder realizar
correctamente el proceso evolutivo es necesario tener en cuenta la
ubicación del mosquito, su estado actual (en que fase de desarrollo se
encuentra y cual es su espectativa de sobrevivencia) y su tiempo de vida
(edad actual).

4- En este paso se verifica si la expectativa de vida del mosquito ha
disminuido a 0. Lo que implica que el mosquito está muerto sea cual sea
su estado. Esto puede darse en un día muy frío o muy caluroso o por un proceso
selección natural.

5- En el caso de que el mosquito haya muerto se lo excluye de la
población de estudio

6- En caso contrario, se establece una serie de heurísticas de reproducción
para dar lugar a nuevos individuos. En el caso de que en la población exista
al menos 1 hembra y 1 macho adulto en proximidad. Datos sobre la tendencia
de ovipostura y cantidad de huevos que una hembra deposita se suman al
análisis para lograr mayor precisión

7- Se realiza una operación que en donde se utilizan variables
georeferenciadas de acuerdo a la posición del mosquito para estimar
la posibilidad de que encuentre criadero fértil para depositar sus huevos

8- Se aplica la operación de poner huevos si se aplica para el mosquito actual

9- Se añaden a la población los nuevos mosquitos. (el valor de la variable
huevos puede ser NULL)



\begin{lstlisting}[caption=Pseudocódigo del proceso evolutivo, label=a_label,  float=t]
for (dia in Periodo) {
    for( individuo in Poblacion){

        individuo.desarrollar(dia)

        if(individuo.es_viejo() or individuo.esta_muerto()){

            Poblacion.remove(individuo)
        }else if(individuo.esta_maduro(dia)){

            cambiar_estado(individuo)
        }else if(individuo.se_reproduce(dia)){

            nuevos_individuos.add(individuo.poner_huevos(dia))
        }
    }

    Poblacion.add(nuevos_individuos)
}
\end{lstlisting}


\subsection{Cálculo de la espectativa de vida}
La expectativa de vida es un valor numérico que indica la vitalida del
mosquito, esta varía de acuerdo a las condiciones climáticas a las que
es sometido el mosquito durante el proceso evolutivo. Cuando la expectativa
de vida es creo el mosquito muere.


\begin{equation}
\text{espectativa vida}(T_{i}) = \text{espectativa vida }(T_{i-1}) - \frac{1}{\text{intervalo dias vida}(T_{i}) * 24 * p}
\end{equation}
\subsection{Cálculo de la madurez}
La madurez es un valor numérico(entre 0 y 100) que varía de acuerdo
a las condiciones climáticas a las que es sometido el mosquito. Cuando la
madurez es igual a 100 el mosquito ya se encuentra listo para un cambio de
estado.

\begin{equation}
\text{madurez}(T_{i}) = \text{madurez}(T_{i-1}) + \frac{1}{\text{intervalo dias madurez}(T_{i}) * 24 * p}
\end{equation}


  %~ * Análisis predictivo
  %~ * Descripción del algoritmo de simulación
  %~ * Representación
  %~ * Resultados

%##Capítulo 5. GeoDengue
\chapter{GeoDengue}
%* Introducción
A finales de los años 90, la información geográfica era mayoritariamente tratada en
supercomputadores, usada casi siempre para mantener registros internos de administraciones y el
software que se utilizaba para su manejo era stand-alone \citep{vgomesAegis2001}. Sin embargo, con
la aparición de internet, la demanda de acceso a la información ha crecido considerablemente
obligando a los sistemas de información geográfica a modificar su paradigma para ofrecer
información de forma distribuida.

Las autoridades sanitarias, en sus tareas de vigilancia en Salud Publica, tienen en los GIS una
herramienta fundamental para conocer como se extiende una enfermedad, estudiar su posible relacion
con un potencial foco de riesgo, o localizar un brote epidemico\citep{vgomesAegis2001}. En esta
sección se presenta a GeoDengue una herramienta para identificación de focos de dengue en un
sistema de información georeferenciada.

%
El modelo solución para este problema se constituye de 4 partes fundamentales:

\begin{enumerate}[style=multiline,leftmargin=1.5cm]
    \item Establecimiento de puntos de control. Como primer paso es necesario fabricar los dispositivos de
     ovipostura (puntos de control, muestras) y distribuirlos en la zona o región que se desea analizazr. En el
     capítulo 5 se describe la estratégia implementada como mejor opción para la distribución de puntos de control.
     Además en el Anexo A se describe con detallada precisión la fabricación de los dispositivos de control;
     materiales, procedimiento, costos y recursos humanos necesarios
    
    \item Conteo. Una vez depositado el conjunto de muestras es necesario hacer un seguimiento para realizar el conteo de larvas que habitan en el dispositivo de control. Este conteo puede llevarse a cabo a los 7 días luego de que el dispositivo de ovipostura fue depositado. Es importante resaltar que un número mayor a 7 días podría significar que potenciales mosquitos hayan podido escapar del dispositivo de control por eso es necesario realizar el conteo antes de los 7 días o al séptimo día inclusive y luego vaciar el recipiente para volver a utilizarlo. 
    
    \item Procesamiento. El paso 3 y 4 constituyen básicamente el registro, procesamiento y análisis de los datos obtenidos de los conteos realizados. Ej: De un conjunto $D$ de 150 muestras, debemos tener 150 valores $z_{i}$ que representen los valores de la cantidad de larvas encontradas en cada dispositivo $d_{i}$. Al conjunto de valores $z_{i}$ se aplica un algoritmo de interpolación espacial que se encarga de generar una matriz de valores a partir de los datos de entrada. Este conjunto de valores contenidos en la matriz es utilizado para generar un mapa térmico que representa la propagación y distribución de las larvas en el contexto geoespacial.

    \item Análisis. El potencial analítico que provee la información generada es amplio y diverso. La primera alternativa de análisis y la más intuitiva es la identificación de los focos de la enfermedad. Además de ser el objetivo principal de la solución propuesta es una de las alternativas de mayor importancia ya que la identificación de focos de la enfermedad permite :
    \begin{itemize}
        \item Visualizar el estado actual y la distribución poblacional del mosquito \textit{Aedes Aegipty}.
        \item Realizar planes preventivos sobre zonas o regiones más afectadas.
        \item Implementar estrategias de fumigación y limpieza según zonas más afectadas.
        \item Otros.
    \end{itemize}
\end{enumerate}

\section{Descripción funcional}
GeoDengue es un sistema de información geográfico-sanitaria basado en una arquitectura cliente
servidor. Esta se presenta efectiva y concretamente todos los requerimientos que describen como el
software debe comportarse para cumplir con sus finalidades y objetivos. La aplicación se encuentra
dividida en sub módulos para, la gestión de datos de entrada, el simulador del proceso evolutivo y
presentación de los datos.


\subsection{Gestión de datos de entrada}
%Muestras
El sistema requiere principalmente de 2 datos de entrada, los putnos de control y los datos
climatólogicos correspondientes a el periodo de tiempo para la simulación

El área de estudio se encuentra determinada por una muestra, que representan a un conjunto de
puntos de control georeferenciados en una zona determinada. Los puntos de control que pertenecen a
una muestra deben ser establecidos en un mismo instante, siendo así la fecha de instalación y
fecha de recolección, de los puntos de control de la muestra, iguales.


Los puntos de control, como resultado de la medición realizada, permiten obtener la densidad de
larvas asociada con la información geográfica.

\section{Presentación de los datos}

%* Diseño y Arquitectura
\section{Diseño y Arquitectura}
GeoDengue está basada en una arquitectura, de tres capas, cliente-servidor, en el que las tareas
se reparten entre los proveedores de recursos o servicios, denominados servidores, y los
demandantes, llamados clientes. La primera capa, la de presentación, es la que se encarga de
interactuar con el usuario final, la segunda capa es la de negocios, esta se encarga de procesar
las solicitudes realizadas por la capa de presentación y definir las reglas que deben aplicase en
para cada solicitud. Por último, se encuentra la capa de datos, donde se almacenan los datos,
porcesan las peticiones de la capa de negocios para persistir o recuperar información.

\begin{figure}
\centering
\includegraphics[width=0.9\textwidth]{capitulo-5/graphics/arquitectura-completa.png}
\caption{\label{fig:arquitectura-completa}Arquitectura de interacción de componentes de GeoDengue.}
\end{figure}

Se optó por un enfoque web debido a la practicidad de estas aplicaciones, el usuario final solo
debe contar con un navegador web. Estas deberían funcionar igual independientemente de la versión
del sistema operativo instalado en el cliente. Las aplicaciones web son catalogadas como
aplicaciones de bajo consumo, debido a que la mayor parte de la aplicación no se encuentra en
nuestro ordenador, y muchas de las tareas de procesamiento que realiza el software no consumen
recursos nuestros porque se realizan en el servidor.

La capa de presentación se encuentra diseñada como un Single Page Application o SPA (en español
Aplicación de una sola página). Un SPA es una aplicación que se ejecuta en una única página, donde
la navegación se realiza mediante cargas parciales, sin recargar el sitio completamente. La
comunicación con el servidor se realiza mendiante peticiones Ajax, para ello se cuenta con una
arquitectura basada en servicios REST.


%* Tecnologías y herramientas utilizadas
\section{Tecnologías y herramientas utilizadas}
Como se mencionó anteriormente, se tomo la desicion de implementar la herramienta como una
aplicación web. En esta sección presentaremos las herramientas y tecnologias empleadas para el
desarrollo.

La capa de presentación fue desarrollada puramente en javascript, html y css donde la comunciación
con la capa de servicios se ecuentra dada por peticiones ajax. Como servidor web se utilizó
Apache\footnote{https://httpd.apache.org/} en su versión 2, aunque se podría utilizarce cualquier
otro servidor web HTTP.

La capa de negocios, se encuentra desarrollada en Python, en donde para la capa de servicios se
utilizó Flask\footnote{http://flask.pocoo.org/} un framework minimalista para Python. Para llevar
a cabo análisis y procesamientos complejos se utilizó la extensión de Python
NumPy\footnote{http://www.numpy.org/} que agrega mayor soporte para vectores y matrices,
constituyendo una biblioteca de funciones matemáticas de alto nivel para operar con esos vectores
o matrices. Como servidor web se utilizó Apache con su módulo
mod\_wsgi\footnote{http://www.modwsgi.org/} que proporciona una interfaz compatible con
WSGI \footnote{WSGI es una especificación para una interfaz simple y universal entre los
servidores web y aplicaciones web o frameworks para el lenguaje programación Python.}

Con respecto al almacenamiento de datos, se decidió utilizar el sistema gestor de bases de datos
PostgreSQL\footnote{http://www.postgresql.org/}, y para la manipulación de datos geográficos se utilizó PostGIS\footnote{http://postgis.net/} que es módulo que añade soporte de objetos
geográficos a PostgreSQL.


Falta hablar de geoserver como mapserver

Falta hablar de como se obtienen los datos climáticos

%\section{Algoritmos y estructuras de datos utilizadas}



\chapter{Descripción de las pruebas}
%~ * Introducción

La población total sobre la cual se realizó el estudio es de 2388 individuos
distribuídos en 85 puntos de control. Cada punto control tiene entre 0 y 133
individuos en el estado larval con edad 0; todas las larvas tienen la misma edad
en el instante inicial de las pruebas.Además, los individuos comparten
las mismas condiciones climáticas, es decir, la temperatura es la misma
para todos los puntos de control \\

\begin{figure}
\centering
\includegraphics[width=0.9\textwidth]{./capitulo-6/graphics/puntoscontroldistribuido.png}
\caption{\label{fig:distribucion-puntos}Ubicación geográfica y distribución de puntos de control.}
\end{figure}


En la \figref{fig:distribucion-puntos} se observa la distribución de los puntos de control.
Geográficamente corresponde a 4 barrios de la ciudad de Asunción, Paraguay. Los barrios son
Nazareth, Terminal, San Pablo e Hipódromo. Es importante aclarar que la ubicación geográfica
exacta no es relevante para el desarrollo de las pruebas, ya que no se utilizaron
condiciones climáticas propias de dicha ubicación. La ubicación geográfica sirve
de referencia y para poder georeferenciar resultados finales. El área de covertura
es de 2.9$km^2$ y los puntos de control se hallan ubicados uniformemente distribuidos.\\


\begin{table}
    \begin{center}
        \caption{ \label{tab:valores-formulas} Valores de parámetros en fórmulas utilizadas}
        \begin{tabular}{p{3cm} c c c c}
            \hline \\
            Variable & Simbolo & Valor & Fórmula \\
            \hline
            \hline \\
            Sitios de reproducción &  BS & 50 & ver fórmula xxx\\
            Inhibición de eclosión de huevos & $A_{0}$ & 0.5 & ver fórmula xxx\\
        \end{tabular}
    \end{center}
\end{table}

La duración total de días en cada prueba es 50, se toma como día 0 el instante
inicial del proceso evolutivo y como día final al día 49. Para observar la variación
de propiedades biológicas del mosquito Aedes Aegypti se realizaron pruebas a distintas
temperaturas, dichas temperaturas se mantienen constantes durante cada prueba y son las siguientes :
15\textcelsius , 18\textcelsius , 20\textcelsius , 22\textcelsius ,
24\textcelsius , 25\textcelsius , 26\textcelsius , 27\textcelsius ,
30\textcelsius , 34\textcelsius \\


\subsection{Descripción general de las pruebas}

Se realizaron N iteraciones del algoritmo evolutivo y se realizaron las
validaciones sobre el conjunto de datos generado por el mismo

Se presentan los resultados para tasa de desarroollo de las etapas inmaduras
(HUEVO, LARVA y PUPA) del Aedes Aegypti, mortalidad en cada una de las etapas,
la ovipostura, duracion del ciclo gonotrofico, alimentacion, promedio de vuelo y
desplazamiento y distribucion de sexos

\section{Tasas de desarrollo y mortalidad}

Como se mencionó anteriormente en la sección \ref{sec:cap-3-cambio-climatico}, las tasas de desarrollo,
supervivencia de las distintas etapas del Aedes Aegypti, son susceptibles a los efectos de la 
temperatura. 

En esta sección se presentan los resultados obtenidos, mediante el proceso el proceso evolutivo, y 
pruebas realizadas para validar el desarrollo y la mortalidad de las distintas etapas del Aedes Aegypti.

La validación de los resultados se realiza mediante la comparación de los resultados obtenidos contra los valores definidos y presentados en distintos trabajos que se utilizaron como referencia. En cada prueba se especifican las condiciones de cada prueba y el error encontrado.

En algunos casos se encontradon diferencias significativa entre los resultados obtenidos en nuestro modelo y los resultados presentados en los trabajos que se utilizaron como referencia debido a diversos factores.
Uno de los principales factores es el tipo de cepa del Aedes Aegypti utilizado, ya que existen estudios que demuestran la diferencia de desarrollo entre las diferentes cepas del mosquito(si bien el trabajo que realizamos no utiliza pruebas de laboratorio los valores utilizados para definir el tipo de mosquito corresponden a una cepa en específica). Otros factores corresponden a variables ambientales y biológicas como la alimentación, humedad, iluminación, contaminación del agua a la cual son sometidos los individuos en los laboratorios.

Por último otro factor importante es el estudio que se realiza del mosquito en su ambiente natural, ya que por ejemplo, un mosquito adulto puede vivir meses en un laboratorio pero sin en un habitat natural puede vivir desde 1 día a pocas semanas. 

\input{capitulo-6/desarrollo-mortalidad/desarrollo-huevo.tex}
\input{capitulo-6/desarrollo-mortalidad/desarrollo-larva-pupa.tex}
\input{capitulo-6/desarrollo-mortalidad/desarrollo-adulto.tex}


\section{Duración del ciclo gonotrófico}
Verificar la duración en promedio,en días, del ciclo gonotrófico. La duración promedio tiene que ser 3
días para temperaturas en promedio 30 C

\begin{itemize}
    \item Temperatura constante 
    \item Cantidad de días entre oviposturas.
    \item Promedio de duración para hembras nuliperas.
    \item Desviación estándar
    \item Calcular el error
\end{itemize}

\section{Promedio de vuelo}
Para el análisis de la distancia recorrida, en metros, del adulto de Aedes aegypti, se realizaron 
pruebas a 10 temperaturas constantes (15-34\textcelsius). En la tabla \ref{tab:pomedio-vuelo-test} se
presentan los resultados obtenidos, en general se obtuvo un promedio de 355,15 metros para distancia
recorrida, y unos 67,821 metros de dispersión, los autores \cite{cabezas2005dengue} señalan que por lo
general mosquito no sobrepasa los 50 a 100 metros durante su vida, ya que tiende a permanecer en el lugar
donde emergió.

\begin{table}
    \begin{center}
    
        \caption{ \label{tab:pomedio-vuelo-test} Análisis de la  la distancia recorrida, durante 
         el vuelo, del adulto de Aedes aegypti diez temperaturas constantes (15-34 \textcelsius).}
    
        \begin{tabular}{p{3cm} c  c }
           \hline \\
            Temperatura \textcelsius   & Distancia recorrida& Desplazamiento máximo\\
            \hline
            \hline \\
               15    &  99,09    &  42,39 \\
               18    &  392,35   &  60,17 \\
               20    &  326      &  85,62 \\
               22    &  417,82   &  77,87 \\
               24    &  367,39   &  74,71 \\
               25    &  380,41   &  67,75 \\
               26    &  427,33   &  66,45 \\
               27    &  362,21   &  65,79 \\
               30    &  389,03   &  68,51 \\
               34    &  389,92   &  68,95 \\
            General  &  355,155  &  67,821 \\
        \end{tabular}

    \end{center}
\end{table}

\section{Distribución de Sexo}
Se realizó un análisis para determinar la distribución del sexo del Aedes Aegypti. Según
\cite{otero2006stochastic}, alrededor de la mitad de los adultos emergentes son hembras,
y se define una proporción de 1.02:1 macho: hembra. Los autores de \cite{manrique1998desarrollo}
la proporción sexual promedio de adultos emergidos es de 3.00 machos por 2.75 hembras, lo cual
no representa una diferencia significativa de una relación 1:1 en las proporciones sexuales.

En general se realizaron pruebas variando la cantidad de individuos de la población, los resultados se
pueden apreciar en la tabla \ref{tab:distribucion-sexo-test}, en se observa que existe una relación 1:1
para la distribución del sexo de los mosquitos.

\begin{table}
    \begin{center}
        \caption{ \label{tab:distribucion-sexo-test} Análisis de la distribución del sexo de Aedes
        aegypti.}
        \begin{tabular}{p{3cm} c c c }
            \hline \\
            Total de & Adultos & Adultos & Relación \\
            adultos  & machos  & hembras & (macho:hembra) \\
            \hline
            \hline \\
            912    &  461    &  451    &  0,99 : 1,01 \\
            1581   &  812    &  769    &  0,97 : 1,03 \\
            4154   &  2084   &  2070   &  1    : 1 \\
            9722   &  4940   &  4782   &  0,98 : 1,02 \\
            9045   &  4472   &  4573   &  1,01 : 0,99 \\
            16248  &  8104   &  8144   &  1    : 1 \\
            30693  &  15418  &  15275  &  1    : 1 \\
            28411  &  14224  &  14187  &  1    : 1 \\
        \end{tabular}

    \end{center}
\end{table}




%~ * Descripción del ambiente de prueba/configuración
%~ * Descripción de los datos de entrada
%~ * Salida esperada
%~ * Salida obtenida
%\input{capitulo-6/salidas-obtenidas/salidas-obtenidas.tex}
%~ * Resultados

\chapter{Conclusiones y Trabajos Futuros}
Este trabajo presentó la implementación de un modelo para predecir e identificar focos de riesgo
del vector del dengue, con el fin de apoyar a la lucha preventiva de esta enfermedad. Donde el
modelo se sustenta en métodos de muestreo para la determinación de la abundancia poblacional. El
modelado e implementación de la herramienta como un sistema de información geográfica permite
realizar análisis complejos de la realidad espacial rápidamente.

El modelo cuenta con una buena base matemática de respaldo gracias a la utilización de los modelos
presentados en \cite{sharpe1977reaction, schoolfield1981non, otero2006stochastic} para el
cálculo de las tasas de desarrollo y mortalidad de las distintas etapas de desarrollo del ciclo de
vida del vector. Se considera que el modelo resultante es genérico, donde los parámetros pueden ser
ajustados para que sean aplicables en cualquier lugar.

En general se pudo observar un buen comportamiento, los resultados obtenidos solo presentan
pequeñas variaciones en comparación con los observados por expertos en laboratorio en en
condiciones controladas. Estas variaciones pueden deberse a los rasgos característicos de las
distintas cepas, utilizadas en los estudios de referencia, que permiten una mayor o menor
tolerancia a ciertas condiciones.

%Limitaciones del modelo%

Como trabajo futuro podría considerarse la extensión del modelo para incluir otras variables
con el fin de analizar su relación y el impacto de las mismas con las zonas de riesgo. Se podrían
incluir variables como: casos reportados de dengue, posibles zonas de riesgo(cementerios, patios
baldíos, etc)y la densidad poblacional del área de estudio.

Para muestras muy grandes que cuenten con una alta densidad de larvas y con un clima muy
favorable, el tiempo de respuesta crece considerablemente. Se podría paralelizar el simulador del
proceso evolutivo para optimizar el tiempo de respuesta.

Un interesante trabajo futuro sería incluir el impacto de las lluvias en la generación de sitios
de reproducción, así como el efecto de la fumigación en el desarrollo de ciclo de vida del vector.
Se pude aprovechar la información generada, por el simulador del proceso evolutivo, para realizar
una optimizacíon de las rutas de fumigación.

Dado que este trabajo se propone realizar el conteo de lavas mediante el procesamiento digital de
imágenes, sería interesante analizar otros métodos, como el presentado en
\cite{gonzalez2008segmentacion}, con el fin de mejorar la confianza y la precisión de conteo de
larvas.


\appendix   % inician los apendices de tu tesis
% los cap'itulos que incluyas a partir de aqu'i aparecen
% como ap'endices
\chapter{Especificaciones de dispositivos de ovipostura}

\section{Larvitrampas}

\subsection{Especificaciones para la colocación e inspección}
Instalarla a una altura de 50 cm (del suelo a la base de la larvitrampa). Protegerla de la luz
directa del sol, el aire, la lluvia, en lugares a media luz o completamente a la sombra. No deben
ubicarse cercanas a depósitos de agua. Debe evitarse su colocación en lugares completamente
pavimentados, u otros que tengan mucha refracción de la luz. Debe estar visible para la
hembra del mosquito. Protegerla de niños y animales domésticos (perros, gatos, roedores, etc.).

Un prerrequisito para cualquier tipo de larvitrampa en secciones de llantas es que facilite la
inspección visual del agua insitu o que transfiera fácilmente los contenidos a otro recipiente
para que sean examinados.

\subsection{Forma de revisión}
Se establece una rutina semanal para revisar las larvitrampas, para lo cual, una vez por semana
debe vaciarse todo su contenido cuidadosamente (para que no quede ninguna larva en sus paredes) en
un recipiente adecuado para realizar la inspección. En caso de ser positivas, se registra como tal
y las larvas serán colectadas en tubos para ser enviadas al laboratorio para su determinación
taxonómica. Luego, el dispositivo se lava y se acondiciona para ser colocadas nuevamente siguiendo
las especificaciones ya descritas.

Antes de la utilización de la larvitrampa, ésta debe cepillarse y flamearse, luego mantenerla
sumergida en agua durante no menos de tres días, para asegurarse que el agua no contenga residuos
de sustancias que puedan actuar como larvicida. De esta manera, además, se garantiza la
destrucción de algún huevo del mosquito que estuviese previamente en el neumático o en
larvitrampas ya utilizadas.

\subsection{Consideración final}


Tener en cuenta que en verano, con condiciones más favorables para el desarrollo de esta especie,
las larvas pueden alcanzar el estadio de adulto entre 6 y 7 días desde la ovipostura, por lo que
es necesaria la inspección de todas las larvitrampas en los tiempos indicados a fin de evitar que
alguna de ellas se transformen en criaderos de adultos.

\section{Ovitrampa}

\subsection{Especificaciones para la colocación e inspección}
La colocación debe realizarse en lugares representativos del municipio,
especialmente en las zonas donde se produjeron casos de dengue autóctonos
o importados. Respecto al número de ovitrampas a colocar, se sugiere no
menor a 10 por localidad. La idea es mantener el mismo circuito (mismos
lugares de colocación), un modelo a “escala ciudad”, para tener la idea de
la "presencia" relacionada con la distribución geográfica del vector, se
basa en el criterio que la información sea independiente. O sea que sea
improbable (más bien imposible) que una hembra pueda poner huevos en dos
ovitrampas contiguas. Además la instalación debería basarse en la capacidad
operativa de trabajo, y para ello se pueden colocar las ovitrampas en una
grilla con puntos más o menos equidistantes de aproximadamente 400 metros
de lado.

Cada ovitrampa se coloca en un lugar accesible, protegido donde predomine
la sombra y haya cierto grado de humedad (ambiente sombreado). Debe asegurarse
la presencia de moradores al retirarla.Sobre un plano de la localidad o
sector a muestrear se seleccionarán los puntos donde se colocarán las
ovitrampas. Una variante sería colocar una por Unidad Sanitaria que el
municipio posea, asumiendo que la ubicación de las mismas brindará una
visión representativa del conjunto. Conviene tener presente que en este
caso, el muestreo puede no ser representativo de viviendas regulares.

Las ovitrampas deben ser inspeccionadas semanalmente y en el caso de detectar
paletas con huevos, cuando no puedan ser leídos en el nivel local, se deberán
remitir para su lectura a los laboratorios de entomología más cercanos,
(Divisiones de Zoonosis Urbanas, División de Zoonosis Rurales, CEPAVE, etc.).
La remisión será en un sobre o bolsita plástica, con los datos para georreferenciar.
La vigilancia entomológica se debe realizar en forma continua anual. Es
importante destacar que una vez detectada la presencia de Aedes Aegypti por
cualquiera de los sistemas de monitoreo (larvitrampas u ovitrampas) se deben
realizar las acciones inmediatas de control focal en la comunidad.

\subsection{Consideración final}
Es importante añadir un identificador a cada ovitrampa que permita la
identificarlas fácilmente. El rótulo se debe colocar sobre la baja-lengua
o paleta de la ovitrampa, debe estar debidamente escrito (con lápiz) el
número y/o código de la ovitrampa. También se rotulará el frasco sobre
su pared con tinta indeleble. Se recomienda numerar cada una de las paletas
o baja-lenguas y agregarle iniciales para identificar el municipio y
detallar en el protocolo común los datos de cada una (lugar físico por.
ej. calle, barrio y zona del municipio como también la fecha del retiro
de las mismas de su lugar para su posterior envío).

\section{Mosquitérica genérica}

Los materiales para la construcción son los siguientes :
\begin{itemize}
    \item Una botella de plástico de 2 litros.
    \item Tijeras.
    \item Una lija para madera.
    \item Un rollo de cinta aislante.
    \item Una pieza de tul o gasa para sellar la boquilla (cuello).
    \item Un poco de arroz o alpiste.
    \item 250 ml de agua no clorada.
\end{itemize}

Hay que cortar el cilindro en dos partes, de modo que la porción de la
boca quede en forma contraria, formando un embudo. Debe retirarse el tapón
de la botella. Asimismo, se debe retirar con cuidado el anillo de precinto
y almacenarlo, también se utilizará en la mosquitérica.

Se debe lijar bien dentro del "embudo". Esto servirá para aumentar el área
de evaporación, y será más fácil para el mosquito localizar la  trampa.
Hay que colocar la  tela en el cuello y asegurarlo con el anillo. Hay que
tener en cuenta que debe existir un tejido muy fino, de modo que las larvas
no puedan pasar;


Hay que colocar en la parte inferior de la botella la comida que se eligió,
puede ser granos de alpiste o tres granos de arroz, pero siempre triturados.
Se inserta la porción de cuello hacia abajo sobre la parte inferior de la
botella; Se usa cinta adhesiva para asegurar las dos partes, sobre el lado
exterior.

Se pone el agua sin cloro en la mosquitérica a pocos centímetros por encima
del cuello. Si no dispone de agua sin cloro, se usa agua de tomar del grifo
y se deja que repose durante dos días.

\subsection{Especificaciones para la colocación e inspección}
Las técnicas de colocación e inspección mencionadas anteriormente se aplican
a este método.

\subsection{Consideración final}
Actualmente la mosquitérica es utilizada en reemplazo a los antiguos
dispositivos de ovipostura en los últimos trabajos realizados en sudamérica
para el control del vector del dengue por presentar resultados más exactos.

\chapter{Lecciones aprendidas}
Durante el periodo de investigación y desarrollo de este proyecto se analizaron varias
alternativas, para llegar al diseño y construcción de un simulador del proceso evolutivo de la
ecología del Aedes aegypti.

\section{Primeros pasos y enfoques adoptados}
Uno de los primeros enfoques adoptados, con el fin diseñar un modelo que permita predecir focos de
dengue, se encontraba orientado a recolectar un conjunto de variables para su posterior análisis,
con el fin de encontrar la relación de las mismas con un posible foco de dengue. Entre las
variables básicas, con las que se debían contar para llevar al cabo el análisis se encontraban :

\begin{itemize}
    \item \textit{Registros de casos de dengue} : consiste en historial de los casos reportados, confirmados, sospechosos, y fatales de dengue registrados.

    \item \textit{Datos climatológicos} : historial de datos climatológicos del Paraguay.
\end{itemize}

Se pretendía utilizar técnicas de regresión lineal, para buscar una relación entre las variaciones
climáticas y los casos de dengue, para su posterior presentación en un SIG. Dado que el contexto
en el cual se deseaba analizar los datos era el de un SIG, se reemplazó la regresión lineal por la
regresión ponderada geográficamente que resultaba más adecuada para el estudio.

Teniendo en cuenta que las autoridades sanitarias del Paraguay no cuentan con datos computables,
geográficamente, relacionados con a casos de dengue que permitan realizar los análisis
estadísticos y espaciales correspondientes, se tuvo que optar por otro enfoque que permita
realizar el análisis.

Un segundo enfoque, consistió en reemplazar el registro de casos de dengue por un conjunto de
variables que permitan caracterizar una zona como más o menos riesgosa. Entre estas variables
tenemos:

\begin{itemize}
    \item \textit{Índices de infestación} : Correspondiente a un historial de los índices de infestación observados. Actualmente, las autoridades sanitarias los utilizan para asociar un nivel de riesgo a una zona.

    \item \textit{Patios baldíos} : Registro de patios baldíos correspondientes al Paraguay. Estos son considerados como uno de los mayores criaderos de dengue, ya que normalmente se caracterizan por encontrase en estado de abandono y en algunos casos cumplen la función de vertederos informales.

    \item \textit{Aglomeración de personas} : Registro de lugares con grandes aglomeración de personas como el mercado, estadios, universidades, escuelas etc. Se consideró esta variable atendiendo que las hembras del Aedes aegypti son atraídas por el olor del dióxido de carbono que exhalan los seres humanos. De modo que los lugares con gran aglomeración de personas generan mayor cantidad de dióxido de carbono, y por ende resultan más atractivas para las hembras del Aedes aegypti.
\end{itemize}


Nuevamente hay que tener en cuenta que no se cuentan con datos computables, geográficamente, que
permitan realizar el análisis.

\section{Problemática y limitaciones}

El principal problema con los primero enfoques, mencionados anteriormente, es que no existen datos
computables para realizar el análisis. Por lo que para realizar un análisis válido, de forma a
determinar la relación existente entre las variables, se debe realizar un trabajo previo para
la recolección de los datos.

Para la recolección de datos, primeramente, se deben diseñar e implementar las herramientas
correspondientes, así como las metodologías para la migración de los datos existentes y las
políticas de utilización de dichas herramientas. Esto implica un trabajo institucional con un alto
costo y cuya ejecución podría llevar un tiempo considerable. Se considera que dicho requerimiento
previo es una limitante para la realización de un análisis válido.


\section{Selección del enfoque final}
La selección del enfoque final, parte de la necesidad de contar con nuevas metodologías que
permitan generar información para el análisis sin la necesidad de grandes requerimientos previos
para realizar estimaciones válidas.

Las autoridades sanitarias, del Paraguay, llevan a cabo acciones para la vigilancia entomológica,
con el fin monitorear la densidad vectorial en zonas endémicas y no endémicas, mediante técnicas
basadas en utilización de indices tradicionales, donde estas son consideradas como una pobre
indicación de la producción de mosquitos adultos, debido a que no reflejan la asociación que
existe entre las densidades de mosquitos y tipo de recipientes presentes, con los riesgos de
transmisión de dengue, además no se puede medir la productividad del recipiente y en consecuencia
se proporciona poca o nula información de aquellas viviendas en las que existe un mayor riesgo de
presencia de mosquitos. Actualmente existen numerosos métodos e indicadores más prácticos,
eficientes y económicos para determinar las poblaciones de Aedes aegypti, como larvitrampas y
ovitrampas.

Los métodos de muestreo, como larvitrampas y ovitrampas, permiten obtener información regionalizada
correspondiente a un área determinada, donde esta información, puede ser combinada con información
ambiental, demográfica o epidemiológica, con el fin de obtener modelos detallados que tengan la
capacidad de monitorear, simular el comportamiento del vector y en consecuencia, predecir una
posible epidemia del dengue.

Teniendo en cuenta que la utilización de las larvitrampas u ovitrampas, no cuenta con mayores
requisitos, más que su construcción, instalación y revisión, se optó por utilizarlas como punto
de partida para el enfoque final. La información obtenida es utilizada como entrada para
modelos térmicos que permiten calcular la tasa de desarrollo y mortalidad del Aedes aegypti en sus
etapas de desarrollo (huevo, larva, pupa y adulto). Esta información combinada con datos
climatológicos permite simular el ciclo de vida del vector, mediante el cálculo de las tasas de
desarrollo y mortalidad, motivo por el cual se optó por la construcción de un simulador del
proceso de la ecología del Aedes aegypti.

% estos comandos generan la bilbiograf'ia
\printbibliography

\end{document}
