\subsection{Adulto}
Como se menciona en la \secref{subsec:cap4-mortalidad}, la tasa de la mortalidad utilizada fue
la definida en \cite{otero2006stochastic}, como una constante de valor 0.09 independiente de la
temperatura. En la \tabref{tab:mortalidad-periodo-adulto-test} se presentan los resultados
obtenidos de la tasa de mortalidad del Aedes aegypti a diez temperaturas constantes (18-34
\textcelsius). En general se  se obtuvo una mortalidad, diaria de $11,08$\% en comparación al 9\%
definido en \cite{otero2006stochastic} y el 10\% señalado en \cite{ThironIzcazaJ2003}.

Los autores de \cite{ThironIzcazaJ2003} señalan que la mortalidad de una población de adultos, se
encuentra entre el 50\% a los 7 días y un 95\% a los 30 días. En la
\tabref{tab:mortalidad-periodo-adulto-test} se presenta los resultados correspondientes a la
mortalidad de los adultos a 7 y 30 días. En general se obtuvo un $54,96$\% a los 7 días y un
$96,66$\% a los 30 días, en comparación al 50\% y el 95\% señalados por señalado por los autores
de \cite{ThironIzcazaJ2003}.

\begin{table}[!htbp]
    \begin{minipage}{\textwidth}
        \centering
        \caption{\label{tab:mortalidad-periodo-adulto-test} Análisis de la tasa de mortalidad del adulto del Aedes aegypti a diez temperaturas constantes (15-34 \textcelsius).}
        \begin{tabular}{p{3cm} c c c }
            \hline \\
            Temperatura  & Mortalidad  & Mortalidad & Mortalidad\\
            \textcelsius & diaria (\%) & 7 días(\%) & 30 días(\%)\\
            \hline
            \hline \\
            15  &  --$^a$ & --$^a$ & --$^a$ \\
            18  &  10,43 & 56,08 &  82,43\\
            20  &  11,07 & 57,89 &  96,58\\
            22  &  12,10 & 54,03 &  98,07\\
            24  &  11,23 & 54,55 &  97,43\\
            25  &  11,14 & 54,67 &  97,35\\
            26  &  11,47 & 55,10 &  97,56\\
            27  &  11,12 & 54,99 &  97,62\\
            30  &  10,89 & 56,72 &  97,07\\
            34  &  11,74 & 56,72 &  98,12\\
            Media general & 10,88 & 54,96 &  96,66\\
        \end{tabular}
        \footnotetext[1]{No se observaron adultos a 15 \textcelsius.}
    \end{minipage}
\end{table}

Los errores obtenidos se originan durante el redondeo realizado al aplicar la tasa de mortalidad
diaria en cada colonia, debido a que el modelo, a la hora de calcular la cantidad de adultos a ser
eliminados de la colonia a la que pertenecen, solo permite eliminar a un número entero de adultos.
Si contamos con un grupo 10 de colonias cada una con 6 adultos cada una, tenemos un total de 60
adultos.Aplicando la tasa de mortalidad $0,09$ en cada colonia se obtiene un total de $0,54$,
aplicando el operador de redondeo, definido por la ecuación \eqref{eq:operador-redondeo}, es de $1$
adulto por colonia. Al eliminar un adulto por colonia, en total se eliminarán 10 adultos en
comparación al $5,4$ que se obtendría al eliminar $0,54$ adultos por colonia. En la
\tabref{tab:mortalidad-adulto-error} se presentan los resultados del análisis realizado.

\begin{table}[!htbp]
    \begin{minipage}{\textwidth}
        \centering
        \caption{ \label{tab:mortalidad-adulto-error}Ejemplo del error generado por la aplicación
        del operador de redondeo en tasa de mortalidad del adulto del Aedes aegypti.}
        \begin{tabular}{l c c c }
            \hline \\
            Número de & Cantidad  & Mortalidad      & Mortalidad \\
            colonia   &de adultos & porcentual$^{a}$ & entera$^{b}$\\
            \hline
            \hline \\
            1       & 6  & 0,54 & 1\\
            2       & 6  & 0,54 & 1\\
            3       & 6  & 0,54 & 1\\
            4       & 6  & 0,54 & 1\\
            5       & 6  & 0,54 & 1\\
            6       & 6  & 0,54 & 1\\
            7       & 6  & 0,54 & 1\\
            8       & 6  & 0,54 & 1\\
            9       & 6  & 0,54 & 1\\
            10      & 6  & 0,54 & 1\\
            Media general   & 60 & 5,4  & 10\\
        \end{tabular}
        \footnotetext[1]{Cantidad de adultos a eliminar al aplicar la de mortalidad de $0,09$definida en \cite{otero2006stochastic}.}
        \footnotetext[2]{Cantidad de adultos a eliminar aplicando el operador de redondeo definido por la ecuación \eqref{eq:operador-redondeo}.}
    \end{minipage}
\end{table}
