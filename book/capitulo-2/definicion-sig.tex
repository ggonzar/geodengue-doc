\section{Definición de Sistema de información geográfica}
\label{sec:cap2-definicion-sig}

Un Sistema de Información geográfica o SIG (por sus siglas en español), es la integración
organizada de hardware, software y datos geográficos diseñada para almacenar, manejar, capturar,
analizar y desplegar la información geográficamente de múltiples formas, con el fin de resolver
problemas de planificación y gestión geográfica\cite{lopezMarcos2007}. Se basan en los principios
formales de matemáticas discretas, modelos de datos y geometría computacional; su desarrollo, en
nuevas tecnologías de la información: estándares e ingeniería de software, almacenes de datos,
SIG Web, metadatos, ambientes y lenguajes visuales entre muchas otras\cite{lunaPaulina2010}.

La característica principal de los SIG es el manejo de datos complejos basados en datos
geométricos (coordenadas e información topológica) y datos de atributos (información nominal) la
cual describe las propiedades de los objetos geométricos tales como punto, lineas y polígonos. En
la actualidad las funciones básicas, y más habitualmente utilizadas de un SIG son el
almacenamiento, visualización, consulta y análisis de datos espaciales \cite{fAlonsoSig2006}. Un
uso más avanzado sería la utilización de un SIG para la toma de decisiones en ordenación
territorial o para el modelado de procesos ambientales. A continuación se describen brevemente
estas funciones básicas del SIG.

\begin{itemize}
    \item \textit{Almacenamiento}: El almacenamiento de datos espaciales implica modelar la
    realidad y codificar de forma cuantitativa este modelo.

    \item \textit{Visualización} : La información se presenta en un espacio de cuatro dimensiones,
    tres de ellas espaciales y una cuarta correspondiente al tiempo. Sin embargo, debido al peso
    que la tradición cartográfica tiene sobres los SIG, una de las formas prioritarias de
    presentación de los datos es en su proyección sobre el espacio bidimensional definido mediante
    coordenadas cartesianas.

    \item \textit{Consultas} : En una base de datos convencional las consulta se basan en
    propiedades temáticas, mientras que en un SIG estas se basan tanto en atributos temáticos como
    en propiedades espaciales.

    \item \textit{Análisis} : El uso de herramientas de análisis espacial y álgebra de mapas para
    el desarrollo y verificación de hipótesis acerca de la distribución espacial de las variables
    y objetos.

    \item \textit{Toma de decisiones} : La utilización de un SIG para analizar y resolver
    problemas de toma de decisiones en la planificación física, ordenación territorial, estudios
    de impacto ambiental, entre los más resaltantes.

    \item \textit{Modelado} : Las aplicaciones más elaboradas de los SIG son aquellas
    relacionadas con la integración de modelos matemáticos de procesos naturales, dinámicos y
    espacialmente distribuidos.
\end{itemize}
