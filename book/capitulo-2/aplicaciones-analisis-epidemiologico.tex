\section{Aplicaciones y análisis epidemiológico}
\label{sec:cap2-aplicaciones-analisis-epidemiologico}

El año 1854 marcó un hito para la epidemiología moderna, en \cite{jCerdaJohnSnow2007} se describe
como el Dr. John Snow, pionero de la epidemiología, cartografió la incidencia de los casos de
cólera en el distrito de Soho en Londres, permitiéndole así localizar con precisión un pozo de
agua contaminado como la fuente causante del brote de cólera. Actualmente, el Dr. John Snow, es
considerado como el padre de la epidemiología moderna por introducir el uso de  metodologías de
investigación epidemiológica como, la implementación de encuestas y la epidemiología espacial
\cite{jCerdaJohnSnow2007}.

El uso de los SIG en el campo de la salud pública es muy reciente \cite{martinez2001sigepi}. En
1993 la Organización Mundial de la Salud propuso utilizar los SIG para elaborar pronósticos
sobre varias patologías de fuerte carga ambiental\cite{curto2003aplicacion}. Según
\cite{martinez2001sigepi}, entre algunos de los usos más comunes podemos encontrar: la
determinación de la situación de salud en un área, la generación y análisis de hipótesis de
investigación, la identificación de grupos de alto riesgo a la salud, la planificación y
programación de actividades y el monitoreo y la evaluación de intervenciones.

Los SIG son capaces de simplificar grandes tareas como la localización de eventos en espacio y
tiempo, el monitoreo de eventos de salud y el comportamiento de factores de riesgo en un período
de tiempo dado, la identificación de áreas geográficas y grupos de población con grandes
necesidades de salud y contribuye a la solución de tales necesidades mediante el análisis de
múltiples variables y la evaluación del impacto de intervenciones en salud
\cite{martinez2001sigepi, iMolinaSigEpidemiologia}. La capacidad de superponer la localización de
los casos de una enfermedad como puntos con información espacial relacionada es una herramienta de
considerable poder \cite{iMolinaSigEpidemiologia}.

La aplicación de los SIG, en el área de epidemiología, se encuentra orientada a problemas de salud
pública, tales como : estimar el riesgo de contraer fasciolasis\footnote{Fasciolasis es una
enfermedad parasitaria causada por dos especies de trematodos digéneos, Fasciola hepatica y
Fasciola gigantica.}\cite{zukowski1993fasciolosis}, el monitoreo en tiempo y espacio de la
enfermedad de Chagas\footnote{La enfermedad de Chagas es una enfermedad parasitaria tropical,
generalmente crónica, causada por el protozoo flagelado Trypanosoma cruzi.}\cite{rogers1993monitoring} y la vigilancia entomológica de enfermedades transmitidas por vectores,
tales como malaria, dengue y borreliosis\footnote{Borreliosis de Lyme, es una enfermedad
infecciosa que afecta varios órganos del ser humano, causada por la espiroqueta Borrelia
burgdorferi, que es trasmitida por las garrapatas.}\cite{su1994framework, nino2008uso, albierispatial, beck1994remote, kitron1994geographic}.

El potencial analítico y la capacidad combinatoria de la información proveniente de, registros
de mortalidad, hospitales, bases de datos oficiales de las entidades de salud, observatorios
meteorológicos y proyectos específicos orientados a la salud, permiten implementar infinitos tipos
de análisis.
