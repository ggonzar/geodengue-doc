\section{Análisis espacial en un SIG}
\label{sec:cap2-analisis-espacial-sig}
Dada la amplia gama de técnicas de análisis espacial que se han desarrollado durante
el último medio siglo, cualquier resumen o revisión sólo puede cubrir el tema a una
profundidad limitada \citep{rojas2012ejecucion}.

El análisis en un SIG puede definirse como el proceso de transformación de los datos geográficos
en información útil para un problema determinado, teniendo como finalidad modelar fenómenos
geográficos, las asociaciones y relaciones existentes en la información geográfica almacenada. Se
encuentra determinada por la existencia de relaciones topológicas entre los elementos y permite
realizar cálculos entre variables y obtener así nuevos datos \citep{bravo2000breve}.

